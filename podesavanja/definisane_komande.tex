

\def\chaptername{}
\def\bibname{Literatura}

\newcommand\kckod[1]{\texttt{#1}}        % kratak c kod
\newcommand\argf[1]{\texttt{#1}}         % argument funkcije
\newcommand\komentar[1]{\textcolor{red}{#1}}         % komentar

\numberwithin{Answer}{chapter}
\numberwithin{Exercise}{chapter}

%Detalji se mogu naci na
%https://en.wikibooks.org/wiki/LaTeX/Source_Code_Listings
\definecolor{mygreen}{rgb}{0,0.6,0} 
\definecolor{mygray}{rgb}{0.5,0.5,0.5} 
\definecolor{mymauve}{rgb}{0.58,0,0.82}

%https://en.wikibooks.org/wiki/LaTeX/Source_Code_Listings
\lstset{ 
backgroundcolor=\color{white},   % choose the background color; you must add \usepackage{color} or \usepackage{xcolor}
 basicstyle=\footnotesize,        % the size of the fonts that are used for the code
 breakatwhitespace=false,         % sets if automatic breaks should only happen at whitespace
 breaklines=true,                 % sets automatic line breaking
 captionpos=b,                    % sets the caption-position to bottom
 commentstyle=\color{mygreen},    % comment style
 escapeinside={\%*}{*)},          % if you want to add LaTeX within your code
 extendedchars=true,              % lets you use non-ASCII characters; for 8-bits encodings only, does not work with UTF-8
 frame=single,	                   % adds a frame around the code
 keepspaces=true,                 % keeps spaces in text, useful for keeping indentation of code (possibly needs columns=flexible)
 keywordstyle=\color{blue},       % keyword style
 language=C,                 % the language of the code
 numbers=left,                    % where to put the line-numbers; possible values are (none, left, right)
 numbersep=5pt,                   % how far the line-numbers are from the code
 numberstyle=\tiny\color{mygray}, % the style that is used for the line-numbers
 rulecolor=\color{black},         % if not set, the frame-color may be changed on line-breaks within not-black text (e.g. comments (green here))
 showspaces=false,                % show spaces everywhere adding particular underscores; it overrides 'showstringspaces'
 showstringspaces=false,          % underline spaces within strings only
 showtabs=false,                  % show tabs within strings adding particular underscores
 stepnumber=2,                    % the step between two line-numbers. If it's 1, each line will be numbered
 stringstyle=\color{mymauve},     % string literal style
 tabsize=2,	                   % sets default tabsize to 2 spaces
}





\lstdefinestyle{customc}{
 belowcaptionskip=1\baselineskip,
 breaklines=true,
 xleftmargin=\parindent,
 language=C,
 showstringspaces=false,
 commentstyle=\color{blue}\ttfamily,
 morecomment=[l][\color{magenta}]{\#},
 basicstyle=\footnotesize\ttfamily,
 keywordstyle=\bfseries\color{green!40!black}\ttfamily\textbf,
 identifierstyle=\bfseries\color{black}\ttfamily,
 stringstyle=\bfseries\color{magenta}\ttfamily,
}

\newcommand{\includecode}[1]{\lstinputlisting[language=C,style=customc]{#1}}

\def\d{{\fontencoding{T1}\selectfont\dj}}
\def\D{{\fontencoding{T1}\selectfont\DJ}}

\lstset{literate=
 {č}{{\v{c}}}1 {ć}{{\'{c}}}1 {š}{{\v{s}}}1 {ž}{{\v{z}}}1 {đ}{{\d}}1 
 {Č}{{\v{C}}}1 {Ć}{{\'{C}}}1 {Š}{{\v{S}}}1 {Ž}{{\v{Z}}}1 {Đ}{{\D}}1  
}


\newcommand{\linkresenje}[1]{\ifpdf \hfill <Rešenje \refAnswer{#1}> \fi}

\newenvironment{ckod}
{\verbatim}
{\endverbatim}

\lstnewenvironment{test}[1]
{ \renewcommand\lstlistingname{Test}\lstset{
% belowcaptionskip=1\baselineskip,
% breaklines=true,
 frame=L,
 backgroundcolor=\color{white},
% xleftmargin=\parindent,
 language=C,
 showstringspaces=false,
 basicstyle=\footnotesize\ttfamily,
 keywordstyle=\bfseries\color{green!40!black},
 commentstyle=\itshape\color{purple!40!black},
 identifierstyle=\color{blue},
 stringstyle=\color{orange},
 numbers=none,
 captionpos=t,
 title={#1}
}
}
{}

\newenvironment{minitest}
{\begin{minipage}[t]{45mm}} 
{\end{minipage}}

\newenvironment{miditest}
{\begin{minipage}[t]{70mm}}
{\end{minipage}}

\newenvironment{maxitest}
{\begin{minipage}[t]{150mm}}
{\end{minipage}}


\def\ExerciseName{Zadatak }
\def\AnswerName{Rešenje }

\def\DifficultyMarker{*}

\renewcommand{\ExerciseHeader}{{\textbf{
\ExerciseHeaderDifficulty \ExerciseName \ExerciseHeaderNB \ExerciseHeaderTitle 
\ExerciseHeaderOrigin}}}

\renewcommand{\AnswerHeader}{{\textbf{
\AnswerName \ExerciseHeaderNB \ExerciseHeaderTitle
\medskip}}}


\RecustomVerbatimCommand{\VerbatimInput}{VerbatimInput}%
{fontsize=\footnotesize,
 %
 frame=lines,  % top and bottom rule only
 framesep=1em, % separation between frame and text
 rulecolor=\color{gray},
 %
% label=\fbox{\color{black}data.txt},
% labelposition=topline,
 %
% commandchars=\|\(\), % escape character and argument delimiters for
                      % commands within the verbatim
 commentchar=*        % comment character
}


\setenumerate[0]{label=(\alph*)}