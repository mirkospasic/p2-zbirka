
\chapter{Pretraživanje}

\section{Zadaci}

%-----------------------------------------------------------------
%-----------------------------------------------------------------
\begin{Exercise}[label=401]
  U datoteci koja se zadaje kao argument komandne linije, nalaze se
  koordinate tačaka. U zavisnosti od prisustva opcija komandne linije
  (\argf{-x} ili \argf{-y}), pronaći onu koja je najbliža \argf{x}
  (ili \argf{y}) osi, ili koordinatnom početku, ako nije prisutna
  nijedna opcija. Broj tačaka u datoteci nije unapred poznat.
  
\begin{miditest}
\begin{test}{Test 1}
Poziv: ./a.out dat.txt -x
Datoteka:
12 53
2.342 34.1
-0.3 23
-1 23.1
123.5 756.12
Izlaz: -0.3 23
\end{test}
\end{miditest}
  
\end{Exercise}

\begin{Answer}[ref=401]
  \includecode{resenja/04_Pretrazivanje/401.c}
\end{Answer}
%-----------------------------------------------------------------
%-----------------------------------------------------------------
\begin{Exercise}[label=402]
  Napisati funkciju koja određuje nulu funkcije \kckod{cos(x)} na
  intervalu \argf{[0,2]} metodom polovljenja intervala (napomena:
  analogija sa binarnom pretragom). Algoritam se završava kada se
  vrednost kosinusne funkcije razlikuje za najviše 0.001 od 0.
  
\begin{minitest}
\begin{test}{Test 1}
Izlaz:
1.571289
\end{test}
\end{minitest}
  
\end{Exercise}

%\begin{Answer}[ref=402]
%  \includecode{resenja/04_Pretrazivanje/402.c}
%\end{Answer}
%-----------------------------------------------------------------
%-----------------------------------------------------------------
\begin{Exercise}[label=403]
  Napisati funkciju koja u sortiranom nizu nalazi prvi element veći od
  0 (napomena: primeniti binarnu pretragu). Napisati i program koji
  testira ovu funkciju za niz elemenata koji se zadaju kao argumenti
  komandne linije.
  
\begin{miditest}
\begin{test}{Test 1}
Poziv:  ./a.out -43 -24 -5 -2 1 4 6 12
Izlaz:  1
\end{test}
\end{miditest}

\begin{miditest}
\begin{test}{Test 2}
Poziv:  ./a.out -32 4 65 123
Izlaz:  4
\end{test}
\end{miditest}
  
\end{Exercise}

%\begin{Answer}[ref=403]
%  \includecode{resenja/04_Pretrazivanje/403.c}
%\end{Answer}
%-----------------------------------------------------------------
%-----------------------------------------------------------------
\begin{Exercise}[label=404]
  Napisati funkciju koja rekurzivno implementira algoritam
  interpolacione pretrage i program koji ovu funkciju testira za
  brojeve koji se unose sa standardnog ulaza. Pretpostaviti da niz
  brojeva koji se unosi neće biti duži od 1024 elemenata. Prvo se
  unosi broj koji se traži, a zatim sortirani elementi niza sve do
  kraja ulaza.
  
\begin{miditest}
\begin{test}{Test 1}
Ulaz:   11 2 5 6 8 10 11 23
Izlaz:  5
\end{test}
\end{miditest}

\begin{miditest}
\begin{test}{Test 2}
Ulaz:   14 10 32 35 43 66 89 100
Izlaz:  -1  
\end{test}
\end{miditest}
  
\end{Exercise}

%\begin{Answer}[ref=404]
%  \includecode{resenja/04_Pretrazivanje/404.c}
%\end{Answer}
%-----------------------------------------------------------------
%-----------------------------------------------------------------
\begin{Exercise}[label=405]
  Napisati funkciju koja određuje ceo deo logaritma za osnovu 2 datog
  neoznačenog celog broja, koristeći samo bitske i relacione
  operatore.
  \begin{enumerate}
  \item Napisati varijantu sa pomeranjem broja udesno dok ne postane 0
    (linearna složenost).
  \item Napisati varijantu sa binarnom pretragom (logaritamska složenost).
  \end{enumerate}
  
\begin{minitest}
\begin{test}{Test 1}
Ulaz:       10
Izlaz:      3
\end{test}
\end{minitest}
\begin{minitest}
\begin{test}{Test 2}
Ulaz:       4
Izlaz:      2
\end{test}
\end{minitest}
\begin{minitest}
\begin{test}{Test 3}
Ulaz:       17
Izlaz:      4
\end{test}
\end{minitest}

\begin{minitest}
\begin{test}{Test 4}
Ulaz:       1031
Izlaz:      10
\end{test}
\end{minitest}
  
\end{Exercise}

%\begin{Answer}[ref=405]
%  \includecode{resenja/04_Pretrazivanje/405.c}
%\end{Answer}
%-----------------------------------------------------------------
%-----------------------------------------------------------------
\begin{Exercise}[label=406]
  U prvom kvadrantu dato je $1 \leq \argf{N} \leq 10000$ duži svojim
  koordinatama (duži mogu da se seku, preklapaju, itd.). Napisati
  program koji pronalazi najmanji ugao $0 \leq \alpha \leq 90^\circ$,
  na dve decimale, takav da je suma dužina duži sa obe strane
  polupoluprave iz koordinatnog početka pod uglom $\alpha$ jednak
  (neke duži bivaju presečene, a neke ne). Program prvo učitava broj
  \argf{N}, a zatim i same koordinate temena duži. Uputstvo: vršiti
  binarnu pretragu intervala $[0, 90^\circ]$.
  
\begin{minitest}
\begin{test}{Test 1}
Ulaz:
2
2 0 2 1
1 2 2 2
Izlaz:
26.57 
\end{test}
\end{minitest}
  
\end{Exercise}

%\begin{Answer}[ref=406]
%  \includecode{resenja/04_Pretrazivanje/406.c}
%\end{Answer}
%-----------------------------------------------------------------
%-----------------------------------------------------------------
\begin{Exercise}[label=407]
  Napisati program u kome se prvo inicijalizuje statički niz struktura
  osoba sa članovima ime i prezime (uređen u rastućem poretku
  prezimena) sa manje od 10 elemenata, a zatim se učitava jedan
  karakter i pronalazi (bibliotečkom funkcijom \verb|bsearch|) i
  štampa jedna struktura iz niza osoba čije prezime počinje tim
  karakterom. Ako takva osoba ne postoji, štampati -1 na standardni
  izlaz.
\begin{ckod}
Osoba niz_osoba[]={{"Mika", "Antic"},
                   {"Dobrica", "Eric"},
                   {"Desanka", "Maksimovic"},
                   {"Dusko", "Radovic"},
                   {"Ljubivoje", "Rsumovic"}};
\end{ckod}
  
\begin{minitest}
\begin{test}{Test 1}
Ulaz:  R
Izlaz: Dusko Radovic
\end{test}
\end{minitest}
  
\end{Exercise}

%\begin{Answer}[ref=407]
%  \includecode{resenja/04_Pretrazivanje/407.c}
%\end{Answer}
%-----------------------------------------------------------------
%-----------------------------------------------------------------

\section{Rešenja}

\shipoutAnswer


