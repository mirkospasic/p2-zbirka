
\chapter{Pretraživanje}

\section{Zadaci}

%-----------------------------------------------------------------
%-----------------------------------------------------------------
\begin{Exercise}[label=401]
  U datoteci koja se zadaje kao argument komandne linije, nalaze se
  koordinate ta\v caka. U zavisnosti od prisustva opcija komandne
  linije ($-x$, $-y$), prona\' ci onu koja je najbli\v za $x$ ($y$)
  osi, ili koordinatnom po\v cetku, ako nije prisutna nijedna
  opcija. Broj ta\v caka u datoteci nije unapred poznat.
  
  \begin{miditest}
    \begin{test}{Test 1}
      Poziv: ./a.out dat.txt -x
      Datoteka:
      12 53
      2.342 34.1
      -0.3 23
      -1 23.1
      123.5 756.12
      Izlaz: -0.3 23
    \end{test}
  \end{miditest}
  
\end{Exercise}

\begin{Answer}[ref=401]
  \includecode{resenja/04_Pretrazivanje/401.c}
\end{Answer}
%-----------------------------------------------------------------
%-----------------------------------------------------------------
\begin{Exercise}[label=402]
  Napisati funkciju koja odre\dj uje nulu funkcije $cos(x)$ na intervalu
  $[0,2]$ metodom polovljenja intervala (napomena: analogija sa binarnom
  pretragom). Algoritam se zavr\v sava kada se vrednost kosinusne
  funkcije razlikuje za najvi\v se 0.001 od 0.
  
  \begin{minitest}
    \begin{test}{Test 1}
      Izlaz:
      1.571289
    \end{test}
  \end{minitest}
  
\end{Exercise}

%\begin{Answer}[ref=402]
%  \includecode{resenja/04_Pretrazivanje/402.c}
%\end{Answer}
%-----------------------------------------------------------------
%-----------------------------------------------------------------
\begin{Exercise}[label=403]
  Napisati funkciju koja u sortiranom nizu nalazi prvi element ve\' ci
  od nule (napomena: primeniti binarnu pretragu). Napisati i program
  koji testira ovu funkciju za niz elemenata koji se zadaju kao
  argumenti komandne linije.
  
  \begin{maxitest}
    \begin{test}{Test 1}
      Poziv:  ./a.out -43 -24 -5 -2 1 4 6 12
      Izlaz:  1
    \end{test}
  \end{maxitest}
  
  \begin{miditest}
    \begin{test}{Test 2}
      Poziv:  ./a.out -32 4 65 123
      Izlaz:  4
    \end{test}
  \end{miditest}
  
\end{Exercise}

%\begin{Answer}[ref=403]
%  \includecode{resenja/04_Pretrazivanje/403.c}
%\end{Answer}
%-----------------------------------------------------------------
%-----------------------------------------------------------------
\begin{Exercise}[label=404]
  Napisati funkciju koja rekurzivno implementira algoritam
  interpolacione pretrage i program koji ovu funkciju testira za
  brojeve koji se unose sa standardnog ulaza (pretpostaviti da niz
  brojeva koji se unosi nije du\v{z}i od 1000 elemenata). Prvo se
  unosi broj koji se tra\v zi, a zatim sortirani elementi niza sve do
  kraja ulaza.
  
  \begin{miditest}
    \begin{test}{Test 1}
      Ulaz:   11 2 5 6 8 10 11 23
      Izlaz:  5
    \end{test}
  \end{miditest}
  
  \begin{miditest}
    \begin{test}{Test 2}
      Ulaz:   14 10 32 35 43 66 89 100
      Izlaz:  -1  
    \end{test}
  \end{miditest}
  
\end{Exercise}

%\begin{Answer}[ref=404]
%  \includecode{resenja/04_Pretrazivanje/404.c}
%\end{Answer}
%-----------------------------------------------------------------
%-----------------------------------------------------------------
\begin{Exercise}[label=405]
  Napisati funkciju koja odre\dj uje ceo deo logaritma za osnovu $2$
  datog neozna\v cenog celog broja, koriste\' ci samo bitske i
  relacione operatore.
  \begin{enumerate}
  \item Napisati varijantu sa pomeranjem broja udesno dok ne postane 0
    (linearna slo\v zenost).
  \item Napisati varijantu sa binarnom pretragom (logaritamska slo\v
    zenost).
  \end{enumerate}
  
  \begin{minitest}
    \begin{test}{Test 1}
      Ulaz:       10
      Izlaz:      3
    \end{test}
  \end{minitest}
  \begin{minitest}
    \begin{test}{Test 2}
      Ulaz:       4
      Izlaz:      2
    \end{test}
  \end{minitest}
  \begin{minitest}
    \begin{test}{Test 3}
      Ulaz:       17
      Izlaz:      4
    \end{test}
  \end{minitest}
  
  \begin{miditest}
    \begin{test}{Test 4}
      Ulaz:       1031
      Izlaz:      10
    \end{test}
  \end{miditest}
  
\end{Exercise}

%\begin{Answer}[ref=405]
%  \includecode{resenja/04_Pretrazivanje/405.c}
%\end{Answer}
%-----------------------------------------------------------------
%-----------------------------------------------------------------
\begin{Exercise}[label=406]
  U prvom kvadrantu dato je $1 \leq N \leq 10000$ du\v zi svojim
  koordinatama (du\v zi mogu da se seku, preklapaju, itd.). Napisati
  program koji pronalazi najmanji ugao $0 \leq \alpha \leq 90^\circ$,
  na dve decimale, takav da je suma du\v zina du\v zi sa obe strane
  polupoluprave iz koordinatnog po\v cetka pod uglom $\alpha$ jednak
  (neke du\v zi bivaju prese\v cene, a neke ne). (Uputstvo: vr\v siti
  binarnu pretragu intervala $[0, 90^\circ]$).
  
  \begin{miditest}
    \begin{test}{Test 1}
     Ulaz:
     2
     2 0 2 1
     1 2 2 2
     Izlaz:
     26.57 
    \end{test}
  \end{miditest}
  
\end{Exercise}

%\begin{Answer}[ref=406]
%  \includecode{resenja/04_Pretrazivanje/406.c}
%\end{Answer}
%-----------------------------------------------------------------
%-----------------------------------------------------------------
\begin{Exercise}[label=407]
  Napisati program u kome se prvo inicijalizuje stati\v{c}ki niz
  struktura osoba sa \v{c}lanovima ime i prezime (ure\dj en u
  rastu\'cem poretku prezimena) sa manje od 10 elemenata, a zatim se
  u\v{c}itava jedan karakter i pronalazi (bibliote\v{c}kom funkcijom
  \verb|bsearch|) i \v{s}tampa jedna struktura iz niza osoba \v{c}ije
  prezime po\v{c}inje tim karakterom (ako takva osoba postoji).
\begin{verbatim}
Osoba niz_osoba[]={{"Mika", "Antic"},
                   {"Dobrica", "Eric"},
                   {"Desanka", "Maksimovic"},
                   {"Dusko", "Radovic"},
                   {"Ljubivoje", "Rsumovic"}};
\end{verbatim}
  
  \begin{miditest}
    \begin{test}{Test 1}
      Ulaz:  R
      Izlaz: Dusko Radovic
    \end{test}
  \end{miditest}
  
\end{Exercise}

%\begin{Answer}[ref=407]
%  \includecode{resenja/04_Pretrazivanje/407.c}
%\end{Answer}
%-----------------------------------------------------------------
%-----------------------------------------------------------------

\section{Rešenja}

\shipoutAnswer


