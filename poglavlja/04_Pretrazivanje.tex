
\chapter{Algoritmi pretrage i sortiranja}

\section{Algoritmi pretrage}

%-----------------------------------------------------------------
%-----------------------------------------------------------------
\begin{Exercise}[label=3-01]
  Napisati iterativne funkcije pretraga nizova. Svaka funkcija treba
  da vrati indeks pozicije na kojoj je pronađen traženi broj ili
  broj $-1$ ukoliko broj nije pronađen.
  \begin{enumerate}  
  \item Napisati funkciju \kckod{linarna\_pretraga} koja vrši linearnu pretragu niza 
    celih brojeva \argf{a}, dužine \argf{n}, tražeći u njemu broj
    \argf{x}.  
  \item Napisati funkciju \kckod{binarna\_pretraga} koja vrši binarnu pretragu
    sortiranog niza \argf{a}, dužine \argf{n}, tražeći u njemu broj \argf{x}.
  \item Napisati funkciju \kckod{interpolaciona\_pretraga} koja vrši interpolacionu pretragu
    sortiranog niza \argf{a}, dužine \argf{n}, tražeći u njemu broj \argf{x}.
  \end{enumerate}
  Napisati i program koji generiše rastući niz slučajnih brojeva dimenzije
  \argf{n} i pozivajući napisane funkcije traži broj \argf{x}. Programu se kao prvi 
  argument komandne linije prosleđuje prirodan broj \argf{n} koji nije veći od $1000000$ i broj \argf{x} kao drugi
  argument komandne linije. Potrebna vremena za izvršavanje ovih
  funkcija dopisati u datoteku \kckod{vremena.txt}.

  
\begin{minitest}
\begin{test}{1}
#\poziv{./a.out 1000000 23542}#
  
#\naslovIzlaz#
#\izlaz{Linearna pretraga:}#
#\izlaz{Element nije u nizu}#
#\izlaz{Binarna pretraga:}#
#\izlaz{Element nije u nizu}#
#\izlaz{Interpolaciona pretraga:}#
#\izlaz{Element nije u nizu}#
\end{test}
\end{minitest}
\begin{minitest}
\begin{test2}{1}
    

  #\naslovDat{vremena.txt}#

  #\datoteka{Dimenzija niza: 1000000}#
  #\datoteka{	Linearna:            3615091 ns}#
  #\datoteka{	Binarna:                1536 ns}#
  #\datoteka{	Interpolaciona:          558 ns}#
\end{test2}
\end{minitest}

\begin{minitest}
\begin{test}{2}
#\poziv{./a.out 100000 37842}#
  
#\naslovIzlaz#
#\izlaz{Linearna pretraga:}#
#\izlaz{Element nije u nizu}#
#\izlaz{Binarna pretraga:}#
#\izlaz{Element nije u nizu}#
#\izlaz{Interpolaciona pretraga:}#
#\izlaz{Element nije u nizu}#
\end{test}
\end{minitest}
\begin{minitest}
\begin{test2}{1}


  #\naslovDat{vremena.txt}#

  #\datoteka{Dimenzija niza: 1000000}#
  #\datoteka{	Linearna:            3615091 ns}#
  #\datoteka{	Binarna:                1536 ns}#
  #\datoteka{	Interpolaciona:          558 ns}#
  #\datoteka{}#
  #\datoteka{Dimenzija niza: 100000}#
  #\datoteka{	Linearna:             360803 ns}#
  #\datoteka{	Binarna:                1187 ns}#
  #\datoteka{	Interpolaciona:          628 ns}#
\end{test2}
\end{minitest}

\linkresenje{3-01}

\end{Exercise}

\begin{Answer}[ref=3-01]
  \includecode{resenja/04_Pretrazivanje/3-01.c}
\end{Answer}

%-----------------------------------------------------------------
%-----------------------------------------------------------------
\begin{Exercise}[label=3-02]
  Napisati rekurzivne funkcije algoritama linearne, binarne i
  interpolacione pretrage i program koji ih testira za brojeve koji se
  unose sa standardnog ulaza. Linearnu pretragu implementirati na dva
  načina, svođenjem pretrage na prefiks i na sufiks
  niza. Prvo se unosi broj koji se traži, a zatim sortirani
  elementi niza sve do kraja ulaza. Pretpostaviti da niz brojeva koji se unosi neće biti duži od
  $1024$ elemenata.

\begin{miditest}
\begin{upotreba}{1}
#\naslovInt#
#\izlaz{Unesite trazeni broj:} \ulaz{11}#
#\izlaz{Unesite sortiran niz elemenata:}# 
#\ulaz{2 5 6 8 10 11 23}#
#\izlaz{Linearna pretraga}#
#\izlaz{Pozicija elementa je 5.}#
#\izlaz{Binarna pretraga}#
#\izlaz{Pozicija elementa je 5.}#
#\izlaz{Interpolaciona pretraga}#
#\izlaz{Pozicija elementa je 5.}#
\end{upotreba}
\end{miditest}
\begin{miditest}
\begin{upotreba}{2}
#\naslovInt#
#\izlaz{Unesite trazeni broj:} \ulaz{14}#
#\izlaz{Unesite sortiran niz elemenata:}#
#\ulaz{10 32 35 43 66 89 100}#
#\izlaz{Linearna pretraga}#
#\izlaz{Element se ne nalazi u nizu.}#
#\izlaz{Binarna pretraga}#
#\izlaz{Element se ne nalazi u nizu.}#
#\izlaz{Interpolaciona pretraga}#
#\izlaz{Element se ne nalazi u nizu.}#
\end{upotreba}
\end{miditest}

\linkresenje{3-02}

\end{Exercise}

\begin{Answer}[ref=3-02]
  \includecode{resenja/04_Pretrazivanje/3-02.c}
\end{Answer}
%-----------------------------------------------------------------
%-----------------------------------------------------------------
\begin{Exercise}[label=3-03]
  Napisati program koji preko argumenta komandne linije dobija ime
  datoteke koja sadrži sortirani spisak studenta po broju indeksa
  rastuće. Za svakog studenta u jednom redu stoje informacije o
  indeksu, imenu i prezimenu. Program učitava spisak studenata u niz i
  traži od korisnika indeks ili prezime studenta čije informacije se
  potom prikazuju na ekranu. U slučaju više studenata sa istim
  prezimenom prikazati informacije o prvom takvom. Odabir kriterijuma
  pretrage se vrši kroz poslednji argument komandne linije, koji može
  biti \kckod{-indeks} ili \kckod{-prezime}. U slučaju neuspešnih
  pretragi, štampati odgovarajuću poruku. Pretrage implementirati u
  vidu iterativnih funkcija što manje složenosti. Pretpostaviti da u
  datoteci neće biti više od $128$ studenata i da su imena i prezimena
  svih kraća od $16$ slova.
  
\begin{miditest}
\begin{upotreba}{1}
#\poziv{./a.out datoteka.txt -indeks}#
  
#\naslovDat{datoteka.txt}#
#\datoteka{20140003 Marina Petrovic}#
#\datoteka{20140012 Stefan Mitrovic}#
#\datoteka{20140032 Dejan Popovic}#
#\datoteka{20140049 Mirko Brankovic}#
#\datoteka{20140076 Sonja Stevanovic}#
#\datoteka{20140104 Ivan Popovic}#
#\datoteka{20140187 Vlada Stankovic}#
#\datoteka{20140234 Darko Brankovic}#
\end{upotreba}
\end{miditest}
\begin{miditest}
\begin{test2}{1}
  
  
#\naslovInt#
#\izlaz{Unesite indeks studenta}# 
#\izlaz{cije informacije zelite:}#
#\ulaz{20140076}#
#\izlaz{Indeks: 20140076,}# 
#\izlaz{Ime i prezime: Sonja Stevanovic}#
\end{test2}
\end{miditest}

\begin{miditest}
\begin{upotreba}{2}
#\poziv{./a.out datoteka.txt -prezime}#
  
#\naslovDat{datoteka.txt}#
#\datoteka{20140003 Marina Petrovic}#
#\datoteka{20140012 Stefan Mitrovic}#
#\datoteka{20140032 Dejan Popovic}#
#\datoteka{20140049 Mirko Brankovic}#
#\datoteka{20140076 Sonja Stevanovic}#
#\datoteka{20140104 Ivan Popovic}#
#\datoteka{20140187 Vlada Stankovic}#
#\datoteka{20140234 Darko Brankovic}#
\end{upotreba}
\end{miditest}
\begin{miditest}
\begin{test2}{1}
  
  
#\naslovInt#
#\izlaz{Unesite prezime studenta }#
#\izlaz{cije informacije zelite:}#
#\ulaz{Popovic}#
#\izlaz{Indeks: 20140032,}# 
#\izlaz{Ime i prezime: Dejan Popovic}#
\end{test2}
\end{miditest}

\linkresenje{3-03}

\end{Exercise}

\begin{Answer}[ref=3-03]
  \includecode{resenja/04_Pretrazivanje/3-03.c}
\end{Answer}
%-----------------------------------------------------------------
%-----------------------------------------------------------------
\begin{Exercise}[label=3-04]
  Modifikovati prethodni zadatak \ref{3-03} tako da tražene funkcije
  budu rekurzivne.

\linkresenje{3-04}

\end{Exercise}

\begin{Answer}[ref=3-04]
  \includecode{resenja/04_Pretrazivanje/3-04.c}
\end{Answer}
%-----------------------------------------------------------------
%-----------------------------------------------------------------
\begin{Exercise}[label=3-05]
  U datoteci koja se zadaje kao prvi argument komandne linije, nalaze
  se koordinate tačaka. U zavisnosti od prisustva opcija komandne
  linije (\argf{-x} ili \argf{-y}), pronaći onu koja je najbliža
  \argf{x}, ili \argf{y} osi, ili koordinatnom početku, ako nije
  prisutna nijedna opcija. Pretpostaviti da je broj tačaka u datateci
  veći od $0$ i ne veći od $1024$.
  
\begin{minitest}
\begin{test}{1}
#\poziv{./a.out dat.txt -x}#

#\naslovDat{dat.txt}#
#\datoteka{12 53}#
#\datoteka{2.342 34.1}#
#\datoteka{-0.3 23}#
#\datoteka{-1 23.1}#
#\datoteka{123.5 756.12}#

#\naslovIzlaz#
#\izlaz{-0.3 23}#
\end{test}
\end{minitest}
\begin{minitest}
\begin{test}{2}
#\poziv{./a.out dat.txt}#
  
#\naslovDat{dat.txt}#
#\datoteka{12 53}#
#\datoteka{2.342 34.1}#
#\datoteka{-0.3 23}#
#\datoteka{-1 2.1}#
#\datoteka{123.5 756.12}#

#\naslovIzlaz#
#\izlaz{-1 2.1}#
\end{test}
\end{minitest}
\begin{minitest}
\begin{test}{3}.
#\poziv{./a.out dat.txt -y}#

#\naslovDat{dat.txt}#
#\datoteka{12 53}#
#\datoteka{2.342 34.1}#
#\datoteka{-0.3 0.23}#
#\datoteka{-1 2.1}#
#\datoteka{123.5 756.12}#
  
#\naslovIzlaz#
#\izlaz{-0.3 0.23}#
\end{test}
\end{minitest}

\linkresenje{3-05}

\end{Exercise}

\begin{Answer}[ref=3-05]
  \includecode{resenja/04_Pretrazivanje/3-05.c}
\end{Answer}
%-----------------------------------------------------------------
%-----------------------------------------------------------------
\begin{Exercise}[label=3-06]
  Napisati funkciju koja određuje nulu funkcije \kckod{cos(x)} na
  intervalu \argf{[0,2]} metodom polovljenja intervala. Algoritam se
  završava kada se vrednost kosinusne funkcije razlikuje za najviše
  $0.001$ od nule. \uputstvo{Korisiti algoritam analogan algoritmu
  binarne pretrage.}
  
% Ovde ne moze biti vise od jednog test primera
\begin{minitest}
\begin{test}{1}
#\naslovIzlaz#
#\izlaz{1.57031}#
\end{test}
\end{minitest}

\linkresenje{3-06}

\end{Exercise}

\begin{Answer}[ref=3-06]
  \includecode{resenja/04_Pretrazivanje/3-06.c}
\end{Answer}
%-----------------------------------------------------------------
%-----------------------------------------------------------------
\begin{Exercise}[label=3-07]
Napisati funkciju koja u rastuće sortiranom nizu celih brojeva
binarnom pretragom pronalazi indeks prvog elementa većeg od
nule. Ukoliko nema elemenata većih od nule, funkcija kao rezultat
vraća \argf{-1}. Napisati program koji testira ovu funkciju za rastući
niz celih brojeva koji se učitavaju sa standardnog ulaza. Niz neće biti
duži od $256$, i njegovi elementi se unose sve do kraja ulaza.

\begin{minitest}
\begin{test}{1}
#\naslovUlaz#
#\ulaz{-151 -44 5 12 13 15}#

#\naslovIzlaz#
#\izlaz{2}#
\end{test}
\end{minitest}
\begin{minitest}
\begin{test}{2}
#\naslovUlaz#
#\ulaz{-100 -15 -11 -8 -7 -5}#

#\naslovIzlaz#
#\izlaz{-1}#
\end{test}
\end{minitest}
\begin{minitest}
\begin{test}{3}
#\naslovUlaz#
#\ulaz{-100 -15 0 13 55 124}#
#\ulaz{258 315 516 7000}#
  
#\naslovIzlaz#
#\izlaz{3}#
\end{test}
\end{minitest}

\linkresenje{3-07}

\end{Exercise}
\begin{Answer}[ref=3-07]
  \includecode{resenja/04_Pretrazivanje/3-07.c}
\end{Answer}
%-----------------------------------------------------------------
%-----------------------------------------------------------------
\begin{Exercise}[label=3-08]
Napisati funkciju koja u opadajuće sortiranom nizu celih brojeva
binarnom pretragom pronalazi indeks prvog elementa manjeg od
nule. Ukoliko nema elemenata manjih od nule, funkcija kao rezultat
vraća \argf{-1}. Napisati program koji testira ovu funkciju za
opadajući niz celih brojeva koji se učitavaju sa standardnog ulaza. Niz
neće biti duži od $256$, i njegovi elementi se unose sve do kraja
ulaza.

\begin{minitest}
\begin{test}{1}
#\naslovUlaz#
#\ulaz{151 44 5 -12 -13 -15}#

#\naslovIzlaz#
#\izlaz{3}#
\end{test}
\end{minitest}
\begin{minitest}
\begin{test}{2}
#\naslovUlaz#
#\ulaz{100 55 15 0 -15 -124}#
#\ulaz{-155 -258 -315 -516}#

#\naslovIzlaz#
#\izlaz{4}#
\end{test}
\end{minitest}
\begin{minitest}
\begin{test}{3}
#\naslovUlaz#
#\ulaz{100 15 11 8 7 5 4 3 2}#

#\naslovIzlaz#
#\izlaz{-1}#
\end{test}
\end{minitest}

\linkresenje{3-08}

\end{Exercise}

\begin{Answer}[ref=3-08]
  \includecode{resenja/04_Pretrazivanje/3-08.c}
\end{Answer}
%-----------------------------------------------------------------
%-----------------------------------------------------------------
\begin{Exercise}[label=3-09]
  Napisati funkciju koja određuje ceo deo logaritma za osnovu 2 datog
  neoznačenog celog broja koristeći samo bitske i relacione
  operatore.
  \begin{enumerate}
  \item Napisati funkciju linearne složenosti koja određuje
    logaritam pomeranjem broja udesno.
  \item Napisati funkciju logaritmske složenosti koja određuje
    logaritam koristeći binarnu pretragu.
  \end{enumerate}
  Tražene funkcije testirati programom koji pozitivan broj učitava sa
  standardnog ulaza, a logaritam ispisuje na standardnom izlazu.

\begin{minitest}
\begin{test}{1}
#\naslovUlaz#
#\ulaz{4}#
  
#\naslovIzlaz#
#\izlaz{2 2}#
\end{test}
\end{minitest}
\begin{minitest}
\begin{test}{2}
#\naslovUlaz#
#\ulaz{17}#
  
#\naslovIzlaz#
#\izlaz{4 4}#
\end{test}
\end{minitest}
\begin{minitest}
\begin{test}{3}
#\naslovUlaz#
#\ulaz{1031}#
  
#\naslovIzlaz#
#\izlaz{10 10}#
\end{test}
\end{minitest}

\linkresenje{3-09}

\end{Exercise}

\begin{Answer}[ref=3-09]
  \includecode{resenja/04_Pretrazivanje/3-09.c}
\end{Answer}
%-----------------------------------------------------------------
%-----------------------------------------------------------------
\begin{Exercise}[difficulty=2, label=3-10]
  U prvom kvadrantu dato je $1 \leq \argf{N} \leq 10000$ duži svojim
  koordinatama (duži mogu da se seku, preklapaju, itd.). Napisati
  program koji pronalazi najmanji ugao $0 \leq \alpha \leq 90^\circ$,
  na dve decimale, takav da je suma dužina duži sa obe strane
  polupoluprave iz koordinatnog početka pod uglom $\alpha$ jednak
  (neke duži bivaju presečene, a neke ne). Program prvo učitava broj
  \argf{N}, a zatim i same koordinate temena duži. \uputstvo{Vršiti
  binarnu pretragu intervala $[0, 90^\circ]$.}
  
\begin{minitest}
\begin{upotreba}{1}
#\naslovInt#
#\izlaz{Unesi broj tacaka:}\ulaz{2}#
#\izlaz{Unesi koordinate tacaka:}#
#\ulaz{2 0 2 1}#
#\ulaz{1 2 2 2}#
#\izlaz{26.57}#
\end{upotreba}
\end{minitest}
\begin{minitest}
\begin{upotreba}{2}
#\naslovInt#
#\izlaz{Unesi broj tacaka:}\ulaz{2}#
#\izlaz{Unesi koordinate tacaka:}#
#\ulaz{1 0 1 1}#
#\ulaz{0 1 1 1}#
#\izlaz{45}#
\end{upotreba}
\end{minitest}
\begin{minitest}
\begin{upotreba}{3}
#\naslovInt#
#\izlaz{Unesi broj tacaka:}\ulaz{3}#
#\izlaz{Unesi koordinate tacaka:}#
#\ulaz{1 0 1 1}#
#\ulaz{2 0 2 1}#
#\ulaz{1 2 2 2}#
#\izlaz{26.57}#
\end{upotreba}
\end{minitest}

\end{Exercise}

%-----------------------------------------------------------------
%-----------------------------------------------------------------

%\section{Rešenja}

%\shipoutAnswer


