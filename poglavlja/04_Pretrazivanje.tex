
\chapter{Pretraživanje}

\section{Zadaci}

%-----------------------------------------------------------------
%-----------------------------------------------------------------
\begin{Exercise}[label=401]
  Napisati iterativne funkcije pretraga nizova. Svaka funkcija treba
  da vrati indeks pozicije na kojoj je pronađen traženi element ili
  broj $-1$ ukoliko element nije pronađen.
  \begin{enumerate}  
  \item Napisati funkciju koja vrši linearnu pretragu niza 
    celih brojeva \argf{a}, dužine \argf{n}, tražeći u njemu broj
    \argf{x}.  
  \item Napisati funkciju koja vrši binarnu pretragu
    sortiranog niza \argf{a}, dužine \argf{n}, tražeći u njemu broj \argf{x}.
  \item Napisati funkciju koja vrši interpoacionu pretragu
    sortiranog niza \argf{a}, dužine \argf{n}, tražeći u njemu broj \argf{x}.
  \end{enumerate}
  Napisati i program koji generiše slučajni rastući niz dimenzije
  \argf{n} (prvi argument komandne linije), i u njemu već napisanim
  funkcijama traži element \argf{x} (drugi argument komandne
  linije). Potrebna vremena za izvršavanje ovih funkcija upisati u
  fajl \kckod{vremena.txt}.
  
\begin{maxitest}
\begin{test}{Test 1}
Poziv: ./a.out 1000000 23532
Izlaz: Element nije pronadjen u nizu linearnom pretragom.
       Element nije pronadjen u nizu binarnom pretragom.
       Element nije pronadjen u nizu interpolacionom pretragom.  
\end{test}
\end{maxitest}
  
\end{Exercise}

\begin{Answer}[ref=401]
  \includecode{resenja/04_Pretrazivanje/401.c}
\end{Answer}

%-----------------------------------------------------------------
%-----------------------------------------------------------------
\begin{Exercise}[label=402]
  Napisati rekurzivne funkcije algoritama linearne, binarne i
  interpolacione pretrage i program koji ih testira za brojeve koji se
  unose sa standardnog ulaza. Pretpostaviti da niz brojeva koji se
  unosi neće biti duži od 1024 elemenata. Prvo se unosi broj koji se
  traži, a zatim sortirani elementi niza sve do kraja ulaza.

\begin{miditest}
\begin{test}{Test 1}
Ulaz:   11 2 5 6 8 10 11 23
Izlaz:  5
\end{test}
\end{miditest}

\begin{miditest}
\begin{test}{Test 2}
Ulaz:   14 10 32 35 43 66 89 100
Izlaz:  -1  
\end{test}
\end{miditest}
  
\end{Exercise}

%\begin{Answer}[ref=402]
%  \includecode{resenja/04_Pretrazivanje/402.c}
%\end{Answer}
%-----------------------------------------------------------------
%-----------------------------------------------------------------
\begin{Exercise}[label=403]
  Napisati program koji preko argumenta komandne linije dobija ime
  datoteke koja sadrži sortirani spisak studenta po broju indeksa
  rastuće. Za svakog studenta u jednom redu stoje informacije o
  indeksu, imenu i prezimenu.  Program učitava spisak studenata u niz
  i traži od korisnika indeks studenta čije informacije se potom
  prikazuju na ekranu.  Zatim, korisnik učitava prezime studenta i
  prikazuju mu se informacije o prvom studentu sa unetim prezimenom.
  Pretrage implementirati u vidu iterativnih funkcija što bolje manje
  složenosti.
  
\begin{maxitest}
\begin{test}{Test 1}
Datoteka:
20140003 Marina Petrovic
20140012 Stefan Mitrovic
20140032 Dejan Popovic
20140049 Mirko Brankovic
20140076 Sonja Stevanovic
20140104 Ivan Popovic
20140187 Vlada Stankovic
20140234 Darko Brankovic
Ulaz:
20140076
Popovic
Izlaz:
Indeks: 20140076, Ime i prezime: Sonja Stevanovic
Indeks: 20140032, Ime i prezime: Dejan Popovic
\end{test}
\end{maxitest}
  
\end{Exercise}

%\begin{Answer}[ref=403]
%  \includecode{resenja/04_Pretrazivanje/403.c}
%\end{Answer}
%-----------------------------------------------------------------
%-----------------------------------------------------------------
\begin{Exercise}[label=404]
  Modifikovati prethodni zadatak tako da tražene funkcije budu
  rekurzivne.
\end{Exercise}

%\begin{Answer}[ref=404]
%  \includecode{resenja/04_Pretrazivanje/404.c}
%\end{Answer}
%-----------------------------------------------------------------
%-----------------------------------------------------------------
\begin{Exercise}[label=405]
  U datoteci koja se zadaje kao argument komandne linije, nalaze se
  koordinate tačaka. U zavisnosti od prisustva opcija komandne linije
  (\argf{-x} ili \argf{-y}), pronaći onu koja je najbliža \argf{x}
  (ili \argf{y}) osi, ili koordinatnom početku, ako nije prisutna
  nijedna opcija. Broj tačaka u datoteci nije unapred poznat.
  
\begin{miditest}
\begin{test}{Test 1}
Poziv: ./a.out dat.txt -x
Datoteka:
12 53
2.342 34.1
-0.3 23
-1 23.1
123.5 756.12
Izlaz: -0.3 23
\end{test}
\end{miditest}
  
\end{Exercise}

%\begin{Answer}[ref=405]
%  \includecode{resenja/04_Pretrazivanje/405.c}
%\end{Answer}
%-----------------------------------------------------------------
%-----------------------------------------------------------------
\begin{Exercise}[label=406]
  Napisati funkciju koja određuje nulu funkcije \kckod{cos(x)} na
  intervalu \argf{[0,2]} metodom polovljenja intervala. Algoritam se završava kada se
  vrednost kosinusne funkcije razlikuje za najviše $0.001$ od nule. Uputstvo: korisiti algoritam analogan algoritmu binarne pretrage.
  
  
\begin{minitest}
\begin{test}{Test 1}
Izlaz:
1.571289
\end{test}
\end{minitest}
  
\end{Exercise}

%\begin{Answer}[ref=406]
%  \includecode{resenja/04_Pretrazivanje/406.c}
%\end{Answer}
%-----------------------------------------------------------------
%-----------------------------------------------------------------
\begin{Exercise}[label=407]
  Napisati funkciju koja u sortiranom nizu nalazi prvi element veći od
  0. Napisati i program koji
  testira ovu funkciju za niz elemenata koji se zadaju kao argumenti
  komandne linije. Uputstvo: primeniti binarnu pretragu.
  
\begin{miditest}
\begin{test}{Test 1}
Poziv:  ./a.out -43 -24 -5 -2 1 4 6 12
Izlaz:  1
\end{test}
\end{miditest}

\begin{miditest}
\begin{test}{Test 2}
Poziv:  ./a.out -32 4 65 123
Izlaz:  4
\end{test}
\end{miditest}
  
\end{Exercise}

%\begin{Answer}[ref=407]
%  \includecode{resenja/04_Pretrazivanje/407.c}
%\end{Answer}
%-----------------------------------------------------------------
%-----------------------------------------------------------------
\begin{Exercise}[label=408]
  Napisati funkciju koja određuje ceo deo logaritma za osnovu 2 datog
  neoznačenog celog broja, koristeći samo bitske i relacione
  operatore.
  \begin{enumerate}
  \item Napisati funkciju, linearne složenosti, koja određuje
    logaritam pomeranjem broja udesno dok ne postane 0.
  \item Napisati funkciju, logaritmske složenosti, koja određuje
    logaritam koristeći binarnu pretragu.
  \end{enumerate}
  Tražene funkcije testirati programom koji broj učitava sa
  standardnog ulaza, a logaritam ispisuje na standardni izlaz.
\begin{minitest}
\begin{test}{Test 1}
Ulaz:       10
Izlaz:      3 3
\end{test}
\end{minitest}
\begin{minitest}
\begin{test}{Test 2}
Ulaz:       4
Izlaz:      2 2
\end{test}
\end{minitest}
\begin{minitest}
\begin{test}{Test 3}
Ulaz:       17
Izlaz:      4 3
\end{test}
\end{minitest}

\begin{minitest}
\begin{test}{Test 4}
Ulaz:       1031
Izlaz:      10 10
\end{test}
\end{minitest}
  
\end{Exercise}

%\begin{Answer}[ref=408]
%  \includecode{resenja/04_Pretrazivanje/408.c}
%\end{Answer}
%-----------------------------------------------------------------
%-----------------------------------------------------------------
\begin{Exercise}[label=409]
  U prvom kvadrantu dato je $1 \leq \argf{N} \leq 10000$ duži svojim
  koordinatama (duži mogu da se seku, preklapaju, itd.). Napisati
  program koji pronalazi najmanji ugao $0 \leq \alpha \leq 90^\circ$,
  na dve decimale, takav da je suma dužina duži sa obe strane
  polupoluprave iz koordinatnog početka pod uglom $\alpha$ jednak
  (neke duži bivaju presečene, a neke ne). Program prvo učitava broj
  \argf{N}, a zatim i same koordinate temena duži. Uputstvo: vršiti
  binarnu pretragu intervala $[0, 90^\circ]$.
  \komentar{Milena: Ko ovde vrsi sortiranje? Da li nesto na tu temu treba reci u zadatku?}
\begin{minitest}
\begin{test}{Test 1}
Ulaz:
2
2 0 2 1
1 2 2 2
Izlaz:
26.57 
\end{test}
\end{minitest}
  
\end{Exercise}

%\begin{Answer}[ref=409]
%  \includecode{resenja/04_Pretrazivanje/409.c}
%\end{Answer}
%-----------------------------------------------------------------
%-----------------------------------------------------------------
\begin{Exercise}[label=410]
  Napisati program u kome se prvo inicijalizuje statički niz struktura
  osoba sa članovima ime i prezime (uređen u rastućem poretku
  prezimena) sa manje od 10 elemenata, a zatim se učitava jedan
  karakter i pronalazi (bibliotečkom funkcijom \kckod{bsearch}) i
  štampa jedna struktura iz niza osoba čije prezime počinje tim
  karakterom. Ako takva osoba ne postoji, štampati $-1$ na standardni
  izlaz.
\begin{ckod}
Osoba niz_osoba[]={{"Mika", "Antic"},
                   {"Dobrica", "Eric"},
                   {"Desanka", "Maksimovic"},
                   {"Dusko", "Radovic"},
                   {"Ljubivoje", "Rsumovic"}};
\end{ckod}
  
\begin{minitest}
\begin{test}{Test 1}
Ulaz:  R
Izlaz: Dusko Radovic
\end{test}
\end{minitest}
  
\end{Exercise}

\komentar{Milena: nesto mi je ovih zadataka sa pretrazivanjem mnogo malo... Ne znam da nismo nesto zaboravili? Kako to da nema ni jedan primer sa lfind? I nesto mi je za bsearch mnogo malo...}
%\begin{Answer}[ref=410]
%  \includecode{resenja/04_Pretrazivanje/410.c}
%\end{Answer}
%-----------------------------------------------------------------
%-----------------------------------------------------------------

\section{Rešenja}

\shipoutAnswer


