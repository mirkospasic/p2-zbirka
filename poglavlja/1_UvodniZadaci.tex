\renewcommand{\chaptermark}[1]{\markboth{\thechapter\ #1}{#1}}
\renewcommand{\sectionmark}[1]{\markright{\thesection\ #1}}

\chapter{Uvodni zadaci}


\iffalse
\pagestyle{fancy}
\fancyhf{}
\fancyhead[RE]{\bfseries\slshape\leftmark}
\fancyhead[LO]{\bfseries\slshape\rightmark}
\fancyfoot[RO,LE]{\thepage}
\fi

\pagestyle{fancyplain}
\renewcommand{\chaptermark}[1]{\markboth{\thechapter\ #1}{#1}}
\renewcommand{\sectionmark}[1]{\markright{\thesection\ #1}}
%\lhead[\fancyplain{}{\bfseries\slshape\thepage}]{\fancyplain{}{\bfseries\slshape\rightmark}}
%\rhead[\fancyplain{}{\bfseries\slshape\leftmark}]{\fancyplain{}{\bfseries\slshape\thepage }}
\lhead[\fancyplain{}{\bfseries\slshape\leftmark}]{\fancyplain{}{}}
\rhead[\fancyplain{}{}]{{\fancyplain{}{\bfseries\slshape\rightmark}}}
%\rhead[]{\fancyplain{}{\bfseries\slshape\rightmark}}
%\lhead[]{\fancyplain{}{\bfseries\slshape\leftmark}}
\cfoot{}

%\fancyhead[LE,RO]{\fancyplain{}{\bfseries\slshape\rightmark}
%\fancyhead[RE,LO]{\fancyplain{}{\bfseries\slshape\leftmark}}
\fancyfoot[LE]{\thepage} 
\fancyfoot[RO]{\thepage} 
 

\pagenumbering{arabic}
\setcounter{page}{1}


\section{Podela koda po datotekama}

\begin{Exercise}[label=1_01] % I sa vezbi
Napisati program za rad sa kompleksnim brojevima.
\begin{enumerate}
\item Definisati strukturu \kckod{KompleksanBroj} koja opisuje kompleksan broj zadat njegovim realnim i imaginarnim delom.
\item Napisati funkciju \kckod{void ucitaj\_kompleksan\_broj(KompleksanBroj * z)} koja učitava kompleksan broj \kckod{z} sa standardnog ulaza.
\item Napisati funkciju \kckod{void ispisi\_kompleksan\_broj(KompleksanBroj z)} koja ispisuje kompleksan broj \kckod{z} na standardni izlaz u odgovarajućem formatu. 
%(npr. broj čiji je realni deo \argf{2}, a imaginarni \argf{-3} treba ispisati kao $(2 - 3 i)$).
\item Napisati funkciju \kckod{float realan\_deo(KompleksanBroj z)} koja vraća vrednost realnog dela broja \kckod{z}.
\item Napisati funkciju \kckod{float imaginaran\_deo(KompleksanBroj z)} koja vraća vrednost imaginarnog dela broja \kckod{z}.
\item Napisati funkciju \kckod{float moduo(KompleksanBroj z)} koja vraća moduo kompleksnog broja \kckod{z}.
\item Napisati funkciju \kckod{KompleksanBroj konjugovan(KompleksanBroj z)} koja vraća konjugovano-kompleksni broj broja \kckod{z}.
\item Napisati funkciju \kckod{KompleksanBroj saberi(KompleksanBroj z1, KompleksanBroj z2)} koja vraća zbir dva kompleksna broja \kckod{z1} i \kckod{z2}.
\item Napisati funkciju \kckod{KompleksanBroj oduzmi(KompleksanBroj z1, KompleksanBroj z2)} koja vraća razliku dva kompleksna broja \kckod{z1} i \kckod{z2}.
\item Napisati funkciju \kckod{KompleksanBroj mnozi(KompleksanBroj z1, KompleksanBroj z2)} koja vraća proizvod dva kompleksna broja \kckod{z1} i \kckod{z2}.
\item Napisati funkciju \kckod{float argument(KompleksanBroj z)} koja vraća argument kompleksnog broja \kckod{z}.
\end{enumerate}

Napisati program koji testira prethodno napisane funkcije. Sa standardnog ulaza uneti dva kompleksna broja $z1$ i $z2$, a zatim ispisati realni deo, imaginarni deo, moduo, konjugovano-kompleksan broj i argument broja koji se dobija kao zbir, razlika ili proizvod brojeva $z1$ i $z2$ u zavisnosti od znaka ('+', '-', '*') koji se unosi sa standardnog ulaza. 

\begin{maxitest}
\begin{upotreba}{1}
#\naslovInt#
#\izlaz{Unesite realni i imaginarni deo kompleksnog broja:} \ulaz{1 -3}#
#\izlaz{(1.00 - 3.00 i)}#
#\izlaz{Unesite realni i imaginarni deo kompleksnog broja:} \ulaz{-1 4}#
#\izlaz{(-1.00 + 4.00 i)}#
#\izlaz{Unesite znak (+,-,*):} \ulaz{-}#
#\izlaz{(1.00 - 3.00 i) - (-1.00 + 4.00 i)  =  (2.00 - 7.00 i)}#
#\izlaz{Realni\_deo: 2}#
#\izlaz{Imaginarni\_deo: -7.000000}#
#\izlaz{Moduo: 7.280110}#
#\izlaz{Konjugovano kompleksan broj: (2.00 + 7.00 i)}#
#\izlaz{Argument kompleksnog broja: - 1.292497}#
\end{upotreba}
\end{maxitest}


\linkresenje{1_01}
\end{Exercise}
\begin{Answer}[ref=1_01]
\includecode{resenja/1_UvodniZadaci/1_01.c}
\end{Answer}

\begin{Exercise}[label=1_02] % III sa vezbi
Uraditi prethodni zadatak tako da su sve napisane funkcije za rad sa kompleksnim brojevima zajedno sa definicijom strukture \kckod{KompleksanBroj} izdvojene u posebnu biblioteku. Napisati program koji testira ovu biblioteku. Sa standardnog ulaza uneti kompeksan broj, a zatim na standardni izlaz ispisati njegov polarni oblik. 

\begin{maxitest}
\begin{upotreba}{1}
#\naslovInt#
#\izlaz{Unesite realni i imaginarni deo kompleksnog broja:} \ulaz{-5 2 }#
#\izlaz{Polarni oblik kompleksnog broja je 5.39 *  e\^{}i * 2.76}#
\end{upotreba}
\end{maxitest}

\linkresenje{1_02}

\end{Exercise}
\begin{Answer}[ref=1_02]
\includecodeLib{resenja/biblioteke/kompleksan_broj/kompleksan_broj.h}{kompleksan\_broj.h} 
\includecodeLib{resenja/biblioteke/kompleksan_broj/kompleksan_broj.c}{kompleksan\_broj.c}
\includecodeLib{resenja/1_UvodniZadaci/1_02.c}{main.c}
\end{Answer}


\begin{Exercise}[label=1_03] % sa praktikuma
Napisati biblioteku za rad sa polinomima.
  \begin{enumerate}
  \item Definisati strukturu \kckod{Polinom} koja opisuje polinom stepena najviše 20 koji je zadat nizom svojih koeficijenata tako da se na i-toj poziciji u nizu nalazi koeficijent uz i-ti stepen polinoma.
  \item Napisati funkciju \kckod{void ispisi(const Polinom * p)} koja ispisuje polinom \kckod{p} na standardni izlaz, od najvišeg ka najnižem stepenu. Ipisati samo koeficijente koji su različiti od nule.
  \item Napisati funkciju \kckod{Polinom ucitaj()} koja učitava polinom sa standardnog
    ulaza. Za polinom najpre uneti stepen, a zatim njegove koeficijente.
  \item Napisati funkciju \kckod{double izracunaj(const Polinom * p, double x)} koja vraća vrednosti polinoma \kckod{p} u
    datoj tački \kckod{x} koristeći Hornerov algoritam.
  \item Napisati funkciju \kckod{Polinom saberi(const Polinom * p, const Polinom * q)} koja vraća zbir dva polinoma \kckod{p} i \kckod{q}.
  \item Napisati funkciju \kckod{Polinom pomnozi(const Polinom * p, const Polinom * q)} koja vraća proizvod dva polinoma \kckod{p} i \kckod{q}.
  \item Napisati funkciju \kckod{Polinom izvod(const Polinom * p)} koja vraća izvod polinoma \kckod{p}.
  \item Napisati funkciju \kckod{Polinom n\_izvod(const Polinom * p, int n)} koja vraća \kckod{n}-ti izvod polinoma \kckod{p}.
  \end{enumerate}

Napisati program koji testira prethodno napisane funkcije. Sa standardnog ulaza učitati polinome \argf{p} i \argf{q}, a zatim ih ispisati na standardni izlaz u odgovarajućem formatu. Izračunati i ispisati zbir \argf{z} i proizvod \argf{r} unetih polinoma \argf{p} i \argf{q}. Sa standardnog ulaza učitati realni broj \argf{x}, a zatim na standardni izlaz ispisati vrednost polinoma \argf{z} u tački \argf{x} zaokruženu na dve decimale. Na kraju, sa standardnog ulaza učitati broj \argf{n} i na izlaz ispisati \argf{n}-ti izvod polinoma \argf{r}.

\begin{maxitest}
\begin{upotreba}{1}
#\naslovInt#
#\izlaz{Unesite polinom p (prvo stepen, pa zatim koeficijente od najveceg stepena do nultog):}#
#\ulaz{3 1.2 3.5 2.1 4.2}#
#\izlaz{Unesite polinom q (prvo stepen, pa zatim koeficijente od najveceg stepena do nultog):}#
#\ulaz{2 2.1 0 -3.9}#
#\izlaz{Zbir polinoma je polinom z:}#
#\izlaz{1.20x\^{}3+5.60x\^{}2+2.10x+0.30}# 
#\izlaz{Prozvod polinoma je polinom r:}#
#\izlaz{2.52x\^{}5+7.35x\^{}4-0.27x\^{}3-4.83x\^{}2-8.19x-16.38}#
#\izlaz{Unesite tacku u kojoj racunate vrednost polinoma z:}#
#\ulaz{0}#
#\izlaz{Vrednost polinoma z u tacki 0.00 je 0.30}#
#\izlaz{Unesite izvod polinoma koji zelite:}#
#\ulaz{3}#
#\izlaz{3. izvod polinoma r je: 151.20x\^{}2+176.40x-1.62}#
\end{upotreba}
\end{maxitest}

\linkresenje{1_03}

\end{Exercise}
\begin{Answer}[ref=1_03]
\includecodeLib{resenja/biblioteke/polinom/polinom.h}{polinom.h} 
\includecodeLib{resenja/biblioteke/polinom/polinom.c}{polinom.c}
\includecodeLib{resenja/1_UvodniZadaci/1_03.c}{main.c}
%\includecode{resenja/1_UvodniZadaci/003/Makefile}
\end{Answer}

\begin{Exercise}[label=1_04] % sa praktikuma
Napisati biblioteku za rad sa razlomcima.

  \begin{enumerate}

  \item Definisati strukturu \kckod{Razlomak} koja opisuje razlomak.
  \item Napisati funkciju \kckod{Razlomak ucitaj()} za učitavanje razlomka.
  \item Napisati funkciju \kckod{void ispisi(const Razlomak * r)} koja ispisuje razlomak \argf{r}.
  \item Napisati funkciju \kckod{int brojilac(const Razlomak * r)} koja vraćaja brojilac razlomka \kckod{r}.
   \item Napisati funkciju \kckod{int imenilac(const Razlomak * r)} koja vraćaja imenilac razlomka \kckod{r}.
  \item Napisati funkciju \kckod{double realna\_vrednost(const Razlomak * r)} koja vraća odgovarajuću realnu vrednost razlomka \kckod{r}.
  \item Napisati funkciju  \kckod{double reciprocna\_vrednost(const Razlomak * r)} koja vraća recipročnu vrednost
    razlomka \kckod{r}.
  \item Napisati funkciju \kckod{Razlomak skrati(const Razlomak * r)} koja vraća skraćenu vrednost datog razlomka \kckod{r}.
  \item Napisati funkciju \kckod{Razlomak saberi(const Razlomak * r1, const Razlomak * r2)} koja vraća zbir dva razlomka \kckod{r1} i \kckod{r2}.
 \item Napisati funkciju \kckod{Razlomak oduzmi(const Razlomak * r1, const Razlomak * r2)} koja vraća razliku dva razlomka \kckod{r1} i \kckod{r2}.
 \item Napisati funkciju \kckod{Razlomak pomnozi(const Razlomak * r1, const Razlomak * r2)} koja vraća proizvod dva razlomka \kckod{r1} i \kckod{r2}.
 \item Napisati funkciju \kckod{Razlomak podeli(const Razlomak * r1, const Razlomak * r2)} koja vraća količnik dva razlomka \kckod{r1} i \kckod{r2}.
\end{enumerate}

Napisati program koji testira prethodne funkcije. Sa standardnog ulaza učitati dva razlomka \argf{r1} i \argf{r2}. Na standardni izlaz ispisati skraćene vrednosti zbira, razlike, proizvoda i količnika
razlomaka \argf{r1} i recipročne vrednosti razlomka \argf{r2}.

\begin{maxitest}
\begin{upotreba}{1}
#\naslovInt#
#\izlaz{Unesite imenilac i brojilac prvog razlomka:}\ulaz{1 2}#
#\izlaz{Unesite imenilac i brojilac drugog razlomka:}\ulaz{2 3}#
#\izlaz{1/2 + 3/2 = 2}#
#\izlaz{1/2 - 3/2 =  -1}#
#\izlaz{1/2 * 3/2 =  3/4}#
#\izlaz{1/2 / 3/2 =  1/3}#
\end{upotreba}
\end{maxitest}


%\linkresenje{1_04}

\end{Exercise}
%\begin{Answer}[ref=1_04]
%%\includecode{resenja/...}
%\end{Answer}

\section{Algoritmi za rad sa bitovima}

\begin{Exercise}[label=1_05] %ex 201 dvestajedan
Napisati biblioteku \kckod{stampanje\_bitova} za rad sa bitovima. Biblioteka treba da sadrži funkcije \kckod{stampanje\_bitova}, 
\kckod{stampanje\_bitova\_short} i \kckod{stampanje\_bitova\_char} za štampanje bitova u binarnom zapisu celog broja 
tipa \kckod{int}, \kckod{short} i \kckod{char}, koji se zadaje kao argument funkcije. 
Napisati program koji testira napisanu biblioteku. Sa standardnog ulaza učitati u heksadekadnom formatu cele brojeve tipa
\kckod{int}, \kckod{short} i \kckod{char} i na standardni izlaz ispisati njihovu binarnu reprezentaciju.

\begin{maxitest}
\begin{upotreba}{1}
#\naslovInt#
#\izlaz{Unesite broj tipa int: }\ulaz{0x4f4f4f4f}#
#\izlaz{Binarna reprezentacija: 01001111010011110100111101001111}#
#\izlaz{Unesite broj tipa short: }\ulaz{0x4f4f}#
#\izlaz{Binarna reprezentacija: 0100111101001111}#
#\izlaz{Unesite broj tipa char: }\ulaz{0x4f}#
#\izlaz{Binarna reprezentacija: 01001111}#
\end{upotreba}
\end{maxitest}

\linkresenje{1_05}
\end{Exercise}
\begin{Answer}[ref=1_05]
\includecodeLib{resenja/biblioteke/stampanje_bitova/stampanje_bitova.h}{stampanje\_bitova.h}
\includecodeLib{resenja/biblioteke/stampanje_bitova/stampanje_bitova.c}{stampanje\_bitova.c}
\includecodeLib{resenja/1_UvodniZadaci/1_05.c}{main.c}
\end{Answer}

\begin{Exercise}[label=1_06]
Napisati funkcije \kckod{\_bitove\_1} i \kckod{prebroj\_bitove\_2} koje vraćaju broj jedinica u 
binarnom zapisu označenog celog broja $x$ koji se zadaje kao argument funkcije. Prebrojavanje bitova ostvariti na dva načina:
\begin{enumerate}
\item formiranjem odgovarajuće maske i njenim pomeranjem (funkcija
\kckod{prebroj\-\_\-bitove\_1})
\item formiranjem odgovarajuće maske i pomeranjem promenljive $x$ (funkcija
\kckod{prebroj\_bitove\_2}).
\end{enumerate} 
 
Napisati program koji testira napisane funkcije. Sa standardnog ulaza učitati ceo broj u heksadekasnom formatu i
redni broj funkcije koju treba primeniti ($1$ ili $2$), a zatim na standardni izlaz ispisati broj jedinica u 
binarnom zapisu učitanog broja pozivom izabrane funkcije. Ukoliko
korisnik ne unese ispravnu vrednost za redni broj funkcije, prekinuti izvršavanje
programa i ispisati odgovarajuću poruku na standardni izlaz za greške.

\begin{miditest}
\begin{upotreba}{1}
#\naslovInt#
#\izlaz{Unesite broj:}\ulaz{0x7F}#
#\izlaz{Unesite redni broj funkcije:}\ulaz{1}#
#\izlaz{Poziva se funkcija prebroj\_bitove\_1}#
#\izlaz{Broj jedinica u zapisu je 7}#
\end{upotreba}
\end{miditest}
\begin{miditest}
\begin{upotreba}{2}
#\naslovInt#
#\izlaz{Unesite broj:}\ulaz{-0x7F}#
#\izlaz{Unesite redni broj funkcije:}\ulaz{2}#
#\izlaz{Poziva se funkcija prebroj\_bitove\_2}#
#\izlaz{Broj jedinica u zapisu je 26}#
\end{upotreba}
\end{miditest}
%\begin{miditest}
%\begin{upotreba}{2}
%#\naslovInt#
%#\izlaz{Unesite broj:}\ulaz{0x80}#
%#\izlaz{Unesite redni broj funkcije:}\ulaz{2}#
%#\izlaz{Broj jedinica u zapisu je}#
%#\izlaz{funkcija prebroj\_bitove\_2: 1}#
%\end{upotreba}
%\end{miditest}

\begin{miditest}
\begin{upotreba}{3}
#\naslovInt#
#\izlaz{Unesite broj:}\ulaz{0x00FF00FF}#
#\izlaz{Unesite redni broj funkcije:}\ulaz{2}#
#\izlaz{Poziva se funkcija prebroj\_bitove\_2}#
#\izlaz{Broj jedinica u zapisu je 16}#
\end{upotreba}
\end{miditest}
\begin{miditest}
\begin{upotreba}{4}
#\naslovInt#
#\izlaz{Unesite broj:}\ulaz{0x00FF00FF}#
#\izlaz{Unesite redni broj funkcije:}\ulaz{3}#
#\naslovIzlazZaGresku#
#\izlaz{Greska: Neodgovarajuci redni broj funkcije.}#
\end{upotreba}
\end{miditest}

\linkresenje{1_06}
\end{Exercise}
\begin{Answer}[ref=1_06]
\includecode{resenja/1_UvodniZadaci/1_06.c}
\end{Answer}


\begin{Exercise}[label=1_07]
Napisati funkcije \kckod{unsigned najveci(unsigned x)} i \kckod{unsigned najmanji(unsigned x)} koje vraćaju najveći, odnosno najmanji neoznačen ceo broj koji se može zapisati istim binarnim ciframa kao broj \argf{x}.

Napisati program koji testira prethodno napisane funkcije. Sa standardnog ulaza učitati neoznačen ceo broj u heksadekadnom formatu, a zatim ispisati binarnu reprezentaciju najvećeg i najmanjeg broja koji se može zapisati istim binarnim ciframa kao učitani broj.

\begin{miditest}
\begin{test}{1}
#\naslovUlaz#
#\ulaz{0x7F}#
#\naslovIzlaz#
#\izlaz{Najveci:}#
#\izlaz{11111110000000000000000000000000}#
#\izlaz{Najmanji:}#
#\izlaz{00000000000000000000000001111111}#
\end{test}
\end{miditest}
\begin{miditest}
\begin{test}{2}
#\naslovUlaz#
#\ulaz{0x80}#
#\naslovIzlaz#
#\izlaz{Najveci:}#
#\izlaz{10000000000000000000000000000000}#
#\izlaz{Najmanji:}#
#\izlaz{00000000000000000000000000000001}#
\end{test}
\end{miditest}

\begin{miditest}
\begin{test}{3}
#\naslovUlaz#
#\ulaz{0x00FF00FF}#
#\naslovIzlaz#
#\izlaz{Najveci:}#
#\izlaz{11111111111111110000000000000000}#
#\izlaz{Najmanji:}#
#\izlaz{00000000000000001111111111111111}#
\end{test}
\end{miditest}
\begin{miditest}
\begin{test}{4}
#\naslovUlaz#
#\ulaz{0xFFFFFFFF}#
#\naslovIzlaz#
#\izlaz{Najveci:}#
#\izlaz{11111111111111111111111111111111}#
#\izlaz{Najmanji:}#
#\izlaz{11111111111111111111111111111111}#
\end{test}
\end{miditest}

% \begin{miditest}
% \begin{test}{5}
% #\naslovUlaz#
% #\ulaz{0xABCDE123}#
% #\naslovIzlaz#
% #\izlaz{Najveci:}#
% #\izlaz{11111111111111111000000000000000}#
% #\izlaz{Najmanji:}#
% #\izlaz{00000000000000011111111111111111}#
% \end{test}
% \end{miditest}

\linkresenje{1_07}
\end{Exercise}
\begin{Answer}[ref=1_07]
\\
\napomena{Rešenje koristi biblioteku za štampanje bitova iz zadatka \ref{1_05}.}
\includecode{resenja/1_UvodniZadaci/1_07.c}
\end{Answer}


\begin{Exercise}[label=1_08]
Napisati funkcije za rad sa bitovima. 
\begin{enumerate}
\item Napisati funkciju \kckod{unsigned postavi\_0(unsigned x, unsigned n, unsigned p)} koja vraća broj koji se dobija kada se \argf{n} bitova datog broja \argf{x}, počevši od pozicije \argf{p}, postave na $0$.
\item Napisati funkciju \kckod{unsigned postavi\_1(unsigned x, unsigned n, unsigned p)} koja vraća broj koji se dobija kada se \argf{n} bitova datog broja \argf{x}, počevši od pozicije \argf{p}, postave na $1$.
\item %Napisati funkciju \kckod{vrati\_bitove} koja određuje broj koji se dobija od $n$ bitova datog broja $x$, počevši od pozicije $p$, i vraća ih kao bitove najmanje težine rezultata.
Napisati funkciju \kckod{unsigned vrati\_bitove(unsigned x, unsigned n, unsigned p)} koja vraća broj u kome se \argf{n} bitova najmanje težine poklapa sa $n$ bitova broja \argf{x} počevši od pozicije \argf{p}, dok su mu ostali bitovi postavljeni na $0$.
\item %Napisati funkciju \kckod{postavi\_1\_n\_bits} koja vraća broj koji se dobija upisivanjem poslednjih $n$ bitova broja $y$ u broj $x$, počevši od pozicije $p$.
Napisati funkciju \kckod{unsigned postavi\_1\_n\_bitova(unsigned x, unsigned n, unsigned p, unsigned y)} koja vraća broj koji se dobija upisivanjem poslednjih \argf{n} bitova najmanje težine broja \argf{y} u broj \argf{x}, počevši od pozicije \argf{p}.
\item Napisati funkciju \kckod{unsigned invertuj(unsigned x, unsigned n, unsigned p)} koja vraća broj koji se dobija invertovanjem \argf{n} bitova broja \argf{x} počevši od pozicije \argf{p}. 
\end{enumerate}

Napisati program koji testira prethodno napisane funkcije za neoznačene cele brojeve $x$, $n$, $p$, $y$ koji se unose sa standardnog ulaza. Na standardni izlaz ispisati binarne reprezenatacije brojeva $x$ i $y$, a zatim i binarne reprezentacije brojeva koji se dobijaju pozivanjem prethodno napisanih funkcija. \napomena{Bit najmanje težine je krajnji desni bit i njegova pozicija se označava nultom dok se pozicije ostalih bitova uvećavaju za jedan, sa desna na levo.}

\begin{maxitest}
\begin{upotreba}{1}
#\naslovInt#
#\izlaz{Unesite neoznacen ceo broj x:}\ulaz{235}#
#\izlaz{Unesite neoznacen ceo broj n:}\ulaz{9}#
#\izlaz{Unesite neoznacen ceo broj p:}\ulaz{24}#
#\izlaz{Unesite neoznacen ceo broj y:}\ulaz{127}#
#\izlaz{x =\ \ \ \ 235\ \ \ \ \ \ \ \ \ \ \ \ \ \ \ \ \ \ \ \ \ \ \ \ \ \ \ \ \ \ \ \ \ \ \ \ \ \ \ = 00000000000000000000000011101011}#
#\izlaz{postavi\_0(\ \ \ 235,\ \ \ \ \ 9,\ \ \ \ 24)\ \ \ \ \ \ \ \ \ \ \ \ \ \ \ \ \ \ = 00000000000000000000000011101011}#
#\izlaz{}#
#\izlaz{x =\ \ \ \ 235\ \ \ \ \ \ \ \ \ \ \ \ \ \ \ \ \ \ \ \ \ \ \ \ \ \ \ \ \ \ \ \ \ \ \ \ \ \ \ = 00000000000000000000000011101011}#
#\izlaz{postavi\_1(\ \ \ 235,\ \ \ \ \ 9,\ \ \ \ 24)\ \ \ \ \ \ \ \ \ \ \ \ \ \ \ \ \ \ = 00000001111111110000000011101011}#
#\izlaz{}#
#\izlaz{x =\ \ \ \ 235\ \ \ \ \ \ \ \ \ \ \ \ \ \ \ \ \ \ \ \ \ \ \ \ \ \ \ \ \ \ \ \ \ \ \ \ \ \ \ = 00000000000000000000000011101011}#
#\izlaz{vrati\_bitove(\ \ \ 235,\ \ \ \ \ 9,\ \ \ \ 24)\ \ \ \ \ \ \ \ \ \ \ \ \ \ \ = 00000000000000000000000000000000}#
#\izlaz{}#
#\izlaz{x =\ \ \ \ 235\ \ \ \ \ \ \ \ \ \ \ \ \ \ \ \ \ \ \ \ \ \ \ \ \ \ \ \ \ \ \ \ \ \ \ \ \ \ \ = 00000000000000000000000011101011}#
#\izlaz{y =\ \ \ \ 127\ \ \ \ \ \ \ \ \ \ \ \ \ \ \ \ \ \ \ \ \ \ \ \ \ \ \ \ \ \ \ \ \ \ \ \ \ \ \ = 00000000000000000000000001111111}#
#\izlaz{postavi\_1\_n\_bitove(\ \ 235,\ \ 9,\ \ 24,\ \ 127) \ \ \ \ \ \ \ \ = 00000000011111110000000011101011}#
#\izlaz{}#
#\izlaz{x =\ \ \ \ 235\ \ \ \ \ \ \ \ \ \ \ \ \ \ \ \ \ \ \ \ \ \ \ \ \ \ \ \ \ \ \ \ \ \ \ \ \ \ \ = 00000000000000000000000011101011}#
#\izlaz{invertuj(\ \ \ 235,\ \ \ \ \ 9,\ \ \ \ 24)\ \ \ \ \ \ \ \ \ \ \ \ \ \ \ \ \ \ \ = 00000001111111110000000011101011}#
\end{upotreba}
\end{maxitest}

\begin{maxitest}
\begin{upotreba}{2}
#\naslovInt#
#\izlaz{Unesite neoznacen ceo broj x:}\ulaz{2882398951}#
#\izlaz{Unesite neoznacen ceo broj n:}\ulaz{5}#
#\izlaz{Unesite neoznacen ceo broj p:}\ulaz{10}#
#\izlaz{Unesite neoznacen ceo broj y:}\ulaz{35156526}#
#\izlaz{x = 2882398951\ \ \ \ \ \ \ \ \ \ \ \ \ \ \ \ \ \ \ \ \ \ \ \ \ \ \ \ \ \ \ \ \ \ = 10101011110011011110101011100111}#
#\izlaz{postavi\_0(2882398951,\ \ \ \ \ 5,\ \ \ \ 10)\ \ \ \ \ \ \ \ \ \ \ \ \ = 10101011110011011110100000100111}#
#\izlaz{}#
#\izlaz{x = 2882398951\ \ \ \ \ \ \ \ \ \ \ \ \ \ \ \ \ \ \ \ \ \ \ \ \ \ \ \ \ \ \ \ \ \ = 10101011110011011110101011100111}#
#\izlaz{postavi\_1(2882398951,\ \ \ \ \ 5,\ \ \ \ 10)\ \ \ \ \ \ \ \ \ \ \ \ \ = 10101011110011011110111111100111}#
#\izlaz{}#
#\izlaz{x = 2882398951\ \ \ \ \ \ \ \ \ \ \ \ \ \ \ \ \ \ \ \ \ \ \ \ \ \ \ \ \ \ \ \ \ \ = 10101011110011011110101011100111}#
#\izlaz{vrati\_bitove(2882398951,\ \ \ \ \ 5,\ \ \ \ 10)\ \ \ \ \ \ \ \ \ \ = 00000000000000000000000000001011}#
#\izlaz{}#
#\izlaz{x = 2882398951\ \ \ \ \ \ \ \ \ \ \ \ \ \ \ \ \ \ \ \ \ \ \ \ \ \ \ \ \ \ \ \ \ \ = 10101011110011011110101011100111}#
#\izlaz{y =\ \ 35156526\ \ \ \ \ \ \ \ \ \ \ \ \ \ \ \ \ \ \ \ \ \ \ \ \ \ \ \ \ \ \ \ \ \ \ = 00000010000110000111001000101110}#
#\izlaz{postavi\_1\_n\_bitove(2882398951, 5, 10, 35156526) = 10101011110011011110101110100111}#
#\izlaz{}#
#\izlaz{x = 2882398951\ \ \ \ \ \ \ \ \ \ \ \ \ \ \ \ \ \ \ \ \ \ \ \ \ \ \ \ \ \ \ \ \ \ = 10101011110011011110101011100111}#
#\izlaz{invertuj(2882398951,\ \ \ \ \ 5,\ \ \ \ 10)\ \ \ \ \ \ \ \ \ \ \ \ \ \ = 10101011110011011110110100100111}#
\end{upotreba}
\end{maxitest}

\linkresenje{1_08}
\end{Exercise}
\begin{Answer}[ref=1_08]
\\
\napomena{Rešenje koristi biblioteku za štampanje bitova iz zadatka \ref{1_05}.}
\includecode{resenja/1_UvodniZadaci/1_08.c}
\end{Answer}

\begin{Exercise}[label=1_09] %ex 205
Pod rotiranjem bitova ulevo podrazumeva se pomeranje svih bitova za jednu poziciju ulevo, s tim što se bit sa pozicije najveće težine pomera na poziciju najmanje težine. Analogno, rotiranje bitova udesno podrazumeva pomeranje svih bitova za jednu poziciju udesno, s tim što se bit sa pozicije najmanje težine pomera na poziciju najveće težine.
\begin{enumerate}
\item Napisati funkciju \kckod{unsigned rotiraj\_ulevo(unsigned x, unsigned n)} koja vraća broj koji se dobija rotiranjem \argf{n} puta ulevo datog celog neoznačenog broja \argf{x}. 
\item Napisati funkciju \kckod{unsigned rotiraj\_udesno(unsigned x, unsigned n)} koja vraća broj koji se dobija rotiranjem \argf{n} puta udesno datog celog neoznačenog broja \argf{x}. 
\item Napisati funkciju \kckod{int rotiraj\_udesno\_oznaceni(int x, unsigned n)} koja vraća broj koji se dobija rotiranjem \argf{n} puta udesno datog celog broja \argf{x}. 
\end{enumerate}
Napisati program koji sa standardnog ulaza učitava neoznačene cele brojeve \argf{x} i \argf{n} koji se unose u heksadekasnom formatu, tatim ispisuje binarnu reprezentaciju vrednosti dobijene pozivanjem tri prethodno napisane funkcije sa argumentima  \argf{x} i \argf{n}, a na kraju ispisuje binarnu reprezentaciju vrednosti dobijene pozivanjem funkcije \kckod{rotiraj\_udesno\_oznaceni} za argumente \argf{-x} i \argf{n}.
%Napisati program koji testira prethodno napisane funkcije za broj \argf{x} i broj \argf{n} koji se unose u heksadekasnom formatu sa standardnog ulaza.

\begin{maxitest}
\begin{upotreba}{1}
#\naslovInt#
#\izlaz{Unesite neoznacen ceo broj x:}\ulaz{ba11a7}#
#\izlaz{Unesite neoznacen ceo broj n:}\ulaz{5}#
#\izlaz{x  = 00000000101110100001000110100111 }#
#\izlaz{rotiraj\_ulevo(ba11a7, 5) \ \ \ \ \ \ \ \ \ \ \ = 00010111010000100011010011100000}#
#\izlaz{rotiraj\_udesno(ba11a7, 5) \ \ \ \ \ \ \ \ \ \ = 00111000000001011101000010001101}#
#\izlaz{rotiraj\_udesno\_oznaceni(ba11a7, 5) \ = 00111000000001011101000010001101}#
#\izlaz{rotiraj\_udesno\_oznaceni(-ba11a7, 5) =  11000111111110100010111101110010}#
\end{upotreba}
\end{maxitest}

%\begin{maxitest}
%\begin{upotreba}{1}
%#\naslovInt#
%#\izlaz{Unesite neoznacen ceo broj x:}\ulaz{b10011a7}#
%#\izlaz{Unesite neoznacen ceo broj n:}\ulaz{5}#
%#\izlaz{x  = 10110001000000000001000110100111 }#
%#\izlaz{rotiraj\_ulevo(b10011a7, 5) \ \ \ \ \ \ \ \ \ \ = 00100000000000100011010011110110}#
%#\izlaz{rotiraj\_udesno(b10011a7, 5) \ \ \ \ \ \ \ \ \ = 00111101100010000000000010001101}#
%#\izlaz{rotiraj\_udesno\_oznaceni(b10011a7, 5) = 00111101100010000000000010001101 }#
%\end{upotreba}
%\end{maxitest}

\linkresenje{1_09}
\end{Exercise}
\begin{Answer}[ref=1_09]
\\
\napomena{Rešenje koristi biblioteku za štampanje bitova iz zadatka \ref{1_05}.}
\includecode{resenja/1_UvodniZadaci/1_09.c}
\end{Answer}

\begin{Exercise}[label=1_10]
Napisati funkciju \kckod{unsigned ogledalo(unsigned x)} koja vraća ceo broj čiji binarni zapis predstavlja sliku u ogledalu binarnog zapisa broja \argf{x}. Napisati program koji testira datu funkciju za broj koji se sa standardnog ulaza zadaje u heksadekadnom formatu. Najpre ispisati binarnu reprezentaciju unetog broja, a zatim i binarnu reprezentaciju broja dobijenog kao njegova slika u ogledalu.

\begin{miditest}
\begin{test}{1}
#\naslovUlaz#
#\ulaz{255 }#
#\naslovIzlaz#
#\izlaz{00000000000000000000001001010101}#
#\izlaz{10101010010000000000000000000000}#
\end{test}
\end{miditest}
\begin{miditest}
\begin{test}{2}
#\naslovUlaz#
#\ulaz{-15}#
#\naslovIzlaz#
#\izlaz{11111111111111111111111111101011}#
#\izlaz{11010111111111111111111111111111}#
\end{test}
\end{miditest}

\linkresenje{1_10}
\end{Exercise}
\begin{Answer}[ref=1_10]
\\
\napomena{Rešenje koristi biblioteku za štampanje bitova iz zadatka \ref{1_05}.}
\includecode{resenja/1_UvodniZadaci/1_10.c}
\end{Answer}

%%%%%%%%%%%%%%%%%%%%%%%%%
% Zadaci sa praktikuma - obavezni zadaci 
%%%%%%%%%%%%%%%%%%%%%%%%%

\begin{Exercise}[label=1_11] %ex 207
Napisati funkciju \kckod{int broj\_01(unsigned int n)} koja za dati broj \argf{n} vraća \argf{1} ako u njegovom binarnom zapisu ima više jednica nego nula, a inače vraća \argf{0}.  Napisati program koji tu funkciju testira za broj koji se zadaje sa standardnog ulaza.

\begin{minitest}
\begin{test}{1}
#\naslovUlaz#
#\ulaz{10}#
#\naslovIzlaz#
#\izlaz{0}#
\end{test}
\end{minitest}
%\begin{minitest}
%\begin{test}{2}
%#\naslovUlaz#
%#\ulaz{1024}#
%#\naslovIzlaz#
%#\izlaz{0}#
%\end{test}
%\end{minitest}
\begin{minitest}
\begin{test}{2}
#\naslovUlaz#
#\ulaz{2147377146}#
#\naslovIzlaz#
#\izlaz{1}#
\end{test}
\end{minitest}
\begin{minitest}
\begin{test}{3}
#\naslovUlaz#
#\ulaz{1111111115}#
#\naslovIzlaz#
#\izlaz{0}#
\end{test}
\end{minitest}

\linkresenje{1_11}
\end{Exercise}
\begin{Answer}[ref=1_11]
\includecode{resenja/1_UvodniZadaci/1_11.c}
\end{Answer}

\begin{Exercise}[label=1_12]
Napisati funkciju \kckod{int broj\_parova(unsigned int x)} koja vraća broj pojava dve uzastopne jedinice u binarnom zapisu celog neoznačenog broja \argf{x}. Napisati program koji tu funkciju testira za broj koji se zadaje sa standardnog ulaza. \napomena{Tri uzastopne jedinice sadrže dve uzastopne jedinice dva puta.}

\begin{minitest}
\begin{test}{1}
#\naslovUlaz#
#\ulaz{11}#
#\naslovIzlaz#
#\izlaz{1}#
\end{test}
\end{minitest}
\begin{minitest}
\begin{test}{2}
#\naslovUlaz#
#\ulaz{1024}#
#\naslovIzlaz#
#\izlaz{0}#
\end{test}
\end{minitest}
\begin{minitest}
\begin{test}{3}
#\naslovUlaz#
#\ulaz{2147377146}#
#\naslovIzlaz#
#\izlaz{22}#
\end{test}
\end{minitest}  

\linkresenje{1_12}
\end{Exercise}
\begin{Answer}[ref=1_12]
\includecode{resenja/1_UvodniZadaci/1_12.c}
\end{Answer}


%%%
%Ovaj 209.c nema resenje za sad...
%%%
\begin{Exercise}[difficulty=1, label=1_13]%\marker+{2}
Napisati program koji sa standardnog ulaza učitava pozitivan ceo broj, a na standardni izlaz ispisuje vrednost tog broja sa razmenjenim vrednostima bitova na pozicijama $i$ i $j$. Pozicije $i$ i $j$ učitati kao parametre komandne linije. Pri rešavanju nije dozvoljeno koristiti ni pomoćni niz ni aritmetičke operatore +, -, /, *, \%.

\begin{minitest}
\begin{upotreba}{1}
#\poziv{./a.out 1 2 }#

#\naslovInt#
#\naslovUlaz#
#\ulaz{11}#
#\naslovIzlaz#
#\izlaz{13}#
\end{upotreba}
\end{minitest}
\begin{minitest}
\begin{upotreba}{2}
#\poziv{./a.out 1 2 }#

#\naslovInt#
#\naslovUlaz#
#\ulaz{1024}#
#\naslovIzlaz#
#\izlaz{1024}#
\end{upotreba}
\end{minitest}
\begin{minitest}
\begin{upotreba}{2}
#\poziv{./a.out 12 12 }#

#\naslovInt#
#\naslovUlaz#
#\ulaz{12345}#
#\naslovIzlaz#
#\izlaz{12345}#
\end{upotreba}
\end{minitest}

\end{Exercise}
%\begin{Answer}[ref=1_13]
%\includecode{resenja/1_UvodniZadaci/1_13.c}
%\end{Answer}

\begin{Exercise}[difficulty=1, label=1_14]
  Napisati funkciju \kckod{void prevod(unsigned int x, char s[])} koja na osnovu neoznačenog broja $x$
  formira nisku $s$ koja sadrži heksadekadni zapis broja
  $x$ koristeći algoritam za brzo prevođenje binarnog u
  heksadekadni zapis (svake $4$ binarne cifre se zamenjuju jednom
  odgovarajućom heksadekadnom cifrom).  Napisati program koji tu
  funkciju testira za broj koji se zadaje sa standardnog ulaza.

\begin{minitest}
\begin{test}{1}
#\naslovUlaz#
#\ulaz{11}#
#\naslovIzlaz#
#\izlaz{0000000B}#
\end{test}
\end{minitest}
\begin{minitest}
\begin{test}{2}
#\naslovUlaz#
#\ulaz{1024}#
#\naslovIzlaz#
#\izlaz{00000400}#
\end{test}
\end{minitest}
\begin{minitest}
\begin{test}{3}
#\naslovUlaz#
#\ulaz{12345}#
#\naslovIzlaz#
#\izlaz{00003039}#
\end{test}
\end{minitest}  

\linkresenje{1_14}
\end{Exercise}
\begin{Answer}[ref=1_14]
\includecode{resenja/1_UvodniZadaci/1_14.c}
\end{Answer}

%%%%%%%%%%%%%%%%%%%%%%%%%
% Zadaci sa praktikuma - dodatni zadaci - oni nemaju rešenja
% možda \subsection{Dodatni zadaci} ili zadaci za vežbu
%%%%%%%%%%%%%%%%%%%%%%%%%

\begin{Exercise}[difficulty=1, label=1_15]%\marker+{2}
Napisati funkciju koja za data dva neoznačena broja $x$
i $y$ invertuje one bitove u broju $x$ koji se poklapaju sa odgovarajućim bitovima
u broju $y$. Ostali bitovi treba da ostanu nepromenjeni. Napisati program
koji testira tu funkciju za brojeve koji se zadaju sa standardnog ulaza.

\begin{minitest}
\begin{test}{1}
#\naslovUlaz#
#\ulaz{123 10}#
#\naslovIzlaz#
#\izlaz{4294967285}#
\end{test}
\end{minitest}
\begin{minitest}
\begin{test}{2}
#\naslovUlaz#
#\ulaz{3251 0}#
#\naslovIzlaz#
#\izlaz{4294967295}#
\end{test}
\end{minitest}
\begin{minitest}
\begin{test}{3}
#\naslovUlaz#
#\ulaz{12541 1024}#
#\naslovIzlaz#
#\izlaz{4294966271}#
\end{test}
\end{minitest}    
  
\end{Exercise}
%\begin{Answer}[ref=1_15]
%\includecode{resenja/1_UvodniZadaci/1_15.c}
%\end{Answer}

\begin{Exercise}[label=1_16]%\marker+{2}
Napisati funkciju koja vraća broj petica u oktalnom zapisu neoznačenog celog broja $x$. Napisati program koji testira tu funkciju za broj koji se zadaje sa standardnog ulaza. \napomena{Zadatak rešiti isključivo korišćenjem bitskih operatora.}

\begin{minitest}
\begin{test}{1}
#\naslovUlaz#
#\ulaz{123}#
#\naslovIzlaz#
#\izlaz{0}#
\end{test}
\end{minitest}
\begin{minitest}
\begin{test}{2}
#\naslovUlaz#
#\ulaz{3245}#
#\naslovIzlaz#
#\izlaz{2}#
\end{test}
\end{minitest}
\begin{minitest}
\begin{test}{3}
#\naslovUlaz#
#\ulaz{100328}#
#\naslovIzlaz#
#\izlaz{1}#
\end{test}
\end{minitest}   
 
\end{Exercise}
%\begin{Answer}[ref=1_16]
%\includecode{resenja/1_UvodniZadaci/1_16.c}
%\end{Answer}


\section{Rekurzija}

%%%\subsection{Rekurzivne funkcije nad brojevima}

\begin{Exercise}[label=1_17]
Napisati rekurzivnu funkciju koja izračunava  $x^k$, za dati ceo broj $x$ i prirodan broj $k$
\begin{enumerate}
\item tako da rešenje bude linearne složenosti,
\item tako da rešenje bude logaritamske složenosti.
\end{enumerate}
Napisati program koji testira napisane funkcije. Sa standardnog ulaza
učitati redni broj funkcije koju treba primeniti ('1' ili '2'), ceo broj $x$ i
prirodan broj $k$, a zatim na standarni izlaz ispisati rezultat primene
izabrane funkcije na unete brojeve. Ukoliko se na ulazu unese pogrešan redni broj
funkcije, ispisati odgovarajuću poruku o grešci na standardni izlaz i
prekinuti izvršavanje programa.

\begin{miditest}
\begin{upotreba}{1}
#\naslovInt#
#\izlaz{Unesite redni broj funkcije (1/2):}# 
#\ulaz{1}#
#\izlaz{Unesite broj x:} \ulaz{ 2}#
#\izlaz{Unesite broj k:} \ulaz{ 10}#
#\izlaz{1024}#
\end{upotreba}
\end{miditest}
\begin{miditest}
\begin{upotreba}{2}
#\naslovInt#
#\izlaz{Unesite redni broj funkcije (1/2):}# 
#\ulaz{2}#
#\izlaz{Unesite broj x:} \ulaz{ 9}#
#\izlaz{Unesite broj k:} \ulaz{ 4}#
#\izlaz{6561}#
\end{upotreba}
\end{miditest}    
 
\linkresenje{1_17}
\end{Exercise}
\begin{Answer}[ref=1_17]
\includecode{resenja/1_UvodniZadaci/1_17.c}
\end{Answer}

\begin{Exercise}[label=1_18]
Koristeći uzajamnu (posrednu) rekurziju napisati:
 \begin{enumerate}
\item funkciju \kckod{unsigned paran(unsigned n)} koja proverava da li je broj cifara broja \argf{x} paran i vraća $1$ ako jeste, a $0$ inače;
\item i funkciju \kckod{unsigned neparan(unsigned n)} koja proverava da li je broj cifara broja \argf{x} neparan i vraća $1$ ako jeste, a $0$ inače.
 \end{enumerate}
Napisati program koji testira napisane funkcije tako što za heksadekadni broj koji se unosi sa standardnog ulaza ispisuje da li je broj njegovih cifara paran ili neparan.
 
\begin{miditest}
\begin{test}{1}
#\naslovUlaz#
#\ulaz{11}#
#\naslovIzlaz#
#\izlaz{Uneti broj ima paran broj cifara.}#
\end{test}
\end{miditest}
\begin{miditest}
\begin{test}{2}
#\naslovUlaz#
#\ulaz{123}#
#\naslovIzlaz#
#\izlaz{Uneti broj ima neparan broj cifara.}#
\end{test}
\end{miditest}
 
\linkresenje{1_18}
\end{Exercise}
\begin{Answer}[ref=1_18]
\includecode{resenja/1_UvodniZadaci/1_18.c}
\end{Answer}

\begin{Exercise}[label=1_19]
Napisati repno-rekurzivnu funkciju koja izračunava faktorijel broja $n$. Napisati program koji testira napisanu funkciju za proizvoljan broj $n$ ($n \le 12$) unet sa standardnog ulaza. \napomena{Gornja vrednost za $n$ je postavljena na $12$ zbog ograničenja veličine broja koji može da stane u promenljivu tipa \kckod{int} i činjenice da niz faktorijela brzo raste.}

\begin{miditest}
\begin{upotreba}{1}
#\naslovInt#
#\izlaz{Unesite n (<= 12):} \ulaz{5}#
#\izlaz{5! = 120}#
\end{upotreba}
\end{miditest}
\begin{miditest}
\begin{upotreba}{2}
#\naslovInt#
#\izlaz{Unesite n (<= 12):} \ulaz{0}#
#\izlaz{0! = 1}#
\end{upotreba}
\end{miditest}

\linkresenje{1_19}
\end{Exercise}
\begin{Answer}[ref=1_19]
\includecode{resenja/1_UvodniZadaci/1_19.c}
\end{Answer}

\begin{Exercise}[label=1_20]
Napisati funkciju koja vraća $n$-ti element u nizu Fibonačijevih brojeva. Elementi niza Fibonačijevih brojeva $F$ izračunavaju se na osnovu sledećih rekurentnih relacija:
 $$F(0) = 0$$
 $$F(1) = 1$$
 $$F(n) = F(n-1) + F(n-2).$$ 
Napisati program koji testira napisanu funkciju. Sa standardnog ulaza učitati prirodan broj $n$ i na standardni izlaz ispisati rezultat primene napisane funkcije na prirodan broj $n$.

\begin{miditest}
\begin{upotreba}{1}
#\naslovInt#
#\izlaz{Unesite koji clan niza se racuna:} \ulaz{ 5}#
#\izlaz{F(5) = 5}#
\end{upotreba}
\end{miditest}
\begin{miditest}
\begin{upotreba}{2}
#\naslovInt#
#\izlaz{Unesite koji clan niza se racuna:} \ulaz{8}#
#\izlaz{F(8) = 21}#
\end{upotreba}
\end{miditest}

%\linkresenje{1_20}
\end{Exercise}
%\begin{Answer}[ref=1_20]
%\includecode{resenja/1_UvodniZadaci/1_20.c}
%\end{Answer}

\begin{Exercise}[label=1_21]
Elementi niza $F$ izračunavaju se na osnovu sledećih rekurentnih relacija:
 $$F(0) = 0$$
 $$F(1) = 1$$
 $$F(n) = a* F(n-1) + b*F(n-2).$$
Napisati funkciju koja računa $n$-ti element u nizu $F$
\begin{enumerate}
\item iterativno,
\item tako da funkcija bude rekurzivna i da koristi navedene rekurentne relacije,
\item tako da funkcija bude rekurzivna ali da se problemi manje dimenzije rešavaju samo jedan put.
\end{enumerate}
Napisati program koji testira napisane funkcije. Sa standardnog ulaza
učitati redni broj funkcije koju treba primeniti ('1','2','3'), vrednosti
koeficijenata $a$ i $b$ i prirodan broj $n$. Na standardni izlaz ispisati rezultat primene odabrane funkcije nad učitanim podacima, a u slučaju unosa pogrešnog rednog broja funkcije ispisati odgovarajuću poruku i prekinuti izvršavanje pograma. \napomena{Niz  $F$ definisan na ovaj način predstavlja uopštenje Fibonačijevih brojeva.}

 
\begin{miditest}
\begin{upotreba}{1}
#\naslovInt#
#\izlaz{Unesite redni broj funkcije:}# 
#\izlaz{1 - iterativna}# 
#\izlaz{2 - rekurzivna}# 
#\izlaz{3 - rekurzivna napredna}# 
#\ulaz{1}#
#\izlaz{Unesite koeficijente:} \ulaz{ 2 3}#
#\izlaz{Unesite koji clan niza se racuna:} \ulaz{ 5}#
#\izlaz{F(5) = 61}#
\end{upotreba}
\end{miditest}
\begin{miditest}
\begin{upotreba}{2}
#\naslovInt#
#\izlaz{Unesite redni broj funkcije:}# 
#\izlaz{1 - iterativna}# 
#\izlaz{2 - rekurzivna}# 
#\izlaz{3 - rekurzivna napredna}# 
#\ulaz{3}#
#\izlaz{Unesite koeficijente:} \ulaz{4 2}#
#\izlaz{Unesite koji clan niza se racuna:} \ulaz{8}#
#\izlaz{F(8) = 31360}#
\end{upotreba}
\end{miditest}

\linkresenje{1_21}
\end{Exercise}
\begin{Answer}[ref=1_21]
\includecode{resenja/1_UvodniZadaci/1_21.c}
\end{Answer}

\begin{Exercise}[label=1_22]
Napisati rekurzivnu funkciju koja sabira dekadne cifre datog celog broja $x$. Napisati program koji testira ovu funkciju za broj koji se unosi sa standardnog ulaza.
  
\begin{minitest}
\begin{test}{1}
#\naslovUlaz#
#\ulaz{123}#
#\naslovIzlaz#
#\izlaz{6}#
\end{test}
\end{minitest}
\begin{minitest}
\begin{test}{2}
#\naslovUlaz#
#\ulaz{23156}#
#\naslovIzlaz#
#\izlaz{17}#
\end{test}
\end{minitest}
\begin{minitest}
\begin{test}{3}
#\naslovUlaz#
#\ulaz{1432}#
#\naslovIzlaz#
#\izlaz{10}#
\end{test}
\end{minitest}      
 
\begin{minitest}
\begin{test}{4}
#\naslovUlaz#
#\ulaz{1}#
#\naslovIzlaz#
#\izlaz{1}#
\end{test}
\end{minitest}
\begin{minitest}
\begin{test}{5}
#\naslovUlaz#
#\ulaz{0}#
#\naslovIzlaz#
#\izlaz{0}#
\end{test}
\end{minitest}      

\linkresenje{1_22}
\end{Exercise}
\begin{Answer}[ref=1_22]
\includecode{resenja/1_UvodniZadaci/1_22.c}
\end{Answer}

%%%\subsection{Rekurzivne funkcije za rad sa nizovima}

\begin{Exercise}[label=1_23]
Napisati rekurzivnu funkciju koja sumira elemente niza celih brojeva
\begin{enumerate}
\item sabirajući elemente počev od početka niza ka kraju niza,
\item sabirajući elemente počev od kraja niza ka početku niza.
\end{enumerate}
Napisati program koji testira napisane funkcije. Sa standardnog ulaza učitati redni broj funkcije ('1' ili '2'), zatim dimenziju $n$  ($0 < n \leq 100$) celobrojnog niza, a potom i elemente niza. Na standardni izlaz ispisati rezultat primene odabrane funkcije nad učitanim nizom, a u slučaju unosa pogrešnog rednog broja funkcije ispisati odgovarajuću poruku i prekinuti izvršavanje pograma.

\begin{miditest}
\begin{upotreba}{1}
#\naslovInt#
#\izlaz{Unesite redni broj funkcije (1 ili 2):}# 
#\ulaz{1}#
#\izlaz{Unesite dimenziju niza:}# 
#\ulaz{5}#
#\izlaz{Unesite elemente niza:}#
#\ulaz{1 2 3 4 5}#
#\izlaz{Suma elemenata je 15}#
\end{upotreba}
\end{miditest}
\begin{miditest}
\begin{upotreba}{2}
#\naslovInt#
#\izlaz{Unesite redni broj funkcije (1 ili 2):}# 
#\ulaz{2}#
#\izlaz{Unesite dimenziju niza:}# 
#\ulaz{4}#
#\izlaz{Unesite elemente niza:}#
#\ulaz{-5 2 -3 6}#
#\izlaz{Suma elemenata je 0}#
\end{upotreba}
\end{miditest}

\linkresenje{1_23}
\end{Exercise}
\begin{Answer}[ref=1_23]
\includecode{resenja/1_UvodniZadaci/1_23.c}
\end{Answer}

\begin{Exercise}[label=1_24]
Napisati rekurzivnu funkciju koja određuje maksimum niza celih brojeva. Napisati program koji testira ovu funkciju za niz koji se unosi sa standardnog ulaza. Elementi niza se unose sve do kraja ulaza (EOF). Pretpostaviti da niz neće imati više od $256$ elemenata. 
  
 \begin{minitest}
\begin{test}{1}
#\naslovUlaz#
#\ulaz{3 2 1 4 21}#
#\naslovIzlaz#
#\izlaz{21}#
\end{test}
\end{minitest}
\begin{minitest}
\begin{test}{2}
#\naslovUlaz#
#\ulaz{ 2 -1 0 -5 -10}#
#\naslovIzlaz#
#\izlaz{2}#
\end{test}
\end{minitest}
\begin{minitest}
\begin{test}{3}
#\naslovUlaz#
#\ulaz{1 11 3 5 8 1}#
#\naslovIzlaz#
#\izlaz{11}#
\end{test}
\end{minitest}
% \begin{minitest}
% \begin{test}{4}
% #\naslovUlaz#
% #\ulaz{ 5}#
% #\naslovIzlaz#
% #\izlaz{5}#
% \end{test}
% \end{minitest}

\linkresenje{1_24}
\end{Exercise}
\begin{Answer}[ref=1_24]
\includecode{resenja/1_UvodniZadaci/1_24.c}
\end{Answer}

\begin{Exercise}[label=1_25]
Napisati rekurzivnu funkciju koja izračunava skalarni proizvod dva vektora celih brojeva. Napisati program koji testira ovu funkciju za nizove (vektore) koji
se unose sa standardnog ulaza. Prvo treba uneti dimenziju nizova, a zatim i
njihove elemente. Na standardni izlaz ispisati  skalarni proizvod unetih
nizova. Pretpostaviti da nizovi neće imati više od $256$ elemenata.

\begin{miditest}
\begin{upotreba}{1}
#\naslovInt#
#\izlaz{Unesite dimenziju nizova:} \ulaz{3}#
#\izlaz{Unesite elemente prvog niza:}#
#\ulaz{1 2 3}#
#\izlaz{Unesite elemente drugog niza:}#
#\ulaz{1 2 3}#
#\izlaz{Skalarni proizvod je 14}#
\end{upotreba}
\end{miditest}
\begin{miditest}
\begin{upotreba}{2}
#\naslovInt#
#\izlaz{Unesite dimenziju nizova:} \ulaz{2}#
#\izlaz{Unesite elemente prvog niza:}#
#\ulaz{3 5}#
#\izlaz{Unesite elemente drugog niza:}#
#\ulaz{2 6}#
#\izlaz{Skalarni proizvod je 36}#
\end{upotreba}
\end{miditest}
  
%\begin{minitest}
%\begin{test}{3}
%#\naslovUlaz#
%#\ulaz{ 0 }#
%#\naslovIzlaz#
%#\izlaz{0}#
%\end{test}
%\end{minitest}  

\linkresenje{1_25}
\end{Exercise}
\begin{Answer}[ref=1_25]
\includecode{resenja/1_UvodniZadaci/1_25.c}
\end{Answer}

\begin{Exercise}[label=1_26]
Napisati rekurzivnu funkciju koja vraća broj pojavljivanja
elementa $x$ u nizu $a$ dužine $n$. Napisati program koji testira ovu funkciju za broj $x$ i niz  $a$ koji se unose sa standardnog ulaza. Prvo se unosi $x$, a zatim elementi niza sve do kraja ulaza. Pretpostaviti da nizovi neće imati više od $256$ elemenata.

\begin{miditest}
\begin{upotreba}{1}
#\naslovInt#
#\izlaz{Unesite ceo broj:}#
#\ulaz{4}#
#\izlaz{Unesite elemente niza:}#
#\ulaz{1 2 3 4}#
#\izlaz{Broj pojavljivanja je 1}#
\end{upotreba}
\end{miditest}
\begin{miditest}
\begin{upotreba}{2}
#\naslovInt#
#\izlaz{Unesite ceo broj:}#
#\ulaz{11}#
#\izlaz{Unesite elemente niza:}#
#\ulaz{3 2 11 14 11 43 1}#
#\izlaz{Broj pojavljivanja je 2}#
\end{upotreba}
\end{miditest}

\begin{miditest}
\begin{upotreba}{3}
#\naslovInt#
#\izlaz{Unesite ceo broj:}#
#\ulaz{1}#
#\izlaz{Unesite elemente niza:}#
#\ulaz{3 21 5 6}#
#\izlaz{Broj pojavljivanja je 0}#
\end{upotreba}
\end{miditest}

\linkresenje{1_26}
\end{Exercise}
\begin{Answer}[ref=1_26]
\includecode{resenja/1_UvodniZadaci/1_26.c}
\end{Answer}

\begin{Exercise}[label=1_27]
Napisati rekurzivnu funkciju kojom se proverava da li su tri data cela broja uzastopni članovi datog celobrojnog niza. Sa standardnog ulaza učitati tri broja, a zatim elemente niza sve do
kraja ulaza. Na standardni izlaz ispisati rezultat primene funkcije nad učitanim podacima. Pretpostaviti da neće biti uneto više od $256$ brojeva.

\begin{miditest}
\begin{upotreba}{1}
#\naslovInt#
#\izlaz{Unesite tri cela broja:}#
#\ulaz{1 2 3}#
#\izlaz{Unesite elemente niza:}#
#\ulaz{4 1 2 3 4 5}#
#\izlaz{Uneti brojevi jesu uzastopni }#
#\izlaz{clanovi niza.}#
\end{upotreba}
\end{miditest}
\begin{miditest}
\begin{upotreba}{2}
#\naslovInt#
#\izlaz{Unesite tri cela broja:}#
#\ulaz{1 2 3}#
#\izlaz{Unesite elemente niza:}#
#\ulaz{11 1 2 4 3 6}#
#\izlaz{Uneti brojevi jesu uzastopni }#
#\izlaz{clanovi niza.}#
\end{upotreba}
\end{miditest}
%
%\begin{miditest}
%\begin{test}{Test 3}
%Ulaz:     1 2 3 1 2
%Izlaz:    ne 
%\end{test}
%\end{miditest}

\linkresenje{1_27}
\end{Exercise}
\begin{Answer}[ref=1_27]
\includecode{resenja/1_UvodniZadaci/1_27.c}
\end{Answer}

%%%\subsection{Rekurzivne funkcije za rad sa bitovima}

\begin{Exercise}[label=1_28]
Napisati rekurzivnu funkciju \kckod{int prebroj(int x)} koja vraća broj bitova postavljenih na $1$ 
u binarnoj reprezentaciji broja \argf{x}.  
Napisati program koji testira napisanu funkciju za broj koji se učitava sa 
standardnog ulaza u heksadekadnom formatu. 

\begin{minitest}
\begin{test}{1}
#\naslovUlaz#
#\ulaz{0x7F}#
#\naslovIzlaz#
#\izlaz{7}#
\end{test}
\end{minitest}
%\begin{minitest}
%\begin{test}{2}
%#\naslovUlaz#
%#\ulaz{0x80}#
%#\naslovIzlaz#
%#\izlaz{1}#
%\end{test}
%\end{minitest}
\begin{minitest}
\begin{test}{2}
#\naslovUlaz#
#\ulaz{0x00FF00FF}#
#\naslovIzlaz#
#\izlaz{16}#
\end{test}
\end{minitest}  
\begin{minitest}
\begin{test}{3}
#\naslovUlaz#
#\ulaz{0xFFFFFFFF}#
#\naslovIzlaz#
#\izlaz{32}#
\end{test}
\end{minitest}  

\linkresenje{1_28}
\end{Exercise}
\begin{Answer}[ref=1_28]
\includecode{resenja/1_UvodniZadaci/1_28.c}
\end{Answer}

\begin{Exercise}[label=1_29]%\marker+{2}
Napisati rekurzivnu funkciju koja štampa bitovsku
reprezentaciju neoznačenog celog broja, i program koji je
testira za vrednost koja se zadaje sa standardnog ulaza.

\begin{miditest}
\begin{test}{1}
#\naslovUlaz#
#\ulaz{10}#
#\naslovIzlaz#
#\izlaz{00000000000000000000000000001010}#
\end{test}
\end{miditest}
\begin{miditest}
\begin{test}{2}
#\naslovUlaz#
#\ulaz{0}#
#\naslovIzlaz#
#\izlaz{00000000000000000000000000000000}#
\end{test}
\end{miditest}
\end{Exercise}

%\begin{Answer}[ref=1_29]
%\includecode{resenja/1_UvodniZadaci/1_29.c}
%\end{Answer}

\begin{Exercise}[label=1_30]
Napisati rekurzivnu funkciju za određivanje
najveće cifre u oktalnom zapisu
neoznačenog celog broja korišćenjem bitskih operatora.
\uputstvo{Binarne cifre grupisati u podgrupe od po tri cifre,
počev od bitova najmanje težine.}

\begin{minitest}
\begin{test}{1}
#\naslovUlaz#
#\ulaz{5}#
#\naslovIzlaz#
#\izlaz{5}#
\end{test}
\end{minitest}
\begin{minitest}
\begin{test}{2}
#\naslovUlaz#
#\ulaz{125}#
#\naslovIzlaz#
#\izlaz{7}#
\end{test}
\end{minitest}
\begin{minitest}
\begin{test}{3}
#\naslovUlaz#
#\ulaz{8}#
#\naslovIzlaz#
#\izlaz{1}#
\end{test}
\end{minitest}  

\linkresenje{1_30}
\end{Exercise}
\begin{Answer}[ref=1_30]
\includecode{resenja/1_UvodniZadaci/1_30.c}
\end{Answer}

\begin{Exercise}[label=1_31]
Napisati rekurzivnu funkciju za određivanje (dekadne vrednosti)
najveće cifre u heksadekadnom zapisu neoznačenog celog broja
korišćenjem bitskih operatora. \uputstvo{Binarne cifre
grupisati u podgrupe od po četiri cifre, počev od bitova
najmanje težine.}

\begin{minitest}
\begin{test}{1}
#\naslovUlaz#
#\ulaz{5}#
#\naslovIzlaz#
#\izlaz{5}#
\end{test}
\end{minitest}
\begin{minitest}
\begin{test}{2}
#\naslovUlaz#
#\ulaz{16}#
#\naslovIzlaz#
#\izlaz{1}#
\end{test}
\end{minitest}
\begin{minitest}
\begin{test}{3}
#\naslovUlaz#
#\ulaz{18}#
#\naslovIzlaz#
#\izlaz{2}#
\end{test}
\end{minitest}  
%\begin{minitest}
%\begin{test}{Test 4}
%Ulaz:  165
%Izlaz: 10
%\end{test}
%\end{minitest}

\linkresenje{1_31}
\end{Exercise}
\begin{Answer}[ref=1_31]
\includecode{resenja/1_UvodniZadaci/1_31.c}
\end{Answer}

%%%\subsection{Rekurzivne funkcije - razni zadaci}

\begin{Exercise}[label=1_32]
Napisati rekurzivnu funkciju \kckod{int palindrom(char s[], int n)} koja ispituje da li je data niska
\argf{s} palindrom. Napisati program koji testira ovu funkciju za nisku koja se zadaje sa standardnog ulaza. Pretpostaviti da niska neće imati više od $31$ karaktera.
  
\begin{minitest}
\begin{test}{1}
#\naslovUlaz#
#\ulaz{a}#
#\naslovIzlaz#
#\izlaz{da}#
\end{test}
\end{minitest}  
\begin{minitest}
\begin{test}{2}
#\naslovUlaz#
#\ulaz{aa}#
#\naslovIzlaz#
#\izlaz{da}#
\end{test}
\end{minitest}
\begin{minitest}
\begin{test}{3}
#\naslovUlaz#
#\ulaz{aba}#
#\naslovIzlaz#
#\izlaz{da}#
\end{test}
\end{minitest}  
 
\begin{minitest}
\begin{test}{4}
#\naslovUlaz#
#\ulaz{programiranje}#
#\naslovIzlaz#
#\izlaz{ne}#
\end{test}
\end{minitest}
\begin{miditest}
\begin{test}{5}
#\naslovUlaz#
#\ulaz{anavolimilovana}#
#\naslovIzlaz#
#\izlaz{da}#
\end{test}
\end{miditest}
 
\linkresenje{1_32}
\end{Exercise}
\begin{Answer}[ref=1_32]
\includecode{resenja/1_UvodniZadaci/1_32.c}
\end{Answer}

\begin{Exercise}[label=1_33, difficulty=1]
Napisati rekurzivnu funkciju koja prikazuje sve permutacije skupa $\{1, 2, ... ,n\}$. Napisati program koji testira napisanu funkciju za proizvoljan prirodan broj $n$ ($n \le 15$) unet sa standardnog ulaza.

\begin{minitest}
\begin{test}{1}
#\naslovUlaz#
#\ulaz{2}#
#\naslovIzlaz#
#\izlaz{1 2}#
#\izlaz{2 1}#
\end{test}
\end{minitest}  
\begin{minitest}
\begin{test}{2}
#\naslovUlaz#
#\ulaz{3}#
#\izlaz{1 2 3}#
#\izlaz{1 3 2}#
#\izlaz{2 1 3}#
#\izlaz{2 3 1}#
#\izlaz{3 1 2}#
#\izlaz{3 2 1}#
\end{test}
\end{minitest}
\begin{minitest}
\begin{test}{3}
#\naslovUlaz#
#\ulaz{-5}#
#\izlaz{Duzina }#
#\izlaz{permutacije}#
#\izlaz{mora biti}#
#\izlaz{broj iz }#
#\izlaz{intervala}#
#\izlaz{[0, 15]!}#
\end{test}
\end{minitest}


\linkresenje{1_33}
\end{Exercise}
\begin{Answer}[ref=1_33]
\includecode{resenja/1_UvodniZadaci/1_33.c}
\end{Answer}

\begin{Exercise}[label=1_34, difficulty=1]
Paskalov trougao sadrži brojeve čije se vrednosti računaju tako što svako polje ima vrednost
 zbira dve vrednosti koje su u susedna dva polja iznad. Izuzetak su jedinice na krajevima. Vrednosti
 brojeva Paskalovog trougla odgovaraju binomnim koeficijentima tj.~vrednost polja \argf{(n, k)}, gde je $n$ redni broj hipotenuze, a $k$ redni broj elementa u tom redu (na toj hipotenuzi) odgovara binomnom koeficijentu $\binom{n}{k}$, pri čemu brojanje počinje od nule. Na primer, vrednost polja \argf{(4, 2)} je $6$. 

\begin{verbatim}
               1
             1   1
           1   2   1
         1   3   3   1
       1   4   6   4   1
     1   5   10  10  5   1
\end{verbatim}

\begin{enumerate}
\item Napisati rekurzivnu funkciju koja izračunava vrednost binomnog koeficijenta $\binom{n}{k}$ koristeći osobine Paskalovog trougla. 
\item Napisati rekurzivnu funkciju koja izračunava $d_n$ kao sumu elemenata $n$-te hipotenuze Paskalovog trougla.
\end{enumerate}

Napisati program koji za unetu veličinu Paskalovog trougla i redni broj hipotenuze
najpre iscrtava Paskalov trougao, a zatim štampa sumu elemenata hipotenuze.

\begin{miditest}
\begin{test}{1}
#\naslovUlaz#
#\ulaz{5 3}#
#\naslovIzlaz#
            #\izlaz{1}#
          #\izlaz{1}#  #\izlaz{1}#
        #\izlaz{1}#  #\izlaz{2}#  #\izlaz{1}#
      #\izlaz{1}#  #\izlaz{3}#  #\izlaz{3}#  #\izlaz{1}#
    #\izlaz{1}#  #\izlaz{4}#  #\izlaz{6}#  #\izlaz{4}#  #\izlaz{1}#
  #\izlaz{1}#  #\izlaz{5}#  #\izlaz{10}# #\izlaz{10}# #\izlaz{5}# #\izlaz{1}#
#\izlaz{}#
#\izlaz{8}#
\end{test}
\end{miditest}
\begin{miditest}
\begin{test}{2}
#\naslovUlaz#
#\ulaz{6 5}#
#\naslovIzlaz#
            #\izlaz{1}#
          #\izlaz{1}#  #\izlaz{1}#
        #\izlaz{1}#  #\izlaz{2}#  #\izlaz{1}#
      #\izlaz{1}#  #\izlaz{3}#  #\izlaz{3}#  #\izlaz{1}#
    #\izlaz{1}#  #\izlaz{4}#  #\izlaz{6}#  #\izlaz{4}#  #\izlaz{1}#
  #\izlaz{1}#  #\izlaz{5}#  #\izlaz{10}# #\izlaz{10}# #\izlaz{5}# #\izlaz{1}#
#\izlaz{1}#  #\izlaz{6}#  #\izlaz{15}# #\izlaz{20}# #\izlaz{15}# #\izlaz{6}# #\izlaz{1}#
#\izlaz{}#
#\izlaz{32}#
\end{test}
\end{miditest}

\linkresenje{1_34}
\end{Exercise}
\begin{Answer}[ref=1_34]
\includecode{resenja/1_UvodniZadaci/1_34.c}
\end{Answer}

%%%%%%%%%%%%%%%%%%%%%%%%%
% Zadaci sa praktikuma - dodatni zadaci - oni nemaju rešenja
% možda \subsection{Dodatni zadaci} ili zadaci za vezbu
%%%%%%%%%%%%%%%%%%%%%%%%%


\begin{Exercise}[difficulty=1, label=1_35]%\marker+{2}
Napisati rekurzivnu funkciju koja prikazuje sve varijacije sa
   ponavljanjem dužine $n$ skupa $\{a, b\}$, i program koji je
   testira, za $n$ koje se unosi sa standardnog ulaza.

\begin{miditest}
\begin{test}{1}
#\naslovUlaz#
#\ulaz{2}#
#\naslovIzlaz#
#\izlaz{a a}#
#\izlaz{a b}#
#\izlaz{b a}#
#\izlaz{b b}#
\end{test}
\end{miditest}
\begin{miditest}
\begin{test}{2}
#\naslovUlaz#
#\ulaz{3}#
#\naslovIzlaz#
#\izlaz{a a a }#
#\izlaz{a a b}#
#\izlaz{a b a}#
#\izlaz{a b b}#
#\izlaz{b a a}#
#\izlaz{b a b}#
#\izlaz{b b a}#
#\izlaz{b b b}#
\end{test}
\end{miditest}
\end{Exercise}
%\begin{Answer}[ref=1_35]
%\includecode{resenja/1_UvodniZadaci/1_35.c}
%\end{Answer}

\begin{Exercise}[difficulty=1, label=1_36]%\marker+{2}
{\em Hanojske kule}: Data su tri
  vertikalna štapa. Na jednom od njih se nalazi $n$ diskova poluprečnika
  $1$, $2$, $3$,... do $n$, tako da se najveći nalazi na dnu, a
  najmanji na vrhu. Ostala dva štapa su prazna. Potrebno je
  premestiti diskove sa jednog na drugi štap tako da budu u istom redosledu, pri čemu se ni u jednom
  trenutku ne sme staviti veći disk preko manjeg. Preostali štap koristiti kao pomoćni štap prilikom
  premeštanja. \\
  Napisati program koji za proizvoljnu vrednost $n$, koja se unosi sa standardnog ulaza, prikazuje proces premeštanja diskova.

\end{Exercise}
%\begin{Answer}[ref=1_36]
%\includecode{resenja/1_UvodniZadaci/1_36.c}
%\end{Answer}

\begin{Exercise}[difficulty=1, label=1_37]%\marker+{2}
{\em Modifikacija Hanojskih kula}: Data su četiri
  vertikalna štapa. Na jednom se nalazi $n$ diskova poluprečnika
  $1$, $2$, $3$,... do $n$, tako da se najveći nalazi na dnu, a
  najmanji na vrhu. Ostala tri štapa su prazna. Potrebno je
  premestiti diskove na drugi štap tako da budu u istom redosledu,
  premestajući jedan po jedan disk, pri čemu se ni u jednom
  trenutku ne sme staviti veći disk preko manjeg. Preostala dva 
  štapa koristiti kao pomoćne štapove prilikom
  premeštanja.\\
  Napisati program koji za proizvoljnu vrednost $n$, koja se unosi sa standardnog ulaza, prikazuje proces premeštanja diskova.

\end{Exercise}
%\begin{Answer}[ref=1_37]
%\includecode{resenja/1_UvodniZadaci/1_37.c}
%\end{Answer}

\section{Rešenja}
\shipoutAnswer
