\chapter{Uvodni zadaci}

\section{Podela koda po datotekama}

\begin{Exercise}[label=1_01] % I sa vezbi
Napisati program za rad sa kompleksnim brojevima.
\begin{enumerate}
\item Definisati strukturu \kckod{KompleksanBroj} koja opisuje kompleksan broj njegovim realnim i imaginarnim delom.
\item Napisati funkciju \kckod{void ucitaj\_kompleksan\_broj(KompleksanBroj * z)} koja učitava kompleksan broj sa standardnog ulaza.
\item Napisati funkciju \kckod{void ispisi\_kompleksan\_broj(KompleksanBroj z)} koja ispisuje kompleksan broj na standardni izlaz u odgovarajućem formatu (npr. broj čiji je realan deo \argf{2}, a imaginarni \argf{-3} ispisati kao $(2 - 3 i)$ na standardni izlaz).
\item Napisati funkciju \kckod{float realan\_deo(KompleksanBroj z)} koja vraća vrednost realnog dela broja.
\item Napisati funkciju \kckod{float imaginaran\_deo(KompleksanBroj z)} koja vraća vrednost imaginarnog dela broja.
\item Napisati funkciju \kckod{float moduo(KompleksanBroj z)} koja računa moduo kompleksnog broja.
\item Napisati funkciju \kckod{KompleksanBroj konjugovan(KompleksanBroj z)} koja računa konjugovano-kompleksni broj svog argumenta \kckod{z}.
\item Napisati funkciju \kckod{KompleksanBroj saberi(KompleksanBroj z1, KompleksanBroj z2)} koja sabira dva kompleksna broja \kckod{z1} i \kckod{z2}.
\item Napisati funkciju \kckod{KompleksanBroj oduzmi(KompleksanBroj z1, KompleksanBroj z2)} koja oduzima dva kompleksna broja \kckod{z1} i \kckod{z2}.
\item Napisati funkciju \kckod{KompleksanBroj mnozi(KompleksanBroj z1, KompleksanBroj z2)} koja množi dva kompleksna broja \kckod{z1} i \kckod{z2}.
\item Napisati funkciju \kckod{float argument(KompleksanBroj z)} koja računa argument kompleksnog broja \kckod{z}.
\end{enumerate}

Napisati program koji testira prethodno napisane funkcije. Program najpre za kompleksan broj $z1$ koji se unosi sa standardnog ulaza ispisuje njegov realni deo, imaginarni deo i moduo. Zatim, za naredni kompleksan broj $z2$ koji se unosi sa standardnog ulaza ispisuje njegov konjugovano-kompleksan broj i argument. Na kraju program ispisuje zbir, razliku i proizvod brojeva $z1$ i $z2$.

\iffalse
Napisati program koji testira prethodno napisane funkcije tako što redom:
\begin{enumerate}
\item pozivanjem funkcije \kckod{ucitaj\_kompleksan\_broj} omogućava da se kompleksan broj $z1$  unese sa standardnog ulaza,
\item pozivanjem funkcije \kckod{ucitaj\_kompleksan\_broj} omogućava da se kompleksan broj $z2$  unosi sa standardnog ulaza,
\item računa i ispisuje kompleksan broj $z$ koji predstavlja zbir, razliku ili proizvod brojeva $z1$ i $z2$ u zavisnosti od znaka ('+', '-', '*') koji se unosi sa standardnog ulaza,
\item ispisuje realni deo, imaginarni deo i moduo kompleksnog broja $z$,
\item ispisuje konjugovano kompleksan broj i argument broja $z$,
\end{enumerate}
\fi


\begin{maxitest}
\begin{upotreba}{1}
#\naslovInt#
#\izlaz{Unesite realan i imaginaran deo kompleksnog broja:} \ulaz{1 -3}#
#\izlaz{(1.00 - 3.00 i)}#
#\izlaz{Unesite realan i imaginaran deo kompleksnog broja:} \ulaz{-1 4}#
#\izlaz{(-1.00 + 4.00 i)}#
#\izlaz{Unesite znak (+,-,*):} \ulaz{-}#
#\izlaz{(1.00 - 3.00 i) - (-1.00 + 4.00 i)  =  (2.00 - 7.00 i)}#
#\izlaz{realan\_deo: 2}#
#\izlaz{imaginaran\_deo: -7.000000}#
#\izlaz{moduo: 7.280110}#
#\izlaz{Njegov konjugovano kompleksan broj: (2.00 + 7.00 i)}#
#\izlaz{Argument kompleksnog broja: - 1.292497}#
\end{upotreba}
\end{maxitest}


\linkresenje{1_01}
\end{Exercise}
\begin{Answer}[ref=1_01]
\includecode{resenja/1_UvodniZadaci/1_01.c}
\end{Answer}

\begin{Exercise}[label=1_02] % III sa vezbi
Uraditi prethodni zadatak tako da su sve napisane funkcije za rad sa kompleksnim brojevima zajedno sa definicijom strukture \kckod{KompleksanBroj} izdvojene u posebnu biblioteku. Test program treba da koristi tu biblioteku da za kompleksan broj unet sa standardnog ulaza ispiše polarni oblik unetog broja.

\begin{maxitest}
\begin{upotreba}{1}
#\naslovInt#
#\izlaz{Unesite realan i imaginaran deo kompleksnog broja:} \ulaz{-5 2 }#
#\izlaz{Polarni oblik kompleksnog broja je 5.39 *  e\^{}i * 2.76}#
\end{upotreba}
\end{maxitest}

\linkresenje{1_02}

\end{Exercise}
\begin{Answer}[ref=1_02]
\includecodeLib{resenja/1_UvodniZadaci/1_02/complex.h}{complex.h}
\includecodeLib{resenja/1_UvodniZadaci/1_02/complex.c}{complex.c}
\includecodeLib{resenja/1_UvodniZadaci/1_02/main.c}{main.c}
\end{Answer}


\begin{Exercise}[label=1_03] % sa praktikuma
Napisati biblioteku za rad sa polinomima.
  \begin{enumerate}
  \item Definisati strukturu \kckod{Polinom} koja opisuje polinom stepena najviše 20. \uputstvo{Struktura sadrži stepen i niz
    koeficijenata. Redosled navođenja koeficijenata u nizu treba da bude takav da na nultoj poziciji u nizu bude koeficijent uz slobodan član, na prvoj koeficijent uz prvi stepen, itd.}
  \item Napisati funkciju \kckod{void ispisi(const Polinom * p)} koja ispisuje polinom \kckod{p} na standardni izlaz. Ispisivanje polinoma počinje od najvišeg stepena ka najnižem. Ipisisuju se samo oni koeficijenti koji su različiti od nule.
  \item Napisati funkciju \kckod{Polinom ucitaj()} koja učitava polinom sa standardnog
    ulaza. Za polinom se najpre unosi stepen pa njegovi koeficijenti.
  \item Napisati funkciju \kckod{double izracunaj(const Polinom * p, double x)} za izračunavanje vrednosti polinoma \kckod{p} u
    datoj tački \kckod{x} koristeći Hornerov algoritam.
  \item Napisati funkciju \kckod{Polinom saberi(const Polinom * p, const Polinom * q)} koja sabira dva polinoma \kckod{p} i \kckod{q}.
  \item Napisati funkciju \kckod{Polinom pomnozi(const Polinom * p, const Polinom * q)} koja množi dva polinoma \kckod{p} i \kckod{q}.
  \item Napisati funkciju \kckod{Polinom izvod(const Polinom * p)} koja računa izvod polinoma \kckod{p}.
  \item Napisati funkciju \kckod{Polinom n\_izvod(const Polinom * p, int n)} koja računa \kckod{n}-ti izvod polinoma \kckod{p}.
  \end{enumerate}

Napisati program koji testira prethodno napisane funkcije. Najpre se polinomi \argf{p} i \argf{q} unose sa standardnog ulaza i ispisuju na standardni izlaz u odgovarajućem obliku. Zatim se računa i ispisuje zbir i proizvod polinoma \argf{p} i \argf{q}. Označimo izračunati proizvod sa  \argf{r}. Nakon toga program računa i ispisuje vrednost polinoma \argf{r} (zaokruženu na dve decimale) u tački koju unosi korisnik. Na kraju se sa standardnog ulaza unosi broj \argf{n}, i ispisuje \argf{n}-ti izvod polinoma \argf{r}.

\begin{maxitest}
\begin{upotreba}{1}
#\naslovInt#
#\izlaz{Unesite polinom p (prvo stepen, pa zatim koeficijente od najveceg stepena do nultog):}#
#\ulaz{3 1.2 3.5 2.1 4.2}#
#\izlaz{Unesite polinom q (prvo stepen, pa zatim koeficijente od najveceg stepena do nultog):}#
#\ulaz{2 2.1 0 -3.9}#
#\izlaz{Zbir polinoma je: 1.20x\^{}3+5.60x\^{}2+2.10x+0.30}#
#\izlaz{Prozvod polinoma je polinom r:}#
#\izlaz{2.52x\^{}5+7.35x\^{}4-0.27x\^{}3-4.83x\^{}2-8.19x-16.38}#
#\izlaz{Unesite tacku u kojoj racunate vrednost polinoma r}#
#\ulaz{0}#
#\izlaz{Vrednost polinoma u tacki je -16.38}#
#\izlaz{Unesite izvod polinoma koji zelite:}#
#\ulaz{3}#
#\izlaz{3. izvod polinoma r je: 151.20x\^{}2+176.40x-1.62}#
\end{upotreba}
\end{maxitest}

\linkresenje{1_03}

\end{Exercise}
\begin{Answer}[ref=1_03]
\includecodeLib{resenja/1_UvodniZadaci/1_03/polinom.h}{polinom.h}
\includecodeLib{resenja/1_UvodniZadaci/1_03/polinom.c}{polinom.c}
\includecodeLib{resenja/1_UvodniZadaci/1_03/main.c}{main.c}
%\includecode{resenja/1_UvodniZadaci/003/Makefile}
\end{Answer}

\begin{Exercise}[label=1_04] % sa praktikuma
Napisati biblioteku za rad sa razlomcima.

  \begin{enumerate}

  \item Definisati strukturu \kckod{Razlomak} za reprezentovanje razlomaka.
  \item Napisati funkciju \kckod{Razlomak ucitaj()} za učitavanje razlomaka.
  \item Napisati funkciju \kckod{void ispisi(const Razlomak * r)} koja ispisuje razlomak.
  \item Napisati funkciju \kckod{int brojilac(const Razlomak * r)} koje vraćaju brojilac razlomka \kckod{r}.
   \item Napisati funkciju \kckod{int imenilac(const Razlomak * r)} koje vraćaju imenilac razlomka \kckod{r}.
  \item Napisati funkciju \kckod{double realna\_vrednost(const Razlomak * r)} koja vraća odgovarajuću realnu vrednost razlomka \kckod{r}.
  \item Napisati funkciju  \kckod{double reciprocna\_vrednost(const Razlomak * r)} koja izračunava recipročnu vrednost
    razlomka \kckod{r}.
  \item Napisati funkciju \kckod{Razlomak skrati(const Razlomak * r)} koja skraćuje dati razlomak \kckod{r}.
  \item Napisati funkciju \kckod{Razlomak saberi(const Razlomak * r1, const Razlomak * r2)} koja sabira dva razlomka \kckod{r1} i \kckod{r2}.
 \item Napisati funkciju \kckod{Razlomak oduzmi(const Razlomak * r1, const Razlomak * r2)} koja oduzima dva razlomka \kckod{r1} i \kckod{r2}.
 \item Napisati funkciju \kckod{Razlomak pomnozi(const Razlomak * r1, const Razlomak * r2)} koja množi
    dva razlomka \kckod{r1} i \kckod{r2}.
 \item Napisati funkciju \kckod{Razlomak podeli(const Razlomak * r1, const Razlomak * r2)} koja deli
    dva razlomka \kckod{r1} i \kckod{r2}.
\end{enumerate}

Napisati program koji testira prethodne funkcije tako što se sa standardnog ulaza unose dva razlomka \argf{r1} i \argf{r2} i na standardni izlaz se ispisuju skraćene vrednosti razlomaka koji su dobijeni kao zbir, razlika, proizvod i količnik razlomka \argf{r1} i recipročne vrednosti razlomka \argf{r2}.

\begin{maxitest}
\begin{upotreba}{1}
#\naslovInt#
#\izlaz{Unesite imenilac i brojilac prvog razlomka:}\ulaz{1 2}#
#\izlaz{Unesite imenilac i brojilac drugog razlomka:}\ulaz{3 2}#
#\izlaz{1/2 + 3/2 = 2}#
#\izlaz{1/2 - 3/2 =  -1}#
#\izlaz{1/2 * 3/2 =  3/4}#
#\izlaz{1/2 / 3/2 =  1/3}#
\end{upotreba}
\end{maxitest}


%\linkresenje{1_04}

\end{Exercise}
%\begin{Answer}[ref=1_04]
%%\includecode{resenja/...}
%\end{Answer}

\section{Algoritmi za rad sa bitovima}

\begin{Exercise}[label=1_05] %ex 201 dvestajedan
Napisati biblioteku \kckod{stampanje\_bitova}  za rad sa bitovima koja sadrži funkciju \kckod{stampaj\_bitove} koja štampa bitove u binarnom zapisu neoznačenog celog broja $x$. Napisati program koja testira funkciju \kckod{stampaj\_bitove} za brojeve koji se sa standardnog ulaza zadaju u heksadekasnom formatu.

\begin{miditest}
\begin{test}{1}
#\naslovUlaz#
#\ulaz{0x7F}#
#\naslovIzlaz#
#\izlaz{00000000000000000000000001111111}#
\end{test}
\end{miditest}
\begin{miditest}
\begin{test}{2}
#\naslovUlaz#
#\ulaz{0x80}#
#\naslovIzlaz#
#\izlaz{00000000000000000000000010000000}#
\end{test}
\end{miditest}

\begin{miditest}
\begin{test}{3}
#\naslovUlaz#
#\ulaz{0x00FF00FF}#
#\naslovIzlaz#
#\izlaz{00000000111111110000000011111111}#
\end{test}
\end{miditest}
% \begin{miditest}
% \begin{test}{4}
% #\naslovUlaz#
% #\ulaz{0xFFFFFFFF}#
% #\naslovIzlaz#
% #\izlaz{11111111111111111111111111111111}#
% \end{test}
% \end{miditest}
\begin{miditest}
\begin{test}{4}
#\naslovUlaz#
#\ulaz{0xABCDE123}#
#\naslovIzlaz#
#\izlaz{10101011110011011110000100100011}#
\end{test}
\end{miditest}

\linkresenje{1_05}
\end{Exercise}
\begin{Answer}[ref=1_05]
\includecodeLib{resenja/biblioteke/stampanje_bitova/stampanje_bitova.h}{stampanje\_bitova.h}
\includecodeLib{resenja/biblioteke/stampanje_bitova/stampanje_bitova.c}{stampanje\_bitova.c}
\includecodeLib{resenja/1_UvodniZadaci/1_05.c}{main.c}
\end{Answer}

\begin{Exercise}[label=1_06]
Napisati funkcije \kckod{prebroj\_bitove\_1} i \kckod{prebroj\_bitove\_2} koje broje bitove sa vrednošću \argf{1} u binarnom zapisu celog broja $x$. Prebrojavanje bitova ostvariti na dva načina:
\begin{enumerate}
\item formiranjem odgovarajuće maske i njenim pomeranjem
\item formiranjem odgovarajuće maske i pomeranjem promenljive $x$.
\end{enumerate} 
 
Napisati program koji za broj koji unosi u heksadekasnom formatu sa standardnog ulaza računa broj bitova sa vrednošću \argf{1} korišćenjem funkcije \kckod{prebroj\_bitove\_1} ili funkcije \kckod{prebroj\_bitove\_2}. Od korisnika sa standardnog ulaza tražiti da izabere koju od ove funkcije treba koristiit u zavisnosti da li unese \argf{1} ili \argf{2}. Ukoliko korisnik ne unese odgovarajuće vrednosti za redni broj funkcije prekinuti izvršavanje programa i ispisati odgovarajuću poruku na standardni izlaz za greške.

\begin{miditest}
\begin{upotreba}{1}
#\naslovInt#
#\izlaz{Unesite broj:}\ulaz{0x7F}#
#\izlaz{Unesite redni broj funkcije:}\ulaz{1}#
#\izlaz{Broj jedinica u zapisu je}#
#\izlaz{funkcija prebroj\_bitove\_1: 7}#
\end{upotreba}
\end{miditest}
\begin{miditest}
\begin{upotreba}{2}
#\naslovInt#
#\izlaz{Unesite broj:}\ulaz{0xABCDE123}#
#\izlaz{Unesite redni broj funkcije:}\ulaz{2}#
#\izlaz{Broj jedinica u zapisu je}#
#\izlaz{funkcija prebroj\_bitove\_2: 17}#
\end{upotreba}
\end{miditest}
%\begin{miditest}
%\begin{upotreba}{2}
%#\naslovInt#
%#\izlaz{Unesite broj:}\ulaz{0x80}#
%#\izlaz{Unesite redni broj funkcije:}\ulaz{2}#
%#\izlaz{Broj jedinica u zapisu je}#
%#\izlaz{funkcija prebroj\_bitove\_2: 1}#
%\end{upotreba}
%\end{miditest}

\begin{miditest}
\begin{upotreba}{3}
#\naslovInt#
#\izlaz{Unesite broj:}\ulaz{0x00FF00FF}#
#\izlaz{Unesite redni broj funkcije:}\ulaz{2}#
#\izlaz{Broj jedinica u zapisu je}#
#\izlaz{funkcija prebroj\_bitove\_2: 16}#
\end{upotreba}
\end{miditest}
\begin{miditest}
\begin{upotreba}{4}
#\naslovInt#
#\izlaz{Unesite broj:}\ulaz{0x00FF00FF}#
#\izlaz{Unesite redni broj funkcije:}\ulaz{3}#
#\naslovIzlazZaGresku#
#\izlaz{Neodgovarajuci redni broj funkcije!}#
\end{upotreba}
\end{miditest}

\linkresenje{1_06}
\end{Exercise}
\begin{Answer}[ref=1_06]
\includecode{resenja/1_UvodniZadaci/1_06.c}
\end{Answer}


\begin{Exercise}[label=1_07]
Napisati funkciju \kckod{najveci} koja određuje najveći broj koji se može zapisati istim binarnim ciframa kao dati broj i funkciju \kckod{najmanji} koja određuje najmanji broj koji se može zapisati istim binarnim ciframa kao dati broj.

Napisati program koji testira prethodno napisane funkcije tako što prikazuje binarnu reprezentaciju brojeva koji se dobijaju nakon poziva funkcije \kckod{najveci}, odnosno \kckod{najmanji} za brojeve koji se zadaju u heksadekasnom formatu sa standardnog ulaza. 

\begin{miditest}
\begin{test}{1}
#\naslovUlaz#
#\ulaz{0x7F}#
#\naslovIzlaz#
#\izlaz{Najveci:}#
#\izlaz{11111110000000000000000000000000}#
#\izlaz{Najmanji:}#
#\izlaz{00000000000000000000000001111111}#
\end{test}
\end{miditest}
\begin{miditest}
\begin{test}{2}
#\naslovUlaz#
#\ulaz{0x80}#
#\naslovIzlaz#
#\izlaz{Najveci:}#
#\izlaz{10000000000000000000000000000000}#
#\izlaz{Najmanji:}#
#\izlaz{00000000000000000000000000000001}#
\end{test}
\end{miditest}

\begin{miditest}
\begin{test}{3}
#\naslovUlaz#
#\ulaz{0x00FF00FF}#
#\naslovIzlaz#
#\izlaz{Najveci:}#
#\izlaz{11111111111111110000000000000000}#
#\izlaz{Najmanji:}#
#\izlaz{00000000000000001111111111111111}#
\end{test}
\end{miditest}
\begin{miditest}
\begin{test}{4}
#\naslovUlaz#
#\ulaz{0xFFFFFFFF}#
#\naslovIzlaz#
#\izlaz{Najveci:}#
#\izlaz{11111111111111111111111111111111}#
#\izlaz{Najmanji:}#
#\izlaz{11111111111111111111111111111111}#
\end{test}
\end{miditest}

% \begin{miditest}
% \begin{test}{5}
% #\naslovUlaz#
% #\ulaz{0xABCDE123}#
% #\naslovIzlaz#
% #\izlaz{Najveci:}#
% #\izlaz{11111111111111111000000000000000}#
% #\izlaz{Najmanji:}#
% #\izlaz{00000000000000011111111111111111}#
% \end{test}
% \end{miditest}

\linkresenje{1_07}
\end{Exercise}
\begin{Answer}[ref=1_07]
\napomena{Rešenje koristi biblioteku za štampanje bitova iz zadatka \ref{1_05}.}
\includecode{resenja/1_UvodniZadaci/1_07.c}
\end{Answer}


\begin{Exercise}[label=1_08]
Napisati funkcije za rad sa bitovima. \napomena{Pozicije se broje počev od pozicije bita najmanje težine, pri čemu je bit najmanje težine na poziciji nula.}
\begin{enumerate}
\item Napisati funkciju \kckod{postavi\_0} koja određuje broj koji se dobija kada se $n$ bitova datog broja $x$, počevši od pozicije $p$, postave na \argf{0}.
\item Napisati funkciju \kckod{postavi\_1} koja određuje broj koji se dobija kada se $n$ bitova datog broja $x$, počevši od pozicije $p$, postave na \argf{1}.
\item %Napisati funkciju \kckod{vrati\_bitove} koja određuje broj koji se dobija od $n$ bitova datog broja $x$, počevši od pozicije $p$, i vraća ih kao bitove najmanje težine rezultata.
Napisati funkciju \kckod{vrati\_bitove} koja određuje broj u kome se $n$ bitova najmanje težine poklapa sa $n$ bitova broja $x$ počevši od pozicije $p$.
\item %Napisati funkciju \kckod{postavi\_1\_n\_bits} koja vraća broj koji se dobija upisivanjem poslednjih $n$ bitova broja $y$ u broj $x$, počevši od pozicije $p$.
Napisati funkciju \kckod{postavi\_1\_n\_bitova} koja vraća broj koji se dobija upisivanjem poslednjih $n$ bitova najmanje težine broja $y$ u broj $x$, počevši od pozicije $p$.
\item Napisati funkciju \kckod{invertuj} koja vraća broj koji se dobija invertovanjem $n$ bitova broja $x$ počevši od pozicije $p$. 
\end{enumerate}
Napisati program koji testira prethodno napisane funkcije za neoznačene cele brojeve $x$, $n$, $p$, $y$ koji se unose sa standardnog ulaza. Program treba nakon učitavanja odgovarajućih vrednosti ispiše najpre binarne reprezenatacije brojeva $x$ i $y$, a potom i binarne reprezentacije brojeva koji se dobijaju pozivanjem prethodno napisanih funkcija.

\begin{maxitest}
\begin{upotreba}{1}
#\naslovInt#
#\izlaz{Unesite neoznacen ceo broj x:}\ulaz{235}#
#\izlaz{Unesite neoznacen ceo broj n:}\ulaz{9}#
#\izlaz{Unesite neoznacen ceo broj p:}\ulaz{24}#
#\izlaz{Unesite neoznacen ceo broj y:}\ulaz{127}#
#\izlaz{x =\ \ \ \ 235\ \ \ \ \ \ \ \ \ \ \ \ \ \ \ \ \ \ \ \ \ \ \ \ \ \ \ \ \ \ \ \ \ \ \ \ \ \ \ = 00000000000000000000000011101011}#
#\izlaz{postavi\_0(\ \ \ 235,\ \ \ \ \ 9,\ \ \ \ 24)\ \ \ \ \ \ \ \ \ \ \ \ \ \ \ \ \ \ = 00000000000000000000000011101011}#
#\izlaz{}#
#\izlaz{x =\ \ \ \ 235\ \ \ \ \ \ \ \ \ \ \ \ \ \ \ \ \ \ \ \ \ \ \ \ \ \ \ \ \ \ \ \ \ \ \ \ \ \ \ = 00000000000000000000000011101011}#
#\izlaz{postavi\_1(\ \ \ 235,\ \ \ \ \ 9,\ \ \ \ 24)\ \ \ \ \ \ \ \ \ \ \ \ \ \ \ \ \ \ = 00000001111111110000000011101011}#
#\izlaz{}#
#\izlaz{x =\ \ \ \ 235\ \ \ \ \ \ \ \ \ \ \ \ \ \ \ \ \ \ \ \ \ \ \ \ \ \ \ \ \ \ \ \ \ \ \ \ \ \ \ = 00000000000000000000000011101011}#
#\izlaz{vrati\_bitove(\ \ \ 235,\ \ \ \ \ 9,\ \ \ \ 24)\ \ \ \ \ \ \ \ \ \ \ \ \ \ \ = 00000000000000000000000000000000}#
#\izlaz{}#
#\izlaz{x =\ \ \ \ 235\ \ \ \ \ \ \ \ \ \ \ \ \ \ \ \ \ \ \ \ \ \ \ \ \ \ \ \ \ \ \ \ \ \ \ \ \ \ \ = 00000000000000000000000011101011}#
#\izlaz{y =\ \ \ \ 127\ \ \ \ \ \ \ \ \ \ \ \ \ \ \ \ \ \ \ \ \ \ \ \ \ \ \ \ \ \ \ \ \ \ \ \ \ \ \ = 00000000000000000000000001111111}#
#\izlaz{postavi\_1\_n\_bitove(\ \ 235,\ \ 9,\ \ 24,\ \ 127) \ \ \ \ \ \ \ \ = 00000000011111110000000011101011}#
#\izlaz{}#
#\izlaz{x =\ \ \ \ 235\ \ \ \ \ \ \ \ \ \ \ \ \ \ \ \ \ \ \ \ \ \ \ \ \ \ \ \ \ \ \ \ \ \ \ \ \ \ \ = 00000000000000000000000011101011}#
#\izlaz{invertuj(\ \ \ 235,\ \ \ \ \ 9,\ \ \ \ 24)\ \ \ \ \ \ \ \ \ \ \ \ \ \ \ \ \ \ \ = 00000001111111110000000011101011}#
\end{upotreba}
\end{maxitest}

\begin{maxitest}
\begin{upotreba}{2}
#\naslovInt#
#\izlaz{Unesite neoznacen ceo broj x:}\ulaz{2882398951}#
#\izlaz{Unesite neoznacen ceo broj n:}\ulaz{5}#
#\izlaz{Unesite neoznacen ceo broj p:}\ulaz{10}#
#\izlaz{Unesite neoznacen ceo broj y:}\ulaz{35156526}#
#\izlaz{x = 2882398951\ \ \ \ \ \ \ \ \ \ \ \ \ \ \ \ \ \ \ \ \ \ \ \ \ \ \ \ \ \ \ \ \ \ = 10101011110011011110101011100111}#
#\izlaz{postavi\_0(2882398951,\ \ \ \ \ 5,\ \ \ \ 10)\ \ \ \ \ \ \ \ \ \ \ \ \ = 10101011110011011110100000100111}#
#\izlaz{}#
#\izlaz{x = 2882398951\ \ \ \ \ \ \ \ \ \ \ \ \ \ \ \ \ \ \ \ \ \ \ \ \ \ \ \ \ \ \ \ \ \ = 10101011110011011110101011100111}#
#\izlaz{postavi\_1(2882398951,\ \ \ \ \ 5,\ \ \ \ 10)\ \ \ \ \ \ \ \ \ \ \ \ \ = 10101011110011011110111111100111}#
#\izlaz{}#
#\izlaz{x = 2882398951\ \ \ \ \ \ \ \ \ \ \ \ \ \ \ \ \ \ \ \ \ \ \ \ \ \ \ \ \ \ \ \ \ \ = 10101011110011011110101011100111}#
#\izlaz{vrati\_bitove(2882398951,\ \ \ \ \ 5,\ \ \ \ 10)\ \ \ \ \ \ \ \ \ \ = 00000000000000000000000000001011}#
#\izlaz{}#
#\izlaz{x = 2882398951\ \ \ \ \ \ \ \ \ \ \ \ \ \ \ \ \ \ \ \ \ \ \ \ \ \ \ \ \ \ \ \ \ \ = 10101011110011011110101011100111}#
#\izlaz{y =\ \ 35156526\ \ \ \ \ \ \ \ \ \ \ \ \ \ \ \ \ \ \ \ \ \ \ \ \ \ \ \ \ \ \ \ \ \ \ = 00000010000110000111001000101110}#
#\izlaz{postavi\_1\_n\_bitove(2882398951, 5, 10, 35156526) = 10101011110011011110101110100111}#
#\izlaz{}#
#\izlaz{x = 2882398951\ \ \ \ \ \ \ \ \ \ \ \ \ \ \ \ \ \ \ \ \ \ \ \ \ \ \ \ \ \ \ \ \ \ = 10101011110011011110101011100111}#
#\izlaz{invertuj(2882398951,\ \ \ \ \ 5,\ \ \ \ 10)\ \ \ \ \ \ \ \ \ \ \ \ \ \ = 10101011110011011110110100100111}#
\end{upotreba}
\end{maxitest}

\linkresenje{1_08}
\end{Exercise}
\begin{Answer}[ref=1_08]
\napomena{Rešenje koristi biblioteku za štampanje bitova iz zadatka \ref{1_05}.}
\includecode{resenja/1_UvodniZadaci/1_08.c}
\end{Answer}

\begin{Exercise}[label=1_09] %ex 205
Pod rotiranjem ulevo podrazumeva se pomeranje svih bitova za jednu poziciju ulevo, s tim što se bit sa pozicije najveće težine pomera na poziciju najmanje težine. Analogno, rotiranje udesno podrazumeva pomeranje svih bitova za jednu poziciju udesno, s tim što se bit sa pozicije najmanje težine pomera na poziciju najviše težine.
\begin{enumerate}
\item Napisati funkciju \kckod{rotiraj\_ulevo} koja određuje broj koji se dobija rotiranjem \argf{k} puta ulevo datog celog broja \argf{x}. 
\item Napisati funkciju \kckod{rotiraj\_udesno} koja određuje broj koji se dobija rotiranjem \argf{k} puta udesno datog celog neoznačenog broja \argf{x}. 
\item Napisati funkciju \kckod{rotiraj\_udesno\_oznaceni} koja određuje broj koji se dobija rotiranjem \argf{k} puta udesno datog celog broja \argf{x}. 
\end{enumerate}
Napisati program koji testira prethodno napisane funkcije za broj \argf{x} i broj \argf{k} koji se unose u heksadekasnom formatu sa standardnog ulaza.


\begin{maxitest}
\begin{upotreba}{1}
#\naslovInt#
#\izlaz{Unesite neoznacen ceo broj x:}\ulaz{B10011A7}#
#\izlaz{Unesite neoznacen ceo broj k:}\ulaz{5}#
#\izlaz{x  = 10110001000000000001000110100111 }#
#\izlaz{rotiraj\_ulevo(2969571751, 5) \ \ \ \ \ \ \ \ \ \ = 00100000000000100011010011110110}#
#\izlaz{rotiraj\_udesno(2969571751, 5) \ \ \ \ \ \ \ \ \ = 00111101100010000000000010001101}#
#\izlaz{rotiraj\_udesno\_oznaceni(2969571751, 5) = 00111101100010000000000010001101 }#
\end{upotreba}
\end{maxitest}

\linkresenje{1_09}
\end{Exercise}
\begin{Answer}[ref=1_09]
\napomena{Rešenje koristi biblioteku za štampanje bitova iz zadatka \ref{1_05}.}
\includecode{resenja/1_UvodniZadaci/1_09.c}
\end{Answer}

\begin{Exercise}[label=1_10]
Napisati funkciju \kckod{ogledalo} koja određuje ceo broj čiji je binarni zapis slika u ogledalu binarnog zapisa argumenta funkcije. Napisati i program koji testira datu funkciju za brojeve koji se sa standardnog ulaza zadaju u heksadekasnom formatu, tako što najpre ispisuje binarnu reprezentaciju unetog broja, a potom i binarnu reprezentaciju broja dobijenog nakon poziva funkcije \kckod{ogledalo} za uneti broj.

\begin{miditest}
\begin{test}{1}
#\naslovUlaz#
#\ulaz{255 }#
#\naslovIzlaz#
#\izlaz{00000000000000000000001001010101}#
#\izlaz{10101010010000000000000000000000}#
\end{test}
\end{miditest}
\begin{miditest}
\begin{test}{2}
#\naslovUlaz#
#\ulaz{-15}#
#\naslovIzlaz#
#\izlaz{11111111111111111111111111101011}#
#\izlaz{11010111111111111111111111111111}#
\end{test}
\end{miditest}

\linkresenje{1_10}
\end{Exercise}
\begin{Answer}[ref=1_10]
\napomena{Rešenje koristi biblioteku za štampanje bitova iz zadatka \ref{1_05}.}
\includecode{resenja/1_UvodniZadaci/1_10.c}
\end{Answer}

%%%%%%%%%%%%%%%%%%%%%%%%%
% Zadaci sa praktikuma - obavezni zadaci 
%%%%%%%%%%%%%%%%%%%%%%%%%

\begin{Exercise}[label=1_11] %ex 207
Napisati funkciju \kckod{int broj\_01(unsigned int n)} koja za dati broj \argf{n} vraća \argf{1} ako u njegovom binarnom zapisu ima više jednica nego nula, a inače vraća \argf{0}.  Napisati program koji tu funkciju testira za broj koji se zadaje sa standardnog ulaza.

\begin{minitest}
\begin{test}{1}
#\naslovUlaz#
#\ulaz{10}#
#\naslovIzlaz#
#\izlaz{0}#
\end{test}
\end{minitest}
%\begin{minitest}
%\begin{test}{2}
%#\naslovUlaz#
%#\ulaz{1024}#
%#\naslovIzlaz#
%#\izlaz{0}#
%\end{test}
%\end{minitest}
\begin{minitest}
\begin{test}{2}
#\naslovUlaz#
#\ulaz{2147377146}#
#\naslovIzlaz#
#\izlaz{1}#
\end{test}
\end{minitest}
\begin{minitest}
\begin{test}{3}
#\naslovUlaz#
#\ulaz{1111111115}#
#\naslovIzlaz#
#\izlaz{0}#
\end{test}
\end{minitest}

\linkresenje{1_11}
\end{Exercise}
\begin{Answer}[ref=1_11]
\includecode{resenja/1_UvodniZadaci/1_11.c}
\end{Answer}

\begin{Exercise}[label=1_12]
Napisati funkciju koja broji koliko se puta dve uzastopne jedinice pojavljuju u binarnom zapisu
  celog neoznačenog broja $x$. Napisati program koji tu funkciju testira za broj koji se
  zadaje sa standardnog ulaza. \napomena{Tri uzastopne jedinice sadrže dve uzastopne jedinice dva puta.}
  
\begin{minitest}
\begin{test}{1}
#\naslovUlaz#
#\ulaz{11}#
#\naslovIzlaz#
#\izlaz{1}#
\end{test}
\end{minitest}
\begin{minitest}
\begin{test}{2}
#\naslovUlaz#
#\ulaz{1024}#
#\naslovIzlaz#
#\izlaz{0}#
\end{test}
\end{minitest}
\begin{minitest}
\begin{test}{3}
#\naslovUlaz#
#\ulaz{2147377146}#
#\naslovIzlaz#
#\izlaz{22}#
\end{test}
\end{minitest}  

\linkresenje{1_12}
\end{Exercise}
\begin{Answer}[ref=1_12]
\includecode{resenja/1_UvodniZadaci/1_12.c}
\end{Answer}


%%%
%Ovaj 209.c nema resenje za sad...
%%%
\begin{Exercise}[label=1_13]%\marker+{2}
Napisati program koji sa standardnog ulaza učitava pozitivan ceo broj, a na standardni izlaz ispisuje vrednost tog broja sa razmenjenim vrednostima bitova na pozicijama $i$ i $j$. Pozicije $i$ i $j$ se učitavaju kao parametri
  komandne linije. Smatrati da je krajnji desni bit binarne
  reprezentacije \argf{0}-ti bit. Pri rešavanju nije dozvoljeno koristiti
  ni pomoćni niz ni aritmetičke operatore +, -, /, *, \%.

\begin{minitest}
\begin{upotreba}{1}
#\poziv{./a.out 1 2 }#

#\naslovInt#
#\naslovUlaz#
#\ulaz{11}#
#\naslovIzlaz#
#\izlaz{13}#
\end{upotreba}
\end{minitest}
\begin{minitest}
\begin{upotreba}{2}
#\poziv{./a.out 1 2 }#

#\naslovInt#
#\naslovUlaz#
#\ulaz{1024}#
#\naslovIzlaz#
#\izlaz{1024}#
\end{upotreba}
\end{minitest}
\begin{minitest}
\begin{upotreba}{2}
#\poziv{./a.out 12 12 }#

#\naslovInt#
#\naslovUlaz#
#\ulaz{12345}#
#\naslovIzlaz#
#\izlaz{12345}#
\end{upotreba}
\end{minitest}

\end{Exercise}
%\begin{Answer}[ref=1_13]
%\includecode{resenja/1_UvodniZadaci/1_13.c}
%\end{Answer}

\begin{Exercise}[label=1_14]
  Napisati funkciju koja na osnovu neoznačenog broja $x$
  formira nisku $s$ koja sadrži heksadekadni zapis broja
  $x$ koristeći algoritam za brzo prevođenje binarnog u
  heksadekadni zapis (svake $4$ binarne cifre se zamenjuju jednom
  odgovarajućom heksadekadnom cifrom).  Napisati program koji tu
  funkciju testira za broj koji se zadaje sa standardnog ulaza.

\begin{minitest}
\begin{test}{1}
#\naslovUlaz#
#\ulaz{11}#
#\naslovIzlaz#
#\izlaz{0000000B}#
\end{test}
\end{minitest}
\begin{minitest}
\begin{test}{2}
#\naslovUlaz#
#\ulaz{1024}#
#\naslovIzlaz#
#\izlaz{00000400}#
\end{test}
\end{minitest}
\begin{minitest}
\begin{test}{3}
#\naslovUlaz#
#\ulaz{12345}#
#\naslovIzlaz#
#\izlaz{00003039}#
\end{test}
\end{minitest}  

\linkresenje{1_14}
\end{Exercise}
\begin{Answer}[ref=1_14]
\includecode{resenja/1_UvodniZadaci/1_14.c}
\end{Answer}

%%%%%%%%%%%%%%%%%%%%%%%%%
% Zadaci sa praktikuma - dodatni zadaci - oni nemaju rešenja
% možda \subsection{Dodatni zadaci} ili zadaci za vežbu
%%%%%%%%%%%%%%%%%%%%%%%%%

\begin{Exercise}[label=1_15]%\marker+{2}
  Napisati funkciju koja za data dva neoznačena broja $x$
  i $y$ invertuje u podatku $x$ one bitove koji se poklapaju
  sa odgovarajućim bitovima u broju $y$. Ostali bitovi ostaju
  nepromenjeni.  Napisati program koji tu funkciju testira za brojeve
  koji se zadaju sa standardnog ulaza.
  
\begin{minitest}
\begin{test}{1}
#\naslovUlaz#
#\ulaz{123 10}#
#\naslovIzlaz#
#\izlaz{4294967285}#
\end{test}
\end{minitest}
\begin{minitest}
\begin{test}{2}
#\naslovUlaz#
#\ulaz{3251 0}#
#\naslovIzlaz#
#\izlaz{4294967295}#
\end{test}
\end{minitest}
\begin{minitest}
\begin{test}{3}
#\naslovUlaz#
#\ulaz{12541 1024}#
#\naslovIzlaz#
#\izlaz{4294966271}#
\end{test}
\end{minitest}    
  
\end{Exercise}
%\begin{Answer}[ref=1_15]
%\includecode{resenja/1_UvodniZadaci/1_15.c}
%\end{Answer}

\begin{Exercise}[label=1_16]%\marker+{2}
Napisati funkciju koja računa koliko petica bi imao ceo
  neoznačen broj $x$ u oktalnom zapisu. Napisati program koji
  tu funkciju testira za broj koji se zadaje sa standardnog ulaza. \napomena{Zadatak rešiti isključivo korišćenjem bitskih operatora.}
  
\begin{minitest}
\begin{test}{1}
#\naslovUlaz#
#\ulaz{123}#
#\naslovIzlaz#
#\izlaz{0}#
\end{test}
\end{minitest}
\begin{minitest}
\begin{test}{2}
#\naslovUlaz#
#\ulaz{3245}#
#\naslovIzlaz#
#\izlaz{2}#
\end{test}
\end{minitest}
\begin{minitest}
\begin{test}{3}
#\naslovUlaz#
#\ulaz{100328}#
#\naslovIzlaz#
#\izlaz{1}#
\end{test}
\end{minitest}   
 
\end{Exercise}
%\begin{Answer}[ref=1_16]
%\includecode{resenja/1_UvodniZadaci/1_16.c}
%\end{Answer}


\section{Rekurzija}

%%%\subsection{Rekurzivne funkcije nad brojevima}

\begin{Exercise}[label=1_17]
Napisati rekurzivnu funkciju koja izračunava  $x^k$, za dati ceo broj $x$ i prirodan broj $k$
\begin{enumerate}
\item tako da rešenje bude linearne složenosti,
\item tako da rešenje bude logaritamske složenosti.
\end{enumerate}
Napisati program koji testira napisane funkciju tako što se sa standardnog ulaza najpre unosi redni broj funckije koja se primenjuje, a zatim i ceo broj $x$ i prirodan broj $k$, a potom se na standarni izlaz ispisuje rezultat primene odgovarajuće funkcije na unete brojeve. Ukoliko se za redni broj funkcije unese broj različit od \argf{1} i \argf{2} ispisati odgovarajuću poruku i prekinuti izvršavanje programa. 
 
\begin{miditest}
\begin{upotreba}{1}
#\naslovInt#
#\izlaz{Unesite redni broj funkcije (1/2):}# 
#\ulaz{1}#
#\izlaz{Unesite broj x:} \ulaz{ 2}#
#\izlaz{Unesite broj k:} \ulaz{ 10}#
#\izlaz{1024}#
\end{upotreba}
\end{miditest}
\begin{miditest}
\begin{upotreba}{2}
#\naslovInt#
#\izlaz{Unesite redni broj funkcije (1/2):}# 
#\ulaz{2}#
#\izlaz{Unesite broj x:} \ulaz{ 9}#
#\izlaz{Unesite broj k:} \ulaz{ 4}#
#\izlaz{6561}#
\end{upotreba}
\end{miditest}    
 
\linkresenje{1_17}
\end{Exercise}
\begin{Answer}[ref=1_17]
\includecode{resenja/1_UvodniZadaci/1_17.c}
\end{Answer}

\begin{Exercise}[label=1_18]
Koristeći uzajamnu (posrednu) rekurziju napisati:
 \begin{enumerate}
\item funkciju \kckod{paran} koja proverava da li je broj cifara nekog broja paran i vraća 1 ako jeste, a 0 inače;
\item i funkciju \kckod{neparan} koja vraća 1, ukoliko je broj cifara nekog broja neparan, a 0 inače.
 \end{enumerate}
Napisati program koji testira napisane funkcije tako što za heksadekadni broj koji se unosi sa standardnog ulaza ispisuje da li je broj njegovih cifara paran ili neparan.
 
 \begin{miditest}
\begin{test}{1}
#\naslovUlaz#
#\ulaz{11}#
#\naslovIzlaz#
#\izlaz{Uneti broj ima paran broj cifara.}#
\end{test}
\end{miditest}
\begin{miditest}
\begin{test}{2}
#\naslovUlaz#
#\ulaz{123}#
#\naslovIzlaz#
#\izlaz{Uneti broj ima neparan broj cifara.}#
\end{test}
\end{miditest}
 
\linkresenje{1_18}
\end{Exercise}
\begin{Answer}[ref=1_18]
\includecode{resenja/1_UvodniZadaci/1_18.c}
\end{Answer}

\begin{Exercise}[label=1_19]
Napisati repno-rekurzivnu funkciju koja izračunava faktorijel broja $n$. Napisati program koji testira napisanu funkciju za proizvoljan broj $n$ ($n \le 12$) unet sa standardnog ulaza. \napomena{Gornje ograničenje $12$ je postavljeno zbog ograničenja za tip podataka \kckod{int} i činjenice da niz faktorijela brzo raste.}

\begin{miditest}
\begin{upotreba}{1}
#\naslovInt#
#\izlaz{Unesite n (<= 12):} \ulaz{5}#
#\izlaz{5! = 120}#
\end{upotreba}
\end{miditest}
\begin{miditest}
\begin{upotreba}{2}
#\naslovInt#
#\izlaz{Unesite n (<= 12):} \ulaz{0}#
#\izlaz{0! = 1}#
\end{upotreba}
\end{miditest}

\linkresenje{1_19}
\end{Exercise}
\begin{Answer}[ref=1_19]
\includecode{resenja/1_UvodniZadaci/1_19.c}
\end{Answer}

\begin{Exercise}[label=1_20]
Napisati funkciju koja računa $n$-ti element u nizu Fibonačijevih brojeva. Elementi niza Fibonačijevih brojeva $F$ izračunavaju se na osnovu sledećih rekurentnih relacija:
 $$F(0) = 0$$
 $$F(1) = 1$$
 $$F(n) = F(n-1) + F(n-2).$$ 
Napisati program koji testira napisane funkciju tako što se sa standardnog ulaza unosi prirodan broj $n$ i na standardni izlaz se ispisuje rezultat primene napisane funkcije na prirodan broj $n$.

\begin{miditest}
\begin{upotreba}{1}
#\naslovInt#
#\izlaz{Unesite koji clan niza se racuna:} \ulaz{ 5}#
#\izlaz{F(5) = 5}#
\end{upotreba}
\end{miditest}
\begin{miditest}
\begin{upotreba}{2}
#\naslovInt#
#\izlaz{Unesite koji clan niza se racuna:} \ulaz{8}#
#\izlaz{F(8) = 21}#
\end{upotreba}
\end{miditest}

%\linkresenje{1_20}
\end{Exercise}
%\begin{Answer}[ref=1_20]
%\includecode{resenja/1_UvodniZadaci/1_20.c}
%\end{Answer}

\begin{Exercise}[label=1_21]
Elementi niza $F$ izračunavaju se na osnovu sledećih rekurentnih relacija:
 $$F(0) = 0$$
 $$F(1) = 1$$
 $$F(n) = a* F(n-1) + b*F(n-2).$$
Napisati funkciju koja računa $n$-ti element u nizu $F$
\begin{enumerate}
\item iterativno,
\item tako da funkcija bude rekurzivna i da koristi navedene rekurentne relacije,
\item tako da funkcija bude rekurzivna ali da se problemi manje dimenzije rešavaju samo jedan put.
\end{enumerate}
Napisati program koji testira napisane funkciju tako što se sa standardnog ulaza najpre unosi redni broj funckije koja se primenjuje, a zatim i vrednosti koeficijenata $a$ i $b$ i prirodan broj $n$. Na standardni izlaz se ispisuje rezultat primene odabrane funkcije na unete vrednosti koeficijenata $a$ i $b$ i prirodan broj  $n$.  \napomena{Niz  $F$ definisan na ovaj način predstavlja uopštenje Fibonačijevih brojeva.}
  
\begin{miditest}
\begin{upotreba}{1}
#\naslovInt#
#\izlaz{Unesite redni broj funkcije koju zelite:}# 
#\izlaz{1 - iterativna}# 
#\izlaz{2 - rekurzivna}# 
#\izlaz{3 - rekurzivna napredna}# 
#\ulaz{1}#
#\izlaz{Unesite koeficijente:} \ulaz{ 2 3}#
#\izlaz{Unesite koji clan niza se racuna:} \ulaz{ 5}#
#\izlaz{F(5) = 61}#
\end{upotreba}
\end{miditest}
\begin{miditest}
\begin{upotreba}{2}
#\naslovInt#
#\izlaz{Unesite redni broj funkcije koju zelite:}# 
#\izlaz{1 - iterativna}# 
#\izlaz{2 - rekurzivna}# 
#\izlaz{3 - rekurzivna napredna}# 
#\ulaz{3}#
#\izlaz{Unesite koeficijente:} \ulaz{4 2}#
#\izlaz{Unesite koji clan niza se racuna:} \ulaz{8}#
#\izlaz{F(8) = 31360}#
\end{upotreba}
\end{miditest}

\linkresenje{1_21}
\end{Exercise}
\begin{Answer}[ref=1_21]
\includecode{resenja/1_UvodniZadaci/1_21.c}
\end{Answer}

\begin{Exercise}[label=1_22]
Napisati rekurzivnu funkciju koja sabira dekadne cifre datog celog broja $x$. Napisati program koji testira ovu funkciju, za broj koji se unosi sa standardnog ulaza.
  
\begin{minitest}
\begin{test}{1}
#\naslovUlaz#
#\ulaz{123}#
#\naslovIzlaz#
#\izlaz{6}#
\end{test}
\end{minitest}
\begin{minitest}
\begin{test}{2}
#\naslovUlaz#
#\ulaz{23156}#
#\naslovIzlaz#
#\izlaz{17}#
\end{test}
\end{minitest}
\begin{minitest}
\begin{test}{3}
#\naslovUlaz#
#\ulaz{1432}#
#\naslovIzlaz#
#\izlaz{10}#
\end{test}
\end{minitest}      
 
\begin{minitest}
\begin{test}{4}
#\naslovUlaz#
#\ulaz{1}#
#\naslovIzlaz#
#\izlaz{1}#
\end{test}
\end{minitest}
\begin{minitest}
\begin{test}{5}
#\naslovUlaz#
#\ulaz{0}#
#\naslovIzlaz#
#\izlaz{0}#
\end{test}
\end{minitest}      

\linkresenje{1_22}
\end{Exercise}
\begin{Answer}[ref=1_22]
\includecode{resenja/1_UvodniZadaci/1_22.c}
\end{Answer}

%%%\subsection{Rekurzivne funkcije za rad sa nizovima}

\begin{Exercise}[label=1_23]
Napisati rekurzivnu funkciju koja sumira elemente niza celih brojeva
\begin{enumerate}
\item sabirajući elemente počev od početka niza ka kraju niza,
\item sabirajući elemente počev od kraja niza ka početku niza.
\end{enumerate}
Napisati program koji testira napisane funkcije. Sa standardnog ulaza učitati redni broj funkcije (\argf{1} ili \argf{2}), zatim dimenziju $n$  ($0 < n \leq 100$) celobrojnog niza, a potom i elemente niza. Na standardni izlaz ispisati rezultat primene odabrane funkcije nad učitanim nizom, a u slučaju unosa pogrešnog rednog broja funkcije ispisati odgovarajuću poruku i prekinuti izvršavanje pograma.

\begin{miditest}
\begin{upotreba}{1}
#\naslovInt#
#\izlaz{Unesite redni broj funkcije (1 ili 2):} \ulaz{1}#
#\izlaz{Unesite dimenziju niza:} \ulaz{5}#
#\izlaz{Unesite elemente niza:}#
#\ulaz{1 2 3 4 5}#
#\izlaz{Suma elemenata je 15}#
\end{upotreba}
\end{miditest}
\begin{miditest}
\begin{upotreba}{2}
#\naslovInt#
#\izlaz{Unesite redni broj funkcije (1 ili 2):} \ulaz{2}#
#\izlaz{Unesite dimenziju niza:} \ulaz{4}#
#\izlaz{Unesite elemente niza:}#
#\ulaz{-5 2 -3 6}#
#\izlaz{Suma elemenata je 0}#
\end{upotreba}
\end{miditest}

\linkresenje{1_23}
\end{Exercise}
\begin{Answer}[ref=1_23]
\includecode{resenja/1_UvodniZadaci/1_23.c}
\end{Answer}

\begin{Exercise}[label=1_24]
Napisati rekurzivnu funkciju koja određuje maksimum niza celih brojeva. Napisati program koji testira ovu funkciju za niz koji se unosi sa standardnog ulaza. Niz neće imati više od $256$ elemenata. Njegovi elementi se unose sve do unosa kraja ulaza (EOF).
  
 \begin{minitest}
\begin{test}{1}
#\naslovUlaz#
#\ulaz{3 2 1 4 21}#
#\naslovIzlaz#
#\izlaz{21}#
\end{test}
\end{minitest}
\begin{minitest}
\begin{test}{2}
#\naslovUlaz#
#\ulaz{ 2 -1 0 -5 -10}#
#\naslovIzlaz#
#\izlaz{2}#
\end{test}
\end{minitest}
\begin{minitest}
\begin{test}{3}
#\naslovUlaz#
#\ulaz{1 11 3 5 8 1}#
#\naslovIzlaz#
#\izlaz{11}#
\end{test}
\end{minitest}
% \begin{minitest}
% \begin{test}{4}
% #\naslovUlaz#
% #\ulaz{ 5}#
% #\naslovIzlaz#
% #\izlaz{5}#
% \end{test}
% \end{minitest}

\linkresenje{1_24}
\end{Exercise}
\begin{Answer}[ref=1_24]
\includecode{resenja/1_UvodniZadaci/1_24.c}
\end{Answer}

\begin{Exercise}[label=1_25]
Napisati rekurzivnu funkciju koja izračunava skalarni proizvod dva data vektora.  Napisati program koji testira ovu funkciju, za nizove koji se unose sa standardnog ulaza. Prvo se unosi dimenzija nizova, a zatim i njihovi elementi. Nizovi neće imati više od $256$ elemenata.

\begin{miditest}
\begin{upotreba}{1}
#\naslovInt#
#\izlaz{Unesite dimenziju nizova:} \ulaz{3}#
#\izlaz{Unesite elemente prvog niza:}#
#\ulaz{1 2 3}#
#\izlaz{Unesite elemente drugog niza:}#
#\ulaz{1 2 3}#
#\izlaz{Skalarni proizvod je 14}#
\end{upotreba}
\end{miditest}
\begin{miditest}
\begin{upotreba}{2}
#\naslovInt#
#\izlaz{Unesite dimenziju nizova:} \ulaz{2}#
#\izlaz{Unesite elemente prvog niza:}#
#\ulaz{3 5}#
#\izlaz{Unesite elemente drugog niza:}#
#\ulaz{2 6}#
#\izlaz{Skalarni proizvod je 36}#
\end{upotreba}
\end{miditest}
  
%\begin{minitest}
%\begin{test}{3}
%#\naslovUlaz#
%#\ulaz{ 0 }#
%#\naslovIzlaz#
%#\izlaz{0}#
%\end{test}
%\end{minitest}  

\linkresenje{1_25}
\end{Exercise}
\begin{Answer}[ref=1_25]
\includecode{resenja/1_UvodniZadaci/1_25.c}
\end{Answer}

\begin{Exercise}[label=1_26]
Napisati rekurzivnu funkciju koja računa broj pojavljivanja
elementa $x$ u nizu $a$ dužine $n$. Napisati program koji testira ovu funkciju za broj $x$ i niz  $a$ koji se unose sa standardnog ulaza. Prvo se unosi $x$, a zatim elementi niza sve do unosa kraja ulaza. Niz neće imati više od $256$ elemenata. 

\begin{miditest}
\begin{upotreba}{1}
#\naslovInt#
#\izlaz{Unesite ceo broj:}#
#\ulaz{4}#
#\izlaz{Unesite elemente niza:}#
#\ulaz{1 2 3 4}#
#\izlaz{Broj pojavljivanja je 1}#
\end{upotreba}
\end{miditest}
\begin{miditest}
\begin{upotreba}{2}
#\naslovInt#
#\izlaz{Unesite ceo broj:}#
#\ulaz{11}#
#\izlaz{Unesite elemente niza:}#
#\ulaz{3 2 11 14 11 43 1}#
#\izlaz{Broj pojavljivanja je 2}#
\end{upotreba}
\end{miditest}


\begin{miditest}
\begin{upotreba}{3}
#\naslovInt#
#\izlaz{Unesite ceo broj:}#
#\ulaz{1}#
#\izlaz{Unesite elemente niza:}#
#\ulaz{3 21 5 6}#
#\izlaz{Broj pojavljivanja je 0}#
\end{upotreba}
\end{miditest}

\linkresenje{1_26}
\end{Exercise}
\begin{Answer}[ref=1_26]
\includecode{resenja/1_UvodniZadaci/1_26.c}
\end{Answer}

\begin{Exercise}[label=1_27]
Napisati rekurzivnu funkciju kojom se proverava da li su tri zadata broja uzastopni članovi niza. Potom, napisati program koji
  je testira. Sa standardnog ulaza se unose najpre tri tražena
  broja, a zatim elementi niza, sve do kraja ulaza. Pretpostaviti da
neće biti uneto više od $256$ brojeva.
  
\begin{miditest}
\begin{upotreba}{1}
#\naslovInt#
#\izlaz{Unesite tri cela broja:}#
#\ulaz{1 2 3}#
#\izlaz{Unesite elemente niza:}#
#\ulaz{4 1 2 3 4 5}#
#\izlaz{Uneti brojevi jesu uzastopni clanovi niza.}#
\end{upotreba}
\end{miditest}
\begin{miditest}
\begin{upotreba}{2}
#\naslovInt#
#\izlaz{Unesite tri cela broja:}#
#\ulaz{1 2 3}#
#\izlaz{Unesite elemente niza:}#
#\ulaz{11 1 2 4 3 6}#
#\izlaz{Uneti brojevi nisu uzastopni clanovi niza.}#
\end{upotreba}
\end{miditest}
%
%\begin{miditest}
%\begin{test}{Test 3}
%Ulaz:     1 2 3 1 2
%Izlaz:    ne 
%\end{test}
%\end{miditest}

\linkresenje{1_27}
\end{Exercise}
\begin{Answer}[ref=1_27]
\includecode{resenja/1_UvodniZadaci/1_27.c}
\end{Answer}

%%%\subsection{Rekurzivne funkcije za rad sa bitovima}

\begin{Exercise}[label=1_28]
Napisati rekurzivnu funkciju koja vraća broj bitova koji su postavljeni na $1$, 
u binarnoj reprezentaciji njenog celobrojnog argumenta.  
Napisati program koji testira napisanu funkciju za broj koji se učitava sa 
standardnog ulaza u heksadekadnom formatu. 

\begin{minitest}
\begin{test}{1}
#\naslovUlaz#
#\ulaz{0x7F}#
#\naslovIzlaz#
#\izlaz{7}#
\end{test}
\end{minitest}
%\begin{minitest}
%\begin{test}{2}
%#\naslovUlaz#
%#\ulaz{0x80}#
%#\naslovIzlaz#
%#\izlaz{1}#
%\end{test}
%\end{minitest}
\begin{minitest}
\begin{test}{2}
#\naslovUlaz#
#\ulaz{0x00FF00FF}#
#\naslovIzlaz#
#\izlaz{16}#
\end{test}
\end{minitest}  
\begin{minitest}
\begin{test}{3}
#\naslovUlaz#
#\ulaz{0xFFFFFFFF}#
#\naslovIzlaz#
#\izlaz{32}#
\end{test}
\end{minitest}  

\linkresenje{1_28}
\end{Exercise}
\begin{Answer}[ref=1_28]
\includecode{resenja/1_UvodniZadaci/1_28.c}
\end{Answer}

\begin{Exercise}[label=1_29]%\marker+{2}
Napisati rekurzivnu funkciju koja štampa bitovsku
  reprezentaciju neoznačenog celog broja, i program koji je
  testira za vrednost koja se zadaje sa standardnog ulaza.

\begin{miditest}
\begin{test}{1}
#\naslovUlaz#
#\ulaz{10}#
#\naslovIzlaz#
#\izlaz{00000000000000000000000000001010}#
\end{test}
\end{miditest}
\begin{miditest}
\begin{test}{2}
#\naslovUlaz#
#\ulaz{0}#
#\naslovIzlaz#
#\izlaz{00000000000000000000000000000000}#
\end{test}
\end{miditest}
\end{Exercise}

%\begin{Answer}[ref=1_29]
%\includecode{resenja/1_UvodniZadaci/1_29.c}
%\end{Answer}

\begin{Exercise}[label=1_30]
Napisati rekurzivnu funkciju za određivanje
najveće cifre u oktalnom zapisu
neoznačenog celog broja korišćenjem bitskih operatora.
\uputstvo{Binarne cifre grupisati u podgrupe od po tri cifre,
počev od bitova najmanje težine.}

\begin{minitest}
\begin{test}{1}
#\naslovUlaz#
#\ulaz{5}#
#\naslovIzlaz#
#\izlaz{5}#
\end{test}
\end{minitest}
\begin{minitest}
\begin{test}{2}
#\naslovUlaz#
#\ulaz{125}#
#\naslovIzlaz#
#\izlaz{7}#
\end{test}
\end{minitest}
\begin{minitest}
\begin{test}{3}
#\naslovUlaz#
#\ulaz{8}#
#\naslovIzlaz#
#\izlaz{1}#
\end{test}
\end{minitest}  

\linkresenje{1_30}
\end{Exercise}
\begin{Answer}[ref=1_30]
\includecode{resenja/1_UvodniZadaci/1_30.c}
\end{Answer}

\begin{Exercise}[label=1_31]
Napisati rekurzivnu funkciju za određivanje (dekadne vrednosti)
najveće cifre u heksadekadnom zapisu neoznačenog celog broja
korišćenjem bitskih operatora. \uputstvo{Binarne cifre
grupisati u podgrupe od po četiri cifre, počev od bitova
najmanje težine.}

\begin{minitest}
\begin{test}{1}
#\naslovUlaz#
#\ulaz{5}#
#\naslovIzlaz#
#\izlaz{5}#
\end{test}
\end{minitest}
\begin{minitest}
\begin{test}{2}
#\naslovUlaz#
#\ulaz{16}#
#\naslovIzlaz#
#\izlaz{1}#
\end{test}
\end{minitest}
\begin{minitest}
\begin{test}{3}
#\naslovUlaz#
#\ulaz{18}#
#\naslovIzlaz#
#\izlaz{2}#
\end{test}
\end{minitest}  
%\begin{minitest}
%\begin{test}{Test 4}
%Ulaz:  165
%Izlaz: 10
%\end{test}
%\end{minitest}

\linkresenje{1_31}
\end{Exercise}
\begin{Answer}[ref=1_31]
\includecode{resenja/1_UvodniZadaci/1_31.c}
\end{Answer}

%%%\subsection{Rekurzivne funkcije - razni zadaci}

\begin{Exercise}[label=1_32]
Napisati rekurzivnu funkciju \kckod{palindrom} koja ispituje da li je data niska
  palindrom. Napisati program koji testira ovu funkciju na nisci koja se unosi sa standardnog
  ulaza. Pretpostaviti da niska neće neće imati više od $31$ karaktera.
  
\begin{minitest}
\begin{test}{1}
#\naslovUlaz#
#\ulaz{a}#
#\naslovIzlaz#
#\izlaz{da}#
\end{test}
\end{minitest}  
\begin{minitest}
\begin{test}{2}
#\naslovUlaz#
#\ulaz{aa}#
#\naslovIzlaz#
#\izlaz{da}#
\end{test}
\end{minitest}
\begin{minitest}
\begin{test}{3}
#\naslovUlaz#
#\ulaz{aba}#
#\naslovIzlaz#
#\izlaz{da}#
\end{test}
\end{minitest}  
 
\begin{minitest}
\begin{test}{4}
#\naslovUlaz#
#\ulaz{programiranje}#
#\naslovIzlaz#
#\izlaz{ne}#
\end{test}
\end{minitest}
\begin{miditest}
\begin{test}{5}
#\naslovUlaz#
#\ulaz{anavolimilovana}#
#\naslovIzlaz#
#\izlaz{da}#
\end{test}
\end{miditest}
 
\linkresenje{1_32}
\end{Exercise}
\begin{Answer}[ref=1_32]
\includecode{resenja/1_UvodniZadaci/1_32.c}
\end{Answer}

\begin{Exercise}[label=1_33, difficulty=1]
Napisati rekurzivnu funkciju koja prikazuje sve permutacije skupa $\{1, 2, ... ,n\}$. Napisati program koji testira napisanu funkciju za proizvoljan prirodan broj $n$ ($n \le 15$) unet sa standardnog ulaza.

\begin{minitest}
\begin{test}{1}
#\naslovUlaz#
#\ulaz{2}#
#\naslovIzlaz#
#\izlaz{1 2}#
#\izlaz{2 1}#
\end{test}
\end{minitest}  
\begin{minitest}
\begin{test}{2}
#\naslovUlaz#
#\ulaz{3}#
#\izlaz{1 2 3}#
#\izlaz{1 3 2}#
#\izlaz{2 1 3}#
#\izlaz{2 3 1}#
#\izlaz{3 1 2}#
#\izlaz{3 2 1}#
\end{test}
\end{minitest}
\begin{minitest}
\begin{test}{3}
#\naslovUlaz#
#\ulaz{-5}#
#\izlaz{Duzina }#
#\izlaz{permutacije}#
#\izlaz{mora biti}#
#\izlaz{broj iz }#
#\izlaz{intervala}#
#\izlaz{[0, 15]!}#
\end{test}
\end{minitest}


\linkresenje{1_33}
\end{Exercise}
\begin{Answer}[ref=1_33]
\includecode{resenja/1_UvodniZadaci/1_33.c}
\end{Answer}

\begin{Exercise}[label=1_34, difficulty=1]
Paskalov trougao sadrži brojeve čije se vrednosti računaju tako što svako polje ima vrednost
 zbira dve vrednosti koje su u susedna dva polja iznad. Izuzetak su jedinice na krajevima. Vrednosti
 brojeva Paskalovog trougla odgovaraju binomnim koeficijentima tj.~vrednost polja \argf{(n, k)}, gde je $n$ redni broj hipotenuze, a $k$ redni broj elementa u tom redu (na toj hipotenuzi) odgovara binomnom koeficijentu $\binom{n}{k}$, pri čemu brojanje počinje od nule. Na primer, vrednost polja \argf{(4, 2)} je $6$. 

\begin{verbatim}
               1
             1   1
           1   2   1
         1   3   3   1
       1   4   6   4   1
     1   5   10  10  5   1
\end{verbatim}

\begin{enumerate}
\item Napisati rekurzivnu funkciju koja izračunava vrednost binomnog koeficijenta $\binom{n}{k}$ koristeći osobine Paskalovog trougla. 
\item Napisati rekurzivnu funkciju koja izračunava $d_n$ kao sumu elemenata $n$-te hipotenuze Paskalovog trougla.
\end{enumerate}

Napisati program koji za unetu veličinu Paskalovog trougla i redni broj hipotenuze
najpre iscrtava Paskalov trougao, a zatim štampa sumu elemenata hipotenuze.

\begin{miditest}
\begin{test}{1}
#\naslovUlaz#
#\ulaz{5 3}#
#\naslovIzlaz#
            #\izlaz{1}#
          #\izlaz{1}#   #\izlaz{1}#
        #\izlaz{1}#   #\izlaz{2}#   #\izlaz{1}#
      #\izlaz{1}#   #\izlaz{3}#   #\izlaz{3}#   #\izlaz{1}#
    #\izlaz{1}#   #\izlaz{4}#   #\izlaz{6}#   #\izlaz{4}#   #\izlaz{1}#
  #\izlaz{1}#   #\izlaz{5}#   #\izlaz{10}#  #\izlaz{10}#  #\izlaz{5}#  #\izlaz{1}#
#\izlaz{}#
#\izlaz{8}#
\end{test}
\end{miditest}
\begin{miditest}
\begin{test}{2}
#\naslovUlaz#
#\ulaz{6 5}#
#\naslovIzlaz#
            #\izlaz{1}#
          #\izlaz{1}#   #\izlaz{1}#
        #\izlaz{1}#   #\izlaz{2}#   #\izlaz{1}#
      #\izlaz{1}#   #\izlaz{3}#   #\izlaz{3}#   #\izlaz{1}#
    #\izlaz{1}#   #\izlaz{4}#   #\izlaz{6}#   #\izlaz{4}#   #\izlaz{1}#
  #\izlaz{1}#   #\izlaz{5}#   #\izlaz{10}#  #\izlaz{10}#  #\izlaz{5}#  #\izlaz{1}#
#\izlaz{1}#   #\izlaz{6}#   #\izlaz{15}#  #\izlaz{20}# #\izlaz{15}#  #\izlaz{6}#  #\izlaz{1}#
#\izlaz{}#
#\izlaz{32}#
\end{test}
\end{miditest}

\linkresenje{1_34}
\end{Exercise}
\begin{Answer}[ref=1_34]
\includecode{resenja/1_UvodniZadaci/1_34.c}
\end{Answer}

%%%%%%%%%%%%%%%%%%%%%%%%%
% Zadaci sa praktikuma - dodatni zadaci - oni nemaju rešenja
% možda \subsection{Dodatni zadaci} ili zadaci za vezbu
%%%%%%%%%%%%%%%%%%%%%%%%%


\begin{Exercise}[label=1_35]%\marker+{2}
Napisati rekurzivnu funkciju koja prikazuje sve varijacije sa
   ponavljanjem dužine $n$ skupa $\{a, b\}$, i program koji je
   testira, za $n$ koje se unosi sa standardnog ulaza.

\begin{miditest}
\begin{test}{1}
#\naslovUlaz#
#\ulaz{2}#
#\naslovIzlaz#
#\izlaz{a a}#
#\izlaz{a b}#
#\izlaz{b a}#
#\izlaz{b b}#
\end{test}
\end{miditest}
\begin{miditest}
\begin{test}{2}
#\naslovUlaz#
#\ulaz{3}#
#\naslovIzlaz#
#\izlaz{a a a }#
#\izlaz{a a b}#
#\izlaz{a b a}#
#\izlaz{a b b}#
#\izlaz{b a a}#
#\izlaz{b a b}#
#\izlaz{b b a}#
#\izlaz{b b b}#
\end{test}
\end{miditest}
\end{Exercise}
%\begin{Answer}[ref=1_35]
%\includecode{resenja/1_UvodniZadaci/1_35.c}
%\end{Answer}

\begin{Exercise}[label=1_36]%\marker+{2}
{\em Hanojske kule}: Data su tri
  vertikalna štapa, na jednom se nalazi $n$ diskova poluprečnika
  $1$, $2$, $3$,... do $n$, tako da se najveći nalazi na dnu, a
  najmanji na vrhu. Ostala dva štapa su prazna. Potrebno je
  premestiti diskove na drugi štap tako da budu u istom redosledu, pri čemu se ni u jednom
  trenutku ne sme staviti veći disk preko manjeg, a preostali štap se koristi kao pomoćni štap prilikom
  premeštanja. \\
  Napisati program koji za proizvoljnu vrednost $n$, koja se unosi sa standardnog ulaza, prikazuje proces premeštanja diskova.

\end{Exercise}
%\begin{Answer}[ref=1_36]
%\includecode{resenja/1_UvodniZadaci/1_36.c}
%\end{Answer}

\begin{Exercise}[label=1_37]%\marker+{2}
{\em Modifikacija Hanojskih kula}: Data su četiri
  vertikalna štapa, na jednom se nalazi $n$ diskova poluprečnika
  $1$, $2$, $3$,... do $n$, tako da se najveći nalazi na dnu, a
  najmanji na vrhu. Ostala tri štapa su prazna. Potrebno je
  premestiti diskove na drugi štap tako da budu u istom redosledu,
  premestajući jedan po jedan disk, pri čemu se ni u jednom
  trenutku ne sme staviti veći disk preko manjeg, pri čemu se
  preostala dva štapa koriste kao pomoćni štapovi prilikom
  premeštanja.\\
  Napisati program koji za proizvoljnu vrednost $n$, koja se unosi sa standardnog ulaza, prikazuje proces premeštanja diskova.

\end{Exercise}
%\begin{Answer}[ref=1_37]
%\includecode{resenja/1_UvodniZadaci/1_37.c}
%\end{Answer}

\section{Rešenja}
\shipoutAnswer

