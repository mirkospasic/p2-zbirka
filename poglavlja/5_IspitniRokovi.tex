\appendix
\chapter{Ispitni rokovi}

\section{Praktični deo ispita, jun 2015.}

\begin{Exercise}[label=A_01]
Kao argument komandne linije zadaje se ime ulazne datoteke u kojoj se nalaze niske. U prvoj liniji datoteke nalazi se informacija o broju niski, a zatim u narednim linijama po jedna niska ne duža od $50$ karaktera.
  
Napisati program u kojem se dinamički alocira memorija za zadati niz niski, a zatim se na standardnom izlazu u redosledu suprotnom od redosleda čitanja ispisuju sve niske koje počinju velikim slovom. 

U slučaju pojave bilo kakve greške na standardnom izlazu za grešku ispisati vrednost $-1$ i prekinuti izvršavanje programa.

\begin{miditest}
\begin{test}{1}
#\poziv{./a.out ulaz.txt}#

#\naslovDat{ulaz.txt}#
#\datoteka{5}#
#\datoteka{Programiranje}#
#\datoteka{Matematika}#
#\datoteka{12345}#
#\datoteka{dInAmiCnArEc}#
#\datoteka{Ispit}#
  
#\naslovIzlaz#
#\izlaz{Ispit}#
#\izlaz{Matematika}#
#\izlaz{Programiranje}#
\end{test}
\end{miditest}
\begin{minitest}
\begin{test}{2}
#\poziv{./a.out ulaz.txt}#

#\naslovDat{ulaz.txt}#
#\datoteka{2}#
#\datoteka{maksimalano}#
#\datoteka{poena}#

#\naslovIzlaz#
#\izlaz{}#
\end{test}
\end{minitest}


\begin{miditest}
\begin{test}{3}
#\poziv{./a.out ulaz.txt}#

#\naslovDat{Datoteka ulaz.txt ne postoji}#

#\naslovIzlazZaGresku#
#\izlaz{-1}#
\end{test}
\end{miditest}
\begin{miditest}
\begin{test}{4}
#\poziv{./a.out}#

#\naslovIzlazZaGresku#
#\izlaz{-1}#
\end{test}
\end{miditest}

\linkresenje{A_01}
\end{Exercise}
\begin{Answer}[ref=A_01]
\includecode{resenja/A_IspitniRokovi/A_01.c}
\end{Answer}


\begin{Exercise}[label=A_02]
Data je biblioteka za rad sa binarnim pretraživačkim stablima čiji čvorovi sadrže cele brojeve. 
Napisati funkciju   \kckod{int sumiraj\_n (Cvor * koren, int n)}
koja izračunava zbir svih čvorova koji se nalaze na $n$-tom nivou stabla (koren se nalazi na nultom nivou, njegova deca na prvom nivou i tako redom). 
Ispravnost napisane funkcije testirati na osnovu zadate  \kckod{main} funkcije i biblioteke za rad sa pretraživačkim stablima.

Napisati program koji sa standardnog ulaza učitava najpre prirodan broj $n$, a potom i brojeve sve do pojave nule koje smešta u stablo i ispisuje rezultat pozivanja funkcije \kckod{sumiraj\_n} za broj $n$ i tako kreirano stablo. U slučaju greške na standardni izlaz za grešku ispisati $-1$.


\begin{miditest}
\begin{test}{1}
#\naslovUlaz#
#\ulaz{2 8 10 3 6 14 13 7 4 0}#
#\naslovIzlaz#
#\izlaz{ 20}#
\end{test}
\end{miditest}
\begin{miditest}
\begin{test}{2}
#\naslovUlaz#
#\ulaz{0 50 14 5 2 4 56 8 52 7 1 0}#
#\naslovIzlaz#
#\izlaz{ 50}#
\end{test}
\end{miditest}

\linkresenje{A_02}
\end{Exercise}
\begin{Answer}[ref=A_02]
%\includecode{resenja/A_IspitniRokovi/A_02/stabla.h}
%\includecodeLib{resenja/A_IspitniRokovi/A_02/stabla.h}{stabla.h}
%\includecodeLib{resenja/A_IspitniRokovi/A_02/stabla.c}{stabla.c}
\\
\napomena{Rešenje koristi biblioteku za rad sa binarnim pretraživačkim stablima iz zadatka \ref{701}.}
\includecodeLib{resenja/A_IspitniRokovi/A_02.c}{main.c}
%\includecode{resenja/A_IspitniRokovi/A_02/stabla.c}
%\includecode{resenja/A_IspitniRokovi/A_02/main.c}
\end{Answer}

\begin{Exercise}[label=A_03]
Sa standardnog ulaza učitava se broj vrsta i broj kolona celobrojne matrice $A$, 
a zatim i elementi matrice $A$. Napisati program koji će ispisati indeks kolone u kojoj se nalazi najviše negativnih elemenata. 
Ukoliko postoji više takvih kolona, ispisati indeks prve kolone. 
Može se pretpostaviti da je broj vrsta i broj kolona manji od $50$. 
U slučaju greške ispisati vrednost $-1$ na standardni izlaz za greške. 

\begin{minitest}
\begin{test}{1}
#\naslovUlaz#
#\ulaz{4 5}#
#\ulaz{1  2  3  4  5}#
#\ulaz{ -1  2 -3  4 -5 }#
#\ulaz{ -5 -4 -3 -2  1}#
#\ulaz{-1  0  0  0  0 }#
#\naslovIzlaz#
#\izlaz{0}#
\end{test}
\end{minitest}
\begin{minitest}
\begin{test}{2}
#\naslovUlaz#
#\ulaz{2 3}#
#\ulaz{0 0 -5}#
#\ulaz{1 2 -4}#
#\naslovIzlaz#
#\izlaz{2}#
\end{test}
\end{minitest}
\begin{minitest}
\begin{test}{3}
#\naslovUlaz#
#\ulaz{-2}#
#\naslovIzlazZaGresku#
#\izlaz{-1}#
\end{test}
\end{minitest}

\linkresenje{A_03}
\end{Exercise}
\begin{Answer}[ref=A_03]
\includecode{resenja/A_IspitniRokovi/A_03.c}
\end{Answer}

\section{Praktični deo ispita, jul 2015.}

\begin{Exercise}[label=A_04]
Napisati program koji kao prvi arugment komandne linije prima ime dokumenta u kome treba prebrojati sva pojavljivanja tražene niske (bez preklapanja) koja se navodi kao drugi argument komandne linije (iskoristiti funkciju standardne biblioteke \kckod{strstr}). U slučaju
bilo kakve greške ispisati $-1$ na standardni izlaz za greške.
Pretpostaviti da linije datoteke neće biti duže od $127$
karaktera.\\
Potpis funkcije \kckod{strstr}:\\
\kckod{char *strstr(const char *haystack, const char *needle);}\\
Funkcija traži prvo pojavljivanje podniske needle u nisci
haystack, i vraća pokazivač na početak podniske, ili
NULL ako podniska nije pronađena.

\begin{miditest}
\begin{test}{1}
#\poziv{./a.out ulaz.txt test}#

#\naslovDat{ulaz.txt}#
#\datoteka{ Ovo je test primer. }#
#\datoteka{ U njemu se rec test javlja}#
#\datoteka{vise puta. testtesttest}#

#\naslovIzlaz#
#\izlaz{5}#
\end{test}
\end{miditest}
\begin{miditest}
\begin{test}{2}
#\poziv{./a.out}#

#\naslovIzlazZaGresku#
#\izlaz{-1}# 
\end{test}
\end{miditest}

\begin{miditest}
\begin{test}{3}
#\poziv{./a.out ulaz.txt foo}#

#\naslovDat{Datoteka ulaz.txt ne postoji}#

#\naslovIzlazZaGresku#
#\izlaz{-1}#
\end{test}
\end{miditest}
\begin{miditest}
\begin{test}{4}
#\poziv{ ./a.out ulaz.txt .  }#

#\naslovDat{Datoteka ulaz.txt je prazna}#

#\naslovIzlaz#
#\izlaz{0}#
\end{test}
\end{miditest}

\linkresenje{A_04}
\end{Exercise}
\begin{Answer}[ref=A_04]
\includecode{resenja/A_IspitniRokovi/A_04.c}
\end{Answer}


\begin{Exercise}[label=A_05]
Na početku datoteke \kckod{trouglovi.txt} nalazi se broj trouglova čije su koordinate temena zapisane u nastavku datoteke. Napisati
  program koji učitva trouglove, i ispisuje ih na standardni izlaz
  sortirane po površini opadajuće (koristiti Heronov obrazac: 
  $P = \sqrt{s*(s-a)*(s-b)*(s-c)}$, gde je $s$ poluobim trougla). U slučaju bilo kakve greške ispisati $-1$ na standardni izlaz za greške. Ne praviti nikave pretpostavke o broju trouglova u datoteci, i proveriti da li je datoteka ispravno zadata.

\begin{miditest}
\begin{test}{1}
#\naslovDat{trouglovi.txt}#
#\datoteka{4}#
#\datoteka{ 0 0 0 1.2 1 0 }#
#\datoteka{ 0.3 0.3 0.5 0.5 0.9 1}#
#\datoteka{-2 0 0 0 0 1}#
#\datoteka{-2 0 0 0 0 1}#

#\naslovIzlaz#
#\izlaz{2 0 2 2 -1 -1}#
#\izlaz{-2 0 0 0 0 1}#
#\izlaz{0 0 0 1.2 1 0}#
#\izlaz{0.3 0.3 0.5 0.5 0.9 1}#
\end{test}
\end{miditest}
\begin{minitest}
\begin{test}{2}
#\naslovDat{trouglovi.txt}#
#\datoteka{3}#
#\datoteka{ 1.2 3.2 1.1 4.3}#

#\naslovIzlazZaGresku#
#\izlaz{-1}#
\end{test}
\end{minitest}

\begin{miditest}
\begin{test}{3}
#\naslovDat{Datoteka trouglovi.txt ne postoji}#

#\naslovIzlazZaGresku#
#\izlaz{-1}#
\end{test}
\end{miditest}
\begin{minitest}
\begin{test}{4}
#\naslovDat{trouglovi.txt}#
#\datoteka{0}#

#\naslovIzlaz#
#\izlaz{}#
\end{test}
\end{minitest}

\linkresenje{A_05}
\end{Exercise}

\begin{Answer}[ref=A_05]
\includecode{resenja/A_IspitniRokovi/A_05.c}
\end{Answer}

\begin{Exercise}[label=A_06]
Data je biblioteka za rad sa binarnim pretraživačkim stablima celih brojeva iz zadatka \ref{701}.} % trebalo bi \ref{4_14}. 
Napisati funkciju\\ 
\kckod{int prebroj\_n(Cvor *koren, int n)}\\
  koja u datom stablu prebrojava čvorove na $n$-tom nivou, koji
  imaju tačno jednog potomka. Pretpostaviti da se koren nalazi na
  nivou $0$. Ispravnost napisane funkcije testirati na osnovu zadate
  \kckod{main} funkcije i biblioteke za rad sa stablima.

\begin{minitest}
\begin{test}{1}
#\naslovUlaz#
#\ulaz{ 1 5 3 6 1 4 7 9}#
#\naslovIzlaz#
#\izlaz{1}#
\end{test}
\end{minitest}
\begin{minitest}
\begin{test}{2}
#\naslovUlaz#
#\ulaz{2 5 3 6 1 0 4 7 9}#
#\naslovIzlaz#
#\izlaz{2}#
\end{test}
\end{minitest}
\begin{minitest}
\begin{test}{3}
#\naslovUlaz#
#\ulaz{0 4 2 5}#
#\naslovIzlaz#
#\izlaz{0}#
\end{test}
\end{minitest}

\begin{minitest}
\begin{test}{4}
#\naslovUlaz#
#\ulaz{ 3}#
#\naslovIzlaz#
#\izlaz{0}#
\end{test}
\end{minitest}
\begin{minitest}
\begin{test}{5}
#\naslovUlaz#
#\ulaz{-1 4 5 1 7}#
#\naslovIzlaz#
#\izlaz{0}#
\end{test}
\end{minitest}

\linkresenje{A_06}
\end{Exercise}
\begin{Answer}[ref=A_06]
%%\includecode{resenja/A_IspitniRokovi/A_06/stabla.h}  
%\includecodeLib{resenja/A_IspitniRokovi/A_06/stabla.h}{stabla.h}
%\includecodeLib{resenja/A_IspitniRokovi/A_06/stabla.c}{stabla.c}
\\
\napomena{Rešenje koristi biblioteku za rad sa binarnim pretraživačkim stablima iz zadatka \ref{701}.} % trebalo bi \ref{4_14}
\includecodeLib{resenja/A_IspitniRokovi/A_06.c}{main.c}
%%\includecode{resenja/A_IspitniRokovi/A_06/stabla.c}
%%\includecode{resenja/A_IspitniRokovi/A_06/main.c}
\end{Answer}


\section{Praktični deo ispita, septembar 2015.}

\begin{Exercise}[label=A_07]
Sa standardnog ulaza se učitavaju neoznačeni celi brojevi $x$ i $n$. Na
   standardni izlaz ispisati neoznačen ceo broj koji se dobija od broja $x$ kada se njegov binarni zapis
   rotira za $n$ mesta udesno (na primer, ako je binarni zapis broja $x$ jednak \texttt{00000000000000000000000000001111},
   i ako je $n=1$ tada na standardni izlaz treba ispisati neočnačen broj čiji je binarni zapis jednak \texttt{10000000000000000000000000000111}).

\begin{minitest}
\begin{test}{1}
#\naslovUlaz#
#\ulaz{6 1}#
#\naslovIzlaz#
#\izlaz{3}#
\end{test}
\end{minitest}
\begin{minitest}
\begin{test}{2}
#\naslovUlaz#
#\ulaz{15 3}#
#\naslovIzlaz#
#\izlaz{3758096385}#
\end{test}
\end{minitest}
\begin{minitest}
\begin{test}{3}
#\naslovUlaz#
#\ulaz{31 100}#
#\naslovIzlaz#
#\izlaz{4026531841}#
\end{test}
\end{minitest}

\begin{minitest}
\begin{test}{4}
#\naslovUlaz#
#\ulaz{4 0}#
#\naslovIzlaz#
#\izlaz{4}#
\end{test}
\end{minitest}
\begin{minitest}
\begin{test}{5}
#\naslovUlaz#
#\ulaz{0 5}#
#\naslovIzlaz#
#\izlaz{0}#
\end{test}
\end{minitest}

\linkresenje{A_07}
\end{Exercise}
\begin{Answer}[ref=A_07]
\includecode{resenja/A_IspitniRokovi/A_07.c}
\end{Answer}


\begin{Exercise}[label=A_08]
Napisati funkciju \kckod{int dopuni\_listu(Cvor ** adresa\_glave)}
koja samo čvorovima koji imaju sledbenika u jednostruko povezanoj listi realnih brojeva,
  dodaje između čvora i njegovog sledbenika nov čvor čija vrednost je aritmetička sredina njihovih vrednosti. Povratna vrednost funkcije treba da bude $1$ ukoliko je došlo greške pri alokaciji memorije, inače $0$.
 Ispravnost napisane funkcije testirati koristeći dostupnu biblioteku za rad sa listama i \kckod{main} funkciju koja najpre
 učitava elemente liste, poziva pomenutu funkciju i ispisuje sadržaj liste.

\begin{maxitest}
\begin{test}{1}
#\naslovUlaz#
#\ulaz{1 2 3 4 5}#
#\naslovIzlaz#
#\izlaz{1.00 1.50 2.00 2.50 3.00 3.50 4.00 4.50 5.00}#
\end{test}
\end{maxitest}

\begin{minitest}
\begin{test}{2}
#\naslovUlaz#
#\ulaz{12}#
#\naslovIzlaz#
#\izlaz{12.00}#
\end{test}
\end{minitest}
\begin{minitest}
\begin{test}{3}
#\naslovUlaz#
#\ulaz{prazna lista}#
#\naslovIzlaz#
#\izlaz{}#
\end{test}
\end{minitest}
\begin{minitest}
\begin{test}{4}
#\naslovUlaz#
#\ulaz{13.3 15.8}#
#\naslovIzlaz#
#\izlaz{13.30 14.55}#
\end{test}
\end{minitest}

\linkresenje{A_08}
\end{Exercise}

\begin{Answer}[ref=A_08]
%\includecode{resenja/A_IspitniRokovi/908/liste.h}
\includecodeLib{resenja/A_IspitniRokovi/A_08/liste.h}{liste.h}
\includecodeLib{resenja/A_IspitniRokovi/A_08/liste.c}{liste.c}
\includecodeLib{resenja/A_IspitniRokovi/A_08/main.c}{main.c}
%\includecode{resenja/A_IspitniRokovi/908/liste.c}
%\includecode{resenja/A_IspitniRokovi/908/main.c}
\end{Answer}

\begin{Exercise}[label=A_09]
Sa standardnog ulaza se učitava dimenzija $n$ kvadratne celobrojne
    matrice $A$ ($n>0$), a zatim i elementi matrice $A$. Napisati program koji
    proverava da li je data kvadratna matrica magični kvadrat
    (magični kvadrat je kvadratna matrica kod koje su sume brojeva
    u svim redovima i kolonama međusobno jednake). Ukoliko jeste, ispisati na
    standardnom izlazu sumu brojeva jedne vrste ili kolone te matrice,
    a ukoliko nije ispisati "-". Broj vrsta i broj kolona matrice nije
    unapred poznat. U slučaju greške ispisati $-1$ na standardni izlaz za greške.
    \napomena{Rešenje koristi biblioteku za rad sa matricama iz zadatka \ref{2_19}.}

\begin{minitest}
\begin{test}{1}
#\naslovUlaz#
#\ulaz{4}#
#\ulaz{1 2 3 4}#
#\ulaz{2 1 4 3}#
#\ulaz{3 4 2 1}#
#\ulaz{4 3 1 2}#
#\naslovIzlaz#
#\izlaz{10}#
\end{test}
\end{minitest}
\begin{minitest}
\begin{test}{2}
#\naslovUlaz#
#\ulaz{3}#
#\ulaz{1 1 1}#
#\ulaz{1 1 1}#
#\ulaz{1 1 1}#
#\naslovIzlaz#
#\izlaz{3}#
\end{test}
\end{minitest}
\begin{minitest}
\begin{test}{3}
#\naslovUlaz#
#\ulaz{2}#
#\ulaz{1 1}#
#\ulaz{2 2}#
#\naslovIzlaz#
#\izlaz{-}#
\end{test}
\end{minitest}

\begin{minitest}
\begin{test}{4}
#\naslovUlaz#
#\ulaz{2}#
#\ulaz{1 2}#
#\ulaz{1 2}#
#\naslovIzlaz#
#\izlaz{-}#
\end{test}
\end{minitest}
\begin{minitest}
\begin{test}{5}
#\naslovUlaz#
#\ulaz{1}#
#\ulaz{5}#
#\naslovIzlaz#
#\izlaz{5}#
\end{test}
\end{minitest}
\begin{minitest}
\begin{test}{6}
#\naslovUlaz#
#\ulaz{0}#
#\naslovIzlazZaGresku#
#\izlaz{-1}#
\end{test}
\end{minitest}

\linkresenje{A_09}
\end{Exercise}
\begin{Answer}[ref=A_09]
\includecode{resenja/A_IspitniRokovi/A_09.c}
\end{Answer}

\section{Praktični deo ispita, januar 2016.}
\begin{Exercise}[label=A_10]
Napisati funkciju \kckod{unsigned int zamena(unsigned int x)} koja u datom broju \argf{x} menja mesta prvom i četvrtom bajtu. Prvi bajt je sačinjen od $8$ bitova najmanje težine. Napisati program koji testira funkciju \kckod{zamena} za ceo broj $x$ unet sa standardnog ulaza. U slučaju da je uneti broj $x$ negativan na standardni izlaz za greške program ispisuje $-1$, a inače ispisaje na standardni izlaz broj dobijen primenom funkcije \kckod{zamena}. 

\begin{minitest}
\begin{test}{1}
#\naslovUlaz#
#\ulaz{285278344}#
#\naslovIzlaz#
#\izlaz{2281766929}#
\end{test}
\end{minitest}
\begin{minitest}
\begin{test}{2}
#\naslovUlaz#
#\ulaz{1024}#
#\naslovIzlaz#
#\izlaz{1024}#
\end{test}
\end{minitest}
\begin{minitest}
\begin{test}{3}
#\naslovUlaz#
#\ulaz{1}#
#\naslovIzlaz#
#\izlaz{16777216}#
\end{test}
\end{minitest}

\begin{minitest}
\begin{test}{4}
#\naslovUlaz#
#\ulaz{0}#
#\naslovIzlaz#
#\izlaz{0}#
\end{test}
\end{minitest}
\begin{minitest}
\begin{test}{5}
#\naslovUlaz#
#\ulaz{-63}#
#\naslovIzlazZaGresku#
#\izlaz{-1}#
\end{test}
\end{minitest}

\linkresenje{A_10}
\end{Exercise}
\begin{Answer}[ref=A_10]
\includecode{resenja/A_IspitniRokovi/A_10.c}
\end{Answer}


\begin{Exercise}[label=A_11]
Data je biblioteka za rad sa binarnim pretraživackim stablima celih brojeva iz zadatka  \ref{701}. % trebalo bi \ref{4_14} 
Napisati funkciju \kckod{int najduzi\_put (Cvor * koren)} koja za dato stablo izračunava dužinu najdužeg puta od korena do nekog lista. Ako je stablo prazno, povratna vrednost funkcije je $-1$. Ako stablo ima samo koren dužina najdužeg puta je $0$. Ispravnost napisane funkcije testirati na osnovu zadate \kckod{main} funkcije i biblioteke za rad sa stablima.

\napomena{Rešenje koristi biblioteku za rad sabinarnim pretraživackim stablima celih brojeva iz zadatka  \ref{701}.} % trebalo bi \ref{4_14} 

\begin{miditest}
\begin{test}{1}
#\naslovUlaz#
#\ulaz{10 5 15 3 2 4 30 12 14 13}#
#\naslovIzlaz#
#\izlaz{4}#
\end{test}
\end{miditest}
\begin{minitest}
\begin{test}{2}
#\naslovUlaz#
#\ulaz{3}#
#\naslovIzlaz#
#\izlaz{0}#
\end{test}
\end{minitest}

\begin{minitest}
\begin{test}{3}
#\naslovUlaz#
#\ulaz{5 6}#
#\naslovIzlaz#
#\izlaz{1}#
\end{test}
\end{minitest}
\begin{minitest}
\begin{test}{4}
#\naslovUlaz#
#\ulaz{7 5 8}#
#\naslovIzlaz#
#\izlaz{1}#
\end{test}
\end{minitest}
\begin{minitest}
\begin{test}{5}
#\naslovUlaz#
#\ulaz{5 7 8}#
#\naslovIzlaz#
#\izlaz{2}#
\end{test}
\end{minitest}


\linkresenje{A_11}
\end{Exercise}

\begin{Answer}[ref=A_11]
%\includecodeLib{resenja/A_IspitniRokovi/A_08/liste.h}{liste.h}
%\includecodeLib{resenja/A_IspitniRokovi/A_08/liste.c}{liste.c}
\napomena{Rešenje koristi biblioteku za rad sa binarnim pretraživačkim stablima iz zadatka \ref{701}.} % trebalo bi \ref{4_14}
\includecode{resenja/A_IspitniRokovi/A_11.c}
\end{Answer}

\begin{Exercise}[label=A_12]
Sa standardnog ulaza zadaje se ime datoteke u kojoj se nalazi matrica realnih brojeva jednostruke tačnosti
i jedan realan broj. Napisati program koji iz datoteke učitava matricu realnih brojeva, a zatim pronalazi i na standardni
izlaz ispisuje indeks vrste matrice u kojoj se uneti realan broj pojavljuje najmanje puta. Ako postoji više takvih vrsta,
ispisati indeks prve vrste. U datoteci su prvo navedena dva cela broja koja predstavljaju dimenzije matrice, redom broj
vrsta i broj kolona, a zatim i elementi matrice vrstu po vrstu. U slucaju greške ispisati $-1$ na standardni izlaz za greške.
Pretpostaviti da ime datoteke neće biti duže od $30$ karaktera.
 \napomena{U zadatku treba koristiti dinamičku alokaciju matrice.}

\begin{minitest}
\begin{test}{1}
#\naslovUlaz#
#\ulaz{brojevi.txt 0}#

#\naslovDat{brojevi.txt}#
#\datoteka{4 4}#
#\datoteka{0 0 0 1.2}#
#\datoteka{1 0 0.3 0.3}#
#\datoteka{0.5 0.5 0.9 -1}#
#\datoteka{-2 0 0 0}#

#\naslovIzlaz#
#\izlaz{2}#
\end{test}
\end{minitest}
\begin{minitest}
\begin{test}{2}
#\naslovUlaz#
#\ulaz{in.txt 2}#

#\naslovDat{in.txt}#
#\datoteka{3 3}#
#\datoteka{2 0 2}#
#\datoteka{-1 2 -1}#
#\datoteka{2 5 3}#

#\naslovIzlaz#
#\izlaz{1}#
\end{test}
\end{minitest}
\begin{minitest}
\begin{test}{3}
#\naslovUlaz#
#\ulaz{matrica.txt 7}#

#\naslovDat{matrica.txt}#
#\datoteka{3 2}#
#\datoteka{1.1 -5.31}#
#\datoteka{-3.7 35.24}#
#\datoteka{1.4 2.09}#

#\naslovIzlaz#
#\izlaz{0}#
\end{test}
\end{minitest}

\begin{minitest}
\begin{test}{4}
#\naslovUlaz#
#\ulaz{brojevi.txt 12}#

#\naslovDat{brojevi.txt}#
#\datoteka{}#

#\naslovIzlazZaGresku#
#\izlaz{-1}#
\end{test}
\end{minitest}

\linkresenje{A_12}
\end{Exercise}
\begin{Answer}[ref=A_12]
\includecode{resenja/A_IspitniRokovi/A_12.c}
\end{Answer}

\section{Rešenja}
\shipoutAnswer
