\chapter{Pokazivači}

\section{Pokazivačka aritmetika}

\begin{Exercise}[label=301]
\komentar{Milen: ovako definisan zadatak zahteva dva programa kao resenja, a ne jedan sa definisane dve funkcije.}
Za dati celobrojni niz dimenzije $n$, napisati funkciju koja obrće njegove elemente:
\begin{enumerate}
\item korišćenjem indeksne sintakse,
\item korišćenjem pokazivačke sintakse.
\end{enumerate}
Napisati program koji testira napisanu funkciju. Sa standardnog ulaza učitati dimenziju 
niza $n$ ($0 < n \leq 100$), a zatim elemente niza. Prikazati sadržaj niza posle 
poziva funkcije za obrtanje elemenata niza. 

\begin{minitest}
\begin{test}{Test 1}
Ulaz:  3
       1 -2 3
Izlaz: 3 -2 1 
\end{test}
\end{minitest}
\begin{maxitest}
\begin{test}{Test 2}
Ulaz:  0
Izlaz: Greska: neodgovarajuca dimenzija niza.
\end{test}
\end{maxitest}

\end{Exercise}
\begin{Answer}[ref=301]
\includecode{resenja/03_Pokazivaci/301.c}
\end{Answer}

\begin{Exercise}[label=302]
Dat je niz realnih brojeva dimenzije $n$. 
\begin{enumerate}
\item Napisati funkciju \kckod{zbir} koja izračunava zbir elemenata niza.
\item Napisati funkciju \kckod{proizvod} koja izračunava proizvod elemenata niza.
\item Napisati funkciju \kckod{min\_element}  koja izračunava najmanji elemenat niza.
\item Napisati funkciju \kckod{max\_element}  koja izračunava najveći elemenat niza.
\end{enumerate}
Napisati program koji testira napisane funkcije. Sa standardnog ulaza 
učitati dimenziju $n$ ($0 < n \leq 100$) realnog niza, a zatim i 
elemente niza. Na standardni izlaz ispisati zbir, proizvod, 
minimalni i maksimalni element učitanog niza.

\begin{miditest}
\begin{test}{Test 1}
Ulaz: 3 
      -1.1 2.2 3.3 
Izlaz: zbir = 4.400
       proizvod = -7.986
       min = -1.100
       max = 3.300
\end{test}
\end{miditest} 

\end{Exercise}
\begin{Answer}[ref=302]
\includecode{resenja/03_Pokazivaci/302.c}
\end{Answer}

\begin{Exercise}[label=303]
Korišćenjem pokazivačke sintakse, napisati funkciju koja
vrednosti elemenata u prvoj polovini niza povećava za jedan, a u
drugoj polovini smanjuje za jedan. Ukoliko niz ima neparan broj
elemenata, onda vrednost srednjeg elementa niza ostaviti
nepromenjenim. Napisati program koji testira napisanu funkciju. Sa
standardnog ulaza učitati dimenziju $n$ ($0 < n \leq 100$)
celobrojong niza, a zatim i elemente niza. Na standardni izlaz
ispisati rezultat primene napisane funkcije nad učitanim nizom.
\komentar{Jelena: Sta kazete na to da prekoracenja dimenzije niza
u razlicitim zadacima razlicito obradjujemo. Na primer, mozemo 
da unosimo dimenziju niza sve dok se ne unese broj koji je u odgovarajucem 
opsegu, ili mozemo da dimenziju postavimo na 1 ako je korisnik uneo broj 
manji od 1, a na MAX ako je korisnik uneo broj veci od MAX, itd?}

\begin{miditest}
\begin{test}{Test 1}
Ulaz: 5 
      1 2 3 4 5
Izlaz: 2 3 3 3 4
\end{test}
\end{miditest} 
\begin{miditest}
\begin{test}{Test 2}
Ulaz: 4 
      4 -3 2 -1
Izlaz: 5 -2 1 -2
\end{test}
\end{miditest} 

\begin{maxitest}
\begin{test}{Test 3}
Ulaz: 0
Izlaz: Greska: neodgovarajuca dimenzija niza.
\end{test}
\end{maxitest}

\begin{maxitest}
\begin{test}{Test 4}
Ulaz: 101
Izlaz: Greska: neodgovarajuca dimenzija niza.
\end{test}
\end{maxitest}
\end{Exercise}
\begin{Answer}[ref=303]
\includecode{resenja/03_Pokazivaci/303.c}
\end{Answer}

\begin{Exercise}[label=304]
Napisati program koji ispisuje broj prihvaćenih argumenata
komandne linije, a zatim i same argumenate kojima prethode njihovi
redni brojevi. Nakon toga ispisati prve karaktere svakog od argumenata.
Zadatak rešiti:
\begin{enumerate}
\item korišćenjem indeksne sintakse,
\item korišćenjem pokazivačke sintakse.
\end{enumerate} 
\komentar{Jelena: Da li je ok da ovaj zadatak pod a i b resim na nacin na 
koji sam resila, odnosno, da jedno od ta dva resenja iskomentarisem?} 
\komentar{Milena: Meni se cini da je bolje bez komentarisanja, vec da su oba prisutna.}

\begin{miditest}
\begin{test}{Test 1}
Poziv: ./a.out prvi 2. treci -4
Izlaz: 5
       0 ./a.out
       1 prvi
       2 2.
       3 treci
       4 -4
       . p 2 -
\end{test}
\end{miditest} 
\begin{minitest}
\begin{test}{Test 2}
Poziv: ./a.out 
Izlaz: 1
       0 ./a.out
       . 
\end{test}
\end{minitest} 

\end{Exercise}
\begin{Answer}[ref=304]
\includecode{resenja/03_Pokazivaci/304.c}
\end{Answer}

\begin{Exercise}[label=305]
Korišćenjem pokazivačke sintakse, napisati funkciju koja
za datu nisku ispituje da li je palindrom. Napisati program koji
vrši prebrojavanje argumenata komandne linije koji su
palindromi.

\begin{maxitest}
\begin{test}{Test 1}
Poziv:  ./a.out programiranje anavolimilovana topot ana anagram t
Izlaz:  4
\end{test}
\end{maxitest}

\begin{miditest}
\begin{test}{Test 2}
Poziv: ./a.out a b 11 212 
Izlaz: 4
\end{test}
\end{miditest}
\begin{minitest}
\begin{test}{Test 3}
Poziv: ./a.out 
Izlaz: 0
\end{test}
\end{minitest}

\end{Exercise}
\begin{Answer}[ref=305]
\includecode{resenja/03_Pokazivaci/305.c}
\end{Answer}

\begin{Exercise}[label=306]
Napisati program koji kao prvi argument komandne linije prihvata
putanju do datoteke za koju treba proveriti koliko reči ima
$n$ karaktera, gde se $n$ zadaje kao drugi argument
komandne linije. Smatrati da reč ne sadrži više od $100$ karaktera.
U zadatku ne koristiti ugrađene funkcije za
rad sa niskama, već implementirati svoje koristeći
pokazivačku sintaksu.

\begin{maxitest}
\begin{test}{Test 1}
Poziv:    ./a.out ulaz.txt 1
ulaz.txt: Ovo je sadrzaj datoteke i u njoj ima reci koje imaju 
          1 karakter
Izlaz:    3
\end{test}
\end{maxitest}

\begin{maxitest}
\begin{test}{Test 2}
Poziv:  ./a.out ulaz.txt 
Izlaz:  Greska: Nedovoljan broj argumenata komandne linije.
        Program se poziva sa ./a.out ime_dat br_karaktera.
\end{test}
\end{maxitest}

\begin{maxitest}
\begin{test}{Test 3}
Poziv:    ./a.out ulaz.txt 2
(ne postoji datoteka ulaz.txt)
Izlaz: Greska: Neuspesno otvaranje datoteke ulaz.txt.
\end{test}
\end{maxitest}

\end{Exercise}
\begin{Answer}[ref=306]
\includecode{resenja/03_Pokazivaci/306.c}
\end{Answer}

\begin{Exercise}[label=307]
Napisati program koji kao prvi argument komandne linije prihvata
putanju do datoteke za koju treba proveriti koliko reči ima
zadati sufiks (ili prefiks), koji se zadaje kao drugi argument
komandne linije. Smatrati da reč ne sadrži više od $100$ karaktera.
Program je neophodno pozvati sa jednom od opcija
\kckod{-s} ili \kckod{-p} u zavisnosti od čega treba proveriti
koliko reči ima zadati sufiks (ili prefiks). U zadatku ne
koristiti ugrađene funkcije za rad sa niskama, već
implementirati svoje koristeći pokazivačku sintaksu.
\komentar{Milena: Umesto komentara "Funkcija strcpy () iz standardne biblioteke" i ostalih slicnih, napisati
"Implementacije funkcije strcpy () iz standardne biblioteke"}

\begin{maxitest}
\begin{test}{Test 1}
Poziv:    ./a.out ulaz.txt ke -s
ulaz.txt: Ovo je sadrzaj datoteke i u njoj ima reci koje se 
          zavrsavaju na ke
Izlaz:    2
\end{test}
\end{maxitest}

\begin{maxitest}
\begin{test}{Test 2}
Poziv:    ./a.out ulaz.txt sa -p
ulaz.txt: Ovo je sadrzaj datoteke i u njoj ima reci koje 
          pocinju sa sa
Izlaz:    3
\end{test}
\end{maxitest}

\begin{maxitest}
\begin{test}{Test 3}
Poziv:  ./a.out ulaz.txt sa -p
(ne postoji datoteka ulaz.txt)
Izlaz:  Greska: Neuspesno otvaranje datoteke ulaz.txt.
\end{test}
\end{maxitest}

\begin{maxitest}
\begin{test}{Test 3}
Poziv:  ./a.out ulaz.txt
Izlaz:  Greska: Nedovoljan broj argumenata komandne linije.
        Program se poziva sa ./a.out ime_dat suf/pref -s/-p.
\end{test}
\end{maxitest}

\end{Exercise}
\begin{Answer}[ref=307]
\includecode{resenja/03_Pokazivaci/307.c}
\end{Answer}

\section{Višedimenzioni nizovi}

\begin{Exercise}[label=314]
Data je kvadratna matrica dimenzije $n$.
\begin{enumerate}
\item Napisati funkciju koja izračunava trag matrice (sumu elemenata na glavnoj dijagonali).
\item Napisati funkciju koja izračunava euklidsku normu matrice (koren sume kvadrata svih elemenata).
\item Napisati funkciju koja izračunava gornju vandijagonalnu normu matrice (sumu apsolutnih vrednosti elemenata iznad glavne dijagonale).
\end{enumerate}
Napisati program koji testira napisane funkcije. Sa standardnog
ulaza učitati dimanziju kvadratne matrice $n$
($0 < n \leq 100$), a zatim i elemente matrice. Na standardni izlaz
ispisati učitanu matricu a zatim trag, euklidsku normu i vandijagonalnu normu 
učitane matrice.

\begin{maxitest}
\begin{test}{Test 1}
Ulaz:  3 1 -2 3 4 -5 6 7 -8 9
Izlaz: 1 -2 3 
       4 -5 6 
       7 -8 9
       trag = 5
       euklidska norma = 16.88
       vandijagonalna norma = 11
\end{test}
\end{maxitest}

\begin{maxitest}
\begin{test}{Test 2}
Ulaz:  0
Izlaz: Greska: neodgovarajuca dimenzija matrice. 
\end{test}
\end{maxitest}
\end{Exercise}
\begin{Answer}[ref=314]
\includecode{resenja/03_Pokazivaci/314.c}
\end{Answer}

\begin{Exercise}[label=315]
Date su dve kvadratne matrice istih dimenzija $n$. 
\begin{enumerate}
\item Napisati funkciju koja proverava da li su matrice jednake.
\item Napisati funkciju koja izračunava zbir matrica.
\item Napisati funkciju koja izračunava proizvod matrica.
\end{enumerate}
Napisati program koji testira napisane funkcije. Sa standardnog
ulaza učitati dimanziju kvadratnih matrica $n$
($0 < n \leq 100$), a zatim i elemente matrica. Na standardni izlaz
ispisati ,,da`` ako su matrice jednake, ,,ne`` ako nisu a zatim ispisati
zbir i proizvod učitanih matrica.

\begin{maxitest}
\begin{test}{Test 1}
Ulaz:   3 
        1 2 3 1 2 3 1 2 3
        1 2 3 1 2 3 1 2 3
Izlaz:  da
        Zbir matrica je:
        2 4 6
        2 4 6
        2 4 6
        Proizvod matrica je:
        6 12 18
        6 12 18
        6 12 18
\end{test}
\end{maxitest}
\end{Exercise}
\begin{Answer}[ref=315]
\includecode{resenja/03_Pokazivaci/315.c}
\end{Answer}

\begin{Exercise}[label=322]
Relacija se može predstaviti kvadratnom matricom nula i
jedinica na sledeći način: dva elementa $i$ i $j$
su u relaciji ukoliko se u preseku $i$-te vrste i $j$-te
kolone matrice nalazi broj $1$, a nisu u relaciji ukoliko se
tu nalazi broj $0$. 
\begin{enumerate}
\item Napisati funkciju koja proverava da li je relacija zadata matricom refleksivna.
\item Napisati funkciju koja proverava da li je relacija zadata matricom simetrična.
\item Napisati funkciju koja proverava da li je relacija zadata matricom tranzitivna.
\item Napisati funkciju koja određuje refleksivno zatvorenje relacije (najmanju refleksivnu relaciju koja sadrži datu).
\item Napisati funkciju koja određuje simetrično zatvorenje relacije (najmanju simetričnu relaciju koja sadrži datu).
\item Napisati funkciju koja određuje refleksivno-tranzitivno zatvorenje relacije (najmanju refleksivnu i tranzitivnu relaciju
koja sadrži datu)(Napomena: koristiti Varšalov algoritam).
\end{enumerate}
Napisati program koji učitava matricu iz datoteke čije se ime zadaje kao prvi argument komandne linije.
U prvoj liniji datoteke nalazi se dimenzija matrice $n$ ($0 < n \leq 64$), a potom i sami elementi matrice.
Na standardni izlaz ispisati rezultat testiranja napisanih funkcija.

\begin{maxitest}
\begin{test}{Test 1}
Poziv: ./a.out ulaz.txt
ulaz.txt:  4
           1 0 0 0
           0 1 1 0
           0 0 1 0
           0 0 0 0
Izlaz:     Refleksivnost: ne
           Simetricnost: ne
           Tranzitivnost: da
           Refleksivno zatvorenje:
           1 0 0 0
           0 1 1 0
           0 0 1 0
           0 0 0 1
           Simetricno zatvorenje:
           1 0 0 0
           0 1 1 0
           0 1 1 0
           0 0 0 0
           Refleksivno-tranzitivno zatvorenje:
           1 0 0 0
           0 1 1 0
           0 0 1 0
           0 0 0 0
\end{test}
\end{maxitest}
\end{Exercise}
\begin{Answer}[ref=322]
\includecode{resenja/03_Pokazivaci/322.c}
\end{Answer}

\begin{Exercise}[label=323]
Data je kvadratna matrica dimenzije $n$.
\begin{enumerate}
\item Napisati funkciju koja određuje najveći element matrice na sporednoj dijagonali.
\item Napisati funkciju koja određuje indeks kolone koja sadrži najmanji element matrice.
\item Napisati funkciju koja određuje indeks vrste koja sadrži najveći element matrice.
\item Napisati funkciju koja određuje broj negativnih elemenata matrice.
\end{enumerate}
Napisati program koji testira napisane funkcije. Sa standardnog
ulaza učitati elemente celobrojne kvadratne matrice čija
se dimenzija $n$ ($0 < n \leq 32$) zadaje kao argument
komandne linije. Na standardni izlaz ispisati najveći element matrice na sporednoj dijagonali,
indeks kolone koja sadrži najmanji element, indeks vrste koja sadrži najveći element i broj 
negativnih elemenata učitane matrice. 

\komentar{Milena: Izbegavala bih komentare koji ulaze u kod i na taj nacin narusavaju citljivost koda, kao sto je to npr u funkciji indeks\_min i indeks\_max. Resenje 2.15 - izbacila bih napomenu iz komentara. Slicno mi se cini i za zadatak 2.17. Zadatak 2.17 - cini mi se da je resenje bez koriscnja biblioteckih funckija visak? Zadatak 2.19 --- izvuci komentare za ucitaj i ispisi ispred funkcija, umesto sto su unutar funkcija. Zadatak 2.21 - cini mi se da komentari unutar funkcije izmeni narusavaju citljivost koda. Resenje 2.26 --- izbaciti nasa slova iz komentara, izbaciti mozda napomenu sa pocetka jer je suvisna}

\begin{miditest}
\begin{test}{Test 1}
Poziv:  ./a.out 3
Ulaz:   1 2 3
        -4 -5 -6
        7 8 9
Izlaz:  7 2 2 3
\end{test}
\end{miditest}
\begin{miditest}
\begin{test}{Test 2}
Poziv:  ./a.out 4
Ulaz:   -1 -2 -3 -4
        -5 -6 -7 -8
        -9 -10 -11 -12
        -13 -14 -15 -16
Izlaz:  -4 3 0 16
\end{test}
\end{miditest}

\begin{maxitest}
\begin{test}{Test 3}
Poziv: ./a.out 
Izlaz: Greska: Nedovoljan broj argumenata komandne linije.
       Program se poziva sa ./a.out dim_matrice.
\end{test}
\end{maxitest}
\end{Exercise}
\begin{Answer}[ref=323]
\includecode{resenja/03_Pokazivaci/323.c}
\end{Answer}

\begin{Exercise}[label=324]
Napisati funkciju kojom se proverava da li je zadata kvadratna
matrica dimenzije $n$ ortonormirana. Matrica je ortonormirana
ako je skalarni proizvod svakog para različitih vrsta jednak
nuli, a skalarni proizvod vrste sa samom sobom jednak jedinici.
Napisati program koji testira napisanu funkciju. Sa standardnog
ulaza učitati dimenziju celobrojne kvadratne matrice $n$
($0 < n \leq 32$), a zatim i njene elemente. Na standardni izlaz
ispisati rezultat primene napisane funkcije na učitanu
matricu.

\begin{minitest}
\begin{test}{Test 1}
Ulaz:  4
       1 0 0 0
       0 1 0 0
       0 0 1 0
       0 0 0 1
Izlaz: da
\end{test}
\end{minitest}
\begin{minitest}
\begin{test}{Test 2}
Ulaz:  3
       1 2 3
       5 6 7
       1 4 2
Izlaz: ne
\end{test}
\end{minitest}

\begin{maxitest}
\begin{test}{Test 3}
Ulaz:  33       
Izlaz: Greska: neodgovarajuca dimenzija matrice.
\end{test}
\end{maxitest}
\end{Exercise}
\begin{Answer}[ref=324]
\includecode{resenja/03_Pokazivaci/324.c}
\end{Answer}


\begin{Exercise}[label=325]
Data je matrica dimenzije $n \times m$.
\begin{enumerate}
\item Napsiati funkciju koja učitava elemente matrice sa standardnog ulaza
\item Napsiati funkciju koja na standardni izlaz spiralno ispisuje elemente matrice.
\end{enumerate}
Napisati program koji testira napisane funkcije. Sa standardnog
ulaza učitati dimenzije matrice $n$ ($0 < n \leq 10$) i
$m$ ($0 < n \leq 10$), a zatim i elemente matrice (pozivom gore
napisane funkcije). Na standardni izlaz spiralno ispisati elemente
učitane matrice.

\begin{miditest}
\begin{test}{Test 1}
Ulaz:  3 3
       1 2 3
       4 5 6
       7 8 9       
Izlaz: 1 2 3 6 9 8 7 4 5
\end{test}
\end{miditest}
\begin{miditest}
\begin{test}{Test 2}
Ulaz:  3 4
       1 2 3 4
       5 6 7 8
       9 10 11 12	   
Izlaz: 1 2 3 4 8 12 11 10 9 5 6 7
\end{test}
\end{miditest}

\begin{maxitest}
\begin{test}{Test 3}
Ulaz:  11 4       	   
Izlaz: Greska: neodgovarajuce dimenzije matrice.
\end{test}
\end{maxitest}
\end{Exercise}
\begin{Answer}[ref=325]
\includecode{resenja/03_Pokazivaci/325.c}
\end{Answer}

\begin{Exercise}[label=327]
Napisati funkciju koja izračunava $k$-ti stepen kvadratne
matrice dimenzije $n$ ($0 < n \leq 32$). Napisati program koji
testira napisanu funkciju. Sa standardnog ulaza učitati
dimenziju celobrojne matrice $n$, elemente matrice i stepen
$k$ ($0 < k \leq 10$). Na standardni izlaz ispisati rezultat
primene napisane funkcije. Napomena: voditi računa da se
prilikom stepenovanja matrice izvrši što manji broj
množenja.

\begin{maxitest}
\begin{test}{Test 1}
Ulaz:  3
       1 2 3
       4 5 6
       7 8 9
       8
Izlaz: 510008400 626654232 743300064
       1154967822 1419124617 1683281412
       1799927244 2211595002 2623262760
\end{test}
\end{maxitest}
\end{Exercise}
\begin{Answer}[ref=327]
%\includecode{resenja/03_Pokazivaci/327.c}
\end{Answer}

\section{Dinamička alokacija memorije}

\begin{Exercise}[label=328]
Napisati program koji sa standardnog ulaza učitava dimenziju  
niza celih brojeva a zatim i njegove elemente. Ne praviti
nikakve pretpostavke o dimenziji niza. Na standardni izlaz 
ispisati ove brojeve u obrnutom poretku. 

\begin{minitest}
\begin{test}{Test 1}
Ulaz:  3
       1 -2 3
Izlaz: 3 -2 1
\end{test}
\end{minitest}
\begin{maxitest}
\begin{test}{Test 2}
Ulaz:  -1       
Izlaz: malloc(): neuspela alokacija memorije.
\end{test}
\end{maxitest}

\end{Exercise}
\begin{Answer}[ref=328]
\includecode{resenja/03_Pokazivaci/328.c}
\end{Answer}

\begin{Exercise}[label=330]
Napisati program koji sa standardnog ulaza učitava niz celih
brojeva. Brojevi se unose sve dok se ne unese nula. Ne praviti
nikakve pretpostavke o dimenziji niza. Na standardni izlaz
ispisati ovaj niz brojeva u obrnutom poretku. Zadatak uraditi na dva načina:
\begin{enumerate}
\item realokaciju memorije niza vršiti korišćenjem \kckod{malloc()} funkcije,
\item realokaciju memorije niza vršiti korišćenjem \kckod{realloc()} funkcije.
\end{enumerate}

\begin{minitest}
\begin{test}{Test 1}
Ulaz:  1 -2 3 -4 0
Izlaz: -4 3 -2 1
\end{test}
\end{minitest}
\begin{minitest}
\begin{test}{Test 2}
Ulaz:  0
Izlaz:
\end{test}
\end{minitest}
\end{Exercise}
\begin{Answer}[ref=330]
\includecode{resenja/03_Pokazivaci/330.c}
\end{Answer}

\begin{Exercise}[label=329]
Napisati funkciju koja kao rezultat vraća nisku koja se dobija
nadovezivanjem dve niske, bez promene njihovog sadržaja.
Napisati program koji testira rad napisane funkcije. Sa
standardnog ulaza učitati dve niske karaktera (pretpostaviti da 
niske nisu duže od $1000$ karaktera i da ne sadrže praznine). Na 
standardni izlaz ispisati nisku koja se dobija njihovim nadovezivanjem. 
Za rezultujuću nisku dinamički alocirati memoriju.

\begin{minitest}
\begin{test}{Test 1}
Ulaz:  Jedan Dva       
Izlaz: JedanDva
\end{test}
\end{minitest}

\end{Exercise}
\begin{Answer}[ref=329]
\includecode{resenja/03_Pokazivaci/329.c}
\end{Answer}

\begin{Exercise}[label=331]
Napisati program koji sa standardnog ulaza učitava matricu
celih brojeva. Prvo se učitavaju dimenzije matrice $n$ i
$m$ (ne praviti nikakve pretpostavke o njihovoj veličini),
a zatim i elementi matrice. Na standardni izlaz ispisati trag
matrice.

\begin{miditest}
\begin{test}{Test 1}
Ulaz:  2 3
       1.2 2.3 3.4
       4.5 5.6 6.7
Izlaz: 6.80
\end{test}
\end{miditest}
\end{Exercise}
\begin{Answer}[ref=331]
\includecode{resenja/03_Pokazivaci/331.c}
\end{Answer}

\begin{Exercise}[label=332]
Data je celobrojna matrica dimenzije $n \times m$ napisati:
\begin{enumerate}
\item Napisati funkciju koja vrši učitavanje matrice sa standardnog ulaza.
\item Napisati funkciju koja ispisuje elemente ispod glavne dijagonale matrice 
(uključujući i glavnu dijagonalu).
\end{enumerate}
Napisati program koji testira napisane funkcije. Sa standardnog
ulaza učitati $n$ i $m$ (ne praviti nikakve
pretpostavke o njihovoj veličini), zatim učitati elemente
matrice i na standardni izlaz ispisati elemente ispod glavne
dijagonale matrice.

\begin{miditest}
\begin{test}{Test 1}
Ulaz:  2 3
       1 -2 3
       -4 5 -6
Izlaz: 1
       -4 5
\end{test}
\end{miditest}
\end{Exercise}
\begin{Answer}[ref=332]
\includecode{resenja/03_Pokazivaci/332.c}
\end{Answer}

\begin{Exercise}[label=336]
Za zadatu matricu dimenzije $n \times m$ napisati funkciju koja
izračunava redni broj kolone matrice čiji je zbir
maksimalan. Napisati program koji testira ovu funkciju. Sa
standardnog ulaza učitati dimenzije matrice $n$ i
$m$ (ne praviti nikakve pretpostavke o njihovoj veličini), 
a zatim elemente matrice. Na standardni izlaz ispisati 
redni broj kolone matrice sa maksimalnim zbirom.

\begin{maxitest}
\begin{test}{Test 1}
Ulaz:  Unesite dimenzije matrice:
       2 3
       Unesite elemente matrice:
       1 2 3
       4 5 6
Izlaz: Kolona pod rednim brojem 3 ima najveci zbir.		   
\end{test}
\end{maxitest}
\end{Exercise}
\begin{Answer}[ref=336]
\includecode{resenja/03_Pokazivaci/336.c}
\end{Answer}

\begin{Exercise}[label=337]
Data je kvadratna realna matrica dimenzije $n$.
\begin{itemize}
\item Napisati funkciju koja izračunava zbir apsolutnih 
vrednosti matrice ispod sporedne dijagonale. 
\item Napisati funkciju koja menja sadržaj matrice tako 
što polovi elemente iznad glavne dijagonale, duplira 
elemente ispod glavne dijagonale, dok elemente na 
glavnoj dijagonali ostavlja nepromenjene.
\end{itemize}
Napisati program koji testira ove funkcije za matricu koja se
učitava iz datoteke čije se ime zadaje kao argument komandne linije. 
U datoteci se nalazi prvo dimenzija matrice, a zatim redom elementi matrice.

\begin{maxitest}
\begin{test}{Test 1}
Poziv: ./a.out matrica.txt
matrica.txt:  3
              1.1 -2.2 3.3
              -4.4 5.5 -6.6
              7.7 -8.8 9.9
Izlaz: Zbir apsolutnih vrednosti ispod sporedne dijagonale je 25.30.
       Transformisana matrica je:
       1.10 -1.10 1.65 
       -8.80 5.50 -3.30 
       15.40 -17.60 9.90 	   
\end{test}
\end{maxitest}
\end{Exercise}
\begin{Answer}[ref=337]
\includecode{resenja/03_Pokazivaci/337.c}
\end{Answer}

% \begin{Exercise}[label=338]
% Za zadatu kvadratnu realnu matricu dimenzije $n$ napisati funkciju
% koja izračunava zbir apsolutnih vrednosti matrice ispod
% sporedne dijagonale. Napisati program koji testira ovu funkciju za
% vrednosti koje se učitavaju iz datoteke čije se ime zadaje
% kao argument komandne linije. U datoteci se nalazi prvo dimenzija
% matrice, a zatim redom elementi matrice.
% \end{Exercise}
% \begin{Answer}[ref=338]
% \includecode{resenja/03_Pokazivaci/338.c}
% \end{Answer}

\begin{Exercise}[label=340]
Petar sakuplja sličice igrača za predstojeće Svetsko
prvenstvo u fudbalu. U datoteci ,,slicice.txt`` se nalaze
informacije o sličicama koje mu nedostaju u formatu:
\kckod{redni\_broj\_sličice
ime\_reprezentacije\_kojoj\_sličica\_pripada}. Pomozite Petru
da otkrije koliko mu sličica ukupno nedostaje, kao i da
pronađe ime reprezentacije čijih sličica ima najmanje.
Dobijene podatke ispisati na standardni izlaz. Napomena: za
realokaciju memorije koristiti \kckod{realloc()} funkciju.
\komentar{Jelena: treba dodati test primer.}
\end{Exercise}
\begin{Answer}[ref=340]
%\includecode{resenja/03_Pokazivaci/340.c}
\end{Answer}

\begin{Exercise}[label=341]
U datoteci ,,temena.txt`` se nalaze tačke koje predstavljaju
temena nekog $n$-tougla. Napisati program koji na osnovu
sadržaja datoteke na standardni izlaz ispisuje o kom
$n$-touglu je reč, a zatim i vrednosti njegovog obima i
površine. Pretpostavka je da će mnogougao biti konveksan.
\komentar{Jelena: treba dodati test primer.}
\end{Exercise}
\begin{Answer}[ref=341]
%\includecode{resenja/03_Pokazivaci/341.c}
\end{Answer}

\begin{Exercise}[label=342]
Napisati program koji na osnovu dve matrice dimenzija $m \times n$
formira matricu dimenzije $2 \cdot m \times n$ tako što
naizmenično kombinuje jednu vrstu prve matrice i jednu vrstu
druge matrice. Matirce su zapisane u datoteci ,,matrice.txt``. U
prvom redu se nalaze dimenzije matrica $m$ i $n$, u
narednih $m$ redova se nalaze vrste prve matrice, a u
narednih $m$ redova vrste druge matrice. Rezultujuću
matricu ispisati na standardni izlaz.
\komentar{Jelena: treba dodati test primer.}
\end{Exercise}
\begin{Answer}[ref=342]
%\includecode{resenja/03_Pokazivaci/342.c}
\end{Answer}

\begin{Exercise}[label=344]
Na ulazu se zadaje niz celih brojeva čiji se unos završava nulom.
Napisati funkciju koja od zadatog niza formira matricu tako da
prva vrsta odgovara unetom nizu, a svaka naredna se dobija
cikličkim pomeranjem elemenata niza za jednu poziciju ulevo.
Napisati program koji testira ovu funkciju. Sa standardnog ulaza
se prvo unosi dimenzija matrice, a zatim redom elementi matrice.
Rezultujuću matricu ispisati na standardni izlaz.
\komentar{Jelena: treba dodati test primer.}
\end{Exercise}
\begin{Answer}[ref=344]
%\includecode{resenja/03_Pokazivaci/344.c}
\end{Answer}

\section{Pokazivači na funkcije}

\begin{Exercise}[label=345]
Napisati program koji tabelarno štampa vrednosti proizvoljne realne funkcije sa jednim realnim
argumentom, odnosno izračunava i ispisuje
vrednosti date funkcije na diskretnoj ekvidistantnoj mreži od
$n$ tačaka intervala $[a, b]$. Realni brojevi
$a$ i $b$ ($a<b$) kao i ceo broj $n$
($n \geq 2$) se učitavaju sa standardnog ulaza. Ime funkcije
se zadaje kao argument komandne linije (\kckod{sin}, \kckod{cos}, \kckod{tan}, \kckod{atan},
\kckod{acos}, \kckod{asin}, \kckod{exp}, \kckod{log}, \kckod{log10}, \kckod{sqrt}, \kckod{floor}, \kckod{ceil}, \kckod{sqr}).

\begin{maxitest}
\begin{test}{Test 1}
Poziv: ./a.out sin
Ulaz: Unesite krajeve intervala:
      -0.5 1
      Koliko tacaka ima na ekvidistantnoj mrezi (ukljucujuci krajeve intervala)?
      4
Izlaz:      x        sin(x)
      -----------------------
      | -0.50000 | -0.47943 |
      |  0.00000 |  0.00000 |
      |  0.50000 |  0.47943 |
      |  1.00000 |  0.84147 |
      -----------------------
\end{test}
\end{maxitest}

\begin{maxitest}
\begin{test}{Test 2}
Poziv: ./a.out cos
Ulaz: Unesite krajeve intervala:
      0 2
      Koliko tacaka ima na ekvidistantnoj mrezi (ukljucujuci krajeve intervala)?
      4
Izlaz:      x        cos(x)
       -----------------------
       |  0.00000 |  1.00000 |
       |  0.66667 |  0.78589 |
       |  1.33333 |  0.23524 |
       |  2.00000 | -0.41615 |
       -----------------------
\end{test}
\end{maxitest}
\end{Exercise}
\begin{Answer}[ref=345]
\includecode{resenja/03_Pokazivaci/345.c}
\end{Answer}

\begin{Exercise}[label=346]
Napisati funkciju koja izračunava limes funkcije \kckod{f(x)} u   
tački $a$. Adresa funkcije \kckod{f} čiji se limes računa
se prenosi kao parametar funkciji za računanje limesa.  Limes se
računa sledećom aproksimacijom (vrednosti $n$ i $a$ uneti sa
standardnog ulaza kao i ime funkcije):
$$ lim_{x \rightarrow a} f(x) = lim_{n \rightarrow \infty} f(a + \frac{1}{n})$$

\begin{miditest}
\begin{test}{Test 1}
Ulaz:   tan 1.570795 10000
Izlaz:  -10134.5
\end{test}
\end{miditest}
\begin{miditest}
\begin{test}{Test 2}
Ulaz:   log 0 1000000
Izlaz:  -13.81551
\end{test}
\end{miditest}
\end{Exercise}
\begin{Answer}[ref=346]
%\includecode{resenja/03_Pokazivaci/346.c}
\end{Answer}

\begin{Exercise}[label=347]
Napisati funkciju koja određuje integral funkcije \kckod{f(x)} na
intervalu $[a, b]$.  Adresa funkcije \kckod{f} se prenosi kao
parametar. Integral se računa prema formuli:
$$ \int_{a}^{b} f(x) = h \cdot (\frac{f(a)+f(b)}{2} + \Sigma_{i=1}^{n}
f(a+i \cdot h))$$ 
Vrednost $h$ se izračunava po formuli $h = (b-a)/n$, dok se vrednosti $n$, $a$ i $b$ unose sa standardnog
ulaza kao i ime funkcije iz zaglavlja
\kckod{math.h}. Na standardni izlaz ispisati vrednost integrala.
\komentar{Jelena: treba dodati test primer.}
\end{Exercise}
\begin{Answer}[ref=347]
%\includecode{resenja/03_Pokazivaci/347.c}
\end{Answer}

\begin{Exercise}[label=348]
Napisati funkciju koja približno izračunava integral      
funkcije \kckod{f(x)} na intervalu $[a, b]$. Funkcija \kckod{f} se prosleđuje
kao parametar, a integral se procenjuje po Simpsonovoj formuli:
$$I = \frac{h}{3}\left(f(a) + 4 \sum_{i=1}^{n/2}f(a+(2i-1)h) + 2  
\sum_{i=1}^{n/2-1}f(a+2ih) + f(b)\right)$$ 
Granice intervala i $n$ su argumenti funkcije. Napisati program, koji kao
argumente komandne linije prihvata ime funkcije iz zaglavlja
\kckod{math.h}, krajeve intervala pretrage i $n$, a na standardni izlaz ispisuje vrednost odgovarajućeg integrala.
\komentar{Jelena: treba dodati test primer.}
\end{Exercise}
\begin{Answer}[ref=348]
%\includecode{resenja/03_Pokazivaci/348.c}
\end{Answer}
\section{Rešenja}
\shipoutAnswer
