\chapter{Pokazivači}

\section{Pokazivači i aritmetika sa pokazivačima}


\begin{Exercise}[label=301]
Za dati celobrojni niz dimenzije $n$, napisati funkcije koje
obrću njegove elemente:
\begin{enumerate}
\item korišćenjem indeksne sintakse,
\item korišćenjem pokazivačke sintakse.
\end{enumerate}
Napisati program koji testira napisane funkcije. Sa standardnog
ulaza učitati dimenziju niza $n$ ($n \leq 100$), a zatim
elemente niza. Prikazati sadržaj niza pre i posle poziva
funkcije za obrtanje elemenata niza.
\end{Exercise}
\begin{Answer}[ref=301]
\includecode{resenja/03_Pokazivaci/301.c}
\end{Answer}

\begin{Exercise}[label=302]
Dat je celobrojni niz dimenzije $n$. 
\begin{enumerate}
\item Napisati funkciju \kckod{zbir} koja izračunava zbir elemenata niza.
\item Napisati funkciju \kckod{proizvod} koja izračunava proizvod elemenata niza.
\item Napisati funkciju \kckod{min\_element}  koja izračunava najmanji elemenat niza.
\item Napisati funkciju \kckod{max\_element}  koja izračunava najveći elemenat niza.
\end{enumerate}
Napisati program koji testira napisane funkcije. Sa standardnog
ulaza učitati dimenziju $n$ 
($0 < n \leq 100$) celobrojong
niza, a zatim i elemente niza. Na standardni izlaz ispisati
rezultat primene svake od napisanih funkcija nad učitanim
nizom.
\end{Exercise}
\begin{Answer}[ref=302]
\includecode{resenja/03_Pokazivaci/302.c}
\end{Answer}


\begin{Exercise}[label=303]
Korišćenjem pokazivačke sintakse, napisati funkciju koja
vrednosti elemenata u prvoj polovini niza povećava za jedan, a u
drugoj polovini smanjuje za jedan. Ukoliko niz ima neparan broj
elemenata, onda vrednost srednjeg elementa niza ostaviti
nepromenjenim. Napisati program koji testira napisanu funkciju. Sa
standardnog ulaza učitati dimenziju $n$ ($0 \le n \leq 100$)
celobrojong niza, a zatim i elemente niza. Na standardni izlaz
ispisati rezultat primene svake od napisanih funkcija nad
učitanim nizom.
\end{Exercise}
\begin{Answer}[ref=303]
\includecode{resenja/03_Pokazivaci/303.c}
\end{Answer}

\begin{Exercise}[label=304]
Napisati program koji ispisuje broj prihvaćenih argumenata
komandne linije, a zatim i same argumenate kojima prethode njihovi
redni brojevi.
\end{Exercise}
\begin{Answer}[ref=304]
\includecode{resenja/03_Pokazivaci/304.c}
\end{Answer}

\begin{Exercise}[label=305]
Korišćenjem pokazivačke sintakse, napisati funkciju koja
za datu nisku ispituje da li je palindrom. Napisati program koji
vrši prebrojavanje argumenata komandne linije koji su
palindromi.

\begin{maxitest}
\begin{test}{Test 1}
Poziv:  ./a.out programiranje anavolimilovana topot ana anagram t
Izlaz:  4
\end{test}
\end{maxitest}
\end{Exercise}
\begin{Answer}[ref=305]
\includecode{resenja/03_Pokazivaci/305.c}
\end{Answer}

\begin{Exercise}[label=306]
Napisati program koji kao prvi argument komandne linije prihvata
putanju do datoteke za koju treba proveriti koliko reči ima
$n$ karaktera, gde se $n$ zadaje kao drugi argument
komandne linije. U zadatku ne koristiti ugrađene funkcije za
rad sa niskama, već implementirati svoje koristeći
pokazivačku sintaksu.

\begin{miditest}
\begin{test}{Test 1}
Poziv:    ./a.out fajl.txt 1
Datoteka: Ovo je sadrzaj datoteke i u 
          njoj ima reci koje imaju 
          1 karakter
Izlaz:    2
\end{test}
\end{miditest}

\end{Exercise}
\begin{Answer}[ref=306]
\includecode{resenja/03_Pokazivaci/306.c}
\end{Answer}

\begin{Exercise}[label=307]
Napisati program koji kao prvi argument komandne linije prihvata
putanju do datoteke za koju treba proveriti koliko reči ima
zadati sufiks (ili prefiks), koji se zadaje kao drugi argument
komandne linije. Program je neophodno pozvati sa jednom od opcija
\kckod{-s} ili \kckod{-p} u zavisnosti od čega treba proveriti
koliko reči ima zadati sufiks (ili prefiks). U zadatku ne
koristiti ugrađene funkcije za rad sa niskama, već
implementirati svoje koristeći pokazivačku sintaksu.

\begin{miditest}
\begin{test}{Test 1}
Poziv:    ./a.out fajl.txt ke -s
Datoteka: Ovo je sadrzaj datoteke 
          i u njoj ima reci koje se 
          zavrsavaju na ke
Izlaz:    2
\end{test}
\end{miditest}
\end{Exercise}
\begin{Answer}[ref=307]
\includecode{resenja/03_Pokazivaci/307.c}
\end{Answer}

\begin{Exercise}[label=308]
\komentar{Milena: Ovo bi trebalo da ide u deo o bitskim operatorima?}
Napisati rekurzivnu funkciju za određivanje
najveće cifre u oktalnom zapisu
neoznačenog celog broja korišćenjem bitskih operatora.
Uputstvo: binarne cifre grupisati u podgrupe od po tri cifre,
počev od bitova najmanje težine.

\begin{minitest}
\begin{test}{Test 1}
Ulaz:  5
Izlaz: 5
\end{test}
\end{minitest}
\begin{minitest}
\begin{test}{Test 2}
Ulaz:  125
Izlaz: 7
\end{test}
\end{minitest}
\begin{minitest}
\begin{test}{Test 3}
Ulaz:  8
Izlaz: 1
\end{test}
\end{minitest}

\begin{minitest}
\begin{test}{Test 4}
Ulaz:  10
Izlaz: 2
\end{test}
\end{minitest}
\end{Exercise}
\begin{Answer}[ref=308]
\includecode{resenja/03_Pokazivaci/308.c}
\end{Answer}

\begin{Exercise}[label=309]
\komentar{Milena: Ovo bi trebalo da ide u deo o bitskim operatorima?}
Napisati rekurzivnu funkciju za određivanje (dekadne vrednosti)
najveće cifre u heksadekadnom zapisu neoznačenog celog broja
korišćenjem bitskih operatora. Uputstvo: binarne cifre
grupisati u podgrupe od po četiri cifre, počev od bitova
najmanje težine.

\begin{minitest}
\begin{test}{Test 1}
Ulaz:  5
Izlaz: 5
\end{test}
\end{minitest}
\begin{minitest}
\begin{test}{Test 2}
Ulaz:  16
Izlaz: 1
\end{test}
\end{minitest}
\begin{minitest}
\begin{test}{Test 3}
Ulaz:  18
Izlaz: 2
\end{test}
\end{minitest}

\begin{minitest}
\begin{test}{Test 4}
Ulaz:  165
Izlaz: 10
\end{test}
\end{minitest}
\end{Exercise}
\begin{Answer}[ref=309]
\includecode{resenja/03_Pokazivaci/309.c}
\end{Answer}

\begin{Exercise}[label=310]
\komentar{Milena: Ovo bi trebalo da ide u pretraživanje/sortiranje?}
Napisati funkciju koja u rastuće sortiranom nizu celih brojeva
binarnom pretragom pronalazi prvi element veći od nule i kao
rezultat vraća njegovu poziciju u nizu. Ukoliko nema elemenata
većih od nule, funkcija kao rezultat vraća \argf{-1}. Napisati
program koji testira ovu funkciju za niz elemenata koji se
učitavaju sa standardnog ulaza. Niz neće biti duži od
$256$, i njegovi elementi se unose sve do kraja ulaza.

\begin{minitest}
\begin{test}{Test 1}
Ulaz:  -151 -44 5 
       12 13 15
Izlaz: 2
\end{test}
\end{minitest}
\begin{minitest}
\begin{test}{Test 2}
Ulaz:  -100 -15 -11 
       -8 -7 -5
Izlaz: -1
\end{test}
\end{minitest}
\begin{minitest}
\begin{test}{Test 3}
Ulaz:  -100 -15 0 13 
       155 124 258 
       315 516 7000
Izlaz: 3
\end{test}
\end{minitest}
\end{Exercise}
\begin{Answer}[ref=310]
\includecode{resenja/03_Pokazivaci/310.c}
\end{Answer}

\begin{Exercise}[label=311]
\komentar{Milena: Ovo bi trebalo da ide u pretraživanje/sortiranje?}
Napisati funkciju koja u opadajuće sortiranom nizu celih brojeva
binarnom pretragom pronalazi prvi element manji od nule i kao
rezultat vraća njegovu poziciju u nizu. Ukoliko nema elemenata
manjih od nule, funkcija kao rezultat vraća \argf{-1}. Napisati
program koji testira ovu funkciju za niz elemenata koji se
učitavaju sa standardnog ulaza. Niz neće biti duži od
$256$, i njegovi elementi se unose sve do kraja ulaza.

\begin{miditest}
\begin{test}{Test 1}
Ulaz:   151 44 5 -12 -13 -15
Izlaz: 3
\end{test}
\end{miditest}

\begin{miditest}
\begin{test}{Test 2}
Ulaz:  100 55 15 0 -15 -124 -155 
       -258 -315 -516 -7000
Izlaz: 4
\end{test}
\end{miditest}

\begin{miditest}
\begin{test}{Test 3}
Ulaz:   100 15 11 8 7 5 4 3 2
Izlaz: -1
\end{test}
\end{miditest}
\end{Exercise}
\begin{Answer}[ref=311]
\includecode{resenja/03_Pokazivaci/311.c}
\end{Answer}

\begin{Exercise}[label=312]
\komentar{Milena: Ovo bi trebalo da ide u pretraživanje/sortiranje?}
Struktura \kckod{Student} čuva podatke o broju
indeksa studenta i poenima sa kolokvijuma, pri čemu su
brojevi indeksa i poeni sa kolokvijuma celi brojevi. Napisati
program koji učitava podatke o studentima iz datoteke
,,kolokvijum.txt`` u kojoj se nalazi najviše $500$ zapisa
o studentima. Sortirati ovaj niz studenata po broju poena
opadajuće, a ako više studenata ima isti broj poena, onda po
broju indeksa rastuće. Ispisati sortiran niz studenata na
standardni izlaz.

\begin{miditest}
\begin{test}{Test 1}
Kolokvijum.txt:   20140015 25
                  20140115 24
                  20130250 3
                  20140001 4
                  20140038 25
Izlaz:            20140015 25
                  20140038 25
                  20140115 24
                  20140001 4
                  20130250 3
\end{test}
\end{miditest}

\begin{miditest}
\begin{test}{Test 2}
Kolokvijum.txt:   20140015 25
                  20110010 12
                  20140105 0
                  20120110 13
Izlaz:            20140015 25
                  20120110 13
                  20110010 12
                  20140105 0
\end{test}
\end{miditest}
\end{Exercise}
\begin{Answer}[ref=312]
\includecode{resenja/03_Pokazivaci/312.c}
\end{Answer}

\begin{Exercise}[label=313]
Napisati strukturu \kckod{Student} koja čuva podatke o broju
indeksa studenta i broju poena osvojenih na testu. Pretpostaviti
da su brojevi indeksa celi brojevi, a poeni sa testa realni
brojevi. Napisati program koji učitava podatke o studentima iz
datoteke ,,studenti.txt`` u kojoj se nalazi najviše $100$
zapisa o studentima. Sortirati ovaj niz studenata po broju poena
rastuće, a ako više studenata ima isti broj poena, onda po
broju indeksa opadajuće. Ispisati sortiran niz studenata na
standardni izlaz.

\begin{miditest}
\begin{test}{Test 1}
Kolokvijum.txt:   20140015 4.5
                  20130115 4.5
                  20140250 3.4
                  20110304 1.2
Izlaz:            20110304 1.2
                  20140250 3.4
                  20140015 4.5
                  20130115 4.5
\end{test}
\end{miditest}

\begin{miditest}
\begin{test}{Test 2}
Kolokvijum.txt:   20130015 9.5
                  20130010 9.5
                  20090103 0.5
                  20140005 10.0
                  20140120 1.3
                  20140038 2.5
Izlaz:            20090103 0.5
                  20140120 1.3
                  20140038 2.5
                  20130015 9.5
                  20130010 9.5
                  20140005 10.0
\end{test}
\end{miditest}
\end{Exercise}
\begin{Answer}[ref=313]
\includecode{resenja/03_Pokazivaci/313.c}
\end{Answer}

\section{Višedimenzioni nizovi}

\begin{Exercise}[label=314]
Data je kvadratna matrica dimenzije $n$.
\begin{enumerate}
\item Napisati funkciju koja izračunava trag matrice.
\item Napisati funkciju koja izračunava euklidsku normu matrice.
\item Napisati funkciju koja izračunava gornju vandijagonalnu normu matrice.
\end{enumerate}
Napisati program koji testira napisane funkcije. Sa standardnog
ulaza učitati dimanziju kvadratne matrice $n$
($0<n\leq 100$), a zatim i elemente matrice. Na standardni izlaz
ispisati rezultat primene svake od napisanih funkcija nad
učitanom matricom.
\end{Exercise}
\begin{Answer}[ref=314]
\includecode{resenja/03_Pokazivaci/314.c}
\end{Answer}

\begin{Exercise}[label=315]
Date su dve kvadratne matrice istih dimenzija $n$. 
\begin{enumerate}
\item Napisati funkciju koja proverava da li su matrice jednake.
\item Napisati funkciju koja izračunava zbir matrica.
\item Napisati funkciju koja izračunava proizvod matrica.
\end{enumerate}
Napisati program koji testira napisane funkcije. Sa standardnog
ulaza učitati dimanziju kvadratnih matrica $n$
($0<n\leq 100$), a zatim i elemente matrica. Na standardni izlaz
ispisati rezultat primene svake od napisanih funkcija nad
učitanim matricama.

\begin{minitest}

\begin{test}{Test 1}
Ulaz:   3
        1 2 3
        1 2 3
        1 2 3
        1 2 3
        1 2 3
        1 2 3
Izlaz:  da
        2 4 6
        2 4 6
        2 4 6
        6 12 18
        6 12 18
        6 12 18
\end{test}
\end{minitest}
\end{Exercise}
\begin{Answer}[ref=315]
\includecode{resenja/03_Pokazivaci/315.c}
\end{Answer}

%\begin{Exercise}[label=320]
%Napisati program u kom se definiše i inicijalizuje kvadratna matrica dimenzije \argf{5} a zatim se
%njeni elementi spiralno ispisuju na standardni izlaz.
%\end{Exercise}
%\begin{Answer}[ref=320]
%\includecode{resenja/03_Pokazivaci/320.c}
%\end{Answer}

\begin{Exercise}[label=322]
Relacija se može predstaviti kvadratnom matricom nula i
jedinica na sledeći način: dva elementa $i$ i $j$
su u relaciji ukoliko se u preseku $i$-te vrste i $j$-te
kolone matrice nalazi broj $1$, a nisu u relaciji ukoliko se
tu nalazi broj $0$. 
\begin{enumerate}
\item Napisati funkciju koja proverava da li je relacija zadata matricom refleksivna.
\item Napisati funkciju koja proverava da li je relacija zadata matricom simetrična.
\item Napisati funkciju koja proverava da li je relacija zadata matricom tranzitivna.
\item Napisati funkciju koja određuje refleksivno zatvorenje relacije (najmanju refleksivnu relaciju koja sadrži datu).
\item Napisati funkciju koja određuje simetrično zatvorenje relacije (najmanju simetričnu relaciju koja sadrži datu).
\item Napisati funkciju koja određuje refleksivno-tranzitivno zatvorenje relacije (najmanju refleksivnu i tranzitivnu relaciju
koja sadrži datu)(Napomena: koristiti Varšalov algoritam).
\end{enumerate}
Napisati program koji učitava matricu iz datoteke čije se ime zadaje kao prvi argument komandne linije.
U prvoj liniji datoteke nalazi se dimenzija matrice $n$ ($0<n\leq 64$), a potom i sami elementi matrice.
Na standardni izlaz ispisati rezultat testiranja napisanih funkcija.

\begin{maxitest}
\begin{test}{Test 1}
Datoteka:  4
           1 0 0 0
           0 1 1 0
           0 0 1 0
           0 0 0 0
Izlaz:     refleksivnost: ne
           simetricnost: ne
           tranzitivnost: da
           refleksivno zatvorenje:
           1 0 0 0
           0 1 1 0
           0 0 1 0
           0 0 0 1
           simetricno zatvorenje:
           1 0 0 0
           0 1 1 0
           0 1 1 0
           0 0 0 0
           refleksivno-tranzitivno zatvorenje:
           1 0 0 0
           0 1 1 0
           0 0 1 0
           0 0 0 1
\end{test}
\end{maxitest}
\end{Exercise}
\begin{Answer}[ref=322]
\includecode{resenja/03_Pokazivaci/322.c}
\end{Answer}

\begin{Exercise}[label=323]
Data je kvadratna matrica dimenzije $n$.
\begin{enumerate}
\item Napisati funkciju koja određuje najveći element matrice na sporednoj dijagonali.
\item Napisati funkciju koja određuje indeks kolone koja sadrži najmanji element matrice.
\item Napisati funkciju koja određuje indeks vrste koja sadrži najveći element matrice.
\item Napisati funkciju koja određuje broj negativnih elemenata matrice.
\end{enumerate}
Napisati program koji testira napisane funkcije. Sa standardnog
ulaza učitati elemente celobrojne kvadratne matrice čija
se dimenzija $n$ ($0<n\leq 32$) zadaje kao argument
komandne linije.

\begin{minitest}
\begin{test}{Test 1}
Poziv:  ./a.out 3
Ulaz:   1 2 3
        -4 -5 -6
        7 8 9
Izlaz:  7 2 2 3
\end{test}
\end{minitest}
\end{Exercise}
\begin{Answer}[ref=323]
\includecode{resenja/03_Pokazivaci/323.c}
\end{Answer}

\begin{Exercise}[label=324]
Napisati funkciju kojom se proverava da li je zadata kvadratna
matrica dimenzije $n$ ortonormirana. Matrica je ortonormirana
ako je skalarni proizvod svakog para različitih vrsta jednak
nuli, a skalarni proizvod vrste sa samom sobom jednak jedinici.
Napisati program koji testira napisanu funkciju. Sa standardnog
ulaza učitati dimenziju celobrojne kvadratne matrice $n$
($n \leq 32$), a zatim i njene elemente. Na standardni izlaz
ispisati rezultat primene napisane funkcije na učitanu
matricu.

\begin{minitest}
\begin{test}{Test 1}
Ulaz:  4
       1 0 0 0
       0 1 0 0
       0 0 1 0
       0 0 0 1
Izlaz: da
\end{test}
\end{minitest}
\begin{minitest}
\begin{test}{Test 2}
Ulaz:  3
       1 2 3
       5 6 7
       1 4 2
Izlaz: ne
\end{test}
\end{minitest}
\end{Exercise}
\begin{Answer}[ref=324]
\includecode{resenja/03_Pokazivaci/324.c}
\end{Answer}


\begin{Exercise}[label=325]
Data je matrica dimenzije $n \times m$.
\begin{enumerate}
\item Napsiati funkciju koja učitava elemente matrice sa standardnog ulaza
\item Napsiati funkciju koja na standardni izlaz spiralno ispisuje elemente matrice.
\end{enumerate}
Napisati program koji testira napisane funkcije. Sa standardnog
ulaza učitati dimenzije matrice $n$ ($0<n\leq 10$) i
$m$ ($0<n\leq 10$), a zatim i elemente matrice (pozivom gore
napisane funkcije). Na standardni izlaz spiralno ispisati elemente
učitane matrice.
\end{Exercise}
\begin{Answer}[ref=325]
\includecode{resenja/03_Pokazivaci/325.c}
\end{Answer}

\begin{Exercise}[label=326]
\komentar{Milena: Ovo bi trebalo da ide u pretraživanje/sortiranje?}
Napisati funkciju koja vrši leksikografsko opadajuće
sortiranje niza niski (pretpostaviti da ima najvise $1000$
niski, od kojih svaka ima najvise $30$ karaktera). Napisati
program koji testira rad napisane funkcije. Niske  učitati iz
datoteke ,,niske.txt`` (prva linija datoteke sadrži broj niski,
a svaka sledeća po jednu nisku). Na standardni izlaz ispisati
leksikografski opadajuće sortirane niske.
\end{Exercise}
\begin{Answer}[ref=326]
\includecode{resenja/03_Pokazivaci/326.c}
\end{Answer}

\begin{Exercise}[label=327]
Napisati funkciju koja izračunava $k$-ti stepen kvadratne
matrice dimenzije $n$ ($n \leq 32$). Napisati program koji
testira napisanu funkciju. Sa standardnog ulaza učitati
dimenziju celobrojne matrice $n$, elemente matrice i stepen
$k$ ($0<k\leq 10$). Na standardni izlaz ispisati rezultat
primene napisane funkcije. Napomena: voditi računa da se
prilikom stepenovanja matrice izvrši što manji broj
množenja.

\begin{miditest}
\begin{test}{Test 1}
Ulaz:  3
       1 2 3
       4 5 6
       7 8 9
       8
Izlaz: 510008400 626654232 743300064
       1154967822 1419124617 1683281412
       1799927244 2211595002 2623262760
\end{test}
\end{miditest}
\end{Exercise}
\begin{Answer}[ref=327]
\includecode{resenja/03_Pokazivaci/327.c}
\end{Answer}

\section{Dinamička alokacija memorije}

\begin{Exercise}[label=328]
Napisati program koji sa standardnog ulaza učitava niz celih
brojeva a zatim na standardni izlaz ispisuje ove brojeve u
obrnutom poretku. Ne praviti nikakavo ograničenje za dimenziju
niza.
\end{Exercise}
\begin{Answer}[ref=328]
\includecode{resenja/03_Pokazivaci/328.c}
\end{Answer}

\begin{Exercise}[label=329]
Napisati funkciju koja kao rezultat vraća nisku koja se dobija
nadovezivanjem dve niske, bez promene njihovog sadržaja.
Napisati program koji testira rad napisane funkcije. Sa
standardnog ulaza učitati dve niske karaktera. Na standardni
izlaz ispisati nisku koja se dobija njihovim nadovezivanjem.
\end{Exercise}
\begin{Answer}[ref=329]
\includecode{resenja/03_Pokazivaci/329.c}
\end{Answer}

\begin{Exercise}[label=330]
Napisati program koji sa standardnog ulaza učitava niz celih
brojeva. Brojevi se unose sve dok se ne unese nula. Ne praviti
nikakve pretpostavke o dimenziji niza. Na standardni izlaz
ispisati ovaj niz brojeva u obrnutom poretku. Zadatak uraditi na dva načina:
\begin{enumerate}
\item realokaciju memorije niza vršiti korišćenjem \kckod{malloc()} funkcije,
\item realokaciju memorije niza vršiti korišćenjem \kckod{realloc()} funkcije.
\end{enumerate}
\end{Exercise}
\begin{Answer}[ref=330]
\includecode{resenja/03_Pokazivaci/330.c}
\end{Answer}

\begin{Exercise}[label=331]
Napisati program koji sa standardnog ulaza učitava matricu
celih brojeva. Prvo se učitavaju dimenzije matrice $n$ i
$m$ (ne praviti nikakve pretpostavke o njihovoj veličini),
a zatim i elementi matrice. Na standardni izlaz ispisati trag
matrice.
\end{Exercise}
\begin{Answer}[ref=331]
\includecode{resenja/03_Pokazivaci/331.c}
\end{Answer}

\begin{Exercise}[label=332]
Data je celobrojna matrica dimenzije $n \times m$ napisati:
\begin{enumerate}
\item Napisati funkciju koja vrši učitavanje matrice sa standardnog ulaza.
\item Napisati funkciju koja ispisuje elemente ispod glavne dijagonale matrice.
\end{enumerate}
Napisati program koji testira napisane funkcije. Sa standardnog
ulaza učitati $n$ i $m$ (ne praviti nikakve
pretpostavke o njihovoj veličini), zatim učitati elemente
matrice i na standardni izlaz ispisati elemente ispod glavne
dijagonale matrice.

\end{Exercise}
\begin{Answer}[ref=332]
\includecode{resenja/03_Pokazivaci/332.c}
\end{Answer}

\begin{Exercise}[label=333]
\komentar{Milena: Ovo bi trebalo da ide u pretraživanje/sortiranje?}
U datoteci ,,pesme.txt`` nalaze se informacije o gledanosti pesama
na Youtube-u. 
Format datoteke sa informacijama je sledeći:
\begin{itemize}
\item U prvoj liniji datoteke može i ne mora da se nalazi ukupan broj pesama prisutnih u datoteci.
\item Svaki naredni red datoteke sadrži
informacije o gledanosti pesama u formatu \kckod{izvodjac - naslov, brojGledanja}. 
\end{itemize}
Napisati program koji učitava
informacije o pesmama i vrši sortiranje pesama u zavisnosti od
argumenata komandne linije na sledeći način:
\begin{itemize}
\item nema opcija, sortiranje se vrši po broju gledanja;
\item prisutna je opcija \kckod{-i}, sortiranje se vrši po imenima izvođača;
\item prisutna je opcija \kckod{-n}, sortiranje se vrši po naslovu pesama.
\end{itemize}
Na standardni izlaz ispisati informacije o pesmama sortirane na opisan način.

\begin{enumerate}
\item Uraditi zadatak uz pretpostavku da 
je maksimalna dužina imena izvođača $20$ karaktera, a imena naslova pesme $50$ kraraktera.
\item Uraditi zadatak bez pravljenja pretpostavki o maksimalnoj dužini imena izvođača i naslova pesme.
\end{enumerate}


%Napomena: ukoliko se u datoteci ,,pesme.txt`` ne nalaze informacije o ukupnom broju pesama, koristiti realociranje memorije.

\begin{miditest}
\begin{test}{Test 1}
Poziv: ./a.out
Datoteka:  5
           Ana - Nebo, 2342
           Laza - Oblaci, 29
           Pera - Ptice, 327
           Jelena - Sunce, 92321
           Mika - Kisa, 5341
Izlaz:     Jelena - Sunce, 92321
           Mika - Kisa, 5341
           Ana - Nebo, 2342
           Pera - Ptice, 327
           Laza - Oblaci, 29
\end{test}
\end{miditest}
\end{Exercise}
\begin{Answer}[ref=333]
\includecode{resenja/03_Pokazivaci/333.c}
\end{Answer}


\begin{Exercise}[label=336]
Za zadatu matricu dimenzije $n \times m$ napisati funkciju koja
izračunava redni broj kolone matrice čiji je zbir
maksimalan. Napisati program koji testira ovu funkciju. Sa
standardnog ulaza učitati dimenzije matrice $n$ i
$m$, a zatim elemente matrice. 

\end{Exercise}
\begin{Answer}[ref=336]
\includecode{resenja/03_Pokazivaci/336.c}
\end{Answer}

\begin{Exercise}[label=337]
Za zadatu kvadratnu matricu dimenzije $n$ napisati funkciju
koja menja njen sadržaj tako što polovi elemente iznad
glavne dijagonale, duplira elemente ispod glavne dijagonale, dok
elemente na glavnoj dijagonali ostavlja nepromenjene. Napisati
program koji testira ovu funkciju za vrednosti koje se
učitavaju iz datoteke ,,matrica.txt``. U datoteci se nalazi
prvo dimenzija matrice, a zatim redom elementi matrice.
\end{Exercise}
\begin{Answer}[ref=337]
\includecode{resenja/03_Pokazivaci/337.c}
\end{Answer}

\begin{Exercise}[label=338]
Za zadatu kvadratnu matricu dimenzije $n$ napisati funkciju
koja izračunava zbir apsolutnih vrednosti matrice ispod
sporedne dijagonale. Napisati program koji testira ovu funkciju za
vrednosti koje se učitavaju iz datoteke čije se ime zadaje
kao argument komandne linije. U datoteci se nalazi prvo dimenzija
matrice, a zatim redom elementi matrice.
\end{Exercise}
\begin{Answer}[ref=338]
\includecode{resenja/03_Pokazivaci/338.c}
\end{Answer}

\begin{Exercise}[label=339]
Za zadatu kvadratnu matricu dimenzije $n$ napisati funkciju
koja vrši sortiranje vrsta matrice, rastuće na osnovu sume
elemenata u vrsti. Napisati program koji testira ovu funkciju. Sa
standardnog ulaza se prvo unosi dimenzija matrice, a zatim redom
elementi matrice. Rezultujuću matricu ispisati na standardni
izlaz.
\end{Exercise}
\begin{Answer}[ref=339]
\includecode{resenja/03_Pokazivaci/339.c}
\end{Answer}

\begin{Exercise}[label=340]
Petar sakuplja sličice igrača za predstojeće Svetsko
prvenstvo u fudbalu. U datoteci ,,slicice.txt`` se nalaze
informacije o sličicama koje mu nedostaju u formatu:
\kckod{redni\_broj\_sličice
ime\_reprezentacije\_kojoj\_sličica\_pripada}. Pomozite Petru
da otkrije koliko mu sličica ukupno nedostaje, kao i da
pronađe ime reprezentacije čijih sličica ima najmanje.
Dobijene podatke ispisati na standardni izlaz. Napomena: za
realokaciju memorije koristiti \kckod{realloc()} funkciju.
\end{Exercise}
\begin{Answer}[ref=340]
\includecode{resenja/03_Pokazivaci/340.c}
\end{Answer}

\begin{Exercise}[label=341]
U datoteci ,,temena.txt`` se nalaze tačke koje predstavljaju
temena nekog $n$-tougla. Napisati program koji na osnovu
sadržaja datoteke na standardni izlaz ispisuje o kom
$n$-touglu je reč, a zatim i vrednosti njegovog obima i
površine. Pretpostavka je da će mnogougao biti konveksan.
\end{Exercise}
\begin{Answer}[ref=341]
\includecode{resenja/03_Pokazivaci/341.c}
\end{Answer}

\begin{Exercise}[label=342]
Napisati program koji na osnovu dve matrice dimenzija $m \times n$
formira matricu dimenzije $2 \cdot m \times n$ tako što
naizmenično kombinuje jednu vrstu prve matrice i jednu vrstu
druge matrice. Matirce su zapisane u datoteci ,,matrice.txt``. U
prvom redu se nalaze dimenzije matrica $m$ i $n$, u
narednih $m$ redova se nalaze vrste prve matrice, a u
narednih $m$ redova vrste druge matrice. Rezultujuću
matricu ispisati na standardni izlaz.
\end{Exercise}
\begin{Answer}[ref=342]
\includecode{resenja/03_Pokazivaci/342.c}
\end{Answer}

\begin{Exercise}[label=343]
Za zadatu kvadratnu matricu dimenzije $n$ napisati funkciju
koja sortira kolone matrice, opadajuće, na osnovu vrednosti
prvog elementa u koloni.

Napisati program koji testira ovu funkciju. Sa standardnog ulaza
se prvo unosi dimenzija matrice, a zatim redom elementi matrice.
Rezultujuću matricu ispisati na standardni izlaz.
\end{Exercise}
\begin{Answer}[ref=343]
\includecode{resenja/03_Pokazivaci/343.c}
\end{Answer}

\begin{Exercise}[label=344]
Na ulazu se zadaje niz celih brojeva čiji se unos završava nulom.
Napisati funkciju koja od zadatog niza formira matricu tako da
prva vrsta odgovara unetom nizu, a svaka naredna se dobija
cikličkim pomeranjem elemenata niza za jednu poziciju ulevo.

Napisati program koji testira ovu funkciju. Sa standardnog ulaza
se prvo unosi dimenzija matrice, a zatim redom elementi matrice.
Rezultujuću matricu ispisati na standardni izlaz.
\end{Exercise}
\begin{Answer}[ref=344]
\includecode{resenja/03_Pokazivaci/344.c}
\end{Answer}

\section{Pokazivači na funkcije}

\begin{Exercise}[label=345]
Napisati program koji tabelarno štampa vrednosti proizvoljne realne funkcije sa jednim realnim
argumentom, odnosno izračunava i ispisuje
vrednosti date funkcije na diskretnoj ekvidistantnoj mreži od
$n$ tačaka intervala $[a, b]$. Realni brojevi
$a$ i $b$ ($a<b$) kao i ceo broj $n$
($n \geq 2$) se učitavaju sa standardnog ulaza. Ime funkcije
se zadaje kao argument komandne linije (\kckod{sin}, \kckod{cos}, \kckod{tan}, \kckod{atan},
\kckod{acos}, \kckod{asin}, \kckod{exp}, \kckod{log}, \kckod{log10}, \kckod{sqrt}, \kckod{floor}, \kckod{ceil}, \kckod{sqr}).
\end{Exercise}
\begin{Answer}[ref=345]
\includecode{resenja/03_Pokazivaci/345.c}
\end{Answer}


\begin{Exercise}[label=346]
Napisati funkciju koja izračunava limes funkcije \kckod{f(x)} u   
tački $a$. Adresa funkcije \kckod{f} čiji se limes računa
se prenosi kao parametar funkciji za računanje limesa.  Limes se
računa sledećom aproksimacijom (vrednosti $n$ i $a$ uneti sa
standardnog ulaza kao i ime funkcije):
$$ lim_{x \rightarrow a} f(x) = lim_{n \rightarrow \infty} f(a + \frac{1}{n})$$
\begin{miditest}
\begin{test}{Test 1}
Ulaz:   tan 1.570795 10000
Izlaz:  -10134.5
\end{test}
\end{miditest}
\begin{miditest}
\begin{test}{Test 2}
Ulaz:   log 0 1000000
Izlaz:  -13.81551
\end{test}
\end{miditest}
\end{Exercise}
\begin{Answer}[ref=346]
\includecode{resenja/03_Pokazivaci/346.c}
\end{Answer}

\begin{Exercise}[label=347]
Napisati funkciju koja određuje integral funkcije \kckod{f(x)} na
intervalu $[a, b]$.  Adresa funkcije \kckod{f} se prenosi kao
parametar. Integral se računa prema formuli:
$$ \int_{a}^{b} f(x) = h \cdot (\frac{f(a)+f(b)}{2} + \Sigma_{i=1}^{n}
f(a+i \cdot h))$$ 
Vrednost $h$ se izračunava po formuli $h = (b-a)/n$, dok se vrednosti $n$, $a$ i $b$ unose sa standardnog
ulaza kao i ime funkcije iz zaglavlja
\kckod{math.h}. Na standardni izlaz ispisati vrednost integrala.
\end{Exercise}
\begin{Answer}[ref=347]
\includecode{resenja/03_Pokazivaci/347.c}
\end{Answer}

\begin{Exercise}[label=348]
Napisati funkciju koja približno izračunava integral      
funkcije \kckod{f(x)} na intervalu $[a, b]$. Funkcija \kckod{f} se prosleđuje
kao parametar, a integral se procenjuje po Simpsonovoj formuli:
$$I = \frac{h}{3}\left(f(a) + 4 \sum_{i=1}^{n/2}f(a+(2i-1)h) + 2  
\sum_{i=1}^{n/2-1}f(a+2ih) + f(b)\right)$$ 
Granice intervala i $n$ su argumenti funkcije. Napisati program, koji kao
argumente komandne linije prihvata ime funkcije iz zaglavlja
\kckod{math.h}, krajeve intervala pretrage i $n$, a na standardni izlaz ispisuje vrednost odgovarajućeg integrala.
\end{Exercise}
\begin{Answer}[ref=348]
\includecode{resenja/03_Pokazivaci/348.c}
\end{Answer}


\section{Rešenja}
\shipoutAnswer
