 \chapter{Pokazivači}

 \section{Pokazivačka aritmetika}

\begin{Exercise}[label=301]
Za dati celobrojni niz dimenzije $n$, napisati funkciju koja obrće njegove elemente:
\begin{enumerate}
\item korišćenjem indeksne sintakse,
\item korišćenjem pokazivačke sintakse.
\end{enumerate}
Napisati program koji testira napisane funkcije. Sa standardnog ulaza učitati dimenziju 
niza $n$ ($0 < n \leq 100$), a zatim elemente niza. Pozvati funkciju koja obrće njegove
elemente korišćenjem indeksne sintakse i prikazati sadržaj niza. Nakon toga pozvati funkciju koja obrće 
njegove elemente korišćenjem pokazivačke sintakse i prikazati sadržaj niza.

\begin{miditest}
\begin{upotreba}{1}
#\naslovInt#
#\izlaz{Unesite dimenziju niza:} \ulaz{3}#
#\izlaz{Unesite elemente niza:}#
#\ulaz{1 -2 3}#
#\izlaz{Nakon obrtanja elemenata, niz je:}#
#\izlaz{3 -2 1}#
#\izlaz{Nakon ponovnog obrtanja elemenata,}#
#\izlaz{niz je:}#
#\izlaz{3 -2 1}#
\end{upotreba}
\end{miditest}
\begin{miditest}
\begin{upotreba}{2}
#\naslovInt#
#\izlaz{Unesite dimenziju niza: } \ulaz{0}#
#\izlaz{Greska: neodgovarajuca dimenzija niza.}#
\end{upotreba}
\end{miditest}

\linkresenje{301}
\end{Exercise}
\begin{Answer}[ref=301]
\includecode{resenja/03_Pokazivaci/301.c}
\end{Answer}

\begin{Exercise}[label=302]
Dat je niz realnih brojeva dimenzije $n$. Korišćenjem pokazivačke sintakse, napisati: 
\begin{enumerate}
\item funkciju \kckod{zbir} koja izračunava zbir elemenata niza,
\item funkciju \kckod{proizvod} koja izračunava proizvod elemenata niza,
\item funkciju \kckod{min\_element}  koja izračunava najmanji elemenat niza,
\item funkciju \kckod{max\_element}  koja izračunava najveći elemenat niza.
\end{enumerate}
Napisati program koji testira napisane funkcije. Sa standardnog ulaza 
učitati dimenziju $n$ ($0 < n \leq 100$) realnog niza, a zatim i 
elemente niza. Na standardni izlaz ispisati zbir, proizvod, 
minimalni i maksimalni element učitanog niza.

\begin{miditest}
\begin{upotreba}{1}
#\naslovInt#
#\izlaz{Unesite dimenziju niza:} \ulaz{3}#
#\izlaz{Unesite elemente niza:}#
#\ulaz{-1.1 2.2 3.3}#
#\izlaz{Zbir elemenata niza je 4.400.}#
#\izlaz{Proizvod elemenata niza je -7.986}#
#\izlaz{Minimalni element niza je -1.100}#
#\izlaz{Maksimalni element niza je 3.300}#
\end{upotreba}
\end{miditest}
\begin{miditest}
\begin{upotreba}{2}
#\naslovInt#
#\izlaz{Unesite dimenziju niza:} \ulaz{5}#
#\izlaz{Unesite elemente niza:}#
#\ulaz{1.2 3.4 0.0 -5.4 2.1}#
#\izlaz{Zbir elemenata niza je 1.300.}#
#\izlaz{Proizvod elemenata niza je -0.000.}#
#\izlaz{Minimalni element niza je -5.400.}#
#\izlaz{Maksimalni element niza je 3.400.}#
\end{upotreba}
\end{miditest}

\linkresenje{302}
\end{Exercise}
\begin{Answer}[ref=302]
\includecode{resenja/03_Pokazivaci/302.c}
\end{Answer}

\begin{Exercise}[label=303]
Korišćenjem pokazivačke sintakse, napisati funkciju koja
vrednosti elemenata u prvoj polovini niza povećava za jedan, a u
drugoj polovini smanjuje za jedan. Ukoliko niz ima neparan broj
elemenata, onda vrednost srednjeg elementa niza ostaviti
nepromenjenim. Napisati program koji testira napisanu funkciju. Sa
standardnog ulaza učitati dimenziju $n$ ($0 < n \leq 100$)
celobrojnog niza, a zatim i elemente niza. Na standardni izlaz
ispisati rezultat primene napisane funkcije nad učitanim nizom.

%\noindent
\begin{miditest}
\begin{upotreba}{1}
#\naslovInt#
#\izlaz{Unesite dimenziju niza:} \ulaz{5}#
#\izlaz{Unesite elemente niza:}#
#\ulaz{1 2 3 4 5}#
#\izlaz{Transformisan niz je:}#
#\izlaz{2 3 3 3 4}#
\end{upotreba}
\end{miditest}
\begin{miditest}
\begin{upotreba}{2}
#\naslovInt#
#\izlaz{Unesite dimenziju niza:} \ulaz{4}#
#\izlaz{Unesite elemente niza:}#
#\ulaz{4 -3 2 -1}#
#\izlaz{Transformisan niz je:}#
#\izlaz{5 -2 1 -2}#
\end{upotreba}
\end{miditest}

\begin{miditest}
\begin{upotreba}{3}
#\naslovInt#
#\izlaz{Unesite dimenziju niza:} \ulaz{0}#
#\izlaz{Greska: neodgovarajuca dimenzija niza.}#
\end{upotreba}
\end{miditest}
\begin{miditest}
\begin{upotreba}{4}
#\naslovInt#
#\izlaz{Unesite dimenziju niza:} \ulaz{101}#
#\izlaz{Greska: neodgovarajuca dimenzija niza.}#
\end{upotreba}
\end{miditest}

\linkresenje{303}
\end{Exercise}
\begin{Answer}[ref=303]
\includecode{resenja/03_Pokazivaci/303.c}
\end{Answer}

\begin{Exercise}[label=304]
Napisati program koji ispisuje broj prihvaćenih argumenata
komandne linije, a zatim i same argumenate kojima prethode njihovi
redni brojevi. Nakon toga ispisati prve karaktere svakog od argumenata.
Zadatak rešiti:
\begin{enumerate}
\item korišćenjem indeksne sintakse,
\item korišćenjem pokazivačke sintakse.
\end{enumerate} 
Od korisnika sa ulaza tražiti da izabere koje od ova dva
rešenja treba koristiti prilikom ispisa.

\begin{miditest}
\begin{upotreba}{1}
#\poziv{./a.out prvi 2. treci -4}#

#\naslovInt#
#\izlaz{Broj argumenata komandne linije je 5.}#
#\izlaz{Kako zelite da ispisete argumente,}# 
#\izlaz{koriscenjem indeksne ili pokazivacke}# 
#\izlaz{sintakse (I ili P)?} \ulaz{I}#
#\izlaz{Argumenti komandne linije su:}#
#\izlaz{0 ./a.out}#
#\izlaz{1 prvi}#
#\izlaz{2 2.}#
#\izlaz{3 treci}#
#\izlaz{4 -4}#
#\izlaz{Pocetna slova argumenata komandne linije:}#
#\izlaz{. p 2 t -}#
\end{upotreba}
\end{miditest}
\begin{miditest}
\begin{upotreba}{2}
#\poziv{./a.out}#

#\naslovInt#
#\izlaz{Broj argumenata komandne linije je 1.}#
#\izlaz{Kako zelite da ispisete argumente,}# 
#\izlaz{koriscenjem indeksne ili pokazivacke}# 
#\izlaz{sintakse (I ili P)?} \ulaz{P}#
#\izlaz{Argumenti komandne linije su:}#
#\izlaz{0 ./a.out}#
#\izlaz{Pocetna slova argumenata komandne linije:}#
#\izlaz{.}#
\end{upotreba}
\end{miditest}

\linkresenje{304}
\end{Exercise}
\begin{Answer}[ref=304]
\includecode{resenja/03_Pokazivaci/304.c}
\end{Answer}

\begin{Exercise}[label=305]
Korišćenjem pokazivačke sintakse, napisati funkciju koja
za datu nisku ispituje da li je palindrom. Napisati program koji
vrši prebrojavanje argumenata komandne linije koji su
palindromi.

%\noindent
\begin{miditest}
\begin{upotreba}{1}
#\poziv{./a.out a b 11 212}#

#\naslovInt#
#\izlaz{Broj argumenata komandne linije}#
#\izlaz{koji su palindromi je 4.}#
\end{upotreba}
\end{miditest}
\begin{miditest}
\begin{upotreba}{2}
#\poziv{./a.out}#

#\naslovInt#
#\izlaz{Broj argumenata komandne linije koji}#
#\izlaz{koji su palindromi je 0.}#
\end{upotreba}
\end{miditest}

\linkresenje{305}
\end{Exercise}
\begin{Answer}[ref=305]
\includecode{resenja/03_Pokazivaci/305.c}
\end{Answer}

\begin{Exercise}[label=306]
Napisati program koji kao prvi argument komandne linije prihvata
putanju do datoteke za koju treba proveriti koliko reči ima
$n$ karaktera, gde se $n$ zadaje kao drugi argument
komandne linije. Smatrati da reč ne sadrži više od $100$ karaktera.
U zadatku ne koristiti ugrađene funkcije za rad sa niskama, 
već implementirati svoje koristeći pokazivačku sintaksu.

%\noindent
\begin{miditest}
\begin{upotreba}{1}
#\poziv{./a.out ulaz.txt 1}#

#\naslovDat{ulaz.txt}#
#\datoteka{Ovo je sadrzaj datoteke i u njoj ima }#
#\datoteka{reci koje imaju 1 karakter}#

#\naslovInt#
#\izlaz{Broj reci ciji je broj karaktera 1 je 3.}#
\end{upotreba}
\end{miditest}
\begin{miditest}
\begin{upotreba}{2}
#\poziv{./a.out ulaz.txt}#

#\naslovDat{ulaz.txt}#
#\datoteka{Ovo je sadrzaj datoteke i u njoj ima }#
#\datoteka{reci koje imaju 1 karakter}#

#\naslovInt#
#\izlaz{Greska: Nedovoljan broj argumenata}#
#\izlaz{komandne linije.}#
#\izlaz{Program se poziva sa }#
#\izlaz{./a.out ime\_dat br\_karaktera.}#
\end{upotreba}
\end{miditest}

\begin{miditest}
\begin{upotreba}{3}
#\poziv{./a.out ulaz.txt 2}#

#\naslovDat{Datoteka ulaz.txt ne postoji}#

#\naslovInt#
#\izlaz{Greska: Neuspesno otvaranje datoteke ulaz.txt.}#
\end{upotreba}
\end{miditest}

\linkresenje{306}
\end{Exercise}
\begin{Answer}[ref=306]
\includecode{resenja/03_Pokazivaci/306.c}
\end{Answer}

\begin{Exercise}[label=307]
Napisati program koji kao prvi argument komandne linije prihvata
putanju do datoteke za koju treba proveriti koliko reči ima
zadati sufiks (ili prefiks), koji se zadaje kao drugi argument
komandne linije. Smatrati da reč ne sadrži više od $100$ karaktera.
Program je neophodno pozvati sa jednom od opcija
\kckod{-s} ili \kckod{-p} u zavisnosti od čega treba proveriti
koliko reči ima zadati sufiks (ili prefiks). U zadatku ne
koristiti ugrađene funkcije za rad sa niskama, već
implementirati svoje koristeći pokazivačku sintaksu.

%\noindent
\begin{miditest}
\begin{upotreba}{1}
#\poziv{./a.out ulaz.txt ke -s}#

#\naslovDat{ulaz.txt}#
#\datoteka{Ovo je sadrzaj datoteke i u njoj ima reci}#
#\datoteka{koje se zavrsavaju na ke}#

#\naslovInt#
#\izlaz{Broj reci koje se zavrsavaju na ke je 2.}#
\end{upotreba}
\end{miditest}
\begin{miditest}
\begin{upotreba}{2}
#\poziv{./a.out ulaz.txt sa -p}#

#\naslovDat{ulaz.txt}#
#\datoteka{Ovo je sadrzaj datoteke i u njoj ima reci }#
#\datoteka{koje pocinju sa sa}#

#\naslovInt#
#\izlaz{Broj reci koje pocinju na sa je 3.}#
\end{upotreba}
\end{miditest}

\begin{miditest}
\begin{upotreba}{3}
#\poziv{./a.out ulaz.txt sa -p}#

#\naslovDat{Datoteka ulaz.txt ne postoji}#

#\naslovInt#
#\izlaz{Greska: Neuspesno otvaranje}#
#\izlaz{datoteke ulaz.txt.}#
\end{upotreba}
\end{miditest}
\begin{miditest}
\begin{upotreba}{4}
#\poziv{./a.out ulaz.txt}#

#\naslovDat{ulaz.txt}#
#\datoteka{Ovo je sadrzaj ulaza.}#

#\naslovInt#
#\izlaz{Greska: Nedovoljan broj argumenata }#
#\izlaz{komandne linije.}#
#\izlaz{Program se poziva sa }#
#\izlaz{./a.out ime\_dat suf/pref -s/-p.}#
\end{upotreba}
\end{miditest}

\linkresenje{307}
\end{Exercise}
\begin{Answer}[ref=307]
\includecode{resenja/03_Pokazivaci/307.c}
\end{Answer}

\section{Višedimenzioni nizovi}

\begin{Exercise}[label=314]
Data je kvadratna matrica dimenzije $n$.
\begin{enumerate}
\item Napisati funkciju koja izračunava trag matrice (sumu elemenata na glavnoj dijagonali).
\item Napisati funkciju koja izračunava euklidsku normu matrice (koren sume kvadrata svih elemenata).
\item Napisati funkciju koja izračunava gornju vandijagonalnu normu matrice (sumu apsolutnih vrednosti elemenata iznad glavne dijagonale).
\end{enumerate}
Napisati program koji testira napisane funkcije. Sa standardnog
ulaza učitati dimanziju kvadratne matrice $n$
($0 < n \leq 100$), a zatim i elemente matrice. Na standardni izlaz
ispisati učitanu matricu, a zatim trag, euklidsku normu i vandijagonalnu normu 
učitane matrice.

%\noindent
\begin{miditest}
\begin{upotreba}{1}
#\naslovInt#
#\izlaz{Unesite dimenziju matrice:} \ulaz{3}#
#\izlaz{Unesite elemente matrice, vrstu po vrstu:}#
#\ulaz{1 -2 3}#
#\ulaz{4 -5 6}#
#\ulaz{7 -8 9}#
#\izlaz{Trag matrice je 5.}#
#\izlaz{Euklidska norma matrice je 16.88.}#
#\izlaz{Vandijagonalna norma matrice je 11.}#
\end{upotreba}
\end{miditest}
\begin{miditest}
\begin{upotreba}{2}
#\naslovInt#
#\izlaz{Unesite dimenziju matrice:} \ulaz{0}#
#\izlaz{Greska: neodgovarajuca dimenzija matrice. }#
\end{upotreba}
\end{miditest}

\linkresenje{314}
\end{Exercise}
\begin{Answer}[ref=314]
\includecode{resenja/03_Pokazivaci/314.c}
\end{Answer}

\begin{Exercise}[label=315]
Date su dve kvadratne matrice istih dimenzija $n$. 
\begin{enumerate}
\item Napisati funkciju koja proverava da li su matrice jednake.
\item Napisati funkciju koja izračunava zbir matrica.
\item Napisati funkciju koja izračunava proizvod matrica.
\end{enumerate}
Napisati program koji testira napisane funkcije. Sa standardnog
ulaza učitati dimenziju kvadratnih matrica $n$
($0 < n \leq 100$), a zatim i elemente matrica. Na standardni izlaz
ispisati da li su matrice jednake, a zatim ispisati
zbir i proizvod učitanih matrica.

\begin{miditest}
\begin{upotreba}{1}
#\naslovInt#
#\izlaz{Unesite dimenziju matrica:} \ulaz{3}#
#\izlaz{Unesite elemente prve matrice, vrstu po vrstu:}#
#\ulaz{1 2 3}#
#\ulaz{1 2 3}#
#\ulaz{1 2 3}#
#\izlaz{Unesite elemente druge matrice, vrstu po vrstu:}#
#\ulaz{1 2 3}#
#\ulaz{1 2 3}#
#\ulaz{1 2 3}#
#\izlaz{Matrice su jednake.}#
#\izlaz{Zbir matrica je:}#
#\izlaz{2 4 6}#
#\izlaz{2 4 6}#
#\izlaz{2 4 6}#
#\izlaz{Proizvod matrica je:}#
#\izlaz{6 12 8}#
#\izlaz{6 12 8}#
#\izlaz{6 12 8}#
\end{upotreba}
\end{miditest}

\linkresenje{315}
\end{Exercise}
\begin{Answer}[ref=315]
\includecode{resenja/03_Pokazivaci/315.c}
\end{Answer}

\begin{Exercise}[label=322]
Relacija se može predstaviti kvadratnom matricom nula i
jedinica na sledeći način: dva elementa $i$ i $j$
su u relaciji ukoliko se u preseku $i$-te vrste i $j$-te
kolone matrice nalazi broj $1$, a nisu u relaciji ukoliko se
tu nalazi broj $0$. 
\begin{enumerate}
\item Napisati funkciju koja proverava da li je relacija zadata matricom refleksivna.
\item Napisati funkciju koja proverava da li je relacija zadata matricom simetrična.
\item Napisati funkciju koja proverava da li je relacija zadata matricom tranzitivna.
\item Napisati funkciju koja određuje refleksivno zatvorenje relacije (najmanju refleksivnu relaciju koja sadrži datu).
\item Napisati funkciju koja određuje simetrično zatvorenje relacije (najmanju simetričnu relaciju koja sadrži datu).
\item Napisati funkciju koja određuje refleksivno-tranzitivno zatvorenje relacije (najmanju refleksivnu i tranzitivnu relaciju
koja sadrži datu).
\end{enumerate}
Napisati program koji učitava matricu iz datoteke čije se ime zadaje kao prvi argument komandne linije.
U prvoj liniji datoteke nalazi se dimenzija matrice $n$ ($0 < n \leq 64$), a potom i sami elementi matrice.
Na standardni izlaz ispisati rezultat testiranja napisanih funkcija.

\begin{miditest}
\begin{upotreba}{1}
#\poziv{./a.out ulaz.txt}#

#\naslovDat{ulaz.txt}#
#\datoteka{4}#
#\datoteka{1 0 0 0}#
#\datoteka{0 1 1 0}#
#\datoteka{0 0 1 0}#
#\datoteka{0 0 0 0}#

#\naslovInt#
#\izlaz{Relacija nije refleksivna.}#
#\izlaz{Relacija nije simetricna.}#
#\izlaz{Relacija jeste tranzitivna.}#
#\izlaz{Refleksivno zatvorenje relacije:}#
#\izlaz{1 0 0 0}#
#\izlaz{0 1 1 0}#
#\izlaz{0 0 1 0}#
#\izlaz{0 0 0 1}#
#\izlaz{Simetricno zatvorenje relacije:}#
#\izlaz{1 0 0 0}#
#\izlaz{0 1 1 0}#
#\izlaz{0 1 1 0}#
#\izlaz{0 0 0 0}#
#\izlaz{Refleksivno-tranzitivno zatvorenje relacije:}#
#\izlaz{1 0 0 0}#
#\izlaz{0 1 1 0}#
#\izlaz{0 0 1 0}#
#\izlaz{0 0 0 1}#
\end{upotreba}
\end{miditest}

\linkresenje{322}
\end{Exercise}
\begin{Answer}[ref=322]
\includecode{resenja/03_Pokazivaci/322.c}
\end{Answer}

\begin{Exercise}[label=323]
Data je kvadratna matrica dimenzije $n$.
\begin{enumerate}
\item Napisati funkciju koja određuje najveći element matrice na sporednoj dijagonali.
\item Napisati funkciju koja određuje indeks kolone koja sadrži najmanji element matrice.
\item Napisati funkciju koja određuje indeks vrste koja sadrži najveći element matrice.
\item Napisati funkciju koja određuje broj negativnih elemenata matrice.
\end{enumerate}
Napisati program koji testira napisane funkcije. Sa standardnog
ulaza učitati elemente celobrojne kvadratne matrice čija
se dimenzija $n$ ($0 < n \leq 32$) zadaje kao argument
komandne linije. Na standardni izlaz ispisati rezultat primene prethodno
napisanih funkcija. 

\begin{miditest}
\begin{upotreba}{1}
#\poziv{./a.out 3}#

#\naslovInt#
#\izlaz{Unesite elemente matrice dimenzije 3:}#
#\ulaz{1 2 3}#
#\ulaz{-4 -5 -6}#
#\ulaz{7 8 9}#
#\izlaz{Najveci element sporedne dijagonale je 7.}#
#\izlaz{Indeks kolone sa najmanjim elementom je 2.}#
#\izlaz{Indeks vrste sa najvecim elementom je 2.}#
#\izlaz{Broj negativnih elemenata matrice je 3.}#
\end{upotreba}
\end{miditest}
\begin{miditest}
\begin{upotreba}{2}
#\poziv{./a.out 4}#

#\naslovInt#
#\izlaz{Unesite elemente matrice dimenzije 4:}#
#\ulaz{-1 -2 -3 -4}#
#\ulaz{-5 -6 -7 -8}#
#\ulaz{-9 -10 -11 -12}#
#\ulaz{-13 -14 -15 -16}#
#\izlaz{Najveci element sporedne dijagonale je -4.}#
#\izlaz{Indeks kolone sa najmanjim elementom je 3.}#
#\izlaz{Indeks vrste sa najvecim elementom je 0.}#
#\izlaz{Broj negativnih elemenata matrice je 16.}#
\end{upotreba}
\end{miditest}

\linkresenje{323}
\end{Exercise}
\begin{Answer}[ref=323]
\includecode{resenja/03_Pokazivaci/323.c}
\end{Answer}

\begin{Exercise}[label=324]
Napisati funkciju kojom se proverava da li je zadata kvadratna
matrica dimenzije $n$ ortonormirana. Matrica je ortonormirana
ako je skalarni proizvod svakog para različitih vrsta jednak
nuli, a skalarni proizvod vrste sa samom sobom jednak jedinici.
Napisati program koji testira napisanu funkciju. Sa standardnog
ulaza učitati dimenziju celobrojne kvadratne matrice $n$
($0 < n \leq 32$), a zatim i njene elemente. Na standardni izlaz
ispisati rezultat primene napisane funkcije na učitanu
matricu.

\begin{miditest}
\begin{upotreba}{1}
#\naslovInt#
#\izlaz{Unesite dimenziju matrice:} \ulaz{4}#
#\izlaz{Unesite elemente matrice, vrstu po vrstu:}#
#\ulaz{1 0 0 0}#
#\ulaz{0 1 0 0}#
#\ulaz{0 0 1 0}#
#\ulaz{0 0 0 1}#
#\izlaz{Matrica je ortonormirana.}#
\end{upotreba}
\end{miditest}
\begin{miditest}
\begin{upotreba}{2}
#\naslovInt#
#\izlaz{Unesite dimenziju matrice:} \ulaz{3}#
#\izlaz{Unesite elemente matrice, vrstu po vrstu:}#
#\ulaz{1 2 3}#
#\ulaz{5 6 7}#
#\ulaz{1 4 2}#
#\izlaz{Matrica nije ortonormirana.}#
\end{upotreba}
\end{miditest}

\linkresenje{324}
\end{Exercise}
\begin{Answer}[ref=324]
\includecode{resenja/03_Pokazivaci/324.c}
\end{Answer}

\begin{Exercise}[label=325]
Data je matrica dimenzije $n \times m$.
\begin{enumerate}
\item Napisati funkciju koja učitava elemente matrice sa standardnog ulaza
\item Napisati funkciju koja na standardni izlaz spiralno ispisuje elemente matrice.
\end{enumerate}
Napisati program koji testira napisane funkcije. Sa standardnog
ulaza učitati dimenzije matrice $n$ ($0 < n \leq 10$) i
$m$ ($0 < n \leq 10$), a zatim i elemente matrice. Na standardni izlaz spiralno ispisati elemente
učitane matrice.


\begin{miditest}
\begin{upotreba}{1}
#\naslovInt#
#\izlaz{Unesite broj vrsta i broj kolona matrice:}#
#\ulaz{3 3}#
#\izlaz{Unesite elemente matrice, vrstu po vrstu:}#
#\ulaz{1 2 3}#
#\ulaz{4 5 6}#
#\ulaz{7 8 9}#
#\izlaz{Spiralno ispisana matrica:}#
#\izlaz{1 2 3 6 9 8 7 4 5}#
\end{upotreba}
\end{miditest}
\begin{miditest}
\begin{upotreba}{2}
#\naslovInt#
#\izlaz{Unesite broj vrsta i broj kolona matrice:}# 
#\ulaz{3 4}#
#\izlaz{Unesite elemente matrice, vrstu po vrstu:}#
#\ulaz{1 2 3 4}#
#\ulaz{5 6 7 8}#
#\ulaz{9 10 11 12}#
#\izlaz{Spiralno ispisana matrica:}# 
#\izlaz{1 2 3 4 8 12 11 10 9 5 6 7}#
\end{upotreba}
\end{miditest}

\linkresenje{325}
\end{Exercise}
\begin{Answer}[ref=325]
\includecode{resenja/03_Pokazivaci/325.c}
\end{Answer}

\begin{Exercise}[label=327]
Napisati funkciju koja izračunava $k$-ti stepen kvadratne
matrice dimenzije $n$ ($0 < n \leq 32$). Napisati program koji
testira napisanu funkciju. Sa standardnog ulaza učitati
dimenziju celobrojne matrice $n$, elemente matrice i stepen
$k$ ($0 < k \leq 10$). Na standardni izlaz ispisati rezultat
primene napisane funkcije. \napomena{Voditi računa da se
prilikom stepenovanja matrice izvrši što manji broj
množenja.}


\begin{miditest}
\begin{upotreba}{1}
#\naslovInt#
#\izlaz{Unesite dimenziju kvadratne matrice:} \ulaz{3}#
#\izlaz{Unesite elemente matrice, vrstu po vrstu:}#
#\ulaz{1 2 3}#
#\ulaz{4 5 6}#
#\ulaz{7 8 9}#
#\izlaz{Unesite stepen koji se racuna:} \ulaz{8}#
#\izlaz{8. stepen matrice je:}#
#\izlaz{510008400 626654232 743300064}#
#\izlaz{1154967822 1419124617 1683281412}#
#\izlaz{1799927244 2211595002 2623262760}#
\end{upotreba}
\end{miditest}
\end{Exercise}
%\begin{Answer}[ref=327]
%\includecode{resenja/03_Pokazivaci/327.c}
%\end{Answer}

\section{Dinamička alokacija memorije}
\begin{Exercise}[label=328]
Napisati program koji sa standardnog ulaza učitava dimenziju  
niza celih brojeva, a zatim i njegove elemente. Ne praviti
nikakve pretpostavke o dimenziji niza. Na standardni izlaz 
ispisati ove brojeve u obrnutom poretku. 


\begin{miditest}
\begin{upotreba}{1}
#\naslovInt#
#\izlaz{Unesite dimenziju niza:} \ulaz{3}#
#\izlaz{Unesite elemente niza:}#
#\ulaz{1 -2 3}#
#\izlaz{Niz u obrnutom poretku je: 3 -2 1}#
\end{upotreba}
\end{miditest}
\begin{miditest}
\begin{upotreba}{2}
#\naslovInt#
#\izlaz{Unesite dimenziju niza:} \ulaz{-1}#
#\izlaz{malloc(): neuspela alokacija memorije.}#
\end{upotreba}
\end{miditest}

\linkresenje{328}
\end{Exercise}
\begin{Answer}[ref=328]
\includecode{resenja/03_Pokazivaci/328.c}
\end{Answer}

\begin{Exercise}[label=330]
Napisati program koji sa standardnog ulaza učitava niz celih
brojeva. Brojevi se unose sve dok se ne unese nula. Ne praviti
nikakve pretpostavke o dimenziji niza. Na standardni izlaz
ispisati ovaj niz brojeva u obrnutom poretku. Zadatak uraditi na dva načina:
\begin{enumerate}
\item realokaciju memorije niza vršiti korišćenjem \kckod{malloc()} funkcije,
\item realokaciju memorije niza vršiti korišćenjem \kckod{realloc()} funkcije.
\end{enumerate}
Od korisnika sa ulaza tražiti da izabere način realokacije memorije. 

\begin{miditest}
\begin{upotreba}{1}
#\naslovInt#
#\izlaz{Unesite zeljeni nacin realokacije (M ili R):}#
#\ulaz{M}#
#\izlaz{Unesite brojeve, nulu za kraj:}#
#\ulaz{1 -2 3 -4 0}#
#\izlaz{Niz u obrnutom poretku je: -4 3 -2 1}#
\end{upotreba}
\end{miditest}
\begin{miditest}
\begin{upotreba}{2}
#\naslovInt#
#\izlaz{Unesite zeljeni nacin realokacije (M ili R):}#
#\ulaz{R}#
#\izlaz{Unesite brojeve, nulu za kraj:}#
#\ulaz{6 -1 5 -2 4 -3 0}#
#\izlaz{Niz u obrnutom poretku je: -3 4 -2 5 -1 6}#
\end{upotreba}
\end{miditest}

\linkresenje{330}
\end{Exercise}
\begin{Answer}[ref=330]
\includecode{resenja/03_Pokazivaci/330.c}
\end{Answer}

\begin{Exercise}[label=329]
Napisati funkciju koja kao rezultat vraća nisku koja se dobija
nadovezivanjem dve niske, bez promene njihovog sadržaja.
Napisati program koji testira rad napisane funkcije. Sa
standardnog ulaza učitati dve niske karaktera (pretpostaviti da 
niske nisu duže od $50$ karaktera i da ne sadrže praznine). Na 
standardni izlaz ispisati nisku koja se dobija njihovim nadovezivanjem. 
Za rezultujuću nisku dinamički alocirati memoriju.


\begin{miditest}
\begin{upotreba}{1}
#\naslovInt#
#\izlaz{Unesite dve niske karaktera:}#
#\ulaz{Jedan Dva}#
#\izlaz{Nadovezane niske: JedanDva}#
\end{upotreba}
\end{miditest}
\begin{miditest}
\begin{upotreba}{2}
#\naslovInt#
#\izlaz{Unesite dve niske karaktera:}#
#\ulaz{Ana Marija}#
#\izlaz{Nadovezane niske: AnaMarija}#
\end{upotreba}
\end{miditest}

\linkresenje{329}
\end{Exercise}
\begin{Answer}[ref=329]
\includecode{resenja/03_Pokazivaci/329.c}
\end{Answer}

\begin{Exercise}[label=331]
Napisati program koji sa standardnog ulaza učitava matricu
realnih brojeva. Prvo se učitavaju dimenzije matrice $n$ i
$m$ (ne praviti nikakve pretpostavke o njihovoj veličini),
a zatim i elementi matrice. Na standardni izlaz ispisati trag
matrice.


\begin{miditest}
\begin{upotreba}{1}
#\naslovInt#
#\izlaz{Unesite broj vrsta i broj kolona matrice:}#
#\ulaz{2 3}#
#\izlaz{Unesite elemente matrice, vrstu po vrstu:}#
#\ulaz{1.2 2.3 3.4}#
#\ulaz{4.5 5.6 6.7}#
#\izlaz{Trag unete matrice je 6.80.}#
\end{upotreba}
\end{miditest}
\begin{miditest}
\begin{upotreba}{2}
#\naslovInt#
#\izlaz{Unesite broj vrsta i broj kolona matrice:}#
#\ulaz{2 2}#
#\izlaz{Unesite elemente matrice, vrstu po vrstu:}#
#\ulaz{-0.1 -0.2}#
#\ulaz{-0.3 -0.4}#
#\izlaz{Trag unete matrice je -0.50.}#
\end{upotreba}
\end{miditest}

\linkresenje{331}
\end{Exercise}
\begin{Answer}[ref=331]
\includecode{resenja/03_Pokazivaci/331.c}
\end{Answer}

\begin{Exercise}[label=332]
Data je celobrojna matrica dimenzije $n \times m$.
Napisati funkciju koja ispisuje elemente ispod glavne dijagonale matrice 
(uključujući i glavnu dijagonalu).
Napisati program koji testira napisanu funkciju. Sa standardnog
ulaza učitati $n$ i $m$ (ne praviti nikakve
pretpostavke o njihovoj veličini), zatim učitati elemente
matrice i na standardni izlaz ispisati elemente ispod glavne
dijagonale matrice.

\begin{miditest}
\begin{upotreba}{1}
#\naslovInt#
#\izlaz{Unesite broj vrsta i broj kolona matrice:}#
#\ulaz{2 3}#
#\izlaz{Unesite elemente matrice, vrstu po vrstu:}#
#\ulaz{1 -2 3}#
#\ulaz{-4 5 -6}#
#\izlaz{Elementi ispod glavne dijagonale matrice:}#
#\izlaz{1}#
#\izlaz{-4 5}#
\end{upotreba}
\end{miditest}

\linkresenje{332}
\end{Exercise}
\begin{Answer}[ref=332]
\includecode{resenja/03_Pokazivaci/332.c}
\end{Answer}

\begin{Exercise}[label=336]
Za zadatu matricu dimenzije $n \times m$ napisati funkciju koja
izračunava redni broj kolone matrice čiji je zbir
maksimalan. Napisati program koji testira ovu funkciju. Sa
standardnog ulaza učitati dimenzije matrice $n$ i
$m$ (ne praviti nikakve pretpostavke o njihovoj veličini), 
a zatim elemente matrice. Na standardni izlaz ispisati 
redni broj kolone matrice sa maksimalnim zbirom. Ukoliko ima
više takvih, ispisati prvu.

\begin{miditest}
\begin{upotreba}{1}
#\naslovInt#
#\izlaz{Unesite broj vrsta i broj kolona matrice:}# 
#\ulaz{2 3}#
#\izlaz{Unesite elemente matrice, vrstu po vrstu:}#
#\ulaz{1 2 3}#
#\ulaz{4 5 6}#
#\izlaz{Kolona pod rednim brojem 3 ima najveci zbir.}#
\end{upotreba}
\end{miditest}
\begin{miditest}
\begin{upotreba}{2}
#\naslovInt#
#\izlaz{Unesite broj vrsta i broj kolona matrice:}#
#\ulaz{2 4}#
#\izlaz{Unesite elemente matrice, vrstu po vrstu:}#
#\ulaz{1 2 3 4}#
#\ulaz{8 7 6 5}#
#\izlaz{Kolona pod rednim brojem 1 ima najveci zbir.}#
\end{upotreba}
\end{miditest}

\end{Exercise}
%\begin{Answer}[ref=336]
%\includecode{resenja/03_Pokazivaci/336.c}
%\end{Answer}

\begin{Exercise}[label=337]
Data je realna kvadratna matrica dimenzije $n$.
\begin{enumerate}
\item Napisati funkciju koja izračunava zbir apsolutnih 
vrednosti matrice ispod sporedne dijagonale. 
\item Napisati funkciju koja menja sadržaj matrice tako 
što polovi elemente iznad glavne dijagonale, duplira 
elemente ispod glavne dijagonale, dok elemente na 
glavnoj dijagonali ostavlja nepromenjene.
\end{enumerate}
Napisati program koji testira ove funkcije za matricu koja se
učitava iz datoteke čije se ime zadaje kao argument komandne linije. 
U datoteci se nalazi prvo dimenzija matrice, a zatim redom elementi matrice.


\begin{miditest}
\begin{upotreba}{1}
#\poziv{./a.out matrica.txt}#

#\naslovDat{matrica.txt}#
#\datoteka{3}#
#\datoteka{1.1 -2.2 3.3}#
#\datoteka{-4.4 5.5 -6.6}#
#\datoteka{7.7 -8.8 9.9}#
\end{upotreba}
\end{miditest}
\begin{miditest}
\begin{test2}{1}



#\naslovInt#
#\izlaz{Zbir apsolutnih vrednosti ispod }#
#\izlaz{sporedne dijagonale je 25.30.}#
#\izlaz{Transformisana matrica je:}#
#\izlaz{1.10 -1.10 1.65}#
#\izlaz{-8.80 5.50 -3.30}#
#\izlaz{15.40 -17.60 9.90}#
\end{test2}
\end{miditest}

\linkresenje{337}
\end{Exercise}
\begin{Answer}[ref=337]
\includecode{resenja/03_Pokazivaci/337.c}
\end{Answer}

\begin{Exercise}[label=342]
Napisati program koji na osnovu dve realne matrice dimenzija $m \times n$
formira matricu dimenzije $2 \cdot m \times n$ tako što
naizmenično kombinuje jednu vrstu prve matrice i jednu vrstu
druge matrice. Matrice su zapisane u datoteci \kckod{matrice.txt}. U
prvom redu se nalaze dimenzije matrica $m$ i $n$, u
narednih $m$ redova se nalaze vrste prve matrice, a u
narednih $m$ redova vrste druge matrice. Rezultujuću
matricu ispisati na standardni izlaz.


\begin{miditest}
\begin{upotreba}{1}
#\poziv{./a.out matrice.txt}#

#\naslovDat{matrice.txt}#
#\datoteka{3}#
#\datoteka{1.1 -2.2 3.3}#
#\datoteka{-4.4 5.5 -6.6}#
#\datoteka{7.7 -8.8 9.9}#
#\datoteka{-1.1 2.2 -3.3}#
#\datoteka{4.4 -5.5 6.6}#
#\datoteka{-7.7 8.8 -9.9}#
\end{upotreba}
\end{miditest}
\begin{miditest}
\begin{test2}{1}



#\naslovInt#
#\izlaz{1.1 -2.2 3.3}#
#\izlaz{-1.1 2.2 -3.3}#
#\izlaz{-4.4 5.5 -6.6}#
#\izlaz{4.4 -5.5 6.6}#
#\izlaz{7.7 -8.8 9.9}#
#\izlaz{-7.7 8.8 -9.9}#
\end{test2}
\end{miditest}
\end{Exercise}
%\begin{Answer}[ref=342]
%\includecode{resenja/03_Pokazivaci/342.c}
%\end{Answer}

\begin{Exercise}[label=344]
Na ulazu se zadaje niz celih brojeva čiji se unos završava nulom.
Napisati funkciju koja od zadatog niza formira matricu tako da
prva vrsta odgovara unetom nizu, a svaka naredna se dobija
cikličkim pomeranjem elemenata niza za jednu poziciju ulevo.
Napisati program koji testira ovu funkciju. 
Rezultujuću matricu ispisati na standardni izlaz.

\begin{miditest}
\begin{upotreba}{1}
#\naslovInt#
#\izlaz{Unesite elemente niza, nulu za kraj:}#
#\ulaz{1 2 3 0}#
#\izlaz{Trazena matrica je:}#
#\izlaz{1 2 3}#
#\izlaz{2 3 1}#
#\izlaz{3 1 2}#
\end{upotreba}
\end{miditest}
\begin{miditest}
\begin{upotreba}{2}
#\naslovInt#
#\izlaz{Unesite elemente niza, nulu za kraj:}#
#\ulaz{-5 -2 -4 -1 0}#
#\izlaz{Trazena matrica je:}#
#\izlaz{-5 -2 -4 -1}#
#\izlaz{-2 -4 -1 -5}#
#\izlaz{-4 -1 -5 -2}#
#\izlaz{-1 -5 -2 -4}#
\end{upotreba}
\end{miditest}

\end{Exercise}
%\begin{Answer}[ref=344]
%\includecode{resenja/03_Pokazivaci/344.c}
%\end{Answer}

\begin{Exercise}[label=340]
Petar sakuplja sličice igrača za predstojeće Svetsko
prvenstvo u fudbalu. U datoteci \kckod{slicice.txt} se nalaze
informacije o sličicama koje mu nedostaju u formatu:\\
\hspace*{5mm}\kckod{redni\_broj\_sličice ime\_reprezentacije\_kojoj\_sličica\_pripada} \\
Pomozite Petru
da otkrije koliko mu sličica ukupno nedostaje, kao i da
pronađe ime reprezentacije čijih sličica ima najmanje.
Dobijene podatke ispisati na standardni izlaz. \napomena{Za
realokaciju memorije koristiti \kckod{realloc()} funkciju.}


\begin{miditest}
\begin{upotreba}{1}
#\naslovDat{slicice.txt}#
#\datoteka{3 Brazil}#
#\datoteka{6 Nemacka}#
#\datoteka{2 Kamerun}#
#\datoteka{1 Brazil}#
#\datoteka{2 Engleska}#
#\datoteka{4 Engleska}#
#\datoteka{5 Brazil}#
\end{upotreba}
\end{miditest}
\begin{miditest}
\begin{test2}{1}



#\naslovInt#
#\izlaz{Petru ukupno nedostaje 7 slicica.}#
#\izlaz{Reprezentacija za koju je sakupio}#
#\izlaz{najmanji broj slicica je Brazil.}#
\end{test2}
\end{miditest}
\end{Exercise}
%\begin{Answer}[ref=340]
%\includecode{resenja/03_Pokazivaci/340.c}
%\end{Answer}

\begin{Exercise}[difficulty=2, label=341]
U datoteci \kckod{temena.txt} se nalaze tačke koje predstavljaju
temena nekog $n$-tougla. Napisati program koji na osnovu
sadržaja datoteke na standardni izlaz ispisuje o kom
$n$-touglu je reč, a zatim i vrednosti njegovog obima i
površine. Pretpostavka je da će mnogougao biti konveksan.

\begin{miditest}
\begin{upotreba}{1}
#\naslovDat{temena.txt}#
#\datoteka{-1 -1}#
#\datoteka{1 -1}#
#\datoteka{1 1}#
#\datoteka{-1 1}#
\end{upotreba}
\end{miditest}
\begin{miditest}
\begin{test2}{1}



#\naslovInt#
#\izlaz{U datoteci su zadata temena cetvorougla.}#
#\izlaz{Obim je 8.}#
#\izlaz{Povrsina je 4.}#
\end{test2}
\end{miditest}

\begin{miditest}
\begin{upotreba}{2}
#\naslovDat{temena.txt}#
#\datoteka{-1.75 -1.5}#
#\datoteka{3 1.5}#
#\datoteka{2.2 3.1}#
#\datoteka{-2 4}#
#\datoteka{-4.1 1}#
\end{upotreba}
\end{miditest}
\begin{miditest}
\begin{test2}{1}



#\naslovInt#
#\izlaz{U datoteci su zadata temena petougla.}#
#\izlaz{Obim je 18.80.}#
#\izlaz{Povrsina je 22.59.}#
\end{test2}
\end{miditest}
\end{Exercise}
%\begin{Answer}[ref=341]
%\includecode{resenja/03_Pokazivaci/341.c}
%\end{Answer}

\section{Pokazivači na funkcije}

\begin{Exercise}[label=345]
Napisati program koji tabelarno štampa vrednosti proizvoljne realne funkcije sa jednim realnim
argumentom, odnosno izračunava i ispisuje
vrednosti date funkcije na diskretnoj ekvidistantnoj mreži od
$n$ tačaka intervala $[a, b]$. Realni brojevi
$a$ i $b$ ($a<b$) kao i ceo broj $n$
($n \geq 2$) se učitavaju sa standardnog ulaza. Ime funkcije
se zadaje kao argument komandne linije (\kckod{sin}, \kckod{cos}, \kckod{tan}, \kckod{atan},
\kckod{acos}, \kckod{asin}, \kckod{exp}, \kckod{log}, \kckod{log10}, \kckod{sqrt}, \kckod{floor}, \kckod{ceil}, \kckod{sqr}).


\begin{miditest}
\begin{upotreba}{1}
#\poziv{./a.out sin}#

#\naslovInt#
#\izlaz{Unesite krajeve intervala:}#
#\ulaz{-0.5 1}#
#\izlaz{Koliko tacaka ima na ekvidistantnoj}#
#\izlaz{mrezi (ukljucujuci krajeve intervala)?}#
#\ulaz{4}#
#\izlaz{     x        sin(x)}#
#\izlaz{-----------------------}#
#\izlaz{| -0.50000 | -0.47943 |}#
#\izlaz{|  0.00000 |  0.00000 |}#
#\izlaz{|  0.50000 |  0.47943 |}#
#\izlaz{|  1.00000 |  0.84147 |}#
#\izlaz{-----------------------}#
\end{upotreba}
\end{miditest}
\begin{miditest}
\begin{upotreba}{2}
#\poziv{./a.out cos}#

#\naslovInt#
#\izlaz{Unesite krajeve intervala:}#
#\ulaz{0 2}#
#\izlaz{Koliko tacaka ima na ekvidistantnoj}#
#\izlaz{mrezi (ukljucujuci krajeve intervala)?}#
#\ulaz{4}#
#\izlaz{     x        cos(x)}#
#\izlaz{-----------------------}#
#\izlaz{|  0.00000 |  1.00000 |}#
#\izlaz{|  0.66667 |  0.78589 |}#
#\izlaz{|  1.33333 |  0.23524 |}#
#\izlaz{|  2.00000 | -0.41615 |}#
#\izlaz{-----------------------}#
\end{upotreba}
\end{miditest}

\linkresenje{345}
\end{Exercise}
\begin{Answer}[ref=345]
\includecode{resenja/03_Pokazivaci/345.c}
\end{Answer}

\begin{Exercise}[label=346]
Napisati funkciju koja izračunava limes funkcije \kckod{f(x)} u   
tački $a$. Adresa funkcije \kckod{f} čiji se limes računa
se prenosi kao parametar funkciji za računanje limesa.  Limes se
računa sledećom aproksimacijom (vrednosti $n$ i $a$ uneti sa
standardnog ulaza kao i ime funkcije):
$$ lim_{x \rightarrow a} f(x) = lim_{n \rightarrow \infty} f(a + \frac{1}{n})$$


\begin{miditest}
\begin{upotreba}{1}
#\naslovInt#
#\izlaz{Unesite ime funkcije, n i a:}#
#\ulaz{tan 10000 1.570795}#
#\izlaz{Limes funkcije tan je -10134.46.}#
\end{upotreba}
\end{miditest}
\begin{miditest}
\begin{upotreba}{2}
#\naslovInt#
#\izlaz{Unesite ime funkcije, n i a:}#
#\ulaz{cos 5000 0.25}#
#\izlaz{Limes funkcije cos je 0.97.}#
\end{upotreba}
\end{miditest}
\end{Exercise}
%\begin{Answer}[ref=346]
%\includecode{resenja/03_Pokazivaci/346.c}
%\end{Answer}

\begin{Exercise}[label=347]
Napisati funkciju koja određuje integral funkcije \kckod{f(x)} na
intervalu $[a, b]$.  Adresa funkcije \kckod{f} se prenosi kao
parametar. Integral se računa prema formuli:
$$ \int_{a}^{b} f(x) = h \cdot (\frac{f(a)+f(b)}{2} + \Sigma_{i=1}^{n}
f(a+i \cdot h))$$ 
Vrednost $h$ se izračunava po formuli $h = (b-a)/n$, dok se vrednosti $n$, $a$ i $b$ unose sa standardnog
ulaza kao i ime funkcije iz zaglavlja
\kckod{math.h}. Na standardni izlaz ispisati vrednost integrala.


\begin{miditest}
\begin{upotreba}{1}
#\naslovInt#
#\izlaz{Unesite ime funkcije, n, a i b:}#
#\ulaz{cos 6000 -1.5 3.5}#
#\izlaz{Vrednost integrala je 0.645931.}#
\end{upotreba}
\end{miditest}
\begin{miditest}
\begin{upotreba}{1}
#\naslovInt#
#\izlaz{Unesite ime funkcije, n, a i b:}#
#\ulaz{sin 10000 -5.2 2.1}#
#\izlaz{Vrednost integrala je 0.973993.}#
\end{upotreba}
\end{miditest}
\end{Exercise}
%\begin{Answer}[ref=347]
%\includecode{resenja/03_Pokazivaci/347.c}
%\end{Answer}

\begin{Exercise}[label=348]
Napisati funkciju koja približno izračunava integral      
funkcije \kckod{f(x)} na intervalu $[a, b]$. Funkcija \kckod{f} se prosleđuje
kao parametar, a integral se procenjuje po Simpsonovoj formuli:
$$I = \frac{h}{3}\left(f(a) + 4 \sum_{i=1}^{n/2}f(a+(2i-1)h) + 2  
\sum_{i=1}^{n/2-1}f(a+2ih) + f(b)\right)$$ 
Granice intervala i $n$ su argumenti funkcije. Napisati program, koji kao
argumente komandne linije prihvata ime funkcije iz zaglavlja
\kckod{math.h}, krajeve intervala i $n$, a na standardni izlaz ispisuje vrednost odgovarajućeg integrala.


\begin{miditest}
\begin{upotreba}{1}
#\naslovInt#
#\izlaz{Unesite ime funkcije, n, a i b:}#
#\ulaz{sin 100 -1.0 3.0}#
#\izlaz{Vrednost integrala je 1.530295.}#
\end{upotreba}
\end{miditest}
\begin{miditest}
\begin{upotreba}{2}
#\naslovInt#
#\izlaz{Unesite ime funkcije, n, a i b:}#
#\ulaz{tan 5000 -4.1 -2.3}#
#\izlaz{Vrednost integrala je -0.147640.}#
\end{upotreba}
\end{miditest}
\end{Exercise}
%\begin{Answer}[ref=348]
%\includecode{resenja/03_Pokazivaci/348.c}
%\end{Answer}
\section{Rešenja}
\shipoutAnswer
