
\chapter{Liste}

\section{Zadaci}
  % mozda iscepkati ove zadatke na pod zadatke <a svako popunjavanje
  %   i za pretraživanje

\begin{Exercise}[label=601]
Napisati program koji testira korišćenje jednostruko povezane liste kao 
dinamičke strukture podataka unošenjem celih brojeva sa standardnog ulaza i 
njihovim smeštanjem u listu. 
Unošenje novih brojeva u listu se prekida učitavanjem kraja ulaza (EOF) sa 
standardnog ulaza. Svako dodavanje novog broja u listu praćeno je ispisivanjem 
trenutnog sadržaja liste. 
\begin{enumerate}
 \item Napisati funkciju koja dodaje nov elemenat na početak liste.
 \item Napisati funkciju koja dodaje nov elemenat na kraj liste.
 \item Napisati funkciju koja dodaje nov elemenat u listu tako da lista ostane rastuće sortirana.
 \item Napisati funkciju koja oslobađa memoriju koju je zauzela lista.
 \item Napisati funkciju koja pretražuje listu za elementom koji ima vrednost koja je argument funkcije.
 \item Napisati funkciju koja briše sve elemente u listi koji imaju vrednost koja je argument funkcije.
\end{enumerate}
Sve funkcije za rad sa listom treba da budu iterativne.
\end{Exercise}
\begin{Answer}[ref=601]
\includecode{resenja/06_Liste/601.c}
\end{Answer}


\begin{Exercise}[label=602]
Prethodni zadatak uraditi tako da sve funkcije za rad sa listom budu rekurzivne.
\end{Exercise}
\begin{Answer}[ref=602]
\includecode{resenja/06_Liste/601.c}
\end{Answer}


\begin{Exercise}[label=603]
Napisati program koji testira korišćenje dvostruko povezane liste kao 
dinamičke strukture podataka unošenjem celih brojeva sa standardnog ulaza i 
njihovim smeštanjem u listu. 
Unošenje novih brojeva u listu se prekida učitavanjem kraja ulaza (EOF) sa 
standardnog ulaza. Svako dodavanje novog broja u listu praćeno je ispisivanjem 
trenutnog sadržaja liste. 
\begin{enumerate}
 \item Napisati funkciju koja dodaje nov elemenat na početak liste.
 \item Napisati funkciju koja dodaje nov elemenat na kraj liste.
 \item Napisati funkciju koja dodaje nov elemenat u listu tako da lista ostane rastuće sortirana.
 \item Napisati funkciju koja oslobađa memoriju koju je zauzela lista.
 \item Napisati funkciju koja pretražuje listu za elementom koji ima vrednost koja je argument funkcije.
 \item Napisati funkciju koja briše sve elemente u listi koji imaju vrednost koja je argument funkcije.
\end{enumerate}
Sve funkcije za rad sa dvostruko povezanom listom treba da budu iterativne.
\end{Exercise}
\begin{Answer}[ref=603]
\includecode{resenja/06_Liste/601.c}
\end{Answer}



\begin{Exercise}[label=604]
Napisati program koji proverava ispravnost uparivanja etiketa u HTML datoteci. 
Za rešenje koristiti stek implementiran korišćenjem listi za smeštanje etiketa.


\begin{maxitest}
    \begin{test}{Test 1}
Poziv: ./a.out datoteka.html
Datoteka.html:                                     Izlaz:
<html>                                             Ispravno
  <head><title>Primer</title></head>               
  <body>                                           
    <h1>Naslov</h1>                                
    Danas je lep i suncan dan. <br>                
    A sutra ce biti jos lepsi.     
    <a link="http://www.google.com"> Link 1</a>    
    <a link="http://www.math.rs"> Link 2</a>
  </body>
</html>
    \end{test}
    
  \begin{test}{Test 2}
Poziv: ./a.out datoteka.html
Datoteka.html:                                     Izlaz:
<html>                                             Neispravno
  <head><title>Primer</title></head>               
  <body>  
</html>
  \end{test}
  
  \begin{test}{Test 3}
Poziv: ./a.out datoteka.html
Datoteka.html:                                     Izlaz:
<html>                                             Neispravno
  <head><title>Primer</title></head>               
  <body>  
  </body>
  \end{test}
  
\end{maxitest}

\end{Exercise}
\begin{Answer}[ref=604]
\includecode{resenja/06_Liste/601.c}
\end{Answer}


\begin{Exercise}[label=605]
Napisati program kojim se simulira rad jednog šaltera na kom se prvo zakazuju 
termini, a potom službenik uslužuje korisnike redom, kako su se prijavljivali.
Korisnik se prijavljuje unošenjem svog jmbg broja i zahteva. Prijavljivanje korisnika se prekida unošenjem karaktera za kraj ulaza (EOF).

Službenik ima mogućnost da prilikom obrade zahteva odgovorom \kckod{Da} na 
postavljeno pitanje, korisnika čiji je to bio zahtev vrati natrag na kraj reda.
Posle svakog 10 usluženog korisnika, službeniku se nudi mogućnost da prekine sa 
radom, nevezano od broja korisnika koji i dalje čekaju u redu.

Za čuvanje korisničkih zahteva koristiti red implementiran korišćenjem listi.
\end{Exercise}
\begin{Answer}[ref=605]
\includecode{resenja/06_Liste/601.c}
\end{Answer}


%%% Obavezni sa praktikuma

\begin{Exercise}[label=606]
Napisati program koji prebrojava pojavljivanja etiketa (bez atributa) HTML 
datoteke čije se ime zadaje kao argument komandne linije. Sadržaj rezultujuće 
liste ispisati na standardni izlaz. Za formiranje liste koristiti strukturu:
\begin{verbatim} 
 typedef struct _Element
 {
   unsigned broj_pojavljivanja;
   char etiketa[20];
   struct _Element *sledeci;
 } Element;
\end{verbatim}

\begin{maxitest}
    \begin{test}{Test 1}
Poziv: ./a.out datoteka.html
Datoteka.html:                                     Izlaz:
<html>                                             a - 4
  <head><title>Primer</title></head>               br - 1
  <body>                                           h1 - 2
    <h1>Naslov</h1>                                body - 2
    Danas je lep i suncan dan. <br>                title - 2
    A sutra ce biti jos lepsi.                     head - 2
    <a link="http://www.google.com"> Link 1</a>    html - 2
    <a link="http://www.math.rs"> Link 2</a>
  </body>
</html>
    \end{test}
\end{maxitest}

\end{Exercise}
\begin{Answer}[ref=606]
\includecode{resenja/06_Liste/601.c}
\end{Answer}



\begin{Exercise}[label=607]
Napisati program koji objedinjuje dve sortirane liste. Funkcija ne treba da 
kreira nove čvorove, već da samo
postojeće čvorove preraspodeli. Prva lista se učitava iz datoteke koja se 
zadaje kao prvi argument komandne
linije, a druga iz druge. Rezultujuću listu ispisati na standardni izlaz.

\begin{miditest}
  \begin{test}{Test 1}
Poziv: ./a.out dat1.txt dat2.txt
dat1.txt: 2 4 6 10 15
dat2.txt: 5 6 11 12 14 16
Izlaz: 2 4 5 6 6 10 11 12 14 15 16
  \end{test}
\end{miditest}
  
\end{Exercise}
\begin{Answer}[ref=607]
\includecode{resenja/06_Liste/601.c}
\end{Answer}



\begin{Exercise}[label=608]
Napisati funkciju koja formira listu studenata tako što se podaci o studentima 
učitavaju iz datoteke čije se ime zadaje kao argument komandne linije. 
U svakom redu datoteke nalaze se podaci o studentu i to broj indeksa, ime
i prezime. Napisati rekurzivnu funkciju koja određuje da li neki student pripada listi ili ne.
Ispisati zatim odgovarajuću poruku i rekurzivno osloboditi memoriju koju je data lista zauzimala.
Student se traži na osnovu broja indeksa, koji se zadaje sa standardnog ulaza.
\begin{maxitest}
    \begin{test}{Test 1}
Poziv: ./a.out studenti.txt
Datoteka:                 Ulaz:       Izlaz:
123/2014 Marko Lukic      3/2014      da: Ana Sokic
3/2014 Ana Sokic          235/2008    ne
43/2013 Jelena Ilic       41/2009     da: Marija Zaric
41/2009 Marija Zaric
13/2010 Milovan Lazic
  \end{test}
\end{maxitest}

\end{Exercise}
\begin{Answer}[ref=608]
\includecode{resenja/06_Liste/601.c}
\end{Answer}


\begin{Exercise}[label=608]
Sadržaj datoteke je aritmetički izraz koji može sadržati zagrade \{, [ i (. 
Napisati program koji učitava sadržaj datoteke i korišćenjem steka 
utvrđuje da li su zagrade u aritmetičkom izrazu 
dobro uparene. Program štampa odgovarajuću poruku na standardni izlaz.

\begin{maxitest}
  \begin{test}{Test 1}
Datoteka: {[23 + 5344] * (24 - 234)} - 23
Izlaz:  ispravno
  \end{test}
  \begin{test}{Test 2}
Datoteka: {[23 + 5] * (9 * 2)} - {23}
Izlaz:  ispravno
  \end{test}
  \begin{test}{Test 3}
Datoteka: {[2 + 54) / (24 * 87)} + (234 + 23)
Izlaz:  neispravno
  \end{test}
\end{maxitest}

\end{Exercise}
\begin{Answer}[ref=608]
\includecode{resenja/06_Liste/601.c}
\end{Answer}

%% Dodatni


\begin{Exercise}[label=609]
Neka su date dve jednostruko povezane liste L1 i L2. Napisati funkciju koja od 
tih lista formira novu listu L koja sadrži alternirajući rasporedene elemente 
lista L1 i L2 (prvi element iz L1, prvi element iz L2, drugi element L1,
drugi element L2, itd). Ne formirati nove čvorove, već samo postojeće čvorove 
rasporediti u jednu listu. Prva lista se učitava iz datoteke koja se zadaje 
kao prvi argument komandne linije, a druga iz druge. Rezultujuću listu ispisati 
na standardni izlaz.

\begin{miditest}
  \begin{test}{Test 1}
Poziv: ./a.out dat1.txt dat2.txt
dat1.txt: 2 4 6 10 15
dat2.txt: 5 6 11 12 14 16
Izlaz:  2 5 4 6 6 11 10 12 15 14 16
  \end{test}
\end{miditest}

\end{Exercise}
\begin{Answer}[ref=609]
\includecode{resenja/06_Liste/601.c}
\end{Answer}



\begin{Exercise}[label=610]
Data je datotka brojevi.txt koja sadrži cele brojeve, po jedan u svakom redu.
\begin{enumerate}
 \item Napisati funkciju koja iz zadate datoteke učitava brojeve i smešta ih u listu.
 \item Napisati funkciju koja u jednom prolazu kroz zadatu listu celih brojeva 
pronalazi maximalan strogo rastući podniz.
 \item Koristeći funkcije pod a) i b) napisati program koji u datoteku 
Rezultat.txt upisuje nadeni strogo rastući podniz.
\end{enumerate}


\begin{maxitest}
  \begin{test}{Test 1}
Ulaz:  brojevi.txt       Izlaz: Rezultat.txt
       43                            12
       12                            15
       15                            16
       16
       4
       2
       8
  \end{test}
\end{maxitest}

\end{Exercise}
\begin{Answer}[ref=610]
\includecode{resenja/06_Liste/601.c}
\end{Answer}



\begin{Exercise}[label=611]
Grupa od n plesača na kostimima imaju brojeve od 1 do n, redom, u smeru kazaljke na satu.
Plesači izvode svoju plesnu tačku tako što formiraju krug iz kog najpre izlazi k-ti plesač.
Odbrojava se počevši od plesača označenog brojem 1 u smeru kretanja kazaljke na satu. 
Preostali plesači obrazuju manji krug iz kog opet izlazi k-ti plesač. Odbrojavanje počinje od
sledećeg suseda prethodno izbačenog, opet u smeru kazaljke na satu. Izlasci iz kruga se nastavljaju
sve dok svi plesači ne budu isključeni. 
Celi brojevi n, k (k < n) se učitavaju sa standardnog ulaza. Napisati program koji će na standardni 
izlaz ispisati redne brojeve plesača u redosledu napuštanja kruga. 
NAPOMENA: Koristiti kružnu listu jer u njoj poslednji element, umesto null 
pokazivača čuva adresu prvog elementa u listi.
\begin{maxitest}
  \begin{test}{Test 1}
Ulaz: 5 3 
Izlaz: 3 1 5 2 4
  \end{test}
\end{maxitest}

\end{Exercise}
\begin{Answer}[ref=611]
\includecode{resenja/06_Liste/601.c}
\end{Answer}



\section{Rešenja}
\shipoutAnswer


