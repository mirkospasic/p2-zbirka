
\chapter{Dinamičke strukture podataka}

\section{Liste}

\begin{Exercise}[label=601]
Napisati program koji koristi jednostruko povezanu listu za čuvanje elemenata koji se unose sa standardnog ulaza. 
Unošenje novih brojeva u listu prekida se učitavanjem kraja ulaza (EOF). 
Svako dodavanje novog broja u listu ispratiti ispisivanjem trenutnog sadržaja liste. 
\begin{enumerate}
\item Definisati strukturu \kckod{Cvor} koja predstavlja čvor liste.
\item Napisati funkciju koja kao argument dobija ceo broj, kreira nov čvor liste sa tom vrednosti i vreća adresu novo kreiranog čvora.
 \item Napisati funkciju koja dodaje novi elemenat na početak liste. 
 \item Napisati funkciju koja ispisuje elemente liste, uokvirene zagradama [, ] i međusobno razdvojene zapetama.
 \item Napisati funkciju koja pretražuje listu za elementom koji ima vrednost koja je argument funkcije.
 \item Napisati funkciju koja briše sve elemente u listi koji imaju vrednost koja je argument funkcije.
 \item Napisati funkciju koja oslobađa dinamički zauzetu memoriju za elemente liste.
 \end{enumerate}

Sve funckije za rad sa listom najpre implementirati iterativno, a zatim i rekurzivno.
\komentar{Ana: Da li bi ovde trebalo da stoje dve reference na rešenja jer imamo nezavisno rekurzivno i interativno rešenje}


\begin{maxitest}
\begin{test}{Upotreba programa 1}
Poziv: ./a.out
Ulaz:
	Unosite elemente liste! (za kraj unesite EOF tj. CTRL+D): 2 3 14 5 3 3 17 3 1 9 3
	Unesite element koji se trazi u listi: 17
	Unesite element koji se brise iz liste: 3
Izlaz:
        Lista: []
        Lista: [2]
        Lista: [3, 2]
        Lista: [14, 3, 2]
        Lista: [5, 14, 3, 2]
        Lista: [3, 5, 14, 3, 2]
        Lista: [3, 3, 5, 14, 3, 2]
        Lista: [17, 3, 3, 5, 14, 3, 2]
        Lista: [3, 17, 3, 3, 5, 14, 3, 2]
        Lista: [1, 3, 17, 3, 3, 5, 14, 3, 2]
        Lista: [9, 1, 3, 17, 3, 3, 5, 14, 3, 2]
        Lista: [3, 9, 1, 3, 17, 3, 3, 5, 14, 3, 2]

        Trazeni broj 17 je u listi!
        Lista nakon brisanja:  [9, 1, 17, 5, 14, 2]
\end{test}
\end{maxitest}



\begin{maxitest}
\begin{test}{Upotreba programa 2}
Poziv: ./a.out
Ulaz:
	Unosite elemente liste! (za kraj unesite EOF tj. CTRL+D): 23 14 35
	Unesite element koji se trazi u listi: 8
	Unesite element koji se brise iz liste: 3
Izlaz:
	Lista: []
        Lista: [23]
        Lista: [14, 23]
        Lista: [35, 14, 23]
        
        Element NIJE u listi!
        Lista nakon brisanja:  [35, 14, 23]
\end{test}
\end{maxitest}

\begin{maxitest}
\begin{test}{Upotreba programa 3}
Poziv: ./a.out
Ulaz:
	Unosite elemente liste! (za kraj unesite EOF tj. CTRL+D): 
	Unesite element koji se trazi u listi: 1
	Unesite element koji se brise iz liste: 12
Izlaz:
        Lista: []
        
        Element NIJE u listi!
        Lista nakon brisanja:  []
\end{test}
\end{maxitest}


\linkresenje{601}
\end{Exercise}

\begin{Answer}[ref=601]
\includecode{resenja/06_Liste/601.c}
\end{Answer}



\begin{Exercise}[label=602]
Napisati program koji koristi jednostruko povezanu listu za čuvanje elemenata koji se unose sa standardnog ulaza. 
Unošenje novih brojeva u listu prekida se učitavanjem kraja ulaza (EOF). 
Svako dodavanje novog broja u listu ispratiti ispisivanjem trenutnog sadržaja liste. 
\begin{enumerate}
\item Definisati strukturu \kckod{Cvor} koja predstavlja čvor liste.
\item Napisati funkciju koja kao argument dobija ceo broj, kreira nov čvor liste sa tom vrednosti i vreća adresu novo kreiranog čvora.
 \item Napisati funkciju koja dodaje novi elemenat na kraj liste. 
 \item Napisati funkciju koja ispisuje elemente liste, uokvirene zagradama [, ] i međusobno razdvojene zapetama.
 \item Napisati funkciju koja oslobađa dinamički zauzetu memoriju za elemente liste.
 \end{enumerate}

 Sve funkcije za rad sa listom najpre implementirati iterativno, a zatim i rekurzivno.
\komentar{Ana: Da li bi ovde trebalo da stoje dve reference na rešenja jer imamo nezavisno rekurzivno i interativno rešenje}


\begin{maxitest}
\begin{test}{Upotreba programa 1}
Poziv: ./a.out
Ulaz:
	Unosite elemente liste! (za kraj unesite EOF tj. CTRL+D): 2 3 14 5 3 3 17 3 1 9 3
Izlaz:
        Lista: []
        Lista: [2]
        Lista: [2, 15]
        Lista: [2, 15, 4]
        Lista: [2, 15, 4, 8]
        Lista: [2, 15, 4, 8, 3]
        Lista: [2, 15, 4, 8, 3, 24]
        Lista: [2, 15, 4, 8, 3, 24, 11]
        Lista: [2, 15, 4, 8, 3, 24, 11, 17]
        Lista: [2, 15, 4, 8, 3, 24, 11, 17, 4]
        Lista: [2, 15, 4, 8, 3, 24, 11, 17, 4, 3]
        Lista: [2, 15, 4, 8, 3, 24, 11, 17, 4, 3, 7]
\end{test}
\end{maxitest}



\begin{maxitest}
\begin{test}{Upotreba programa 2}
Poziv: ./a.out
Ulaz:
	Unosite elemente liste! (za kraj unesite EOF tj. CTRL+D): 
Izlaz:
        Lista: []
\end{test}
\end{maxitest}

\linkresenje{602}
\end{Exercise}

\begin{Answer}[ref=602]
\includecode{resenja/06_Liste/602.c}
\end{Answer}





\begin{Exercise}[label=603]
Napisati program koji koristi jednostruko povezanu listu za čuvanje elemenata koji se unose sa standardnog ulaza. 
Unošenje novih brojeva u listu prekida se učitavanjem kraja ulaza (EOF). 
Svako dodavanje novog broja u listu ispratiti ispisivanjem trenutnog sadržaja liste. 
\begin{enumerate}
\item Definisati strukturu \kckod{Cvor} koja predstavlja čvor liste.
\item Napisati funkciju koja kao argument dobija ceo broj, kreira nov čvor liste sa tom vrednosti i vreća adresu novo kreiranog čvora.
 \item Napisati funkciju koja dodaje novi elemenat u listu tako da lista ostane rastuće sortirana.
 \item Napisati funkciju koja oslobađa memoriju koju je zauzela lista.
 \item Napisati funkciju koja ispisuje elemente liste, uokvirene zagradama \[, \] i međusobno razdvojene zapetama.
 \item Napisati funkciju koja pretražuje listu za elementom koji ima vrednost koja je argument funkcije.
 \item Napisati funkciju koja briše sve elemente u listi koji imaju vrednost koja je argument funkcije.
 \item Napisati funkciju koja oslobađa dinamički zauzetu memoriju za elemente liste.
 \end{enumerate}

 Sve funkcije za rad sa listom najpre implementirati iterativno, a zatim i rekurzivno.
\komentar{Ana: Da li bi ovde trebalo da stoje dve reference na rešenja jer imamo nezavisno rekurzivno i interativno rešenje}



\begin{maxitest}
\begin{test}{Upotreba programa 1}
Poziv: ./a.out
Ulaz:
	Unosite elemente liste! (za kraj unesite EOF tj. CTRL+D): 2 3 14 5 3 3 17 3 1 9 3
	Unesite element koji se trazi u listi: 5
	Unesite element koji se brise iz liste: 3
Izlaz:
        Lista: []
        Lista: [2]
        Lista: [2, 3]
        Lista: [2, 3, 14]
        Lista: [2, 3, 5, 14]
        Lista: [2, 3, 3, 5, 14]
        Lista: [2, 3, 3, 3, 5, 14]
        Lista: [2, 3, 3, 3, 5, 14, 17]
        Lista: [2, 3, 3, 3, 3, 5, 14, 17]
        Lista: [1, 2, 3, 3, 3, 3, 5, 14, 17]
        Lista: [1, 2, 3, 3, 3, 3, 5, 9, 14, 17]
        Lista: [1, 2, 3, 3, 3, 3, 3, 5, 9, 14, 17]
        
        Trazeni broj 5 je u listi!
        Lista nakon brisanja:  [1, 2, 5, 9, 14, 17]
\end{test}
\end{maxitest}



\begin{maxitest}
\begin{test}{Upotreba programa 2}
Poziv: ./a.out
Ulaz:
	Unosite elemente liste! (za kraj unesite EOF tj. CTRL+D): 23 14 35
	Unesite element koji se trazi u listi: 8
	Unesite element koji se brise iz liste: 3
Izlaz:
	Lista: []
        Lista: [23]
        Lista: [14, 23]
        Lista: [14, 23, 35]
        
        Element NIJE u listi!
        Lista nakon brisanja:  [14, 23, 35]
\end{test}
\end{maxitest}

\begin{maxitest}
\begin{test}{Upotreba programa 3}
Poziv: ./a.out
Ulaz:
	Unosite elemente liste! (za kraj unesite EOF tj. CTRL+D): 
	Unesite element koji se trazi u listi: 1
	Unesite element koji se brise iz liste: 12
Izlaz:
        Lista: []
        
        Element NIJE u listi!
        Lista nakon brisanja:  []
\end{test}
\end{maxitest}

\linkresenje{603}
\end{Exercise}
\begin{Answer}[ref=603]
\includecode{resenja/06_Liste/603.c}
\end{Answer}







\begin{Exercise}[label=604]
Napisati program koji koristi dvostruko povezanu listu za čuvanje
celih brojeva koji se unose sa standardnog ulaza. 
Unošenje novih brojeva u listu se prekida učitavanjem kraja ulaza (EOF). 
Svako dodavanje novog broja u listu ispratiti ispisivanjem trenutnog sadržaja liste. 
\komentar{I ovde isto mozda razdvojiti sortiranost od obične liste.}
\begin{enumerate}
 \item Napisati funkciju koja dodaje novi elemenat na početak liste.
 \item Napisati funkciju koja dodaje novi elemenat na kraj liste.
 \item Napisati funkciju koja dodaje novi elemenat u listu tako da lista ostane rastuće sortirana.
 \item Napisati funkciju koja pretražuje listu za elementom koji ima vrednost koja je argument funkcije.
 \item Napisati funkciju koja briše sve elemente u listi koji imaju vrednost koja je argument funkcije.
 \item Napisati funkciju koja oslobađa dinamički zauzetu memoriju za elemente liste.

\end{enumerate}
Sve funckije za rad sa listom implementirati iterativno.
\end{Exercise}
\begin{Answer}[ref=604]
% \includecode{resenja/06_Liste/604.c}
\end{Answer}


\begin{Exercise}[label=605]
Sadržaj datoteke je aritmetički izraz koji može sadržati zagrade \{, [ i (. 
Napisati program koji učitava sadržaj datoteke i korišćenjem steka 
utvrđuje da li su zagrade u aritmetičkom izrazu 
dobro uparene. Program štampa odgovarajuću poruku na standardni izlaz.

\begin{maxitest}
  \begin{test}{Test 1}
Datoteka: {[23 + 5344] * (24 - 234)} - 23
Izlaz:  Zagrade su ispravno uparene.
  \end{test}
\end{maxitest}

\begin{maxitest}
  \begin{test}{Test 2}
Datoteka: {[23 + 5] * (9 * 2)} - {23}
Izlaz:  Zagrade su ispravno uparene.
  \end{test}
\end{maxitest}

\begin{maxitest}
  \begin{test}{Test 3}
Datoteka: {[2 + 54) / (24 * 87)} + (234 + 23)
Izlaz:  Zagrade nisu ispravno uparene.
  \end{test}
\end{maxitest}

\begin{maxitest}
  \begin{test}{Test 3}
Datoteka: {(2 - 14) / (23 + 11)}} * (2 + 13)
Izlaz:  Zagrade nisu ispravno uparene.
  \end{test}
\end{maxitest}

\begin{maxitest}
  \begin{test}{Test 4}
Datoteka je prazna.
Izlaz:  Zagrade su ispravno uparene.
  \end{test}
\end{maxitest}

\begin{maxitest}
  \begin{test}{Test 5}
Datoteka ne postoji. 
Izlaz:  Greska prilikom otvaranja datoteke izraz.txt!
  \end{test}
\end{maxitest}

\end{Exercise}
\begin{Answer}[ref=605]
% \includecode{resenja/06_Liste/605.c}
\end{Answer}





\begin{Exercise}[label=606]
Napisati program koji proverava ispravnost uparivanja etiketa u HTML datoteci. Ime datoteke se zadaje kao argument komandne linije . \komentar{Milena: A sta ako se ne navede argument komandne linije?}
Uputstvo: za rešavanje problema koristiti stek implementiran preko listi čiji su čvorovi HTML etikete.

\begin{maxitest}
    \begin{test}{Test 1}
Poziv: ./a.out datoteka.html
Datoteka.html:                          
<html>                                  
  <head><title>Primer</title></head>               
  <body>                                           
    <h1>Naslov</h1>                                
    Danas je lep i suncan dan. <br>                
    A sutra ce biti jos lepsi.     
    <a link="http://www.google.com"> Link 1</a>    
    <a link="http://www.math.rs"> Link 2</a>
  </body>
</html>
Izlaz: Ispravno uparene etikete.
    \end{test}
\end{maxitest}

\begin{maxitest}    
  \begin{test}{Test 2}
Poziv: ./a.out datoteka.html
Datoteka.html:                         
<html>                                 
  <head><title>Primer</title></head>               
  <body>  
</html>
Izlaz: Neispravno uparene etikete.
  \end{test}
\end{maxitest}

\begin{maxitest}      
  \begin{test}{Test 3}
Poziv: ./a.out datoteka.html
Datoteka.html:                         
<html>                                  
  <head><title>Primer</title></head>               
  <body>  
  </body>
Izlaz: Neispravno uparene etikete.
  \end{test}
\end{maxitest}


\begin{maxitest}
    \begin{test}{Test 4}
Poziv: ./a.out 
Izlaz: Greska! Program se poziva sa: ./a.out datoteka.html!
    \end{test}
\end{maxitest}


\begin{maxitest}
    \begin{test}{Test 5}
Poziv: ./a.out datoteka.html
Datoteka.html ne postoji.
Izlaz: Greska prilikom otvaranja datoteke datoteka.html.
    \end{test}
\end{maxitest}

\begin{maxitest}
    \begin{test}{Test 6}
Poziv: ./a.out datoteka.html
Datoteka.html je prazna
Izlaz: Ispravno uparene etikete.
    \end{test}
\end{maxitest}

\end{Exercise}
\begin{Answer}[ref=606]
% \includecode{resenja/06_Liste/601.c}
\end{Answer}



\begin{Exercise}[label=607]
\komentar{Milena: Problem sa ovim zadatkom je sto je program najpre na usluzi korisnicima, 
a zatim na usluzi sluzbeniku i to nekako zbunjuje u formulaicji. Formulacija mi nije bila jasna bez citanja resenja, 
pokusala sam da je preciziran, u nastavku je izmenjena formulacija. \\
Medjutim, ja i dalje nisam bas zadovoljna i zato predlazem da se formulacija izmeni tako da je program stalno 
na usluzi sluzbeniku. Program ucitva podatke o prijavljenim korisnicima iz datoteke. 
Sluzbenik odlucuje da li ce da obradjuje redom korisnike, ili ce u nekim situacijama da odlozi rad sa korisnikom
i stavi ga na kraj reda. Program ga uvek pita da na osnovu jmbg-a i zahteva odluci da li ce ga staviti na kraj reda, 
ako hoce, on ide na kraj reda, ako nece, onda  sluzbenik daje odgovor na zahtev i jmbg, zahtev i odgovor se upisuju u izlaznu datoteku.}

Napisati program kojim se simulira rad jednog šaltera na kojem se prvo zakazuju 
termini, a potom službenik uslužuje korisnike redom, kako su se prijavljivali.

Korisnik se prijavljuje unošenjem svog \kckod{jmbg} broja (niska koja sadrži $13$ karaktera) i zahteva (niska koja sadrži najviše $999$ karaktera). Prijavljivanje korisnika se prekida unošenjem karaktera za kraj ulaza (\kckod{EOF}).

Službenik redom proziva korisnike čitanjem njihovog \kckod{jmbg} broja, a zatim odlučuje da li 
korisnika vraća na kraj reda ili ga odmah uslužuje. Službeniku se postavlja pitanje 
\kckod{Da li korisnika  vracate na kraj reda?} i ukoliko on da odgovor \kckod{Da}, 
korisnik se vraća na kraj reda. Ukoliko odgovor nije \kckod{Da}, tada službenik čita korisnikov zahtev.  
Posle svakog $10$ usluženog korisnika, službeniku se nudi 
mogućnost da prekine sa radom, nevezano od broja korisnika koji i dalje čekaju u redu. 

Za čuvanje korisničkih zahteva koristiti red implementiran korišćenjem listi.
\end{Exercise}
\begin{Answer}[ref=607]
% \includecode{resenja/06_Liste/601.c}
\end{Answer}


%%% Obavezni sa praktikuma

\begin{Exercise}[label=608]
Napisati program koji prebrojava pojavljivanja etiketa HTML 
datoteke čije se ime zadaje kao argument komandne linije. Rezultat prebrojavanja 
ispisati na standardni izlaz. Etikete smeštati u listu, a za formiranje liste koristiti strukturu:
\begin{ckod} 
 typedef struct _Element
 {
   unsigned broj_pojavljivanja;
   char etiketa[20];
   struct _Element *sledeci;
 } Element;
\end{ckod}

\begin{maxitest}
    \begin{test}{Test 1}
Poziv: ./a.out datoteka.html
Datoteka.html:                                     
<html>                                             
  <head><title>Primer</title></head>               
  <body>                                           
    <h1>Naslov</h1>                                
    Danas je lep i suncan dan. <br>                
    A sutra ce biti jos lepsi.                     
    <a link="http://www.google.com"> Link 1</a>    
    <a link="http://www.math.rs"> Link 2</a>
  </body>
</html>

Izlaz:  a - 4
        br - 1
        h1 - 2
        body - 2
        title - 2
        head - 2
        html - 2
    \end{test}
\end{maxitest}

\begin{maxitest}
    \begin{test}{Test 2}
Poziv: ./a.out 
Izlaz: Greska! Program se poziva sa: ./a.out datoteka.html!
    \end{test}
\end{maxitest}


\begin{maxitest}
    \begin{test}{Test 3}
Poziv: ./a.out datoteka.html
Datoteka.html ne postoji.
Izlaz: Greska prilikom otvaranja datoteke datoteka.html.
    \end{test}
\end{maxitest}

\begin{maxitest}
    \begin{test}{Test 4}
Poziv: ./a.out datoteka.html
Datoteka.html je prazna.
Izlaz: 
    \end{test}
\end{maxitest}



\end{Exercise}
\begin{Answer}[ref=608]
% \includecode{resenja/06_Liste/601.c}
\end{Answer}



\begin{Exercise}[label=609]
\komentar{Milena: malo me muci u ovom zadatku sto nema neki smisao. Naime, ako se samo vrsi ucitavanje iz datoteka i ispisivanje, 
onda su ove liste zapravo visak jer isti rezultat moze da se dobije i bez koriscenja listi. 
Zato mi fali da program uradi nesto sto ne bi mogao da uradi bez koriscenja listi,
npr da na osnovu unetog broja ispisuje svaki n-ti broj rezultujuce liste pa to u nekoj petlji 
da korisnik moze da ispisuje za razlicite unete n ili tako nesto... }

Napisati program koji objedinjuje dve sortirane liste. Funkcija ne treba da 
kreira nove čvorove, već da samo
postojeće čvorove preraspodeli. Prva lista se učitava iz datoteke koja se 
zadaje kao prvi argument komandne
linije, a druga iz datoteke čije se ime zadaje kao drugi argument komandne linije. Rezultujuću listu ispisati na standardni izlaz.

\begin{miditest}
  \begin{test}{Test 1}
Poziv: ./a.out dat1.txt dat2.txt
dat1.txt: 2 4 6 10 15
dat2.txt: 5 6 11 12 14 16
Izlaz: 2 4 5 6 6 10 11 12 14 15 16
  \end{test}
\end{miditest}
  
\begin{miditest}
  \begin{test}{Test 2}
Poziv: ./a.out
Izlaz: Greska! Program se poziva sa: ./a.out dat1.txt dat2.txt!
  \end{test}
\end{miditest}  
 
 
\begin{miditest}
  \begin{test}{Test 3}
Poziv: ./a.out dat1.txt 
Izlaz: Greska! Program se poziva sa: ./a.out dat1.txt dat2.txt!
  \end{test}
\end{miditest}


\begin{miditest}
  \begin{test}{Test 4}
Poziv: ./a.out dat1.txt dat2.txt
dat1.txt: 2 4 6 10 15
dat2.txt ne postoji
Izlaz: Greska prilikom otvaranja datoteke dat2.txt.
  \end{test}
\end{miditest}

\begin{miditest}
  \begin{test}{Test 5}
Poziv: ./a.out dat1.txt dat2.txt
dat1.txt ne postoji
dat2.txt: 2 4 6 10 15
Izlaz: Greska prilikom otvaranja datoteke dat1.txt.
  \end{test}
\end{miditest}
  
  
\begin{miditest}
  \begin{test}{Test 6}
Poziv: ./a.out dat1.txt dat2.txt
dat1.txt prazna
dat2.txt: 2 4 6 10 15
Izlaz: 2 4 6 10 15
  \end{test}
\end{miditest}
  
\end{Exercise}
\begin{Answer}[ref=609]
\includecode{resenja/06_Liste/601.c}
\end{Answer}



\begin{Exercise}[label=610]
Napisati funkciju koja formira listu studenata tako što se podaci o studentima 
učitavaju iz datoteke čije se ime zadaje kao argument komandne linije. 
U svakom redu datoteke nalaze se podaci o studentu i to broj indeksa, ime
i prezime. Napisati rekurzivnu funkciju koja određuje da li neki student pripada listi ili ne.
Ispisati zatim odgovarajuću poruku i rekurzivno osloboditi memoriju koju je data lista zauzimala.
Student se traži na osnovu broja indeksa, koji se zadaje sa standardnog ulaza.
\begin{maxitest}
    \begin{test}{Test 1}
Poziv: ./a.out studenti.txt
Datoteka:                 Ulaz:       Izlaz:
123/2014 Marko Lukic      3/2014      da: Ana Sokic
3/2014 Ana Sokic          235/2008    ne
43/2013 Jelena Ilic       41/2009     da: Marija Zaric
41/2009 Marija Zaric
13/2010 Milovan Lazic
  \end{test}
\end{maxitest}

\begin{miditest}
  \begin{test}{Test 2}
Poziv: ./a.out
Izlaz: Greska! Program se poziva sa: ./a.out studenti.txt!
  \end{test}
\end{miditest}  


\begin{miditest}
  \begin{test}{Test 5}
Poziv: ./a.out  studenti.txt
studenti.txt ne postoji
Izlaz: Greska prilikom otvaranja datoteke studenti.txt
  \end{test}
\end{miditest}
  
  
\begin{miditest}
  \begin{test}{Test 5}
Poziv: ./a.out  studenti.txt
studenti.txt prazna
Izlaz: ???  videti sta ce tacno biti 
  \end{test}
\end{miditest}

\end{Exercise}
\begin{Answer}[ref=610]
% \includecode{resenja/06_Liste/601.c}
\end{Answer}

%% Dodatni


\begin{Exercise}[label=611]
Neka su date dve jednostruko povezane liste L1 i L2. Napisati funkciju koja od 
tih lista formira novu listu L koja sadrži alternirajući raspoređene elemente 
lista L1 i L2 (prvi element iz L1, prvi element iz L2, drugi element L1,
drugi element L2, itd). Ne formirati nove čvorove, već samo postojeće čvorove 
rasporediti u jednu listu. Prva lista se učitava iz datoteke koja se zadaje 
kao prvi argument komandne linije, a druga iz datoteke čije se ime zadaje kao 
drugi argument komandne linije. Rezultujuću listu ispisati 
na standardni izlaz. \komentar{Milena: Sta ako je neka lista duza? To precizirati. I ovde me muci
sto nedostaje neki smisao zadatku, nesto sto ne bi moglo da se uradi da nismo koristili liste. }

\begin{miditest}
  \begin{test}{Test 1}
Poziv: ./a.out dat1.txt dat2.txt
dat1.txt: 2 4 6 10 15
dat2.txt: 5 6 11 12 14 16
Izlaz:  2 5 4 6 6 11 10 12 15 14 16
  \end{test}
\end{miditest}

\begin{miditest}
  \begin{test}{Test 2}
Poziv: ./a.out
Izlaz: Greska! Program se poziva sa: ./a.out dat1.txt dat2.txt!
  \end{test}
\end{miditest}  
 
 
\begin{miditest}
  \begin{test}{Test 3}
Poziv: ./a.out dat1.txt 
Izlaz: Greska! Program se poziva sa: ./a.out dat1.txt dat2.txt!
  \end{test}
\end{miditest}


\begin{miditest}
  \begin{test}{Test 4}
Poziv: ./a.out dat1.txt dat2.txt
dat1.txt: 2 4 6 10 15
dat2.txt ne postoji
Izlaz: Greska prilikom otvaranja datoteke dat2.txt.
  \end{test}
\end{miditest}

\begin{miditest}
  \begin{test}{Test 5}
Poziv: ./a.out dat1.txt dat2.txt
dat1.txt ne postoji
dat2.txt: 2 4 6 10 15
Izlaz: Greska prilikom otvaranja datoteke dat1.txt.
  \end{test}
\end{miditest}
  
  
\begin{miditest}
  \begin{test}{Test 6}
Poziv: ./a.out dat1.txt dat2.txt
dat1.txt prazna
dat2.txt: 2 4 6 10 15
Izlaz: 2 4 6 10 15
  \end{test}
\end{miditest}
  
\end{Exercise}
\begin{Answer}[ref=611]
% \includecode{resenja/06_Liste/601.c}
\end{Answer}



\begin{Exercise}[label=612]
Data je datoteka brojevi.txt koja sadrži cele brojeve.
\begin{enumerate}
 \item Napisati funkciju koja iz zadate datoteke učitava brojeve i smešta ih u listu.
 \item Napisati funkciju koja u jednom prolazu kroz zadatu listu celih brojeva 
pronalazi maximalan strogo rastući podniz.
\end{enumerate}
Napisati program koji u datoteku \kckod{Rezultat.txt} upisuje nađeni strogo rastući podniz.
\komentar{Milena: I ovde me muci sto bi zadatak mogao da se resi i bez koriscenja listi... }

\komentar{Milena: Prirodni oblik testa ovde bi bio horizontalan, a ne ovako vertikalan.}
\begin{miditest}
  \begin{test}{Test 1}
Poziv: ./a.out
brojevi.txt       
       43 12 15 16 4 2 8
Izlaz: 
Rezultat.txt : 12 15 16
  \end{test}
\end{miditest}



\begin{miditest}
  \begin{test}{Test 2}
Poziv: ./a.out
brojevi.txt ne postoji
Izlaz:
Rezultat.txt : Greska prilikom otvaranja datoteke dat2.txt.
  \end{test}
\end{miditest}

\begin{miditest}
  \begin{test}{Test 3}
Poziv: ./a.out 
brojevi.txt prazna
Izlaz: 
Rezultat.txt  je prazna.
  \end{test}
\end{miditest}
  
  
\begin{miditest}
  \begin{test}{Test 6}
Poziv: ./a.out dat1.txt dat2.txt
dat1.txt prazna
dat2.txt: 2 4 6 10 15
Izlaz: 2 4 6 10 15
  \end{test}
\end{miditest}


\end{Exercise}
\begin{Answer}[ref=612]
\includecode{resenja/06_Liste/601.c}
\end{Answer}



\begin{Exercise}[label=613]
Grupa od $n$ plesača na kostimima imaju brojeve od $1$ do $n$, redom, u smeru kazaljke na satu.
Plesači izvode svoju plesnu tačku tako što formiraju krug iz kog najpre izlazi $k$-ti plesač.
Odbrojava se počevši od plesača označenog brojem $1$ u smeru kretanja kazaljke na satu. 
Preostali plesači obrazuju manji krug iz kog opet izlazi $k$-ti plesač. Odbrojavanje počinje od
sledećeg suseda prethodno izbačenog, opet u smeru kazaljke na satu. Izlasci iz kruga se nastavljaju
sve dok svi plesači ne budu isključeni. 
Celi brojevi $n$, $k$ ($k < n$) se učitavaju sa standardnog ulaza. 

Napisati program koji će na standardni izlaz ispisati redne brojeve plesača u redosledu napuštanja kruga. 
Uputstvo: u implementaciji koristiti kružnu listu.
\begin{maxitest}
  \begin{test}{Test 1}
Ulaz: 5 3 
Izlaz: 3 1 5 2 4
  \end{test}
\end{maxitest}

\end{Exercise}
\begin{Answer}[ref=613]
% \includecode{resenja/06_Liste/601.c}
\end{Answer}

\begin{Exercise}[label=614]
Grupa od $n$ plesača na kostimima imaju brojeve od $1$ do $n$, redom, u smeru kazaljke na satu.
Plesači izvode svoju plesnu tačku tako što formiraju krug iz kog najpre izlazi $k$-ti plesač.
Odbrojava se počevši od plesača označenog brojem $1$ u smeru kretanja kazaljke na satu. 
Preostali plesači obrazuju manji krug iz kog opet izlazi $k$-ti plesač. Odbrojavanje počinje od
sledećeg suseda prethodno izbačenog, uz promenu smera. Ukoliko se prilikom prethodnog izbacivanja odbrojavalo 
u smeru kazaljke na satu sada će se obrojavati u suprotnom smeru, i obrnuto. Izlasci iz kruga se nastavljaju
sve dok svi plesači ne budu isključeni. 
Celi brojevi $n$, $k$ ($k < n$) se učitavaju sa standardnog ulaza. 

Napisati program koji će na standardni izlaz ispisati redne brojeve plesača u redosledu napuštanja kruga. 
Uputstvo: u implementaciji koristiti dvostruko povezanu kružnu listu.

\begin{minitest}
  \begin{test}{Test 1}
Ulaz: 5 3 
Izlaz: 3 5 4 2 1
  \end{test}
\end{minitest}

\begin{minitest}
  \begin{test}{Test 1}
Ulaz: 2 7 
Izlaz: n mora biti uvek vece od k, a 2 < 7!
  \end{test}
\end{minitest}

\end{Exercise}
\begin{Answer}[ref=614]
% \includecode{resenja/06_Liste/601.c}
\end{Answer}

