
\chapter{Dinamičke strukture podataka}

\section{Liste}

%=========================================================================
\begin{Exercise}[label=601]
Napisati biblioteku za rad sa jednostruko povezanom listom, čiji elementi sadrže cele brojeve. 
\begin{enumerate}
\item Definisati strukturu \kckod{Cvor} koja predstavlja čvor liste.
\item Napisati funkciju koja kao argument dobija ceo broj, kreira nov čvor liste sa tom vrednosti i vreća adresu kreiranog čvora.
 \item Napisati funkciju koja dodaje novi elemenat na početak liste. 
 \item Napisati funkciju koja dodaje novi elemenat na kraj liste. 
 \item Napisati funkciju koja dodaje novi elemenat u neopadajuće uređenu listu tako da se očuva postojeće uređenje.
 \item Napisati funkciju koja ispisuje elemente liste, uokvirene zagradama [, ] i međusobno razdvojene zapetama.
 \item Napisati funkciju koja proverava da li se u listi nalazi element čija se vrednost zadaje kao argument funkcije. 
 \item Napisati funkciju koja proverava da li se u listi nalayi element čija se vrednost zadaje kao argument funkcije, pri čemu se pretpostavlja da se pretraživanje vrši nad neopadajuće uređenoj listi.
 \item Napisati funkciju koja briše sve elemente u listi koji imaju vrednost koja  se zadaje kao argument funkcije.
 \item Napisati funkciju koja briše sve elemente u listi koji imaju vrednost koja  se zadaje kao argument funkcije, pri čemu se pretpostavlja da se pretraživanje vrši nad neopadajuće uređenoj listi.
 \item Napisati funkciju koja oslobađa dinamički zauzetu memoriju za elemente liste.
 \end{enumerate}
\napomena{sve funkcije za rad sa listom implementirati iterativno.}

Napisati programe koji koriste jednostruko povezanu listu za čuvanje elemenata koji se unose sa standardnog ulaza.  Unošenje novih brojeva u listu prekida se učitavanjem kraja ulaza (EOF). Svako dodavanje novog broja u listu ispratiti ispisivanjem trenutnog sadržaja liste. 

\begin{enumerate}
\item[(1)] U programu se učitani celi brojevi dodaju na početak liste. 
    Unosi se ceo broj koji se traži u unetoj listi, i na ekran se ispisuje rezultat pretrage. 

\begin{maxitest}
\begin{test}{Upotreba programa 1}
Poziv: ./a.out
Ulaz:
  Unosite elemente liste (za kraj unesite CTRL+D): 2 3 14 5 3 3 17 3 1 9 
  Unesite element koji se trazi u listi: 17
Izlaz:
  Lista: []
  Lista: [2]
  Lista: [3, 2]
  Lista: [14, 3, 2]
  Lista: [5, 14, 3, 2]
  Lista: [3, 5, 14, 3, 2]
  Lista: [3, 3, 5, 14, 3, 2]
  Lista: [17, 3, 3, 5, 14, 3, 2]
  Lista: [3, 17, 3, 3, 5, 14, 3, 2]
  Lista: [1, 3, 17, 3, 3, 5, 14, 3, 2]
  Lista: [9, 1, 3, 17, 3, 3, 5, 14, 3, 2]

  Trazeni broj 17 je u listi!
\end{test}
\end{maxitest}
\begin{maxitest}
\begin{test}{Upotreba programa 2}
Poziv: ./a.out
Ulaz:
  Unosite elemente liste (za kraj unesite CTRL+D): 23 14 35
  Unesite element koji se trazi u listi: 8
Izlaz:
  Lista: []
  Lista: [23]
  Lista: [14, 23]
  Lista: [35, 14, 23]
  
  Element nije u listi!
\end{test}
\end{maxitest}
\begin{maxitest}
\begin{test}{Upotreba programa 3}
Poziv: ./a.out
Ulaz:
  Unosite elemente liste (za kraj unesite CTRL+D): 
  Unesite element koji se trazi u listi: 1
Izlaz:
  Lista: []
  
  Element nije u listi!
\end{test}
\end{maxitest}


\item[(2)] U programu se učitani celi brojevi dodaju na kraj liste. 
    Unosi se ceo broj čija se sva pojavljivanja u listi brišu. Na ekran se ispisuje sadržaj liste nakon brisanja.

\begin{maxitest}
\begin{test}{Upotreba programa 1}
Poziv: ./a.out
Ulaz:
  Unosite elemente liste (za kraj unesite CTRL+D): 2 3 14 5 3 3 17 3 1 3
  Unesite element koji se brise iz liste: 3
Izlaz:
  Lista: []
  Lista: [2]
  Lista: [2, 3]
  Lista: [2, 3, 14]
  Lista: [2, 3, 14, 5]
  Lista: [2, 3, 14, 5, 3]
  Lista: [2, 3, 14, 5, 3, 3]
  Lista: [2, 3, 14, 5, 3, 3, 17]
  Lista: [2, 3, 14, 5, 3, 3, 17, 3]
  Lista: [2, 3, 14, 5, 3, 3, 17, 3, 1]
  Lista: [2, 3, 14, 5, 3, 3, 17, 3, 1, 3]
  
  Lista nakon brisanja:  [2, 14, 5, 17, 1]
\end{test}
\end{maxitest}
\begin{maxitest}
\begin{test}{Upotreba programa  2}
Poziv: ./a.out
Ulaz:
  Unosite elemente liste (za kraj unesite CTRL+D): 23 14 35
  Unesite element koji se brise iz liste: 3
Izlaz:
  Lista: []
  Lista: [23]
  Lista: [23, 14]
  Lista: [23, 14, 35]

  Lista nakon brisanja:  [23, 14, 35]
\end{test}
\end{maxitest}
\begin{maxitest}
\begin{test}{Upotreba programa 3}
Poziv: ./a.out
Ulaz:
  Unosite elemente liste (za kraj unesite CTRL+D): 
  Unesite element koji se brise iz liste: 12
Izlaz:
  Lista: []
  
  Lista nakon brisanja:  []
\end{test}
\end{maxitest}



\item[(3)] U glavnom programu se učitani celi brojevi dodaju u listu tako da je elementi budu uređeni u neopadajućem poretku. 
    Unosi ceo broj koji se traži u unetoj listi, i na ekran se ispisuje rezultat pretrage. 
    Potom se unosi još jedan ceo broj čija se sva pojavljivanja u listi brišu i prikazuje se aktuelni sadržaj liste nakon brisanja.
    \napomena{prilikom pretraživanja i brisanja elementa liste koristiti činjenicu da je lista uređena.}
\begin{maxitest}
\begin{test}{Upotreba programa 1}
Poziv: ./a.out
Ulaz:
  Unosite elemente liste (za kraj unesite CTRL+D): 2 3 14 5 3 3 17 3 1 9
  Unesite element koji se trazi u listi: 5
  Unesite element koji se brise iz liste: 3
Izlaz:
  Lista: []
  Lista: [2]
  Lista: [2, 3]
  Lista: [2, 3, 14]
  Lista: [2, 3, 5, 14]
  Lista: [2, 3, 3, 5, 14]
  Lista: [2, 3, 3, 3, 5, 14]
  Lista: [2, 3, 3, 3, 5, 14, 17]
  Lista: [2, 3, 3, 3, 3, 5, 14, 17]
  Lista: [1, 2, 3, 3, 3, 3, 5, 14, 17]
  Lista: [1, 2, 3, 3, 3, 3, 5, 9, 14, 17]
  
  Trazeni broj 5 je u listi!
  Lista nakon brisanja:  [1, 2, 5, 9, 14, 17]
\end{test}
\end{maxitest}
\begin{maxitest}
\begin{test}{Upotreba programa 2}
Poziv: ./a.out
Ulaz:
  Unosite elemente liste (za kraj unesite CTRL+D): 
  Unesite element koji se trazi u listi: 1
  Unesite element koji se brise iz liste: 12
Izlaz:
  Lista: []
  
  Element nije u listi!
  Lista nakon brisanja:  []
\end{test}
\end{maxitest}
\begin{maxitest}
\begin{test}{Upotreba programa 3}
Poziv: ./a.out
Ulaz:
  Unosite elemente liste (za kraj unesite CTRL+D): 23 14 35
  Unesite element koji se trazi u listi: 8
  Unesite element koji se brise iz liste: 3
Izlaz:
  Lista: []
  Lista: [23]
  Lista: [14, 23]
  Lista: [14, 23, 35]
  
  Element nije u listi!
  Lista nakon brisanja:  [14, 23, 35]
\end{test}
\end{maxitest}
\end{enumerate}

\linkresenje{601}
\end{Exercise}
\begin{Answer}[ref=601]
\includecode{resenja/06_Liste/601/lista.h}
\includecode{resenja/06_Liste/601/lista.c}
\includecode{resenja/06_Liste/601/main_a.c}
\includecode{resenja/06_Liste/601/main_b.c}
\includecode{resenja/06_Liste/601/main_c.c}
\end{Answer}

%=========================================================================
\begin{Exercise}[label=602]
Napisati biblioteku za rad sa jednosturko povezanim listama koja sadrži sve funkcije iz zadatka \ref{601}, ali tako da funkcije budu implementirane rekurzivno. 
\napomena{koristiti iste \kckod{main} programe i upotrebe programa iz zadatka \ref{601}.}
%\uputstvo{U slučaju da je u rekurzivnom pozivu funkcija koje dodaju nove elemente u listu došlo do greške pri alokaciji, funkcija treba da vrati 1 višem rekurzivnom pozivu koji tu informaciju prosleđuje u rekurzivni poziv iznad, i tako sve dok se ta informacija ne vrati u main. Ako je poziv funkcije u glavnom programu   vratio 0, onda nije bilo greške. Ukoliko je vraćeno 1, dogodila se greška, i tad treba iz main funkcije pristupiti pravom početku liste i osloboditi je celu.}
\linkresenje{602}
\end{Exercise}
\begin{Answer}[ref=602]
\includecode{resenja/06_Liste/602/lista.h}
\includecode{resenja/06_Liste/602/lista.c}
\includecode{resenja/06_Liste/602/main_a.c}
\includecode{resenja/06_Liste/602/main_b.c}
\includecode{resenja/06_Liste/602/main_c.c}
\end{Answer}

%=========================================================================
\begin{Exercise}[label=603]
Napisati biblioteku za rad sa dvostruko povezanom listom celih brojeva, koja ima iste funkcionalnosti kao biblioteka iz zadatka \ref{601}. 
Dopuniti bibilioteku novim funkcijama.
\begin{enumerate}
 \item Napisati funkciju koja iz prosleđene liste briše element na koju pokazuje pokazivač koji se funkciji šalje kao argument.
 \item Napisati funkciju koja ispisuje sadržaj liste od poslednjeg elementa ka glavi liste.
\end{enumerate}

Sve funkcije za rad sa listom implementirati iterativno.
\napomena{koristiti iste \kckod{main} programe i upotrebe programa iz zadatka \ref{601}. Ove programe dopuniti pozivom funkcije koja ispisuje listu u nazad. }
\linkresenje{603}
\end{Exercise}
\begin{Answer}[ref=603]
\includecode{resenja/06_Liste/603/lista.h}
\includecode{resenja/06_Liste/603/lista.c}
\includecode{resenja/06_Liste/603/main_a.c}
\includecode{resenja/06_Liste/603/main_b.c}
\includecode{resenja/06_Liste/603/main_c.c}
\end{Answer}
%=========================================================================

\begin{Exercise}[label=604]
Sadržaj datoteke je aritmetički izraz koji može sadržati zagrade \{, [ i (. 
Napisati program koji učitava sadržaj datoteke \kckod{dat.txt} i korišćenjem steka 
utvrđuje da li su zagrade u aritmetičkom izrazu dobro uparene. Program štampa odgovarajuću poruku na standardni izlaz.

\noindent
\begin{miditest}
  \begin{test}{Test 1}
Datoteka: 
  {[23 + 5344] * (24 - 234)} - 23
Izlaz:  
  Zagrade su ispravno uparene.
\end{test}
\end{miditest}
\begin{miditest}
\begin{test}{Test 2}
Datoteka: 
  {[23 + 5] * (9 * 2)} - {23}
Izlaz:  
  Zagrade su ispravno uparene.
\end{test}
\end{miditest}
\begin{miditest}
\begin{test}{Test 3}
Datoteka: 
  {[2 + 54) / (24 * 87)} + (234 + 23)
Izlaz:  
  Zagrade nisu ispravno uparene.
\end{test}
\end{miditest}
\begin{miditest}
\begin{test}{Test 3}
Datoteka: 
  {(2 - 14) / (23 + 11)}} * (2 + 13)
Izlaz:  
  Zagrade nisu ispravno uparene.
\end{test}
\end{miditest}
\begin{miditest}
\begin{test}{Test 4}
Datoteka je prazna.
Izlaz:  
  Zagrade su ispravno uparene.
\end{test}
\end{miditest}
\begin{miditest}
\begin{test}{Test 5}
Datoteka ne postoji. 
Izlaz:  
  Greska prilikom otvaranja 
  datoteke izraz.txt!
\end{test}
\end{miditest}
\linkresenje{604}
\end{Exercise}
\begin{Answer}[ref=604]
\includecode{resenja/06_Liste/604.c}
\end{Answer}
%=========================================================================

\begin{Exercise}[label=605]
Napisati program koji proverava ispravnost uparivanja etiketa u \kckod{HTML} datoteci. Ime datoteke se zadaje kao argument komandne linije.
Poruke o greškama ispisivati na standardni izlaz za greške.
\uputstvo{za rešavanje problema koristiti stek implementiran preko liste čiji su čvorovi \kckod{HTML} etikete.}
\begin{maxitest}
\begin{test}{Test 1}
Poziv: ./a.out datoteka.html
Datoteka.html:                          
<html>                                  
  <head>
    <title>Primer</title>
  </head>               
  <body>                                           
    <h1>Naslov</h1>                                
    Danas je lep i suncan dan. <br>                
    A sutra ce biti jos lepsi.     
    <a link="http://www.google.com"> Link 1</a>    
    <a link="http://www.math.rs"> Link 2</a>
  </body>
</html>
Izlaz: 
  Ispravno uparene etikete.
\end{test}
\end{maxitest}
\begin{miditest}
\begin{test}{Test 2}
Poziv: ./a.out 
Izlaz: 
  Greska! Program se poziva 
  sa: ./a.out datoteka.html!
\end{test}
\end{miditest}
\begin{miditest}
\begin{test}{Test 3}
Poziv: ./a.out datoteka.html
Datoteka.html ne postoji.
Izlaz: 
  Greska prilikom otvaranja 
  datoteke datoteka.html.
\end{test}
\end{miditest}
\begin{miditest}    
\begin{test}{Test 4}
Poziv: ./a.out datoteka.html
Datoteka.html:                         
<html>                                 
  <head>
  	<title>Primer</title>
  </head>               
  <body>  
</html>
Izlaz: 
  Neispravno uparene etikete.
\end{test}
\end{miditest}
\begin{miditest}      
\begin{test}{Test 5}
Poziv: ./a.out datoteka.html
Datoteka.html:                         
<html>                                  
  <head>
  	<title>Primer</title>
  </head>               
  <body>  
  </body>
Izlaz: 
  Neispravno uparene etikete.
\end{test}
\end{miditest}
\begin{miditest}
\begin{test}{Test 6}
Poziv: ./a.out datoteka.html
Datoteka.html je prazna
Izlaz: 
  Ispravno uparene etikete.
\end{test}
\end{miditest}
\linkresenje{605}
\end{Exercise}
\begin{Answer}[ref=605]
\includecode{resenja/06_Liste/605.c}
\end{Answer}
%=========================================================================

\begin{Exercise}[label=606]
Napisati program kojim se simulira rad jednog šaltera na kojem se prvo kod službenika zakazuju 
termini, a potom službenik uslužuje korisnike. 
Službenik evidentira korisničke \kckod{jmbg} brojeve (niske koje sadrže po $13$ karaktera) i zahteve (niska koja sadrži najviše $999$ karaktera). 
Prijem zahteva korisnika se prekida unošenjem karaktera za kraj ulaza, (\kckod{EOF}).
Službenik redom pregleda zahteve i odlučuje da li zahtev obrađuje odmah ili kasnije, tj.~službeniku se postavlja pitanje 
\kckod{Da li korisnika  vracate na kraj reda?} i ukoliko on da odgovor \kckod{Da}, 
korisnik se stavlja na kraj reda, čime se obrada njegovog zahteva odlaže. Ukoliko odgovor nije \kckod{Da}, tada službenik obrađuje zahtev i podatke o korisniku dopisuje na kraj datoteke \kckod{izvestaj.txt}. Ova datoteka, za svaki obrađen zahtev, sadrži \kckod{jmbg} i zahtev usluženog korisnika.
Posle svakog $5$ usluženog korisnika, službeniku se nudi mogućnost da prekine sa radom, nevezano od broja korisnika koji i dalje čekaju u redu. 
\uputstvo{Za čuvanje korisničkih zahteva koristiti red implementiran korišćenjem listi.}

\komentar{Milena: Ova upotreba mora da se skrati tako da stane na jednu stranu. To znaci da treba da se eliminise bar 5 redova.}

\begin{maxitest}
\begin{test}{Upotreba programa 1}
Sluzbenik evidentira korisnicke zahteve unosenjem 
njihovog JMBG broja i opisa potrebne usluge:

Novi zahtev [CTRL+D za kraj]
        JMBG: 1234567890123
        Opis problema: Otvaranje racuna

Novi zahtev [CTRL+D za kraj]
        JMBG: 2345678901234
        Opis problema: Podizanje novca

Novi zahtev [CTRL+D za kraj]
        JMBG: 3456789012345
        Opis problema: Reklamacija

Novi zahtev [CTRL+D za kraj]
        JMBG: 4567890123456
        Opis problema: Zatvaranje racuna

Novi zahtev [CTRL+D za kraj]
        JMBG: 5678901234567
        Opis problema: Podizanje kredita

Novi zahtev [CTRL+D za kraj]
        JMBG: 
Sledeci je korisnik sa JMBG brojem: 1234567890123
sa zahtevom: Otvaranje racuna
        Da li ga vracate na kraj reda? [Da/Ne] Da

Sledeci je korisnik sa JMBG brojem: 2345678901234
sa zahtevom: Podizanje novca
        Da li ga vracate na kraj reda? [Da/Ne] Ne

Sledeci je korisnik sa JMBG brojem: 3456789012345
sa zahtevom: Reklamacija
        Da li ga vracate na kraj reda? [Da/Ne] Da

Sledeci je korisnik sa JMBG brojem: 4567890123456
sa zahtevom: Zatvaranje racuna
        Da li ga vracate na kraj reda? [Da/Ne] Ne

Sledeci je korisnik sa JMBG brojem: 5678901234567
sa zahtevom: Podizanje kredita
        Da li ga vracate na kraj reda? [Da/Ne] Da

Da li je kraj smene? [Da/Ne] Ne

Sledeci je korisnik sa JMBG brojem: 1234567890123
sa zahtevom: Otvaranje racuna
        Da li ga vracate na kraj reda? [Da/Ne] Ne

Sledeci je korisnik sa JMBG brojem: 3456789012345
sa zahtevom: Reklamacija
        Da li ga vracate na kraj reda? [Da/Ne] Ne

Sledeci je korisnik sa JMBG brojem: 5678901234567
sa zahtevom: Podizanje kredita
        Da li ga vracate na kraj reda? [Da/Ne] Ne

izvestaj.txt: 
  JMBG: 2345678901234     Zahtev: Podizanje novca
  JMBG: 4567890123456     Zahtev: Zatvaranje racuna
  JMBG: 1234567890123     Zahtev: Otvaranje racuna
  JMBG: 3456789012345     Zahtev: Reklamacija
  JMBG: 5678901234567     Zahtev: Podizanje kredita
\end{test}
\end{maxitest}
\linkresenje{606}
\end{Exercise}
\begin{Answer}[ref=606]
\includecode{resenja/06_Liste/606/red.h}
\includecode{resenja/06_Liste/606/red.c}
\includecode{resenja/06_Liste/606/main_a.c}
\includecode{resenja/06_Liste/606/main_b.c}
\end{Answer}

%%% Obavezni sa praktikuma
%=========================================================================
\begin{Exercise}[label=607]
Napisati program koji prebrojava pojavljivanja etiketa \kckod{HTML} 
datoteke čije se ime zadaje kao argument komandne linije. Rezultat prebrojavanja 
ispisati na standardni izlaz. Etikete smeštati u listu, a za formiranje liste koristiti strukturu:
\begin{ckod} 
 typedef struct _Element
 {
   unsigned broj_pojavljivanja;
   char etiketa[20];
   struct _Element *sledeci;
 } Element;
\end{ckod}
\begin{maxitest}
\begin{test}{Test 1}
Poziv: ./a.out datoteka.html                             Izlaz:  
Datoteka.html:                                             a - 4
<html>                                                     br - 1
  <head><title>Primer</title></head>                       h1 - 2
  <body>                                                   body - 2
    <h1>Naslov</h1>                                        title - 2
    Danas je lep i suncan dan. <br>                        head - 2
    A sutra ce biti jos lepsi.                             html - 2
    <a link="http://www.google.com"> Link 1</a>    
    <a link="http://www.math.rs"> Link 2</a>
  </body>
</html>
\end{test}
\end{maxitest}
\begin{miditest}
\begin{test}{Test 2}
Poziv: ./a.out 
Izlaz: 
  Greska! Program se poziva 
  sa: ./a.out datoteka.html!
\end{test}
\end{miditest}
\begin{miditest}
\begin{test}{Test 3}
Poziv: ./a.out datoteka.html
Datoteka.html ne postoji.
Izlaz: 
  Greska prilikom otvaranja 
  datoteke datoteka.html.
\end{test}
\end{miditest}
\begin{miditest}
\begin{test}{Test 4}
Poziv: ./a.out datoteka.html
Datoteka.html je prazna.
Izlaz: 
\end{test}
\end{miditest}
\linkresenje{607}
\end{Exercise}
\begin{Answer}[ref=607]
\includecode{resenja/06_Liste/607.c}
\end{Answer}

%=========================================================================
\begin{Exercise}[label=608]
%\komentar{Milena: malo me muci u ovom zadatku sto nema neki smisao. Naime, ako se samo vrsi ucitavanje iz datoteka i ispisivanje,  onda su ove liste zapravo visak jer isti rezultat moze da se dobije i bez koriscenja listi. Zato mi fali da program uradi nesto sto ne bi mogao da uradi bez koriscenja listi, npr da na osnovu unetog broja ispisuje svaki n-ti broj rezultujuce liste pa to u nekoj petlji da korisnik moze da ispisuje za razlicite unete n ili tako nesto... }
Napisati program koji objedinjuje dve sortirane liste. Funkcija ne treba da 
kreira nove čvorove, već da samo
postojeće čvorove preraspodeli. Prva lista se učitava iz datoteke koja se 
zadaje kao prvi argument komandne
linije, a druga iz datoteke čije se ime zadaje kao drugi argument komandne linije. Rezultujuću listu ispisati na standardni izlaz.
\napomena{koristiti biblioteku za rad sa listama celih brojeva iz zadatka \ref{601}.}

\noindent
\begin{miditest}
\begin{test}{Test 1}
Poziv: ./a.out dat1.txt dat2.txt
dat1.txt: 
  2 4 6 10 15
dat2.txt: 
  5 6 11 12 14 16
Izlaz: 
  2 4 5 6 6 10 11 12 14 15 16
\end{test}
\end{miditest}
\begin{miditest}
\begin{test}{Test 2}
Poziv: ./a.out
Izlaz: 
  Greska! Program se poziva sa: 
  ./a.out dat1.txt dat2.txt!
\end{test}
\end{miditest}  
\begin{miditest}
\begin{test}{Test 3}
Poziv: ./a.out dat1.txt 
Izlaz: 
  Greska! Program se poziva sa: 
  ./a.out dat1.txt dat2.txt!
\end{test}
\end{miditest}
\begin{miditest}
\begin{test}{Test 4}
Poziv: ./a.out dat1.txt dat2.txt
dat1.txt: 
  2 4 6 10 15
dat2.txt ne postoji
Izlaz: 
  Greska prilikom otvaranja datoteke 
  dat2.txt.
\end{test}
\end{miditest}
\begin{miditest}
\begin{test}{Test 5}
Poziv: ./a.out dat1.txt dat2.txt
dat1.txt ne postoji
dat2.txt: 2 4 6 10 15
Izlaz: 
  Greska prilikom otvaranja datoteke 
  dat1.txt.
\end{test}
\end{miditest}  
\begin{miditest}
\begin{test}{Test 6}
Poziv: ./a.out dat1.txt dat2.txt
dat1.txt prazna
dat2.txt: 2 4 6 10 15
Izlaz:
  2 4 6 10 15
\end{test}
\end{miditest}
\linkresenje{608}  
\end{Exercise}
\begin{Answer}[ref=608]
\includecode{resenja/06_Liste/608.c}
\end{Answer}

%=========================================================================
\begin{Exercise}[label=609]
Napisati funkciju koja dodaje na početak liste studenata novog studenta za koga su nam poznati broj indeksa, ime
i prezime. 
Napisati rekurzivnu funkciju koja određuje da li neki student pripada listi ili ne.
U glavnom programu učitati u listu sve podatke o studentima iz datoteke čije se ime zadaje kao argument komandne linije. U svakom redu datoteke nalaze se podaci o studentu i to broj indeksa, ime
i prezime.
Nakon toga se sa standardnog ulaza unose, jedan po jedan, indeksi studenata koji se traže u učitanoj listi. Posle svakog unetog indeksa, program ispisuje poruku da ili ne,
u zavisnosti od toga da li u listi postoji student sa unetim indeksom ili ne. Prekid unosa indeksa se vrši unošenjem karaktera za kraj ulaza (EOF).
Nakon završenog pretraživanja rekurzivnom funkcijom osloboditi memoriju koju je data lista zauzimala.
\uputstvo{Pretpostaviti da je 10 karaktera dovoljno za zapis indeks i da je 20 karaketera maksimalna dužina bilo imena bilo prezimena studenta.}

\komentar{Milena: lepse formulisati zadatak. Poslednji test primer je neispravan.}

\noindent
\begin{maxitest}
\begin{test}{Test 1}
Poziv: ./a.out studenti.txt
studenti.txt:                 Ulaz:           Izlaz:
123/2014 Marko Lukic          3/2014          da: Ana Sokic
3/2014 Ana Sokic              235/2008        ne
43/2013 Jelena Ilic           41/2009         da: Marija Zaric
41/2009 Marija Zaric
13/2010 Milovan Lazic
\end{test}
\end{maxitest}
\begin{miditest}
\begin{test}{Test 2}
Poziv: ./a.out
Izlaz: 
  Greska! Program se poziva sa: 
  ./a.out studenti.txt!
\end{test}
\end{miditest}
\begin{miditest}
\begin{test}{Test 5}
Poziv: ./a.out  studenti.txt
studenti.txt ne postoji
Izlaz: 
  Greska prilikom otvaranja 
  datoteke studenti.txt
\end{test}
\end{miditest}	
\begin{maxitest}
\begin{test}{Test 5}
Poziv: ./a.out  studenti.txt
studenti.txt:             Ulaz:       Izlaz:
  prazna                  3/2014      ne
                          235/2008    ne
                          41/2009     ne
\end{test}
\end{maxitest}
\linkresenje{609}
\end{Exercise}
\begin{Answer}[ref=609]
\includecode{resenja/06_Liste/609.c}
\end{Answer}

%% Dodatni
%=========================================================================
\begin{Exercise}[label=610]
Date su dve jednostruko povezane liste \kckod{L1} i \kckod{L2}. Napisati funkciju koja od 
tih lista formira novu listu \kckod{L} koja sadrži alternirajući raspoređene elemente 
lista \kckod{L1} i \kckod{L2} (prvi element iz \kckod{L1}, prvi element iz \kckod{L2}, drugi element \kckod{L1},
drugi element \kckod{L2}, itd). Ne formirati nove čvorove, već samo postojeće čvorove 
rasporediti u jednu listu. Prva lista se učitava iz datoteke čije se ime zadaje kao prvi argument komandne linije, a druga iz datoteke čije se ime zadaje kao 
drugi argument komandne linije. Rezultujuću listu ispisati 
na standardni izlaz. 
%\komentar{Milena: Sta ako je neka lista duza? To precizirati. I ovde me muci sto nedostaje neki smisao zadatku, nesto sto ne bi moglo da se uradi da nismo koristili liste. }

\noindent
\begin{miditest}
\begin{test}{Test 1}
Poziv: ./a.out dat1.txt dat2.txt
dat1.txt: 2 4 6 10 15
dat2.txt: 5 6 11 12 14 16
Izlaz:  
  2 5 4 6 6 11 10 12 15 14 16
\end{test}
\end{miditest}
\begin{miditest}
\begin{test}{Test 2}
Poziv: ./a.out
Izlaz: 
  Greska! Program se poziva sa: 
  ./a.out dat1.txt dat2.txt!
\end{test}
\end{miditest}  
\begin{miditest}
\begin{test}{Test 3}
Poziv: ./a.out dat1.txt 
Izlaz: 
  Greska! Program se poziva sa: 
  ./a.out dat1.txt dat2.txt!
\end{test}
\end{miditest}
\begin{miditest}
\begin{test}{Test 4}
Poziv: ./a.out dat1.txt dat2.txt
dat1.txt: 2 4 6 10 15
dat2.txt ne postoji
Izlaz: 
  Greska prilikom otvaranja datoteke 
  dat2.txt.
\end{test}
\end{miditest}
\begin{miditest}
\begin{test}{Test 5}
Poziv: ./a.out dat1.txt dat2.txt
dat1.txt ne postoji
dat2.txt: 2 4 6 10 15
Izlaz: 
  Greska prilikom otvaranja datoteke 
  dat1.txt.
\end{test}
\end{miditest}
\begin{miditest}
\begin{test}{Test 6}
Poziv: ./a.out dat1.txt dat2.txt
dat1.txt prazna
dat2.txt: 2 4 6 10 15
Izlaz: 
  2 4 6 10 15
\end{test}
\end{miditest}
\end{Exercise}
\begin{Answer}[ref=610]
% \includecode{resenja/06_Liste/601.c}
\end{Answer}

%=========================================================================
\begin{Exercise}[label=611]
Data je datoteka \kckod{brojevi.txt} koja sadrži cele brojeve.
\begin{enumerate}
 \item Napisati funkciju koja iz zadate datoteke učitava brojeve i smešta ih u listu.
 \item Napisati funkciju koja u jednom prolazu kroz zadatu listu celih brojeva 
pronalazi maksimalan strogo rastući podniz.
\end{enumerate}
Napisati program koji u datoteku \kckod{Rezultat.txt} upisuje nađeni strogo rastući podniz.
%\komentar{Milena: I ovde me muci sto bi zadatak mogao da se resi i bez koriscenja listi... }

\noindent
\begin{miditest}
\begin{test}{Test 1}
Poziv: ./a.out
brojevi.txt:
  43 12 15 16 4 2 8
Izlaz: 
Rezultat.txt : 
  12 15 16
\end{test}
\end{miditest}
\begin{miditest}
\begin{test}{Test 2}
Poziv: ./a.out
brojevi.txt ne postoji
Izlaz:
Rezultat.txt: 
  Greska prilikom otvaranja datoteke 
  dat2.txt.
\end{test}
\end{miditest}
\begin{miditest}
\begin{test}{Test 3}
Poziv: ./a.out 
brojevi.txt prazna
Izlaz: 
Rezultat.txt  je prazna.
\end{test}
\end{miditest}
\begin{miditest}
\begin{test}{Test 4}
Poziv: ./a.out dat1.txt dat2.txt
dat1.txt prazna
dat2.txt: 
  2 4 6 10 15
Izlaz:
  2 4 6 10 15
\end{test}
\end{miditest}

\end{Exercise}
\begin{Answer}[ref=611]
% \includecode{resenja/06_Liste/601.c}
\end{Answer}

%=========================================================================
\begin{Exercise}[label=612]
Grupa od $n$ plesača na kostimima imaju brojeve od $1$ do $n$, redom, u smeru kazaljke na satu.
Plesači izvode svoju plesnu tačku tako što formiraju krug iz kog najpre izlazi $k$-ti plesač.
Odbrojava se počevši od plesača označenog brojem $1$ u smeru kretanja kazaljke na satu. 
Preostali plesači obrazuju manji krug iz kog opet izlazi $k$-ti plesač. Odbrojavanje počinje od
sledećeg suseda prethodno izbačenog, opet u smeru kazaljke na satu. Izlasci iz kruga se nastavljaju
sve dok svi plesači ne budu isključeni. 
Celi brojevi $n$, $k$ ($k < n$) se učitavaju sa standardnog ulaza. 

Napisati program koji će na standardni izlaz ispisati redne brojeve plesača u redosledu napuštanja kruga. 
\uputstvo{Pri implementaciji koristiti dvostruko povezanu kružnu listu.}



\komentar{Milena: Dodati jos najmanje jedan test primer.}
\komentar{Milena: Da li je ovde potrebna dvostrukopovezna lista ili moze obicna kruzna lista?}

\begin{minitest}
\begin{test}{Test 1}
Ulaz: 
  5 3 
Izlaz: 
  3 1 5 2 4
\end{test}
\end{minitest}
\end{Exercise}
\begin{Answer}[ref=612]
% \includecode{resenja/06_Liste/601.c}
\end{Answer}

%=========================================================================
\begin{Exercise}[label=613]
Grupa od $n$ plesača na kostimima imaju brojeve od $1$ do $n$, redom, u smeru kazaljke na satu.
Plesači izvode svoju plesnu tačku tako što formiraju krug iz kog najpre izlazi $k$-ti plesač.
Odbrojava se počevši od plesača označenog brojem $1$ u smeru kretanja kazaljke na satu. 
Preostali plesači obrazuju manji krug iz kog opet izlazi $k$-ti plesač. Odbrojavanje počinje od
sledećeg suseda prethodno izbačenog, uz promenu smera. Ukoliko se prilikom prethodnog izbacivanja odbrojavalo 
u smeru kazaljke na satu sada će se obrojavati u suprotnom smeru, i obrnuto. Izlasci iz kruga se nastavljaju
sve dok svi plesači ne budu isključeni. 
Celi brojevi $n$, $k$ ($k < n$) se učitavaju sa standardnog ulaza. 
Napisati program koji će na standardni izlaz ispisati redne brojeve plesača u redosledu napuštanja kruga. 
\uputstvo{Pri implementaciji koristiti dvostruko povezanu kružnu listu.}

\begin{miditest}
\begin{test}{Test 1}
Ulaz: 
  5 3 
Izlaz: 
  3 5 4 2 1
\end{test}
\end{miditest}
\begin{miditest}
\begin{test}{Test 2}
Ulaz: 
  2 7 
Izlaz: 
  n mora biti uvek vece od k, a 2 < 7!
\end{test}
\end{miditest}

\end{Exercise}
\begin{Answer}[ref=613]
% \includecode{resenja/06_Liste/601.c}
\end{Answer}