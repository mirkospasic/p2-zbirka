
\chapter{Dinamičke strukture podataka}

\section{Liste}

%=========================================================================
\begin{Exercise}[label=601]
Napisati biblioteku za rad sa jednostruko povezanom listom, čiji čvorovi sadrže cele brojeve. 
\begin{enumerate}
\item Definisati strukturu \kckod{Cvor} kojom se predstavlja čvor liste. Treba da sadrži ceo broj \kckod{vrednost} i pokazivač na sledeći čvor liste.
\item Napisati funkciju \kckod{Cvor* napravi\_cvor(int broj)} koja kao argument dobija ceo broj, kreira nov čvor liste, inicijalizuje polja novog čvora i vreća njegovu adresu.
 \item Napisati funkciju \kckod{int dodaj\_na\_pocetak\_liste(Cvor** adresa\_glave, int broj)} koja dodaje novi čvor sa vrednošću \kckod{broj} na početak liste, čija glava se nalazi na adresi \kckod{adresa\_glave}.
 \item Napisati funkciju \kckod{Cvor * pronadji\_poslednji(Cvor * glava)} koja pronalazi poslednji čvor u listi.
 \item Napisati funkciju \kckod{int dodaj\_na\_kraj\_liste(Cvor ** adresa\_glave, int broj)} koja dodaje novi čvor sa vrednošću \kckod{broj} na kraj liste. 
 \item Napisati funkciju \kckod{Cvor * pronadji\_mesto\_umetanja(Cvor * glava, int broj)} koja vraća adresu čvora u neopadajuće uređenoj listi iza koga bi trebalo da dodati nov čvor sa vrednošću \kckod{broj}.
 \item Napisati funkciju \kckod{void dodaj\_iza(Cvor * tekuci, Cvor * novi)} koja uvezuje u postojeću listu čvor \kckod{novi} iza čvora \kckod{tekuci}.
 \item Napisati funkciju \kckod{int dodaj\_sortirano(Cvor ** adresa\_glave, int broj)} koja dodaje novi elemenat u neopadajuće uređenu listu tako da se očuva postojeće uređenje.
 \item Napisati funkciju \kckod{void ispisi\_listu(Cvor * glava)} koja ispisuje čvorove liste, uokvirene zagradama [, ] i međusobno razdvojene zapetama.
 \item Napisati funkciju \kckod{Cvor * pretrazi\_listu(Cvor * glava, int broj)} koja proverava da li se u listi nalazi čvor čija se vre-dnost zadaje kao argument funkcije. 
 \item Napisati funkciju \kckod{Cvor * pretrazi\_sortiranu\_listu(Cvor * glava, int broj)} koja proverava da li se u listi nalazi čvor čija se vre-dnost zadaje kao argument funkcije, pri čemu se pretpostavlja da se pretraživanje vrši nad neopadajuće uređenoj listi.
 \item Napisati funkciju \kckod{void obrisi\_cvor(Cvor ** adresa\_glave, int broj)} koja briše sve čvorove u listi koji imaju vrednost koja se zadaje kao argument funkcije.
 \item Napisati funkciju \kckod{void obrisi\_cvor\_sortirane\_liste(Cvor ** adresa\_glave, int broj)} koja briše sve čvorove u listi koji imaju vrednost koja se zadaje kao argument funkcije, pri čemu se pretpostavlja da se pretraživanje vrši nad neopadajuće uređenoj listi.
 \item Napisati funkciju \kckod{void oslobodi\_listu(Cvor** adresa\_glave)} koja oslobađa dinamički zauzetu memoriju za čvorove liste.
 \end{enumerate}
\napomena{Sve funkcije za rad sa listom implementirati iterativno.}

Napisati programe koji koriste jednostruko povezanu listu za čuvanje elemenata koji se unose sa standardnog ulaza.  Unošenje novih brojeva u listu prekida se učitavanjem kraja ulaza (EOF). Svako dodavanje novog broja u listu ispratiti ispisivanjem trenutnog sadržaja liste. 

\begin{enumerate}
\item[(1)] U programu se učitani celi brojevi dodaju na početak liste. 
    Unosi se ceo broj koji se traži u unetoj listi, i na ekran se ispisuje rezultat pretrage. 

\begin{maxitest}
\begin{upotreba}{1}
#\poziv{./a.out}#

#\naslovInt#
#\izlaz{Unosite brojeve: (za kraj unesite CTRL+D)}#
#\ulaz{2}#
#\izlaz{Lista: [2]}#
#\ulaz{3}#
#\izlaz{Lista: [3, 2]}#
#\ulaz{14}#
#\izlaz{Lista: [14, 3, 2]}#
#\ulaz{5}#
#\izlaz{Lista: [5, 14, 3, 2]}#
#\ulaz{3}#
#\izlaz{Lista: [3, 5, 14, 3, 2]}#
#\ulaz{3}#
#\izlaz{Lista: [3, 3, 5, 14, 3, 2]}#
#\ulaz{17}#
#\izlaz{Lista: [17, 3, 3, 5, 14, 3, 2]}#
#\ulaz{3}#
#\izlaz{Lista: [3, 17, 3, 3, 5, 14, 3, 2]}#
#\ulaz{1}#
#\izlaz{Lista: [1, 3, 17, 3, 3, 5, 14, 3, 2]}#
#\ulaz{9}#
#\izlaz{Lista: [9, 1, 3, 17, 3, 3, 5, 14, 3, 2]}#

#\izlaz{Unesite broj koji se trazi u listi:} \ulaz{17}#
#\izlaz{Trazeni broj 17 je u listi!}#
\end{upotreba}
\end{maxitest}
\begin{maxitest}
\begin{upotreba}{2}
#\poziv{./a.out}#

#\naslovInt#
#\izlaz{Unosite brojeve: (za kraj unesite CTRL+D)}#
#\ulaz{23}#
#\izlaz{Lista: [23]}#
#\ulaz{14}#
#\izlaz{Lista: [14, 23]}#
#\ulaz{35}#
#\izlaz{Lista: [35, 14, 23]}#

#\izlaz{Unesite broj koji se trazi u listi:} \ulaz{8}#
#\izlaz{Broj 8 se ne nalazi u listi!}#
\end{upotreba}
\end{maxitest}
\begin{maxitest}
\begin{upotreba}{3}
#\poziv{./a.out}#

#\naslovInt#
#\izlaz{Unosite brojeve: (za kraj unesite CTRL+D)}#

#\izlaz{Unesite broj koji se trazi u listi:} \ulaz{1}#
#\izlaz{Broj 1 se ne nalazi u listi!}#
\end{upotreba}
\end{maxitest}


\item[(2)] U programu se učitani celi brojevi dodaju na kraj liste. 
    Unosi se ceo broj čija se sva pojavljivanja u listi brišu. Na ekran se ispisuje sadržaj liste nakon brisanja.

\begin{maxitest}
\begin{upotreba}{1}
#\poziv{./a.out}#

#\naslovInt#
#\izlaz{Unosite brojeve: (za kraj unesite CTRL+D)}#
#\ulaz{2}#
#\izlaz{Lista: [2]}#
#\ulaz{3}#
#\izlaz{Lista: [2, 3]}#
#\ulaz{14}#
#\izlaz{Lista: [2, 3, 14]}#
#\ulaz{5}#
#\izlaz{Lista: [2, 3, 14, 5]}#
#\ulaz{3}#
#\izlaz{Lista: [2, 3, 14, 5, 3]}#
#\ulaz{3}#
#\izlaz{Lista: [2, 3, 14, 5, 3, 3]}#
#\ulaz{17}#
#\izlaz{Lista: [2, 3, 14, 5, 3, 3, 17]}#
#\ulaz{3}#
#\izlaz{Lista: [2, 3, 14, 5, 3, 3, 17, 3]}#
#\ulaz{1}#
#\izlaz{Lista: [2, 3, 14, 5, 3, 3, 17, 3, 1]}#
#\ulaz{3}#
#\izlaz{Lista: [2, 3, 14, 5, 3, 3, 17, 3, 1, 3]}#

#\izlaz{Unesite broj koji se brise iz liste:} \ulaz{3}#
#\izlaz{Lista nakon brisanja:  [2, 14, 5, 17, 1]}#
\end{upotreba}
\end{maxitest}
\begin{maxitest}
\begin{upotreba}{2}
#\poziv{./a.out}#

#\naslovInt#
#\izlaz{Unesite brojeve: (za kraj unesite CTRL+D)}#
#\ulaz{23}#
#\izlaz{Lista: [23]}#
#\ulaz{14}#
#\izlaz{Lista: [23, 14]}#
#\ulaz{35}#
#\izlaz{Lista: [23, 14, 35]}#

#\izlaz{Unesite broj koji se brise iz liste:} \ulaz{3}#
#\izlaz{Lista nakon brisanja:  [23, 14, 35]}#
\end{upotreba}
\end{maxitest}  
\begin{maxitest}
\begin{upotreba}{3}
#\poziv{./a.out}#

#\naslovInt#
#\izlaz{Unesite brojeve: (za kraj unesite CTRL+D)}#

#\izlaz{Unesite broj koji se brise iz liste:} \ulaz{12}#
#\izlaz{Lista nakon brisanja:  []}#
\end{upotreba}
\end{maxitest} 


\item[(3)] U glavnom programu se učitani celi brojevi dodaju u listu tako da je vrednosti budu uređene u neopadajućem poretku. 
    Unosi ceo broj koji se traži u unetoj listi, i na ekran se ispisuje rezultat pretrage. 
    Potom se unosi još jedan ceo broj čija se sva pojavljivanja u listi brišu i prikazuje se aktuelni sadržaj liste nakon brisanja.
    \napomena{Prilikom pretraživanja liste i brisanja cvora liste koristiti činjenicu da je lista uređena.}
    
\begin{maxitest}
\begin{upotreba}{1}
#\poziv{./a.out}#

#\naslovInt#
#\izlaz{Unesite brojeve: (za kraj unesite CTRL+D)}#
#\ulaz{2}#
#\izlaz{Lista: [2]}#
#\ulaz{3}#
#\izlaz{Lista: [2, 3]}#
#\ulaz{14}#
#\izlaz{Lista: [2, 3, 14]}#
#\ulaz{5}#
#\izlaz{Lista: [2, 3, 5, 14]}#
#\ulaz{3}#
#\izlaz{Lista: [2, 3, 3, 5, 14]}#
#\ulaz{3}#
#\izlaz{Lista: [2, 3, 3, 3, 5, 14]}#
#\ulaz{17}#
#\izlaz{Lista: [2, 3, 3, 3, 5, 14, 17]}#
#\ulaz{3}#
#\izlaz{Lista: [2, 3, 3, 3, 3, 5, 14, 17]}#
#\ulaz{1}#
#\izlaz{Lista: [1, 2, 3, 3, 3, 3, 5, 14, 17]}#
#\ulaz{9}#
#\izlaz{Lista: [1, 2, 3, 3, 3, 3, 5, 9, 14, 17]}#

#\izlaz{Unesite broj koji se trazi u listi:} \ulaz{5}#
#\izlaz{Trazeni broj 5 je u listi!}#

#\izlaz{Unesite broj koji se brise iz liste:} \ulaz{3}#
#\izlaz{Lista nakon brisanja:  [1, 2, 5, 9, 14, 17]}#
\end{upotreba}
\end{maxitest}
\begin{maxitest}
\begin{upotreba}{3}
#\poziv{./a.out}#

#\naslovInt#
#\izlaz{Unesite brojeve: (za kraj unesite CTRL+D)}#

#\izlaz{Unesite broj koji se trazi u listi:} \ulaz{1}#
#\izlaz{Broj 1 se ne nalazi u listi!}#

#\izlaz{Unesite broj koji se brise iz liste:} \ulaz{12}#
#\izlaz{Lista nakon brisanja:  []}#
\end{upotreba}
\end{maxitest}
\begin{maxitest}
\begin{upotreba}{2}
#\poziv{./a.out}#

#\naslovInt#
#\izlaz{Unesite brojeve: (za kraj unesite CTRL+D)}#
#\ulaz{23}#
#\izlaz{Lista: [23]}#
#\ulaz{14}#
#\izlaz{Lista: [14, 23]}#
#\ulaz{35}#
#\izlaz{Lista: [14, 23, 35]}#

#\izlaz{Unesite broj koji se trazi u listi:} \ulaz{8}#
#\izlaz{Broj 8 se ne nalazi u listi!}#

#{Unesite broj koji se brise iz liste:} \ulaz{3}#
#{Lista nakon brisanja:  [14, 23, 35]}#
\end{upotreba}
\end{maxitest}

\end{enumerate}
\linkresenje{601}
\end{Exercise}
\begin{Answer}[ref=601]
\includecode{resenja/06_Liste/601/lista.h}
\includecode{resenja/06_Liste/601/lista.c}
\includecode{resenja/06_Liste/601/main_a.c}
\includecode{resenja/06_Liste/601/main_b.c}
\includecode{resenja/06_Liste/601/main_c.c}
\end{Answer}

%=========================================================================
\begin{Exercise}[label=602]
Napisati biblioteku za rad sa jednostruko povezanim listama koja sadrži sve funkcije iz zadatka \ref{601}, ali tako da funkcije budu implementirane rekurzivno. 
\napomena{Koristiti iste \kckod{main} programe i upotrebe programa iz zadatka \ref{601}.}
\linkresenje{602}
\end{Exercise}
\begin{Answer}[ref=602]
\includecode{resenja/06_Liste/602/lista.h}
\includecode{resenja/06_Liste/602/lista.c}
\end{Answer}

%=========================================================================
\begin{Exercise}[label=603]
Napisati biblioteku za rad sa dvostruko povezanom listom celih brojeva, koja ima iste funkcionalnosti kao biblioteka iz zadatka \ref{601}. 
Dopuniti bibilioteku novim funkcijama.
\begin{enumerate}
 \item Napisati funkciju \kckod{void obrisi\_tekuci(Cvor ** adresa\_glave, Cvor * tekuci)} koja iz liste čija se glava nalazi na adresi \kckod{adresa\_glave} briše čvor na koji pokazuje pokazivač \kckod{tekuci}.
 \item Napisati funkciju \kckod{void ispisi\_listu\_u\_nazad(Cvor * glava)} koja ispisuje sadržaj liste od poslednjeg čvora ka glavi liste.
\end{enumerate}

Sve funkcije za rad sa listom implementirati iterativno.
\napomena{Koristiti iste \kckod{main} programe i upotrebe programa iz zadatka \ref{601}. Ove programe dopuniti pozivom funkcije koja ispisuje listu u nazad. }
\linkresenje{603}
\end{Exercise}
\begin{Answer}[ref=603]
\includecode{resenja/06_Liste/603/lista.h}
\includecode{resenja/06_Liste/603/lista.c}
\includecode{resenja/06_Liste/603/main_a.c}
\includecode{resenja/06_Liste/603/main_b.c}
\includecode{resenja/06_Liste/603/main_c.c}
\end{Answer}

%=========================================================================
\begin{Exercise}[label=604]
Sadržaj datoteke je aritmetički izraz koji može sadržati zagrade \{, [ i (. 
Napisati program koji učitava sadržaj datoteke \kckod{izraz.txt} i korišćenjem steka 
utvrđuje da li su zagrade u aritmetičkom izrazu dobro uparene. Program štampa odgovarajuću poruku na standardni izlaz.

\noindent
\begin{miditest}
\begin{test}{1}
#\poziv{./a.out}#

#\naslovDat{izraz.txt}#
#\datoteka{\{[23 + 5344] * (24 - 234)\} - 23}#
  
#\naslovIzlaz#
#\izlaz{Zagrade su ispravno uparene.}#
\end{test}
\end{miditest}
\begin{miditest}
\begin{test}{2}
#\poziv{./a.out}#

#\naslovDat{izraz.txt}#
#\datoteka{\{[23 + 5] * (9 * 2)\} - \{23\}}#

#\naslovIzlaz#
#\izlaz{Zagrade su ispravno uparene.}# 
\end{test}
\end{miditest}
\begin{miditest}
\begin{test}{3}
#\poziv{./a.out}#

#\naslovDat{izraz.txt}#
#\datoteka{\{[2 + 54) / (24 * 87)\} + (234 + 23)}#

#\naslovIzlaz#
#\izlaz{Zagrade nisu ispravno uparene.}#
\end{test}
\end{miditest}
\begin{miditest}
\begin{test}{3}
#\poziv{./a.out}#

#\naslovDat{izraz.txt}#
#\datoteka{\{(2 - 14) / (23 + 11)\}\} * (2 + 13)}#

#\naslovIzlaz#
#\izlaz{Zagrade nisu ispravno uparene.}#
\end{test}
\end{miditest}
\begin{miditest}
\begin{test}{4}
#\poziv{./a.out}#

#\naslovDat{izraz.txt}#
#\datoteka{Datoteka je prazna.}#

#\naslovIzlaz#
#\izlaz{Zagrade su ispravno uparene.}#
\end{test}
\end{miditest}
\begin{miditest}
\begin{test}{5}
#\poziv{./a.out}#

#\naslovDat{izraz.txt}#
#\datoteka{Datoteka ne postoji.}# 

#\naslovIzlaz#
#\izlaz{Greska prilikom otvaranja }#
#\izlaz{datoteke izraz.txt!}#
\end{test}
\end{miditest}
\linkresenje{604}
\end{Exercise}
\begin{Answer}[ref=604]
\includecode{resenja/06_Liste/604.c}
\end{Answer}

%=========================================================================
\begin{Exercise}[label=605]
Napisati program koji proverava ispravnost uparivanja etiketa u \kckod{HTML} datoteci. Ime datoteke se zadaje kao argument komandne linije.
Poruke o greškama ispisivati na standardni izlaz za greške.
\uputstvo{Za rešavanje problema koristiti stek implementiran preko liste čiji su čvorovi \kckod{HTML} etikete.}


\begin{miditest}      
\begin{test}{1}
#\poziv{./a.out datoteka.html}#

#\naslovDat{datoteka.html}#                     
#\datoteka{<html>}#                                
#\datoteka{  <head>}#
#\datoteka{    <title>Primer</title>}#
#\datoteka{  </head>}#
#\datoteka{  <body>}#
#\datoteka{  </body>}#

#\naslovIzlaz#
#\izlaz{Etikete nisu pravilno uparene}#
#\izlaz{(etiketa <html> nije zatvorena)}# 
\end{test}
\end{miditest}
\begin{miditest}
\begin{test}{2}
#\poziv{./a.out datoteka.html}#

#\naslovDat{datoteka.html}#                     
#\datoteka{Datoteka ne postoji.}#

#\naslovIzlaz#
#\izlaz{Greska prilikom otvaranja}# 
#\izlaz{datoteke datoteka.html.}#
\end{test}
\end{miditest}
\begin{miditest}
\begin{test}{3}
#\poziv{./a.out datoteka.html}#

#\naslovDat{datoteka.html}#                     
#\datoteka{<html>}#
#\datoteka{  <head>}#
#\datoteka{    <title>Primer</title>}#
#\datoteka{  </head>}#
#\datoteka{  <body>}#
#\datoteka{    <h1>Naslov</h1>}#
#\datoteka{    Danas je lep i suncan dan. <br>}#
#\datoteka{    A sutra ce biti jos lepsi.}#
#\datoteka{    <a link="http://www.google.com" > Link 1</a>}#
#\datoteka{    <a link="http://www.math.rs" > Link 2</a>}#
#\datoteka{  </body>}#
#\datoteka{</html>}#

#\naslovIzlaz#
#\izlaz{Etikete su pravilno uparene!}#
\end{test}
\end{miditest}
\begin{miditest}    
\begin{test}{4}
#\poziv{./a.out datoteka.html}#

#\naslovDat{datoteka.html}#                     
#\datoteka{<html>}#
#\datoteka{  <head>}#
#\datoteka{    <title>Primer</title>}#
#\datoteka{  </head>}#
#\datoteka{  <body>}#
#\datoteka{</html>}#

#\naslovIzlaz#
#\izlaz{Etikete nisu pravilno uparene}#
#\izlaz{(nadjena etiketa </html>, a poslednja }#
#\izlaz{otvorena etiketa je <body>)}#

\end{test}
\end{miditest}
\begin{miditest}
\begin{test}{5}
#\poziv{./a.out}#

#\naslovIzlaz#
#\izlaz{Greska! Program se poziva}# 
#\izlaz{sa: ./a.out datoteka.html!}#
\end{test}
\end{miditest}
\begin{miditest}
\begin{test}{6}
#\poziv{./a.out datoteka.html}#

#\naslovDat{datoteka.html}#                     
#\datoteka{Datoteka je prazna.}#

#\naslovIzlaz#
#\izlaz{Etikete su pravilno uparene!}#
\end{test}
\end{miditest}
\linkresenje{605}
\end{Exercise}
\begin{Answer}[ref=605]
\includecode{resenja/06_Liste/605.c}
\end{Answer}
%=========================================================================
\begin{Exercise}[label=606]
Napisati program kojim se simulira rad jednog šaltera na kojem se prvo kod službenika zakazuju 
termini, a potom službenik uslužuje korisnike. 
Službenik evidentira korisničke \kckod{JMBG} brojeve (niske koje sadrže po $13$ karaktera) i zahteve (niska koja sadrži najviše $999$ karaktera). 
Prijem zahteva korisnika se prekida unošenjem karaktera za kraj ulaza, (\kckod{EOF}).
Službenik redom pregleda zahteve i odlučuje da li zahtev obrađuje odmah ili kasnije, tj.~službeniku se postavlja pitanje 
\kckod{Da li korisnika  vracate na kraj reda?} i ukoliko on da odgovor \kckod{Da}, 
korisnik se stavlja na kraj reda, čime se obrada njegovog zahteva odlaže. Ukoliko odgovor nije \kckod{Da}, tada službenik obrađuje zahtev i podatke o korisniku dopisuje na kraj datoteke \kckod{izvestaj.txt}. Ova datoteka, za svaki obrađen zahtev, sadrži \kckod{jmbg} i zahtev usluženog korisnika.
Posle svakog $5$ usluženog korisnika, službeniku se nudi mogućnost da prekine sa radom, nevezano od broja korisnika koji i dalje čekaju u redu. 
\uputstvo{Za čuvanje korisničkih zahteva koristiti red implementiran korišćenjem listi.}

\begin{maxitest}
\begin{upotreba}{1}
#\poziv{./a.out}#

#\naslovInt#
#\izlaz{Sluzbenik evidentira korisnicke zahteve unosenjem}# 
#\izlaz{njihovog JMBG broja i opisa potrebne usluge:}#
#\izlaz{Novi zahtev [CTRL+D za kraj]}#
#\izlaz{  JMBG:} \ulaz{1234567890123}#
#\izlaz{  Opis problema:} \ulaz{Otvaranje racuna}#

#\izlaz{Novi zahtev [CTRL+D za kraj]}#
#\izlaz{  JMBG:} \ulaz{2345678901234}#
#\izlaz{  Opis problema:} \ulaz{Podizanje novca}#

#\izlaz{Novi zahtev [CTRL+D za kraj]}#
#\izlaz{  JMBG:} \ulaz{3456789012345}#
#\izlaz{  Opis problema:} \ulaz{Reklamacija}#

#\izlaz{Novi zahtev [CTRL+D za kraj]}#
#\izlaz{  JMBG:} \ulaz{4567890123456}#
#\izlaz{  Opis problema:} \ulaz{Zatvaranje racuna}#

#\izlaz{Novi zahtev [CTRL+D za kraj]}#
#\izlaz{  JMBG:}#

#\izlaz{Sledeci je korisnik sa JMBG brojem: 1234567890123}#
#\izlaz{sa zahtevom: Otvaranje racuna}#
#\izlaz{  Da li ga vracate na kraj reda? [Da/Ne]} \ulaz{Da}#

#\izlaz{Sledeci je korisnik sa JMBG brojem: 2345678901234}#
#\izlaz{sa zahtevom: Podizanje novca}#
#\izlaz{  Da li ga vracate na kraj reda? [Da/Ne]} \ulaz{Ne}#

#\izlaz{Sledeci je korisnik sa JMBG brojem: 3456789012345}#
#\izlaz{sa zahtevom: Reklamacija}#
#\izlaz{  Da li ga vracate na kraj reda? [Da/Ne]} \ulaz{Da}#

#\izlaz{Sledeci je korisnik sa JMBG brojem: 4567890123456}#
#\izlaz{sa zahtevom: Zatvaranje racuna}#
#\izlaz{  Da li ga vracate na kraj reda? [Da/Ne]} \ulaz{Ne}#

#\izlaz{Sledeci je korisnik sa JMBG brojem: 1234567890123}#
#\izlaz{sa zahtevom: Otvaranje racuna}#
#\izlaz{  Da li ga vracate na kraj reda? [Da/Ne]} \ulaz{Ne}#

#\izlaz{Da li je kraj smene? [Da/Ne]} \ulaz{Ne}#

#\izlaz{Sledeci je korisnik sa JMBG brojem: 3456789012345}#
#\izlaz{sa zahtevom: Reklamacija}#
#\izlaz{  Da li ga vracate na kraj reda? [Da/Ne]} \ulaz{Ne}#

#\naslovDat{izvestaj.txt}#
#\datoteka{  JMBG: 2345678901234     Zahtev: Podizanje novca}#
#\datoteka{  JMBG: 4567890123456     Zahtev: Zatvaranje racuna}#
#\datoteka{  JMBG: 1234567890123     Zahtev: Otvaranje racuna}#
#\datoteka{  JMBG: 3456789012345     Zahtev: Reklamacija}#
\end{upotreba}
\end{maxitest}
\linkresenje{606}
\end{Exercise}
\begin{Answer}[ref=606]
\includecode{resenja/06_Liste/606/red.h}
\includecode{resenja/06_Liste/606/red.c}
\includecode{resenja/06_Liste/606/main_a.c}
\end{Answer}

%%% Obavezni sa praktikuma
%=========================================================================
\begin{Exercise}[label=607]
Napisati program koji prebrojava pojavljivanja etiketa \kckod{HTML} 
datoteke čije se ime zadaje kao argument komandne linije. Rezultat prebrojavanja 
ispisati na standardni izlaz. Etikete smeštati u listu, a za formiranje liste koristiti strukturu:
\begin{ckod} 
 typedef struct _Element
 {
   unsigned broj_pojavljivanja;
   char etiketa[20];
   struct _Element *sledeci;
 } Element;
\end{ckod}
\begin{miditest}
\begin{test}{1}
#\poziv{./a.out datoteka.html}#

#\naslovDat{datoteka.html}#
#\datoteka{<html>}#                       
#\datoteka{  <head><title>Primer</title></head>}#
#\datoteka{  <body>}#
#\datoteka{    <h1>Naslov</h1>}#
#\datoteka{    Danas je lep i suncan dan. <br>}#
#\datoteka{    A sutra ce biti jos lepsi.}#
#\datoteka{    <a link="http://www.google.com"> Link 1</a>}#     
#\datoteka{    <a link="http://www.math.rs"> Link 2</a>}# 
#\datoteka{  </body>}# 
#\datoteka{</html>}# 

#\naslovIzlaz#
#\izlaz{a - 4}#
#\izlaz{br - 1}#
#\izlaz{h1 - 2}#
#\izlaz{body - 2}#
#\izlaz{title - 2}#
#\izlaz{head - 2}#
#\izlaz{html - 2}#
\end{test}
\end{miditest}
\begin{miditest}
\begin{test}{2}
#\poziv{./a.out datoteka.html}#

#\naslovDat{datoteka.html}#
#\datoteka{Datoteka ne postoji.}#

#\naslovIzlaz#
#\izlaz{Greska prilikom otvaranja}#
#\izlaz{datoteke datoteka.html.}#
\end{test}
\end{miditest}
\begin{miditest}
\begin{test}{3}
#\poziv{./a.out}#

#\naslovIzlaz#
#\izlaz{Greska! Program se poziva }#
#\izlaz{sa: ./a.out datoteka.html!}#
\end{test}
\end{miditest}
\begin{miditest}
\begin{test}{4}
#\poziv{./a.out datoteka.html}#

#\naslovDat{datoteka.html}#
#\datoteka{Datoteka je prazna.}#
\end{test}
\end{miditest}
\linkresenje{607}
\end{Exercise}
\begin{Answer}[ref=607]
\includecode{resenja/06_Liste/607.c}
\end{Answer}

%=========================================================================
\begin{Exercise}[label=608]
U datoteci se nalaze podaci o studentima. U svakom redu datoteke nalazi se indeks, ime i prezime studenta. 
Napisati program kome se preko argumenata komandne linije prosleđuje ime datoteke sa studentskim podacima, koje program treba da pročita i smesti u listu. 
Nakon završenog učitavanja svih studenata, sa standardnog ulaza unose se, jedan po jedan, indeksi studenata koji se traže u učitanoj listi. 
Posle svakog unetog indeksa, program ispisuje poruku da ili ne, u zavisnosti od toga da li u listi postoji student sa unetim indeksom ili ne. 
Prekid unosa indeksa se vrši unošenjem karaktera za kraj ulaza (EOF).
\uputstvo{Dodavanje novog studenta izdvojiti u funkciju \kckod{void dodaj\_na\_pocetak\_liste(Cvor ** glava, char *broj\_indeksa, char *ime, char *prezime)}.
Napisati rekurzivnu funkciju \kckod{Cvor *pretrazi\_listu(Cvor * glava, char *broj\_indeksa)} koja određuje da li student sa zadatim brojem indeksa pripada listi ili ne.
Nakon završenog pretraživanja rekurzivnom funkcijom \kckod{void oslobodi\_listu(Cvor ** glava)} osloboditi memoriju koju je lista sa studentima zauzimala.
Pretpostaviti da je 10 karaktera dovoljno za zapis indeksa i da je 20 karaketera maksimalna dužina bilo imena bilo prezimena studenta.}

\noindent
\begin{miditest}
\begin{upotreba}{1}
#\poziv{./a.out studenti.txt}#

#\naslovDat{studenti.txt}#
#\datoteka{123/2014 Marko Lukic}#
#\datoteka{3/2014   Ana Sokic}#
#\datoteka{43/2013  Jelena Ilic}#
#\datoteka{41/2009  Marija Zaric}#
#\datoteka{13/2010  Milovan Lazic}#

#\naslovInt#           
#\ulaz{3/2014} \izlaz{da: Ana Sokic}#
#\ulaz{235/2008} \izlaz{ne}#
#\ulaz{41/2009} \izlaz{da: Marija Zaric}#
\end{upotreba}
\end{miditest}
\begin{miditest}
\begin{upotreba}{2}
#\poziv{./a.out  studenti.txt}#

#\naslovDat{studenti.txt}#
#\datoteka{Datoteka je prazna.}#

#\naslovInt#
#\ulaz{3/2014} \izlaz{ne}#
#\ulaz{235/2008} \izlaz{ne}#
#\ulaz{41/2009} \izlaz{ne}#
\end{upotreba}
\end{miditest}
\begin{miditest}
\begin{upotreba}{3}
#\poziv{./a.out}#

#\naslovIzlaz# 
#\izlaz{Greska! Program se poziva sa: }#
#\izlaz{./a.out studenti.txt!}#
\end{upotreba}
\end{miditest}
\begin{miditest}
\begin{upotreba}{4}
#\poziv{./a.out  studenti.txt}#

#\naslovDat{studenti.txt}#
#\datoteka{Datoteka ne postoji.}#

#\naslovIzlaz#
#\izlaz{Greska prilikom otvaranja datoteke}#
#\izlaz{studenti.txt.}#
\end{upotreba}
\end{miditest}  

\linkresenje{608}
\end{Exercise}
\begin{Answer}[ref=608]
\includecode{resenja/06_Liste/608.c}
\end{Answer}

%=========================================================================
\begin{Exercise}[label=609]
%\komentar{Milena: malo me muci u ovom zadatku sto nema neki smisao. Naime, ako se samo vrsi ucitavanje iz datoteka i ispisivanje,  onda su ove liste zapravo visak jer isti rezultat moze da se dobije i bez koriscenja listi. Zato mi fali da program uradi nesto sto ne bi mogao da uradi bez koriscenja listi, npr da na osnovu unetog broja ispisuje svaki n-ti broj rezultujuce liste pa to u nekoj petlji da korisnik moze da ispisuje za razlicite unete n ili tako nesto... }
Napisati program koji objedinjuje dve sortirane liste u jednu sortiranu listu. Funkcija ne treba da 
kreira nove čvorove, već da samo
postojeće čvorove preraspodeli. Prva lista se učitava iz datoteke koja se 
zadaje kao prvi argument komandne
linije, a druga iz datoteke čije se ime zadaje kao drugi argument komandne linije. Rezultujuću listu ispisati na standardni izlaz.
\napomena{Koristiti biblioteku za rad sa listama celih brojeva iz zadatka \ref{601}.}

\noindent
\begin{miditest}
\begin{test}{1}
#\poziv{./a.out dat1.txt dat2.txt}#

#\naslovDat{dat1.txt}#
#\datoteka{2 4 6 10 15}#

#\naslovDat{dat2.txt}#
#\datoteka{5 6 11 12 14 16}#

#\naslovIzlaz#
#\izlaz{2 4 5 6 6 10 11 12 14 15 16}#
\end{test}
\end{miditest}
\begin{miditest}
\begin{test}{2}
#\poziv{./a.out dat1.txt dat2.txt}#

#\naslovDat{dat1.txt}#
#\datoteka{2 4 6 10 15}#

#\naslovDat{dat2.txt}#
#\datoteka{Datoteka ne postoji.}#

#\naslovIzlaz#
#\izlaz{Greska prilikom otvaranja datoteke}#
#\izlaz{dat2.txt.}#
\end{test}
\end{miditest}
\begin{miditest}
\begin{test}{3}
#\poziv{./a.out dat1.txt dat2.txt}#

#\naslovDat{dat1.txt}#
#\datoteka{Datoteka ne postoji.}#

#\naslovDat{dat2.txt}#
#\datoteka{5 6 11 12 14 16}#

#\naslovIzlaz#
#\izlaz{Greska prilikom otvaranja datoteke}#
#\izlaz{dat1.txt.}#
\end{test}
\end{miditest}  
\begin{miditest}
\begin{test}{4}
#\poziv{./a.out dat1.txt dat2.txt}#

#\naslovDat{dat1.txt}#
#\datoteka{2 4 6 10 15}#

#\naslovDat{dat2.txt}#
#\datoteka{Datoteka je prazna.}#

#\naslovIzlaz#
#\izlaz{2 4 6 10 15}#
\end{test}
\end{miditest}
\begin{miditest}
\begin{test}{5}
#\poziv{./a.out dat1.txt dat2.txt}#

#\naslovDat{dat1.txt}#
#\datoteka{Datoteka je prazna.}#

#\naslovDat{dat2.txt}#
#\datoteka{5 6 11 12 14 16}#

#\naslovIzlaz#
#\izlaz{5 6 11 12 14 16}#
\end{test}
\end{miditest}
\begin{miditest}
\begin{test}{6}
#\poziv{./a.out}#

#\naslovIzlaz# 
#\izlaz{Greska! Program se poziva sa:}#
#\izlaz{./a.out dat1.txt dat2.txt!}#
\end{test}
\end{miditest}  
\begin{miditest}
\begin{test}{7}
#\poziv{./a.out dat1.txt}#

#\naslovIzlaz# 
#\izlaz{Greska! Program se poziva sa:}# 
#\izlaz{./a.out dat1.txt dat2.txt!}#
\end{test}
\end{miditest}

\linkresenje{609}  
\end{Exercise}
\begin{Answer}[ref=609]
\includecode{resenja/06_Liste/609.c}
\end{Answer}

%% Dodatni
%=========================================================================
\begin{Exercise}[label=610]
Date su dve jednostruko povezane liste \kckod{L1} i \kckod{L2}. Napisati funkciju koja od 
tih lista formira novu listu \kckod{L} koja sadrži alternirajući raspoređene čvorove 
lista \kckod{L1} i \kckod{L2} (prvi čvor iz \kckod{L1}, prvi čvor iz \kckod{L2}, drugi čvor \kckod{L1},
drugi čvor \kckod{L2}, itd). Ne formirati nove čvorove, već samo postojeće čvorove 
rasporediti u jednu listu. Prva lista se učitava iz datoteke čije se ime zadaje kao prvi argument komandne linije, a druga iz datoteke čije se ime zadaje kao 
drugi argument komandne linije. Rezultujuću listu ispisati na standardni izlaz. 
%\komentar{Milena: Sta ako je neka lista duza? To precizirati. I ovde me muci sto nedostaje neki smisao zadatku, nesto sto ne bi moglo da se uradi da nismo koristili liste. }

\napomena{Iskoristiti testove 2 - 7 za zadatak \ref{609}.}

\noindent
\begin{miditest}
\begin{test}{1}
#\poziv{./a.out dat1.txt dat2.txt}#

#\naslovDat{dat1.txt}#
#\datoteka{2 4 6 10 15}#

#\naslovDat{dat2.txt}#
#\datoteka{5 6 11 12 14 16}#

#\naslovIzlaz#
#\izlaz{2 5 4 6 6 11 10 12 15 14 16}#
\end{test}
\end{miditest}
\end{Exercise}
\begin{Answer}[ref=610]
% \includecode{resenja/06_Liste/601.c}
\end{Answer}

%=========================================================================
\begin{Exercise}[label=611]
Data je datoteka \kckod{brojevi.txt} koja sadrži cele brojeve.
\begin{enumerate}
 \item Napisati funkciju koja iz zadate datoteke učitava brojeve i smešta ih u listu.
 \item Napisati funkciju koja u jednom prolazu kroz zadatu listu celih brojeva 
pronalazi maksimalan strogo rastući podniz.
\end{enumerate}
Napisati program koji u datoteku \kckod{Rezultat.txt} upisuje nađeni strogo rastući podniz.
%\komentar{Milena: I ovde me muci sto bi zadatak mogao da se resi i bez koriscenja listi... }

\noindent
\begin{minitest}
\begin{test}{1}
#\poziv{./a.out}#

#\naslovDat{brojevi.txt}#
#\datoteka{43 12 15 16 4 2 8}#

#\naslovIzlaz#
#\naslovDat{rezultat.txt}#
#\datoteka{12 15 16}#
\end{test}
\end{minitest}
\begin{minitest}
\begin{test}{2}
#\poziv{./a.out}#

#\naslovDat{brojevi.txt}#
#\datoteka{Datoteka ne postoji.}#

#\naslovIzlaz#
#\naslovDat{rezultat.txt}#
#\datoteka{Greska prilikom otvaranja}#
#\datoteka{datoteke brojevi.txt.}#
\end{test}
\end{minitest}
\begin{minitest}
\begin{test}{3}
#\poziv{./a.out }#

#\naslovDat{brojevi.txt}#
#\datoteka{Datoteka je prazna.}#

#\naslovIzlaz#
#\naslovDat{rezultat.txt}#
#\datoteka{Rezultat.txt ce biti prazna.}#
\end{test}
\end{minitest}
\end{Exercise}

\begin{Answer}[ref=611]
% \includecode{resenja/06_Liste/601.c}
\end{Answer}

%=========================================================================
\begin{Exercise}[label=612]
Grupa od $n$ plesača na kostimima imaju brojeve od $1$ do $n$, redom, u smeru kazaljke na satu.
Plesači izvode svoju plesnu tačku tako što formiraju krug iz kog najpre izlazi $k$-ti plesač.
Odbrojava se počevši od plesača označenog brojem $1$ u smeru kretanja kazaljke na satu. 
Preostali plesači obrazuju manji krug iz kog opet izlazi $k$-ti plesač. Odbrojavanje počinje od
sledećeg suseda prethodno izbačenog, opet u smeru kazaljke na satu. Izlasci iz kruga se nastavljaju
sve dok svi plesači ne budu isključeni. 
Celi brojevi $n$, $k$ ($k < n$) se učitavaju sa standardnog ulaza. 
Napisati program koji će na standardni izlaz ispisati redne brojeve plesača u redosledu napuštanja kruga. 
\uputstvo{Pri implementaciji koristiti jednostruko povezanu kružnu listu.}

\noindent
\begin{minitest}
\begin{test}{1}
#\poziv{./a.out}#

#\naslovUlaz#
#\ulaz{5 3}#

#\naslovIzlaz# 
#\izlaz{3 1 5 2 4}#
\end{test}
\end{minitest}
\begin{minitest}
\begin{test}{2}
#\poziv{./a.out}#

#\naslovUlaz#
#\ulaz{8 4}#

#\naslovIzlaz# 
#\izlaz{4 8 5 2 1 3 7 6}# 
\end{test}
\end{minitest}
\begin{minitest}
\begin{test}{3}
#\poziv{./a.out}#

#\naslovUlaz#
#\ulaz{3 8}#

#\naslovIzlaz# 
#\izlaz{n mora biti uvek vece}#
#\izlaz{od k, a 3 < 8!}#
\end{test}
\end{minitest}
\begin{minitest}
\begin{test}{4}
#\poziv{./a.out}#

#\naslovUlaz#
#\ulaz{0 0}#

#\naslovIzlaz# 
#\izlaz{Broj plesaca mora biti}#
#\izlaz{veci od 0!}#
\end{test}
\end{minitest}
\end{Exercise}
\begin{Answer}[ref=612]
% \includecode{resenja/06_Liste/601.c}
\end{Answer}

%=========================================================================
\begin{Exercise}[label=613]
Grupa od $n$ plesača na kostimima imaju brojeve od $1$ do $n$, redom, u smeru kazaljke na satu.
Plesači izvode svoju plesnu tačku tako što formiraju krug iz kog najpre izlazi $k$-ti plesač.
Odbrojava se počevši od plesača označenog brojem $1$ u smeru kretanja kazaljke na satu. 
Preostali plesači obrazuju manji krug iz kog opet izlazi $k$-ti plesač. Odbrojavanje počinje od
sledećeg suseda prethodno izbačenog, uz promenu smera. Ukoliko se prilikom prethodnog izbacivanja odbrojavalo 
u smeru kazaljke na satu sada će se obrojavati u suprotnom smeru, i obrnuto. Izlasci iz kruga se nastavljaju
sve dok svi plesači ne budu isključeni. 
Celi brojevi $n$, $k$ ($k < n$) se učitavaju sa standardnog ulaza. 
Napisati program koji će na standardni izlaz ispisati redne brojeve plesača u redosledu napuštanja kruga. 
\uputstvo{Pri implementaciji koristiti dvostruko povezanu kružnu listu.}
\napomena{Iskoristiti 3. i 4. test za \ref{612}. zadatak.}

\begin{minitest}
\begin{test}{1}
#\poziv{./a.out}#

#\naslovUlaz#
#\ulaz{5 3}#

#\naslovIzlaz# 
#\izlaz{3 5 4 2 1}#
\end{test}
\end{minitest}
\begin{minitest}
\begin{test}{2}
#\poziv{./a.out}#

#\naslovUlaz#
#\ulaz{8 4}#

#\naslovIzlaz# 
#\izlaz{4 8 5 7 6 3 2 1}# 
\end{test}
\end{minitest}
\end{Exercise}
\begin{Answer}[ref=613]
% \includecode{resenja/06_Liste/601.c}
\end{Answer}