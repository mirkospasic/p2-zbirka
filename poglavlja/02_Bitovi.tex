\chapter{Uvodni zadaci}

\section{Podela koda po datotekama}

\begin{Exercise}[label=001] % I sa vezbi
Napisati program za rad sa kompleksnim brojevima.
\begin{enumerate}
\item Definisati strukturu \kckod{KompleksanBroj} koja predstavlja kompleksan broj i sadrži realan i imaginaran deo kompleksnog broja.
\item Napisati funkciju \kckod{ucitaj\_kompleksan\_broj} koja učitava kompleksan broj sa standardnog ulaza.
\item Napisati funkciju \kckod{ispisi\_kompleksan\_broj} koja ispisuje kompleksan broj na standardni izlaz u odgovarajućem fomatu (npr. broj čiji je realan deo \argf{2} a imaginarni \argf{-3} ispisati kao $(2 - 3 i)$ na standardni izlaz).
\item Napisati funkciju \kckod{realan\_deo} koja računa vrednosti realnog dela broja.
\item Napisati funkciju \kckod{imaginaran\_deo} koja računa vrednosti imaginarnog dela broja.
\item Napisati funkciju \kckod{moduo} koja računa moduo kompleksnog broja.
\item Napisati funkciju \kckod{konjugovan} koja računa konjugovano-kompleksni broj svog argumenta.
\item Napisati funkciju \kckod{saberi} koja sabira dva kompleksna broja.
\item Napisati funkciju \kckod{oduzmi} koja oduzima dva kompleksna broja.
\item Napisati funkciju \kckod{mnozi} koja množi dva kompleksna broja.
\item Napisati funkciju \kckod{argument} koja računa argument kompleksnog broja.
\end{enumerate}
Napisati program koji testira prethodno napisane funkcije tako što redom:
\begin{enumerate}
\item pozivanjem funkcije \kckod{ucitaj\_kompleksan\_broj} omogućava da se kompleksan broj $z1$  unosi sa standardnog ulaza,
\item ispisuje realni deo, imaginarni deo i moduo kompleksnog broja $z1$,
\item pozivanjem funkcije \kckod{ucitaj\_kompleksan\_broj} omogućava da se kompleksan broj $z2$  unosi sa standardnog ulaza,
\item ispisuje konjugovano kompleksan broj i argument broja $z2$,
\item ispisuje zbir, razliku i proizvod brojeva $z1$ i $z2$.
\end{enumerate}

\begin{maxitest}
\begin{upotreba}{1}

#\naslovInt#
#\izlaz{Unesite realan i imaginaran deo kompleksnog broja:} \ulaz{1 -3}#
#\izlaz{(1.00 - 3.00 i)}#
#\izlaz{realan_deo: 1}#
#\izlaz{imaginaran_deo: -3.000000}#
#\izlaz{moduo 3.162278}#
#\izlaz{Unesite realan i imaginaran deo kompleksnog broja:} \ulaz{-1 4}#
#\izlaz{ (-1.00 + 4.00 i)}#
#\izlaz{Njegov konjugovano kompleksan broj: (-1.00 - 4.00 i)}#
#\izlaz{Argument kompleksnog broja: 1.815775}#
#\izlaz{(1.00 - 3.00 i) + (-1.00 + 4.00 i)  =  (1.00 i)}#
#\izlaz{(1.00 - 3.00 i) - (-1.00 + 4.00 i)  =  (2.00 - 7.00 i)}#
#\izlaz{(1.00 - 3.00 i) * (-1.00 + 4.00 i)  =  (11.00 + 7.00 i)}#
\end{upotreba}
\end{maxitest}


\linkresenje{001}

\end{Exercise}
\begin{Answer}[ref=001]
\includecode{resenja/00_PodelaPoDatotekama/001.c}
\end{Answer}

\begin{Exercise}[label=002] % III sa vezbi
Uraditi prethodni zadatak tako da su sve napisane funkcije za rad sa kompleksnim brojevima zajedno sa definicijom strukture \kckod{KompleksanBroj} izdvojene u posebnu biblioteku. Test program treba da koristi tu biblioteku da za kompleksan broj unet sa standardnog ulaza ispiše polarni oblik unetog broja.

\begin{maxitest}
\begin{test}{1}

#\naslovInt#
#\izlaz{Unesite realan i imaginaran deo kompleksnog broja:} \ulaz{-5 2 }#
#\izlaz{Polarni oblik kompleksnog broja je 5.39 *  e^i * 2.76}#
\end{upotreba}
\end{maxitest}

\linkresenje{002}	

\end{Exercise}
\begin{Answer}[ref=002]
\includecode{resenja/00_PodelaPoDatotekama/002/complex.c}
\includecode{resenja/00_PodelaPoDatotekama/002/complex.h}
\includecode{resenja/00_PodelaPoDatotekama/002/main.c}
\end{Answer}


\begin{Exercise}[label=003] % sa praktikuma
Napisati malu biblioteku za rad sa polinomima.

  \begin{enumerate}
  \item Definisati strukturu \kckod{Polinom} koja predstavlja polinom
    (stepena najviše 20). Struktura sadrži stepen i niz
    koeficijenata. Redosled navođenja koeficijenata u nizu treba da bude takav da na nultoj poziciji u nizu bude koeficijent uz slobodan član, na prvoj koeficijent uz prvi stepen, itd.
  \item Napisati funkciju koja ispisuje polinom na standardni izlaz u
    što lepšem obliku.
  \item Napisati funkciju koja učitava polinom sa standardnog
    ulaza.
  \item Napisati funkciju za izračunavanje vrednosti polinoma u
    datoj tački koristeći Hornerov algoritam.
  \item Napisati funkciju koja sabira dva polinoma.
  \item Napisati funkciju koja množi dva polinoma.
  \end{enumerate}

Napisati program koji testira prethodno napisane funkcije tako što se najpre unosi polinom \argf{p} (stepen polinoma, a zatim i koeficijenti) i ispisuje na standardan izlaz u odgovarajućem obliku. Nakon toga se od korisnika traži da unese tačku u kojoj se računa vrednost tog polinoma a zatim se ispisuje iztačunata vrednost zaokružena na dve decimale. Nakon toga se unosi polinom \argf{q}, a potom se ispisuju zbir i proizvod polinoma \argf{p} i \argf{q}. Na kraju se sa standardnog ulaza unosi broj \argf{n}, a potom se ispisuje \argf{n}-ti izvod polinoma \argf{p}.


\begin{maxitest}
\begin{upotreba}{1}

#\naslovInt#
#\izlaz{ Unesite polinom (prvo stepen, pa zatim koeficijente od najveceg stepena do nultog):}#
#\ulaz{3 1.2 3.5 2.1 4.2}#
#\izlaz{Unesite tacku u kojoj racunate vrednost polinoma}#
#\ulaz{5}#
#\izlaz{Vrednost polinoma u tacki je 252.20}#
#\izlaz{ Unesite polinom (prvo stepen, pa zatim koeficijente od najveceg stepena do nultog):}#
#\ulaz{2 2.1 0 -3.9}#
#\izlaz{Zbir polinoma je: 1.20x^3+5.60x^2+2.10x+0.30}#
#\izlaz{Prozvod polinoma je: 2.52x^5+7.35x^4-0.27x^3-4.83x^2-8.19x-16.38}#
#\izlaz{Unesite izvod polinoma koji zelite:}#
#\ulaz{2}#
#\izlaz{2. izvod prvog polinoma je: 7.20x+7.00}#
\end{upotreba}
\end{maxitest}

\linkresenje{003}

\end{Exercise}
\begin{Answer}[ref=003]
\includecode{resenja/00_PodelaPoDatotekama/003/polinom.c}
\includecode{resenja/00_PodelaPoDatotekama/003/polinom.h}
\includecode{resenja/00_PodelaPoDatotekama/003/main.c}
%\includecode{resenja/02_Bitovi/003/Makefile}
\end{Answer}

\begin{Exercise}[label=004] % sa praktikuma
Napraviti biblioteku za rad sa razlomcima.

  \begin{enumerate}

  \item Definisati strukturu za reprezentovanje razlomaka.

  \item Napisati funkcije za učitavanje i ispis razlomaka.

  \item Napisati funkcije koje vraćaju brojilac i imenilac.

  \item Napisati funkciju koja vraća vrednost razlomka kao  \kckod{double}
    vrednost.

  \item Napisati funkciju koja izračunava recipročnu vrednost
    razlomka.

  \item Napisati funkciju koja skraćuje dati razlomak.

  \item Napisati funkcije koje sabiraju, oduzimaju, množe i dele
    dva razlomka.

  \end{enumerate}

Napisati program koji testira prethodne funkcije tako što se sa standardnog ulaza unose dva razlomka \argf{r1} i \argf{r2} i na standardni izlaz se ispisuju skraćene vrednoste razlomaka koji koji su dobijeni kao zbir, razlika, proizvod i količnik razlomka \argf{r1} i recipročne vrednosti razlomka \argf{r2}.


\begin{maxitest}
\begin{upotreba}{1}

#\naslovInt#
#\izlaz{ Unesite imenilac i brojilac prvog razlomka:}\ulaz{1 2}#
#\izlaz{ Unesite imenilac i brojilac drugog razlomka:}\ulaz{3 1}#
#\izlaz{Zbir je 5/6}#
#\izlaz{ Razlika je 1/6}#
#\izlaz{Zbir je 5/6}#
#\izlaz{Kolicnik je 3/2}#
\end{upotreba}
\end{maxitest}


%\linkresenje{004}

\end{Exercise}
%\begin{Answer}[ref=004]
%%\includecode{resenja/02_Bitovi/004.c}
%\end{Answer}

\section{Algoritmi za rad sa bitovima}

\begin{Exercise}[label=201]
Napisati funkciju \kckod{print\_bits} koja štampa bitove u binarnom zapisu neoznačenog celog broja $x$. Napisati program koja testira funkciju \kckod{print\_bits} za brojeve koji se sa standardnog ulaza zadaju u heksadekasnom formatu.

\begin{miditest}
\begin{test}{1}
#\naslovUlaz#
#\ulaz{0x7F}#
#\naslovIzlaz#
#\izlaz{00000000000000000000000001111111}#
\end{test}
\end{miditest}
\begin{miditest}
\begin{test}{2}
#\naslovUlaz#
#\ulaz{0x80}#
#\naslovIzlaz#
#\izlaz{00000000000000000000000010000000}#
\end{test}
\end{miditest}

\begin{miditest}
\begin{test}{3}
#\naslovUlaz#
#\ulaz{0x00FF00FF}#
#\naslovIzlaz#
#\izlaz{00000000111111110000000011111111}#
\end{test}
\end{miditest}
\begin{miditest}
\begin{test}{4}
#\naslovUlaz#
#\ulaz{0xFFFFFFFF}#
#\naslovIzlaz#
#\izlaz{ 11111111111111111111111111111111}#
\end{test}
\end{miditest}

\begin{miditest}
\begin{test}{5}
#\naslovUlaz#
#\ulaz{ 0xABCDE123}#
#\naslovIzlaz#
#\izlaz{10101011110011011110000100100011}#
\end{test}
\end{miditest}

\linkresenje{201}
\end{Exercise}
\begin{Answer}[ref=201]
\includecode{resenja/02_Bitovi/201.c}
\end{Answer}

\begin{Exercise}[label=202]

Napisati funkcije \kckod{count\_bits1} i \kckod{count\_bits2} koje broje bitove sa vrednošću \argf{1} u binarnom zapisu celog broja $x$. Prebrojavanje bitova ostvariti na dva načina:
\begin{enumerate}
\item formiranjem odgovarajuće maske i njenim pomeranjem
\item formiranjem odgovarajuće maske i pomeranjem promenljive $x$.
\end{enumerate} 
 
 Napisati program koji testira tu funkciju za brojeve koji se sa standardnog ulaza zadaju u heksadekasnom formatu.


\begin{miditest}
\begin{test}{1}
#\naslovUlaz#
#\ulaz{0x7F}#
#\naslovIzlaz#
#\izlaz{Broj jedinica u zapisu je}#
#\izlaz{funkcija count_bits1: 7}#
#\izlaz{funkcija count_bits2: 7}#
\end{test}
\end{miditest}
\begin{miditest}
\begin{test}{2}
#\naslovUlaz#
#\ulaz{0x80}#
#\naslovIzlaz#
#\izlaz{Broj jedinica u zapisu je}#
#\izlaz{funkcija count_bits1: 1}#
#\izlaz{funkcija count_bits2: 1}#
\end{test}
\end{miditest}

\begin{miditest}
\begin{test}{3}
#\naslovUlaz#
#\ulaz{0x00FF00FF}#
#\naslovIzlaz#
#\izlaz{Broj jedinica u zapisu je}#
#\izlaz{funkcija count_bits1: 16}#
#\izlaz{funkcija count_bits2: 16}#
\end{test}
\end{miditest}
\begin{miditest}
\begin{test}{4}
#\naslovUlaz#
#\ulaz{0xFFFFFFFF}#
#\naslovIzlaz#
#\izlaz{Broj jedinica u zapisu je}#
#\izlaz{funkcija count_bits1: 32}#
#\izlaz{funkcija count_bits2: 32}#
\end{test}
\end{miditest}

\begin{miditest}
\begin{test}{5}
#\naslovUlaz#
#\ulaz{0xABCDE123}#
#\naslovIzlaz#
#\izlaz{Broj jedinica u zapisu je}#
#\izlaz{funkcija count_bits1: 17}#
#\izlaz{funkcija count_bits2: 17}#
\end{test}
\end{miditest}

\linkresenje{202}
\end{Exercise}
\begin{Answer}[ref=202]
\includecode{resenja/02_Bitovi/202.c}		
%\includecode{resenja/02_Bitovi/202b.c}
\end{Answer}


\begin{Exercise}[label=203]
Napisati funkciju \kckod{najveci} koja određuje najveći broj koji se može zapisati istim binarnim ciframa kao dati broj i funkciju \kckod{najmanji} koja određuje najmanji broj koji se može zapisati istim binarnim ciframa kao dati broj.

Napisati program koji testira prethodno napisane funkcije tako što prikazuje binarnu reprezentaciju brojeva koji se dobijaju nakon poziva funkcije \kckod{najveci}, ondosno \kckod{najmanji} za brojeve koji se sa standardnog ulaza zadaju u heksadekasnom formatu. 


\begin{miditest}
\begin{test}{1}
#\naslovUlaz#
#\ulaz{0x7F}#
#\naslovIzlaz#
#\izlaz{ Najveci:}#
#\izlaz{11111110000000000000000000000000}#
#\izlaz{ Najmanji:}#
#\izlaz{00000000000000000000000001111111}#
\end{test}
\end{miditest}
\begin{miditest}
\begin{test}{2}
#\naslovUlaz#
#\ulaz{0x80}#
#\naslovIzlaz#
#\izlaz{ Najveci:}#
#\izlaz{10000000000000000000000000000000}#
#\izlaz{ Najmanji:}#
#\izlaz{00000000000000000000000000000001}#
\end{test}
\end{miditest}

\begin{miditest}
\begin{test}{3}
#\naslovUlaz#
#\ulaz{0x00FF00FF}#
#\naslovIzlaz#
#\izlaz{ Najveci:}#
#\izlaz{11111111111111110000000000000000}#
#\izlaz{ Najmanji:}#
#\izlaz{00000000000000001111111111111111}#
\end{test}
\end{miditest}
\begin{miditest}
\begin{test}{4}
#\naslovUlaz#
#\ulaz{0xFFFFFFFF}#
#\naslovIzlaz#
#\izlaz{ Najveci:}#
#\izlaz{11111111111111111111111111111111}#
#\izlaz{ Najmanji:}#
#\izlaz{11111111111111111111111111111111}#
\end{test}
\end{miditest}

\begin{miditest}
\begin{test}{5}
#\naslovUlaz#
#\ulaz{0xABCDE123}#
#\naslovIzlaz#
#\izlaz{ Najveci:}#
#\izlaz{11111111111111111000000000000000}#
#\izlaz{ Najmanji:}#
#\izlaz{00000000000000011111111111111111}#
\end{test}
\end{miditest}

\linkresenje{203}
\end{Exercise}
\begin{Answer}[ref=203]
\includecode{resenja/02_Bitovi/203.c}
\end{Answer}


\begin{Exercise}[label=204]
Napisati program za rad sa bitovima.
\begin{enumerate}
\item Napisati funkciju koja određuje broj koji se dobija kada se $n$ bitova datog broja, počevši od pozicije $p$ postave na \argf{0}.
\item Napisati funkciju koja određuje broj koji se dobija kada se $n$ bitova datog broja, počevši od pozicije $p$ postave na \argf{1}.
\item Napisati funkciju koja određuje broj koji se dobija od $n$ bitova datog broja, počevši od pozicije $p$ i vraća ih kao bitove najmanje težine rezultata.
\item Napisati funkciju koja vraća broj koji se dobija upisivanjem poslednjih $n$ bitova broja $y$ u broj $x$, počevši od pozicije $p$.
\item Napisati funkciju koja vraća broj koji se dobija invertovanjem $n$ bitova broja $x$ počevši od pozicije $p$.
\item Napisati program koji testira prethodno napisane funkcije.
\end{enumerate}
Program treba da testira prethodno napisane funkcije nad neoznačenim celim brojem koji se unosi sa standardnog ulaza.
\napomena{pozicije se broje počev od pozicije bita najmanje težine, pri čemu je pozicija bita najmanje težine nula.}

\begin{maxitest}
\begin{test}{1}
#\naslovUlaz#
#\ulaz{235 5 10 127}#
#\naslovIzlaz#
#\izlaz{ Broj   235                          = 00000000000000000000000011101011}#
#\izlaz{reset(  235,    5,   10)            = 00000000000000000000000000101011}#
#\izlaz{set(  235,    5,   10)              = 00000000000000000000011111101011}#
#\izlaz{get_bits(  235,    5,   10)         = 00000000000000000000000000000011}#
#\izlaz{y =                             127 = 00000000000000000000000001111111}#
#\izlaz{set_n_bits(  235,    5,   10,  127) = 00000000000000000000011111101011}#
#\izlaz{invert(  235,    5,   10)           = 00000000000000000000011100101011}#
\end{test}
\end{maxitest}

\linkresenje{204}
\end{Exercise}
\begin{Answer}[ref=204]
\includecode{resenja/02_Bitovi/204.c}
\end{Answer}


\begin{Exercise}[label=205]
Rotiranje ulevo podrazumeva pomeranje svih bitova za jednu poziciju ulevo, s tim što se bit sa pozicije najviše težine pomera na poziciju najmanje težine. Analogno, rotiranje udesno podrazumeva pomeranje svih bitova za jednu poziciju udesno, s tim što se bit sa pozicije najmanje težine pomera na poziciju najviše težine.
\begin{enumerate}
\item Napisati funkciju \kckod{rotate\_left} koja određuje broj koji se dobija rotiranjem \argf{k} puta u levo datog celog broja \argf{x}. 
\item Napisati funkciju \kckod{rotate\_right} koja određuje broj koji se dobija rotiranjem \argf{k} puta u desno datog celog neoznačenog broja \argf{x}. 
\item Napisati funkciju \kckod{rotate\_right\_signed} koja određuje broj koji se dobija rotiranjem \argf{k} puta u desno datog celog broja \argf{x}. 
\end{enumerate}
Napisati program koji testira prethodno napisane funkcije za broj \argf{x} i broj \argf{k} koji se sa standardnog ulaza unose u heksadekasnom formatu.

\begin{maxitest}
\begin{test}{1}
#\naslovUlaz#
#\ulaz{B10011A7 5}#
#\naslovIzlaz#
#\izlaz{ x                                      = 10110001000000000001000110100111 }#
#\izlaz{ rotate_left(2969571751,     5)         = 00100000000000100011010011110110}#
#\izlaz{ rotate_right(2969571751,     5)        = 00111101100010000000000010001101}#
#\izlaz{ rotate_right_signed(2969571751,     5) = 00111101100010000000000010001101 }#
\end{test}
\end{maxitest}

\linkresenje{205}
\end{Exercise}
\begin{Answer}[ref=205]
\includecode{resenja/02_Bitovi/205.c}
\end{Answer}

\begin{Exercise}[label=206]
Napisati funkciju \kckod{mirror} koja određuje ceo broj čiji je binarni zapis slika u ogledalu binarnog zapisa argumenta funkcije. Napisati i program koji testira datu funkciju za brojeve koji se sa standardnog ulaza zadaju u heksadekasnom formatu, tako što najpre ispisuje binarnu reprezentaciju unetog broja, a potom i binarnu reprezentaciju broja dobijenog nakon poziva funkcije \kckod{mirror} za uneti broj.

\begin{maxitest}
\begin{test}{1}
#\naslovUlaz#
#\ulaz{255 }#
#\naslovIzlaz#
#\izlaz{00000000000000000000001001010101}#
#\izlaz{10101010010000000000000000000000}#
\end{test}
\end{maxitest}

\linkresenje{206}
\end{Exercise}
\begin{Answer}[ref=206]
\includecode{resenja/02_Bitovi/206.c}
\end{Answer}

%%%%%%%%%%%%%%%%%%%%%%%%%
% Zadaci sa praktikuma - obavezni zadaci 
%%%%%%%%%%%%%%%%%%%%%%%%%

\begin{Exercise}[label=207]
Napisati funkciju \kckod{int Broj01(unsigned int n)} koja za dati broj \argf{n} vraća \argf{1} ako u njegovom binarnom zapisu ima više jednica nego nula, a inače vraća \argf{0}.  Napisati program koji tu funkciju testira za broj koji se zadaje sa standardnog ulaza.

\begin{minitest}
\begin{test}{1}
#\naslovUlaz#
#\ulaz{10}#
#\naslovIzlaz#
#\izlaz{0}#
\end{test}
\end{minitest}
\begin{minitest}
\begin{test}{2}
#\naslovUlaz#
#\ulaz{1024}#
#\naslovIzlaz#
#\izlaz{0}#
\end{test}
\end{minitest}
\begin{minitest}
\begin{test}{3}
#\naslovUlaz#
#\ulaz{2147377146}#
#\naslovIzlaz#
#\izlaz{1}#
\end{test}
\end{minitest}

\begin{minitest}
\begin{test}{4}
#\naslovUlaz#
#\ulaz{1111111115}#
#\naslovIzlaz#
#\izlaz{0}#
\end{test}
\end{minitest}

\linkresenje{207}
\end{Exercise}
\begin{Answer}[ref=207]
\includecode{resenja/02_Bitovi/207.c}
\end{Answer}

\begin{Exercise}[label=208]
Napisati funkciju koja broji koliko se puta kombinacija
  \argf{11} (dve uzastopne jedinice) pojavljuje u binarnom zapisu
  celog neoznačenog broja $x$. Tri uzastopne jedinice se broje
  dva puta.  Napisati program koji tu funkciju testira za broj koji se
  zadaje sa standardnog ulaza.
  
\begin{minitest}
\begin{test}{1}
#\naslovUlaz#
#\ulaz{11}#
#\naslovIzlaz#
#\izlaz{1}#
\end{test}
\end{minitest}
\begin{minitest}
\begin{test}{2}
#\naslovUlaz#
#\ulaz{1024}#
#\naslovIzlaz#
#\izlaz{0}#
\end{test}
\end{minitest}
\begin{minitest}
\begin{test}{3}
#\naslovUlaz#
#\ulaz{2147377146}#
#\naslovIzlaz#
#\izlaz{22}#
\end{test}
\end{minitest}  

\linkresenje{208}
\end{Exercise}
\begin{Answer}[ref=208]
\includecode{resenja/02_Bitovi/208.c}
\end{Answer}


%%%
%Ovaj 209.c nema resenje za sad...
%%%
\begin{Exercise}[label=209]\marker+{2}
Napisati program koji sa standardnog ulaza učitava pozitivan
  ceo broj, a na standardni izlaz ispisuje vrednost tog broja sa
  razmenjenim vrednostima bitova na pozicijama $i$,
  $j$. Pozicije $i$, $j$ se učitavaju kao parametri
  komandne linije. Smatrati da je krajnji desni bit binarne
  reprezentacije \argf{0}-ti bit. Pri rešavanju nije dozvoljeno koristiti
  pomoćni niz niti aritmetičke operatore +,-,/,*,\%.


\begin{minitest}
\begin{upotreba}{1}
#\poziv{./a.out 1 2 }#

#\naslovInt#
#\ulaz{11}#
#\izlaz{13}#
\end{upotreba}
\end{minitest}
\begin{minitest}
\begin{upotreba}{2}
#\poziv{./a.out 1 2 }#

#\naslovInt#
#\ulaz{1024}#
#\izlaz{1024}#
\end{upotreba}
\end{minitest}
\begin{minitest}
\begin{upotreba}{2}
#\poziv{./a.out 12 12 }#

#\naslovInt#
#\ulaz{12345}#
#\izlaz{12345}#
\end{upotreba}
\end{minitest}

\end{Exercise}
\begin{Answer}[ref=209]
%\includecode{resenja/02_Bitovi/209.c}
\end{Answer}

\begin{Exercise}[label=210]
  Napisati funkciju koja na osnovu neoznačenog broja $x$
  formira nisku $s$ koja sadrži heksadekadni zapis broja
  $x$, koristeći algoritam za brzo prevođenje binarnog u
  heksadekadni zapis (svake $4$ binarne cifre se zamenjuju jednom
  odgovarajućom heksadekadnom cifrom).  Napisati program koji tu
  funkciju testira za broj koji se zadaje sa standardnog ulaza.

\begin{minitest}
\begin{test}{1}
#\naslovUlaz#
#\ulaz{11}#
#\naslovIzlaz#
#\izlaz{0000000B}#
\end{test}
\end{minitest}
\begin{minitest}
\begin{test}{2}
#\naslovUlaz#
#\ulaz{1024}#
#\naslovIzlaz#
#\izlaz{00000400}#
\end{test}
\end{minitest}
\begin{minitest}
\begin{test}{3}
#\naslovUlaz#
#\ulaz{12345}#
#\naslovIzlaz#
#\izlaz{00003039}#
\end{test}
\end{minitest}  

\linkresenje{210}
\end{Exercise}
\begin{Answer}[ref=210]
\includecode{resenja/02_Bitovi/210.c}
\end{Answer}

%%%%%%%%%%%%%%%%%%%%%%%%%
% Zadaci sa praktikuma - dodatni zadaci - oni nemaju rešenja
% možda \subsection{Dodatni zadaci} ili zadaci za vežbu
%%%%%%%%%%%%%%%%%%%%%%%%%

\begin{Exercise}[label=211]\marker+{2}
  Napisati funkciju koja za dva data neoznačena broja $x$
  i $y$ invertuje u podatku $x$ one bitove koji se poklapaju
  sa odgovarajućim bitovima u broju $y$. Ostali bitovi ostaju
  nepromenjeni.  Napisati program koji tu funkciju testira za brojeve
  koji se zadaju sa standardnog ulaza.
  
\begin{minitest}
\begin{test}{1}
#\naslovUlaz#
#\ulaz{123 10}#
#\naslovIzlaz#
#\izlaz{4294967285}#
\end{test}
\end{minitest}
\begin{minitest}
\begin{test}{2}
#\naslovUlaz#
#\ulaz{3251 0}#
#\naslovIzlaz#
#\izlaz{4294967295}#
\end{test}
\end{minitest}
\begin{minitest}
\begin{test}{3}
#\naslovUlaz#
#\ulaz{12541 1024}#
#\naslovIzlaz#
#\izlaz{4294966271}#
\end{test}
\end{minitest}    
  
\end{Exercise}
%\begin{Answer}[ref=211]
%\includecode{resenja/02_Bitovi/211.c}
%\end{Answer}

\begin{Exercise}[label=212]\marker+{2}
Napisati funkciju koja računa koliko petica bi imao ceo
  neoznačen broj $x$ u oktalnom zapisu. Napisati program koji
  tu funkciju testira za broj koji se zadaje sa standardnog ulaza.
  
\begin{minitest}
\begin{test}{1}
#\naslovUlaz#
#\ulaz{123}#
#\naslovIzlaz#
#\izlaz{0}#
\end{test}
\end{minitest}
\begin{minitest}
\begin{test}{2}
#\naslovUlaz#
#\ulaz{3245}#
#\naslovIzlaz#
#\izlaz{2}#
\end{test}
\end{minitest}
\begin{minitest}
\begin{test}{3}
#\naslovUlaz#
#\ulaz{100328}#
#\naslovIzlaz#
#\izlaz{1}#
\end{test}
\end{minitest}   
 
\end{Exercise}
%\begin{Answer}[ref=212]
%\includecode{resenja/02_Bitovi/212.c}
%\end{Answer}
