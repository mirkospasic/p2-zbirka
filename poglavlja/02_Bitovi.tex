\chapter{Uvodni zadaci}

\section{Podela koda po datotekama}

\begin{Exercise}[label=001] % I sa vezbi
Napisati program za rad sa kompleksnim brojevima.
\begin{enumerate}
\item Definisati strukturu \kckod{KompleksanBroj} koja opisuje kompleksan broj njegovim realnim i imaginarnim delom.
\item Napisati funkciju \kckod{void ucitaj\_kompleksan\_broj(KompleksanBroj * z)} koja učitava kompleksan broj sa standardnog ulaza.
\item Napisati funkciju \kckod{void ispisi\_kompleksan\_broj(KompleksanBroj z)} koja ispisuje kompleksan broj na standardni izlaz u odgovarajućem formatu (npr. broj čiji je realan deo \argf{2}, a imaginarni \argf{-3} ispisati kao $(2 - 3 i)$ na standardni izlaz).
\item Napisati funkciju \kckod{float realan\_deo(KompleksanBroj z)} koja vraća vrednost realnog dela broja.
\item Napisati funkciju \kckod{float imaginaran\_deo(KompleksanBroj z)} koja vraća vrednost imaginarnog dela broja.
\item Napisati funkciju \kckod{float moduo(KompleksanBroj z)} koja računa moduo kompleksnog broja.
\item Napisati funkciju \kckod{KompleksanBroj konjugovan(KompleksanBroj z)} koja računa konjugovano-kompleksni broj svog argumenta \kckod{z}.
\item Napisati funkciju \kckod{KompleksanBroj saberi(KompleksanBroj z1, KompleksanBroj z2)} koja sabira dva kompleksna broja \kckod{z1} i \kckod{z2}.
\item Napisati funkciju \kckod{KompleksanBroj oduzmi(KompleksanBroj z1, KompleksanBroj z2)} koja oduzima dva kompleksna broja \kckod{z1} i \kckod{z2}.
\item Napisati funkciju \kckod{KompleksanBroj mnozi(KompleksanBroj z1, KompleksanBroj z2)} koja množi dva kompleksna broja \kckod{z1} i \kckod{z2}.
\item Napisati funkciju \kckod{float argument(KompleksanBroj z)} koja računa argument kompleksnog broja \kckod{z}.
\end{enumerate}

Napisati program koji testira prethodno napisane funkcije. Program najpre za kompleksan broj $z1$ koji se unosi sa standardnog ulaza ispisuje njegov realni deo, imaginarni deo i moduo. Zatim, za naredni kompleksan broj $z2$ koji se unosi sa standardnog ulaza ispisuje njegov konjugovano-kompleksan broj i argument. Na kraju program ispisuje zbir, razliku i proizvod brojeva $z1$ i $z2$.

\iffalse
Napisati program koji testira prethodno napisane funkcije tako što redom:
\begin{enumerate}
\item pozivanjem funkcije \kckod{ucitaj\_kompleksan\_broj} omogućava da se kompleksan broj $z1$  unese sa standardnog ulaza,
\item pozivanjem funkcije \kckod{ucitaj\_kompleksan\_broj} omogućava da se kompleksan broj $z2$  unosi sa standardnog ulaza,
\item računa i ispisuje kompleksan broj $z$ koji predstavlja zbir, razliku ili proizvod brojeva $z1$ i $z2$ u zavisnosti od znaka ('+', '-', '*') koji se unosi sa standardnog ulaza,
\item ispisuje realni deo, imaginarni deo i moduo kompleksnog broja $z$,
\item ispisuje konjugovano kompleksan broj i argument broja $z$,
\end{enumerate}
\fi


\begin{maxitest}
\begin{upotreba}{1}
#\naslovInt#
#\izlaz{Unesite realan i imaginaran deo kompleksnog broja:} \ulaz{1 -3}#
#\izlaz{(1.00 - 3.00 i)}#
#\izlaz{Unesite realan i imaginaran deo kompleksnog broja:} \ulaz{-1 4}#
#\izlaz{(-1.00 + 4.00 i)}#
#\izlaz{Unesite znak (+,-,*):} \ulaz{-}#
#\izlaz{(1.00 - 3.00 i) - (-1.00 + 4.00 i)  =  (2.00 - 7.00 i)}#
#\izlaz{realan\_deo: 2}#
#\izlaz{imaginaran\_deo: -7.000000}#
#\izlaz{moduo: 7.280110}#
#\izlaz{Njegov konjugovano kompleksan broj: (2.00 + 7.00 i)}#
#\izlaz{Argument kompleksnog broja: - 1.292497}#
\end{upotreba}
\end{maxitest}


\linkresenje{001}
\end{Exercise}
\begin{Answer}[ref=001]
\includecode{resenja/00_PodelaPoDatotekama/001.c}
\end{Answer}

\begin{Exercise}[label=002] % III sa vezbi
Uraditi prethodni zadatak tako da su sve napisane funkcije za rad sa kompleksnim brojevima zajedno sa definicijom strukture \kckod{KompleksanBroj} izdvojene u posebnu biblioteku. Test program treba da koristi tu biblioteku da za kompleksan broj unet sa standardnog ulaza ispiše polarni oblik unetog broja.

\begin{maxitest}
\begin{upotreba}{1}
#\naslovInt#
#\izlaz{Unesite realan i imaginaran deo kompleksnog broja:} \ulaz{-5 2 }#
#\izlaz{Polarni oblik kompleksnog broja je 5.39 *  e\^{}i * 2.76}#
\end{upotreba}
\end{maxitest}

\linkresenje{002}

\end{Exercise}
\begin{Answer}[ref=002]
\includecodeLib{resenja/00_PodelaPoDatotekama/002/complex.h}{complex.h}
\includecodeLib{resenja/00_PodelaPoDatotekama/002/complex.c}{complex.c}
\includecodeLib{resenja/00_PodelaPoDatotekama/002/main.c}{main.c}
\end{Answer}


\begin{Exercise}[label=003] % sa praktikuma
Napisati biblioteku za rad sa polinomima.
  \begin{enumerate}
  \item Definisati strukturu \kckod{Polinom} koja opisuje polinom stepena najviše 20. \uputstvo{Struktura sadrži stepen i niz
    koeficijenata. Redosled navođenja koeficijenata u nizu treba da bude takav da na nultoj poziciji u nizu bude koeficijent uz slobodan član, na prvoj koeficijent uz prvi stepen, itd.}
  \item Napisati funkciju \kckod{void ispisi(const Polinom * p)} koja ispisuje polinom \kckod{p} na standardni izlaz. Ispisivanje polinoma počinje od najvišeg stepena ka najnižem. Ipisisuju se samo oni koeficijenti koji su različiti od nule.
  \item Napisati funkciju \kckod{Polinom ucitaj()} koja učitava polinom sa standardnog
    ulaza. Za polinom se najpre unosi stepen pa njegovi koeficijenti.
  \item Napisati funkciju \kckod{double izracunaj(const Polinom * p, double x)} za izračunavanje vrednosti polinoma \kckod{p} u
    datoj tački \kckod{x} koristeći Hornerov algoritam.
  \item Napisati funkciju \kckod{Polinom saberi(const Polinom * p, const Polinom * q)} koja sabira dva polinoma \kckod{p} i \kckod{q}.
  \item Napisati funkciju \kckod{Polinom pomnozi(const Polinom * p, const Polinom * q)} koja množi dva polinoma \kckod{p} i \kckod{q}.
  \item Napisati funkciju \kckod{Polinom izvod(const Polinom * p)} koja računa izvod polinoma \kckod{p}.
  \item Napisati funkciju \kckod{Polinom nIzvod(const Polinom * p, int n)} koja računa \kckod{n}-ti izvod polinoma \kckod{p}.
  \end{enumerate}

Napisati program koji testira prethodno napisane funkcije. Najpre se polinomi \argf{p} i \argf{q} unose sa standardnog ulaza i ispisuju na standardni izlaz u odgovarajućem obliku. Zatim se računa i ispisuje zbir i proizvod polinoma \argf{p} i \argf{q}. Označimo izračunati proizvod sa  \argf{r}. Nakon toga program računa i ispisuje vrednost polinoma \argf{r} (zaokruženu na dve decimale) u tački koju unosi korisnik. Na kraju se sa standardnog ulaza unosi broj \argf{n}, i ispisuje \argf{n}-ti izvod polinoma \argf{r}.

\begin{maxitest}
\begin{upotreba}{1}
#\naslovInt#
#\izlaz{Unesite polinom p (prvo stepen, pa zatim koeficijente od najveceg stepena do nultog):}#
#\ulaz{3 1.2 3.5 2.1 4.2}#
#\izlaz{Unesite polinom q (prvo stepen, pa zatim koeficijente od najveceg stepena do nultog):}#
#\ulaz{2 2.1 0 -3.9}#
#\izlaz{Zbir polinoma je: 1.20x\^{}3+5.60x\^{}2+2.10x+0.30}#
#\izlaz{Prozvod polinoma je polinom r:}#
#\izlaz{2.52x\^{}5+7.35x\^{}4-0.27x\^{}3-4.83x\^{}2-8.19x-16.38}#
#\izlaz{Unesite tacku u kojoj racunate vrednost polinoma r}#
#\ulaz{0}#
#\izlaz{Vrednost polinoma u tacki je -16.38}#
#\izlaz{Unesite izvod polinoma koji zelite:}#
#\ulaz{3}#
#\izlaz{3. izvod polinoma r je: 151.20x\^{}2+176.40x-1.62}#
\end{upotreba}
\end{maxitest}

\linkresenje{003}

\end{Exercise}
\begin{Answer}[ref=003]
\includecodeLib{resenja/00_PodelaPoDatotekama/003/polinom.h}{polinom.h}
\includecodeLib{resenja/00_PodelaPoDatotekama/003/polinom.c}{polinom.c}
\includecodeLib{resenja/00_PodelaPoDatotekama/003/main.c}{main.c}
%\includecode{resenja/02_Bitovi/003/Makefile}
\end{Answer}

\begin{Exercise}[label=004] % sa praktikuma
Napisati biblioteku za rad sa razlomcima.

  \begin{enumerate}

  \item Definisati strukturu \kckod{Razlomak} za reprezentovanje razlomaka.
  \item Napisati funkciju \kckod{Razlomak ucitaj()} za učitavanje razlomaka.
  \item Napisati funkciju \kckod{void ispisi(const Razlomak * r)} koja ispisuje razlomak.
  \item Napisati funkciju \kckod{int brojilac(const Razlomak * r)} koje vraćaju brojilac razlomka \kckod{r}.
   \item Napisati funkciju \kckod{int imenilac(const Razlomak * r)} koje vraćaju imenilac razlomka \kckod{r}.
  \item Napisati funkciju \kckod{double realna\_vrednost(const Razlomak * r)} koja vraća odgovarajuću realnu vrednost razlomka \kckod{r}.
  \item Napisati funkciju  \kckod{double reciprocna\_vrednost(const Razlomak * r)} koja izračunava recipročnu vrednost
    razlomka \kckod{r}.
  \item Napisati funkciju \kckod{Razlomak skrati(const Razlomak * r)} koja skraćuje dati razlomak \kckod{r}.
  \item Napisati funkciju \kckod{Razlomak saberi(const Razlomak * r1, const Razlomak * r2)} koja sabira dva razlomka \kckod{r1} i \kckod{r2}.
 \item Napisati funkciju \kckod{Razlomak oduzmi(const Razlomak * r1, const Razlomak * r2)} koja oduzima dva razlomka \kckod{r1} i \kckod{r2}.
 \item Napisati funkciju \kckod{Razlomak pomnozi(const Razlomak * r1, const Razlomak * r2)} koja množi
    dva razlomka \kckod{r1} i \kckod{r2}.
 \item Napisati funkciju \kckod{Razlomak podeli(const Razlomak * r1, const Razlomak * r2)} koja deli
    dva razlomka \kckod{r1} i \kckod{r2}.
\end{enumerate}

Napisati program koji testira prethodne funkcije tako što se sa standardnog ulaza unose dva razlomka \argf{r1} i \argf{r2} i na standardni izlaz se ispisuju skraćene vrednosti razlomaka koji su dobijeni kao zbir, razlika, proizvod i količnik razlomka \argf{r1} i recipročne vrednosti razlomka \argf{r2}.

\begin{maxitest}
\begin{upotreba}{1}
#\naslovInt#
#\izlaz{Unesite imenilac i brojilac prvog razlomka:}\ulaz{1 2}#
#\izlaz{Unesite imenilac i brojilac drugog razlomka:}\ulaz{3 2}#
#\izlaz{1/2 + 3/2 = 2}#
#\izlaz{1/2 - 3/2 =  -1}#
#\izlaz{1/2 * 3/2 =  3/4}#
#\izlaz{1/2 / 3/2 =  1/3}#
\end{upotreba}
\end{maxitest}


%\linkresenje{004}

\end{Exercise}
%\begin{Answer}[ref=004]
%%\includecode{resenja/02_Bitovi/004.c}
%\end{Answer}

\section{Algoritmi za rad sa bitovima}

\begin{Exercise}[label=201]
Napisati biblioteku \kckod{stampanje\_bitova}  za rad sa bitovima koja sadrži funkciju \kckod{print\_bits} koja štampa bitove u binarnom zapisu neoznačenog celog broja $x$. Napisati program koja testira funkciju \kckod{print\_bits} za brojeve koji se sa standardnog ulaza zadaju u heksadekasnom formatu.

\begin{miditest}
\begin{test}{1}
#\naslovUlaz#
#\ulaz{0x7F}#
#\naslovIzlaz#
#\izlaz{00000000000000000000000001111111}#
\end{test}
\end{miditest}
\begin{miditest}
\begin{test}{2}
#\naslovUlaz#
#\ulaz{0x80}#
#\naslovIzlaz#
#\izlaz{00000000000000000000000010000000}#
\end{test}
\end{miditest}

\begin{miditest}
\begin{test}{3}
#\naslovUlaz#
#\ulaz{0x00FF00FF}#
#\naslovIzlaz#
#\izlaz{00000000111111110000000011111111}#
\end{test}
\end{miditest}
% \begin{miditest}
% \begin{test}{4}
% #\naslovUlaz#
% #\ulaz{0xFFFFFFFF}#
% #\naslovIzlaz#
% #\izlaz{11111111111111111111111111111111}#
% \end{test}
% \end{miditest}
\begin{miditest}
\begin{test}{4}
#\naslovUlaz#
#\ulaz{0xABCDE123}#
#\naslovIzlaz#
#\izlaz{10101011110011011110000100100011}#
\end{test}
\end{miditest}

\linkresenje{201}
\end{Exercise}
\begin{Answer}[ref=201]
\includecodeLib{resenja/02_Bitovi/stampanje_bitova.h}{stampanje\_bitova.h}
\includecodeLib{resenja/02_Bitovi/stampanje_bitova.c}{stampanje\_bitova.c}
\includecodeLib{resenja/02_Bitovi/201.c}{main.c}
\end{Answer}

\begin{Exercise}[label=202]
Napisati funkcije \kckod{count\_bits1} i \kckod{count\_bits2} koje broje bitove sa vrednošću \argf{1} u binarnom zapisu celog broja $x$. Prebrojavanje bitova ostvariti na dva načina:
\begin{enumerate}
\item formiranjem odgovarajuće maske i njenim pomeranjem
\item formiranjem odgovarajuće maske i pomeranjem promenljive $x$.
\end{enumerate} 
 
Napisati program koji za broj koji unosi u heksadekasnom formatu sa standardnog ulaza računa broj bitova sa vrednošću \argf{1} korišćenjem funkcije \kckod{count\_bits1} ili funkcije \kckod{count\_bits2}. Od korisnika sa standardnog ulaza tražiti da izabere koju od ove funkcije treba koristiit u zavisnosti da li unese \argf{1} ili \argf{2} .

\begin{miditest}
\begin{upotreba}{1}
#\naslovInt#
#\izlaz{Unesite broj:}\ulaz{0x7F}#
#\izlaz{Unesite redni broj funkcije:}\ulaz{1}#
#\izlaz{Broj jedinica u zapisu je}#
#\izlaz{funkcija count\_bits1: 7}#
\end{upotreba}
\end{miditest}
\begin{miditest}
\begin{upotreba}{2}
#\naslovInt#
#\izlaz{Unesite broj:}\ulaz{0x80}#
#\izlaz{Unesite redni broj funkcije:}\ulaz{2}#
#\izlaz{Broj jedinica u zapisu je}#
#\izlaz{funkcija count\_bits2: 1}#
\end{upotreba}
\end{miditest}

\begin{miditest}
\begin{upotreba}{3}
#\naslovInt#
#\izlaz{Unesite broj:}\ulaz{0x00FF00FF}#
#\izlaz{Unesite redni broj funkcije:}\ulaz{2}#
#\izlaz{Broj jedinica u zapisu je}#
#\izlaz{funkcija count\_bits2: 16}#
\end{upotreba}
\end{miditest}
% \begin{miditest}
% \begin{test}{4}
% #\naslovUlaz#
% #\ulaz{0xFFFFFFFF}#
% #\naslovIzlaz#
% #\izlaz{Broj jedinica u zapisu je}#
% #\izlaz{funkcija count\_bits1: 32}#
% #\izlaz{funkcija count\_bits2: 32}#
% \end{test}
% \end{miditest}
\begin{miditest}
\begin{upotreba}{4}
#\naslovInt#
#\izlaz{Unesite broj:}\ulaz{0xABCDE123}#
#\izlaz{Unesite redni broj funkcije:}\ulaz{1}#
#\izlaz{Broj jedinica u zapisu je}#
#\izlaz{funkcija count\_bits1: 17}#
\end{upotreba}
\end{miditest}

\linkresenje{202}
\end{Exercise}
\begin{Answer}[ref=202]
\includecode{resenja/02_Bitovi/202.c}
\end{Answer}

\begin{Exercise}[label=203]
Napisati funkciju \kckod{najveci} koja određuje najveći broj koji se može zapisati istim binarnim ciframa kao dati broj i funkciju \kckod{najmanji} koja određuje najmanji broj koji se može zapisati istim binarnim ciframa kao dati broj.

Napisati program koji testira prethodno napisane funkcije tako što prikazuje binarnu reprezentaciju brojeva koji se dobijaju nakon poziva funkcije \kckod{najveci}, odnosno \kckod{najmanji} za brojeve koji se zadaju u heksadekasnom formatu sa standardnog ulaza. 

\begin{miditest}
\begin{test}{1}
#\naslovUlaz#
#\ulaz{0x7F}#
#\naslovIzlaz#
#\izlaz{Najveci:}#
#\izlaz{11111110000000000000000000000000}#
#\izlaz{Najmanji:}#
#\izlaz{00000000000000000000000001111111}#
\end{test}
\end{miditest}
\begin{miditest}
\begin{test}{2}
#\naslovUlaz#
#\ulaz{0x80}#
#\naslovIzlaz#
#\izlaz{Najveci:}#
#\izlaz{10000000000000000000000000000000}#
#\izlaz{Najmanji:}#
#\izlaz{00000000000000000000000000000001}#
\end{test}
\end{miditest}

\begin{miditest}
\begin{test}{3}
#\naslovUlaz#
#\ulaz{0x00FF00FF}#
#\naslovIzlaz#
#\izlaz{Najveci:}#
#\izlaz{11111111111111110000000000000000}#
#\izlaz{Najmanji:}#
#\izlaz{00000000000000001111111111111111}#
\end{test}
\end{miditest}
\begin{miditest}
\begin{test}{4}
#\naslovUlaz#
#\ulaz{0xFFFFFFFF}#
#\naslovIzlaz#
#\izlaz{Najveci:}#
#\izlaz{11111111111111111111111111111111}#
#\izlaz{Najmanji:}#
#\izlaz{11111111111111111111111111111111}#
\end{test}
\end{miditest}

% \begin{miditest}
% \begin{test}{5}
% #\naslovUlaz#
% #\ulaz{0xABCDE123}#
% #\naslovIzlaz#
% #\izlaz{Najveci:}#
% #\izlaz{11111111111111111000000000000000}#
% #\izlaz{Najmanji:}#
% #\izlaz{00000000000000011111111111111111}#
% \end{test}
% \end{miditest}

\linkresenje{203}
\end{Exercise}
\begin{Answer}[ref=203]
\napomena{Rešenje koristi biblioteku za štampanje bitova iz zadatka \ref{201}.}
\includecode{resenja/02_Bitovi/203.c}
\end{Answer}


\begin{Exercise}[label=204]
Napisati funkcije za rad sa bitovima. \napomena{Pozicije se broje počev od pozicije bita najmanje težine, pri čemu je bit najmanje težine na poziciji nula.}
\begin{enumerate}
\item Napisati funkciju \kckod{reset} koja određuje broj koji se dobija kada se $n$ bitova datog broja $x$, počevši od pozicije $p$, postave na \argf{0}.
\item Napisati funkciju \kckod{set} koja određuje broj koji se dobija kada se $n$ bitova datog broja $x$, počevši od pozicije $p$, postave na \argf{1}.
\item %Napisati funkciju \kckod{get\_bits} koja određuje broj koji se dobija od $n$ bitova datog broja $x$, počevši od pozicije $p$, i vraća ih kao bitove najmanje težine rezultata.
Napisati funkciju \kckod{get\_bits} koja određuje broj u kome se $n$ bitova najmanje težine poklapa sa $n$ bitova broja $x$ počevši od pozicije $p$.
\item %Napisati funkciju \kckod{set\_n\_bits} koja vraća broj koji se dobija upisivanjem poslednjih $n$ bitova broja $y$ u broj $x$, počevši od pozicije $p$.
Napisati funkciju \kckod{set\_n\_bits} koja vraća broj koji se dobija upisivanjem poslednjih $n$ bitova najmanje težine broja $y$ u broj $x$, počevši od pozicije $p$.
\item Napisati funkciju \kckod{invert} koja vraća broj koji se dobija invertovanjem $n$ bitova broja $x$ počevši od pozicije $p$. 
\end{enumerate}
Napisati program koji testira prethodno napisane funkcije za neoznačene cele brojeve $x$, $n$, $p$, $y$ koji se unose sa standardnog ulaza. Program treba nakon učitavanja odgovarajućih vrednosti ispiše najpre binarne reprezenatacije brojeva $x$ i $y$, a potom i binarne reprezentacije brojeva koji se dobijaju pozivanjem prethodno napisanih funkcija.

\begin{maxitest}
\begin{upotreba}{1}
#\naslovInt#
#\izlaz{Unesite neoznacen ceo broj x:}\ulaz{235}#
#\izlaz{Unesite neoznacen ceo broj n:}\ulaz{9}#
#\izlaz{Unesite neoznacen ceo broj p:}\ulaz{24}#
#\izlaz{Unesite neoznacen ceo broj y:}\ulaz{127}#
#\izlaz{x =\ \ \ \ 235\ \ \ \ \ \ \ \ \ \ \ \ \ \ \ \ \ \ \ \ \ \ \ \ \ \ \ \ \ \ = 00000000000000000000000011101011}#
#\izlaz{reset(\ \ \ 235,\ \ \ \ \ 9,\ \ \ \ 24)\ \ \ \ \ \ \ \ \ \ \ \ \ = 00000000000000000000000011101011}#
#\izlaz{}#
#\izlaz{x =\ \ \ \ 235\ \ \ \ \ \ \ \ \ \ \ \ \ \ \ \ \ \ \ \ \ \ \ \ \ \ \ \ \ \ = 00000000000000000000000011101011}#
#\izlaz{set(\ \ \ 235,\ \ \ \ \ 9,\ \ \ \ 24)\ \ \ \ \ \ \ \ \ \ \ \ \ \ \ = 00000001111111110000000011101011}#
#\izlaz{}#
#\izlaz{x =\ \ \ \ 235\ \ \ \ \ \ \ \ \ \ \ \ \ \ \ \ \ \ \ \ \ \ \ \ \ \ \ \ \ \ = 00000000000000000000000011101011}#
#\izlaz{get\_bits(\ \ \ 235,\ \ \ \ \ 9,\ \ \ \ 24)\ \ \ \ \ \ \ \ \ \ = 00000000000000000000000000000000}#
#\izlaz{}#
#\izlaz{x =\ \ \ \ 235\ \ \ \ \ \ \ \ \ \ \ \ \ \ \ \ \ \ \ \ \ \ \ \ \ \ \ \ \ \ = 00000000000000000000000011101011}#
#\izlaz{y =\ \ \ \ 127\ \ \ \ \ \ \ \ \ \ \ \ \ \ \ \ \ \ \ \ \ \ \ \ \ \ \ \ \ \ = 00000000000000000000000001111111}#
#\izlaz{set\_n\_bits(\ \ \ 235,\ \ \ \ \ 9,\ \ \ \ 24,\ \ \ 127) = 00000000011111110000000011101011}#
#\izlaz{}#
#\izlaz{x =\ \ \ \ 235\ \ \ \ \ \ \ \ \ \ \ \ \ \ \ \ \ \ \ \ \ \ \ \ \ \ \ \ \ \ = 00000000000000000000000011101011}#
#\izlaz{invert(\ \ \ 235,\ \ \ \ \ 9,\ \ \ \ 24)\ \ \ \ \ \ \ \ \ \ \ \ = 00000001111111110000000011101011}#
\end{upotreba}
\end{maxitest}
\begin{maxitest}
\begin{upotreba}{2}
#\naslovInt#
#\izlaz{Unesite neoznacen ceo broj x:}\ulaz{2882398951}#
#\izlaz{Unesite neoznacen ceo broj n:}\ulaz{5}#
#\izlaz{Unesite neoznacen ceo broj p:}\ulaz{10}#
#\izlaz{Unesite neoznacen ceo broj y:}\ulaz{35156526}#
#\izlaz{x = 2882398951\ \ \ \ \ \ \ \ \ \ \ \ \ \ \ \ \ \ \ \ \ \ \ \ \ \ \ \ \ \ = 10101011110011011110101011100111}#
#\izlaz{reset(2882398951,\ \ \ \ \ 5,\ \ \ \ 10)\ \ \ \ \ \ \ \ \ \ \ \ \ = 10101011110011011110100000100111}#
#\izlaz{}#
#\izlaz{x = 2882398951\ \ \ \ \ \ \ \ \ \ \ \ \ \ \ \ \ \ \ \ \ \ \ \ \ \ \ \ \ \ = 10101011110011011110101011100111}#
#\izlaz{set(2882398951,\ \ \ \ \ 5,\ \ \ \ 10)\ \ \ \ \ \ \ \ \ \ \ \ \ \ \ = 10101011110011011110111111100111}#
#\izlaz{}#
#\izlaz{x = 2882398951\ \ \ \ \ \ \ \ \ \ \ \ \ \ \ \ \ \ \ \ \ \ \ \ \ \ \ \ \ \ = 10101011110011011110101011100111}#
#\izlaz{get\_bits(2882398951,\ \ \ \ \ 5,\ \ \ \ 10)\ \ \ \ \ \ \ \ \ \ = 00000000000000000000000000001011}#
#\izlaz{}#
#\izlaz{x = 2882398951\ \ \ \ \ \ \ \ \ \ \ \ \ \ \ \ \ \ \ \ \ \ \ \ \ \ \ \ \ \ = 10101011110011011110101011100111}#
#\izlaz{y =\ \ 35156526\ \ \ \ \ \ \ \ \ \ \ \ \ \ \ \ \ \ \ \ \ \ \ \ \ \ \ \ \ \ \ = 00000010000110000111001000101110}#
#\izlaz{set\_n\_bits(2882398951, \ \ 5, \ 10, \ 35156526) = 10101011110011011110101110100111}#
#\izlaz{}#
#\izlaz{x = 2882398951\ \ \ \ \ \ \ \ \ \ \ \ \ \ \ \ \ \ \ \ \ \ \ \ \ \ \ \ \ \ = 10101011110011011110101011100111}#
#\izlaz{invert(2882398951,\ \ \ \ \ 5,\ \ \ \ 10)\ \ \ \ \ \ \ \ \ \ \ \ = 10101011110011011110110100100111}#
\end{upotreba}
\end{maxitest}

\linkresenje{204}
\end{Exercise}
\begin{Answer}[ref=204]
\napomena{Rešenje koristi biblioteku za štampanje bitova iz zadatka \ref{201}.}
\includecode{resenja/02_Bitovi/204.c}
\end{Answer}

\begin{Exercise}[label=205]
Pod rotiranjem ulevo podrazumeva se pomeranje svih bitova za jednu poziciju ulevo, s tim što se bit sa pozicije najveće težine pomera na poziciju najmanje težine. Analogno, rotiranje udesno podrazumeva pomeranje svih bitova za jednu poziciju udesno, s tim što se bit sa pozicije najmanje težine pomera na poziciju najviše težine.
\begin{enumerate}
\item Napisati funkciju \kckod{rotate\_left} koja određuje broj koji se dobija rotiranjem \argf{k} puta ulevo datog celog broja \argf{x}. 
\item Napisati funkciju \kckod{rotate\_right} koja određuje broj koji se dobija rotiranjem \argf{k} puta udesno datog celog neoznačenog broja \argf{x}. 
\item Napisati funkciju \kckod{rotate\_right\_signed} koja određuje broj koji se dobija rotiranjem \argf{k} puta udesno datog celog broja \argf{x}. 
\end{enumerate}
Napisati program koji testira prethodno napisane funkcije za broj \argf{x} i broj \argf{k} koji se unose u heksadekasnom formatu sa standardnog ulaza.


\begin{maxitest}
\begin{upotreba}{1}
#\naslovInt#
#\izlaz{Unesite neoznacen ceo broj x:}\ulaz{B10011A7}#
#\izlaz{Unesite neoznacen ceo broj k:}\ulaz{5}#
#\izlaz{x  = 10110001000000000001000110100111 }#
#\izlaz{rotate\_left(2969571751, 5) \ \ \ \ \ \ \ \ = 00100000000000100011010011110110}#
#\izlaz{rotate\_right(2969571751, 5) \ \ \ \ \ \ \ = 00111101100010000000000010001101}#
#\izlaz{rotate\_right\_signed(2969571751, 5) = 00111101100010000000000010001101 }#
\end{upotreba}
\end{maxitest}

\linkresenje{205}
\end{Exercise}
\begin{Answer}[ref=205]
\napomena{Rešenje koristi biblioteku za štampanje bitova iz zadatka \ref{201}.}
\includecode{resenja/02_Bitovi/205.c}
\end{Answer}

\begin{Exercise}[label=206]
Napisati funkciju \kckod{mirror} koja određuje ceo broj čiji je binarni zapis slika u ogledalu binarnog zapisa argumenta funkcije. Napisati i program koji testira datu funkciju za brojeve koji se sa standardnog ulaza zadaju u heksadekasnom formatu, tako što najpre ispisuje binarnu reprezentaciju unetog broja, a potom i binarnu reprezentaciju broja dobijenog nakon poziva funkcije \kckod{mirror} za uneti broj.

\begin{miditest}
\begin{test}{1}
#\naslovUlaz#
#\ulaz{255 }#
#\naslovIzlaz#
#\izlaz{00000000000000000000001001010101}#
#\izlaz{10101010010000000000000000000000}#
\end{test}
\end{miditest}
\begin{miditest}
\begin{test}{2}
#\naslovUlaz#
#\ulaz{-15}#
#\naslovIzlaz#
#\izlaz{11111111111111111111111111101011}#
#\izlaz{11010111111111111111111111111111}#
\end{test}
\end{miditest}

\linkresenje{206}
\end{Exercise}
\begin{Answer}[ref=206]
\napomena{Rešenje koristi biblioteku za štampanje bitova iz zadatka \ref{201}.}
\includecode{resenja/02_Bitovi/206.c}
\end{Answer}

%%%%%%%%%%%%%%%%%%%%%%%%%
% Zadaci sa praktikuma - obavezni zadaci 
%%%%%%%%%%%%%%%%%%%%%%%%%

\begin{Exercise}[label=207]
Napisati funkciju \kckod{int Broj01(unsigned int n)} koja za dati broj \argf{n} vraća \argf{1} ako u njegovom binarnom zapisu ima više jednica nego nula, a inače vraća \argf{0}.  Napisati program koji tu funkciju testira za broj koji se zadaje sa standardnog ulaza.

\begin{minitest}
\begin{test}{1}
#\naslovUlaz#
#\ulaz{10}#
#\naslovIzlaz#
#\izlaz{0}#
\end{test}
\end{minitest}
%\begin{minitest}
%\begin{test}{2}
%#\naslovUlaz#
%#\ulaz{1024}#
%#\naslovIzlaz#
%#\izlaz{0}#
%\end{test}
%\end{minitest}
\begin{minitest}
\begin{test}{2}
#\naslovUlaz#
#\ulaz{2147377146}#
#\naslovIzlaz#
#\izlaz{1}#
\end{test}
\end{minitest}
\begin{minitest}
\begin{test}{3}
#\naslovUlaz#
#\ulaz{1111111115}#
#\naslovIzlaz#
#\izlaz{0}#
\end{test}
\end{minitest}

\linkresenje{207}
\end{Exercise}
\begin{Answer}[ref=207]
\includecode{resenja/02_Bitovi/207.c}
\end{Answer}

\begin{Exercise}[label=208]
Napisati funkciju koja broji koliko se puta dve uzastopne jedinice pojavljuju u binarnom zapisu
  celog neoznačenog broja $x$. Napisati program koji tu funkciju testira za broj koji se
  zadaje sa standardnog ulaza. \napomena{Tri uzastopne jedinice sadrže dve uzastopne jedinice dva puta.}
  
\begin{minitest}
\begin{test}{1}
#\naslovUlaz#
#\ulaz{11}#
#\naslovIzlaz#
#\izlaz{1}#
\end{test}
\end{minitest}
\begin{minitest}
\begin{test}{2}
#\naslovUlaz#
#\ulaz{1024}#
#\naslovIzlaz#
#\izlaz{0}#
\end{test}
\end{minitest}
\begin{minitest}
\begin{test}{3}
#\naslovUlaz#
#\ulaz{2147377146}#
#\naslovIzlaz#
#\izlaz{22}#
\end{test}
\end{minitest}  

\linkresenje{208}
\end{Exercise}
\begin{Answer}[ref=208]
\includecode{resenja/02_Bitovi/208.c}
\end{Answer}


%%%
%Ovaj 209.c nema resenje za sad...
%%%
\begin{Exercise}[label=209]%\marker+{2}
Napisati program koji sa standardnog ulaza učitava pozitivan ceo broj, a na standardni izlaz ispisuje vrednost tog broja sa razmenjenim vrednostima bitova na pozicijama $i$ i $j$. Pozicije $i$ i $j$ se učitavaju kao parametri
  komandne linije. Smatrati da je krajnji desni bit binarne
  reprezentacije \argf{0}-ti bit. Pri rešavanju nije dozvoljeno koristiti
  ni pomoćni niz ni aritmetičke operatore +, -, /, *, \%.

\begin{minitest}
\begin{upotreba}{1}
#\poziv{./a.out 1 2 }#

#\naslovInt#
#\naslovUlaz#
#\ulaz{11}#
#\naslovIzlaz#
#\izlaz{13}#
\end{upotreba}
\end{minitest}
\begin{minitest}
\begin{upotreba}{2}
#\poziv{./a.out 1 2 }#

#\naslovInt#
#\naslovUlaz#
#\ulaz{1024}#
#\naslovIzlaz#
#\izlaz{1024}#
\end{upotreba}
\end{minitest}
\begin{minitest}
\begin{upotreba}{2}
#\poziv{./a.out 12 12 }#

#\naslovInt#
#\naslovUlaz#
#\ulaz{12345}#
#\naslovIzlaz#
#\izlaz{12345}#
\end{upotreba}
\end{minitest}

\end{Exercise}
%\begin{Answer}[ref=209]
%\includecode{resenja/02_Bitovi/209.c}
%\end{Answer}

\begin{Exercise}[label=210]
  Napisati funkciju koja na osnovu neoznačenog broja $x$
  formira nisku $s$ koja sadrži heksadekadni zapis broja
  $x$ koristeći algoritam za brzo prevođenje binarnog u
  heksadekadni zapis (svake $4$ binarne cifre se zamenjuju jednom
  odgovarajućom heksadekadnom cifrom).  Napisati program koji tu
  funkciju testira za broj koji se zadaje sa standardnog ulaza.

\begin{minitest}
\begin{test}{1}
#\naslovUlaz#
#\ulaz{11}#
#\naslovIzlaz#
#\izlaz{0000000B}#
\end{test}
\end{minitest}
\begin{minitest}
\begin{test}{2}
#\naslovUlaz#
#\ulaz{1024}#
#\naslovIzlaz#
#\izlaz{00000400}#
\end{test}
\end{minitest}
\begin{minitest}
\begin{test}{3}
#\naslovUlaz#
#\ulaz{12345}#
#\naslovIzlaz#
#\izlaz{00003039}#
\end{test}
\end{minitest}  

\linkresenje{210}
\end{Exercise}
\begin{Answer}[ref=210]
\includecode{resenja/02_Bitovi/210.c}
\end{Answer}

%%%%%%%%%%%%%%%%%%%%%%%%%
% Zadaci sa praktikuma - dodatni zadaci - oni nemaju rešenja
% možda \subsection{Dodatni zadaci} ili zadaci za vežbu
%%%%%%%%%%%%%%%%%%%%%%%%%

\begin{Exercise}[label=211]%\marker+{2}
  Napisati funkciju koja za data dva neoznačena broja $x$
  i $y$ invertuje u podatku $x$ one bitove koji se poklapaju
  sa odgovarajućim bitovima u broju $y$. Ostali bitovi ostaju
  nepromenjeni.  Napisati program koji tu funkciju testira za brojeve
  koji se zadaju sa standardnog ulaza.
  
\begin{minitest}
\begin{test}{1}
#\naslovUlaz#
#\ulaz{123 10}#
#\naslovIzlaz#
#\izlaz{4294967285}#
\end{test}
\end{minitest}
\begin{minitest}
\begin{test}{2}
#\naslovUlaz#
#\ulaz{3251 0}#
#\naslovIzlaz#
#\izlaz{4294967295}#
\end{test}
\end{minitest}
\begin{minitest}
\begin{test}{3}
#\naslovUlaz#
#\ulaz{12541 1024}#
#\naslovIzlaz#
#\izlaz{4294966271}#
\end{test}
\end{minitest}    
  
\end{Exercise}
%\begin{Answer}[ref=211]
%\includecode{resenja/02_Bitovi/211.c}
%\end{Answer}

\begin{Exercise}[label=212]%\marker+{2}
Napisati funkciju koja računa koliko petica bi imao ceo
  neoznačen broj $x$ u oktalnom zapisu. Napisati program koji
  tu funkciju testira za broj koji se zadaje sa standardnog ulaza. \napomena{Zadatak rešiti isključivo korišćenjem bitskih operatora.}
  
\begin{minitest}
\begin{test}{1}
#\naslovUlaz#
#\ulaz{123}#
#\naslovIzlaz#
#\izlaz{0}#
\end{test}
\end{minitest}
\begin{minitest}
\begin{test}{2}
#\naslovUlaz#
#\ulaz{3245}#
#\naslovIzlaz#
#\izlaz{2}#
\end{test}
\end{minitest}
\begin{minitest}
\begin{test}{3}
#\naslovUlaz#
#\ulaz{100328}#
#\naslovIzlaz#
#\izlaz{1}#
\end{test}
\end{minitest}   
 
\end{Exercise}
%\begin{Answer}[ref=212]
%\includecode{resenja/02_Bitovi/212.c}
%\end{Answer}
