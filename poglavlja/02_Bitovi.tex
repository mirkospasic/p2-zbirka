
\chapter{Bitovi}

\section{Zadaci}



\begin{Exercise}[label=201]
Napisati:
\begin{enumerate}
\item funkciju \kckod{print\_bits} koja štampa bitove u binarnom zapisu celog broja $x$.
\item program koja testira funkciju \kckod{print\_bits} za proizvoljne brojeve.
\end{enumerate}
\end{Exercise}
\begin{Answer}[ref=201]
\includecode{resenja/02_Bitovi/201.c}
\end{Answer}

\begin{Exercise}[label=202]
 Napisati funkciju koja broji bitove postavljene na \argf{1} u zapisu broja $x$. Napisati program koji testira tu funkciju za broj \argf{0xABCDE123}.
\end{Exercise}
\begin{Answer}[ref=202]
%\includecode{resenja/02_Bitovi/202.c}
\end{Answer}


\begin{Exercise}[label=203]
Napisati:
\begin{enumerate}
\item funkciju koja određuje najveći broj koji se može zapisati istim binarnim ciframa kao dati broj.
\item funkciju koja određuje najmanji broj koji se može zapisati istim binarnim ciframa kao dati broj.
\item program koji testira prethodno napisane funkcije.
\end{enumerate}
\end{Exercise}
\begin{Answer}[ref=203]
%\includecode{resenja/02_Bitovi/203.c}
\end{Answer}


\begin{Exercise}[label=204]
Napisati:
\begin{enumerate}
\item funkciju funkciju koja određuje broj koji se dobija kada se $n$ bitova datog broja, počevši od pozicije $p$ postave na \argf{0}.
\item funkciju koja određuje broj koji se dobija kada se $n$ bitova datog broja, počevši od pozicije $p$ postave na \argf{1}.
\item funkciju koja određuje broj koji se dobija kada se $n$ bitova datog broja, počevši od pozicije $p$ i vraća ih kao bitove najmanje težine rezultata.
\item funkciju koja vraća broj koji se dobija upisivanjem poslednjih $n$ bitova broja $y$ u broj $x$, počevši od pozicije $p$.
\item funkciju koja vraća broj koji se dobija invertovanjem $n$ bitova broja $x$ počevši od pozicije $p$.
\item program koji testira prethodno napisane funkcije.
\end{enumerate}
\end{Exercise}
\begin{Answer}[ref=204]
%\includecode{resenja/02_Bitovi/204.c}
\end{Answer}


\begin{Exercise}[label=205]
Napisati:
\begin{enumerate}
\item funkciju funkciju koja određuje broj koji se dobija rotiranjem u levo datog celog broja.
\item funkciju koja određuje broj koji se dobija rotiranjem u desno datog celog neoznacenog broja. 
\item funkciju koja određuje broj koji se dobija rotiranjem u desno datog celog broja.
\item program koji testira prethodno napisane funkcije.
\end{enumerate}
\end{Exercise}
\begin{Answer}[ref=205]
%\includecode{resenja/02_Bitovi/205.c}
\end{Answer}

\begin{Exercise}[label=206]
Napisati funkciju koja određuje ceo broj čiji binarni zapis je slika u ogledalu binarnog zapisa argumenta funkcije, a potom i program koji testira datu funkciju za argument \argf{0xABCDE123}.
\end{Exercise}
\begin{Answer}[ref=206]
%\includecode{resenja/02_Bitovi/206.c}
\end{Answer}

%%%%%%%%%%%%%%%%%%%%%%%%%
% Zadaci sa praktikuma - obavezni zadaci 
%%%%%%%%%%%%%%%%%%%%%%%%%

\begin{Exercise}[label=207]
Napisati funkciju \kckod{int Broj01(unsigned int n)} koja za dati broj \argf{n} vraća \argf{1} ako u njegovom binarnom zapisu ima više jednica nego nula, a inače vraća \argf{0}.  Napisati program koji tu funkciju testira za broj koji se zadaje sa standardnog ulaza.

\begin{minitest}
\begin{test}{Test 1}
Ulaz:   10
Izlaz:  0 
\end{test}
\end{minitest}
\begin{minitest}
\begin{test}{Test 2}
Ulaz:   1024
Izlaz:  0 
\end{test}
\end{minitest}
\begin{minitest}
\begin{test}{Test 3}
Ulaz:   2147377146
Izlaz:  1 
\end{test}
\end{minitest}

\begin{minitest}
\begin{test}{Test 4}
Ulaz:   1111111115
Izlaz:  0 
\end{test}
\end{minitest}

\end{Exercise}
\begin{Answer}[ref=207]
%\includecode{resenja/02_Bitovi/207.c}
\end{Answer}

\begin{Exercise}[label=208]
Napisati funkciju koja broji koliko se puta kombinacija
  \argf{11} (dve uzastopne jedinice) pojavljuje u binarnom zapisu
  celog neoznačenog broja $x$. Tri uzastopne jedinice se broje
  dva puta.  Napisati program koji tu funkciju testira za broj koji se
  zadaje sa standardnog ulaza.
  
\begin{minitest}
\begin{test}{Test 1}
Ulaz:   11  
Izlaz:  1    
\end{test}
\end{minitest}
\begin{minitest}
\begin{test}{Test 2}
Ulaz:   1024
Izlaz:  0 
\end{test}
\end{minitest}
\begin{minitest}
\begin{test}{Test 3}
Ulaz:   2147377146
Izlaz:  22
\end{test}
\end{minitest}

\begin{minitest}
\begin{test}{Test 4}
Ulaz:   1111111115
Izlaz:  6 
\end{test}
\end{minitest}
\end{Exercise}
\begin{Answer}[ref=208]
%\includecode{resenja/02_Bitovi/208.c}
\end{Answer}


%%%
%Ovaj 209.c nema resenje za sad...
%%%
\begin{Exercise}[label=209]
Napisati program koji sa standardnog ulaza učitava pozitivan
  ceo broj, a na standardni izlaz ispisuje vrednost tog broja sa
  razmenjenim vrednostima bitova na pozicijama $i$,
  $j$. Pozicije $i$, $j$ se učitavaju kao parametri
  komandne linije. Smatrati da je krajnji desni bit binarne
  reprezentacije \argf{0}-ti bit. Pri rešavanju nije dozvoljeno koristiti
  pomoćni niz niti aritmetičke operatore +,-,/,*,\%.

\begin{minitest}
\begin{test}{Test 1}
Poziv:  ./a.out 1 2 
Ulaz:   11             
Izlaz:  13            
\end{test}
\end{minitest}
\begin{minitest}
\begin{test}{Test 2}
Poziv: ./a.out 1 2
Ulaz:   1024     
Izlaz:  1024        
\end{test}
\end{minitest}
\begin{minitest}
\begin{test}{Test 3}
Poziv: ./a.out 12 12
Ulaz:   12345
Izlaz:  12345
\end{test}
\end{minitest}
\end{Exercise}
\begin{Answer}[ref=209]
%\includecode{resenja/02_Bitovi/209.c}
\end{Answer}

\begin{Exercise}[label=210]
  Napisati funkciju koja na osnovu neoznačenog broja $x$
  formira nisku $s$ koja sadrži heksadekadni zapis broja
  $x$, koristeći algoritam za brzo prevođenje binarnog u
  heksadekadni zapis (svake $4$ binarne cifre se zamenjuju jednom
  odgovarajućom heksadekadnom cifrom).  Napisati program koji tu
  funkciju testira za broj koji se zadaje sa standardnog ulaza.

\begin{minitest}
\begin{test}{Test 1}
Ulaz:   11             
Izlaz:  0000000B      
\end{test}
\end{minitest}
\begin{minitest}
\begin{test}{Test 2}
Ulaz:  1024        
Izlaz: 00000400  
\end{test}
\end{minitest}
\begin{minitest}
\begin{test}{Test 3}
Ulaz:  12345
Izlaz: 00003039
\end{test}
\end{minitest}
\end{Exercise}
\begin{Answer}[ref=210]
%\includecode{resenja/02_Bitovi/210.c}
\end{Answer}

%%%%%%%%%%%%%%%%%%%%%%%%%
% Zadaci sa praktikuma - dodatni zadaci - oni nemaju rešenja
% možda \subsection{Dodatni zadaci} ili zadaci za vežbu
%%%%%%%%%%%%%%%%%%%%%%%%%

\begin{Exercise}[label=211]
  Napisati funkciju koja za dva data neoznačena broja $x$
  i $y$ invertuje u podatku $x$ one bitove koji se poklapaju
  sa odgovarajućim bitovima u broju $y$. Ostali bitovi ostaju
  nepromenjeni.  Napisati program koji tu funkciju testira za brojeve
  koji se zadaju sa standardnog ulaza.
  
\begin{minitest}
\begin{test}{Test 1}
Ulaz:   123 10        
Izlaz:  4294967285    
\end{test}
\end{minitest}
\begin{minitest}
\begin{test}{Test 2}
Ulaz:  3251 0    
Izlaz: 4294967295    
\end{test}
\end{minitest}
\begin{minitest}
\begin{test}{Test 3}
Ulaz:   12541 1024
Izlaz:  4294966271
\end{test}
\end{minitest}
\end{Exercise}
\begin{Answer}[ref=211]
%\includecode{resenja/02_Bitovi/211.c}
\end{Answer}

\begin{Exercise}[label=212]
Napisati funkciju koja računa koliko petica bi imao ceo
  neoznačen broj $x$ u oktalnom zapisu. Napisati program koji
  tu funkciju testira za broj koji se zadaje sa standardnog ulaza.
  
\begin{minitest}
\begin{test}{Test 1}
Ulaz:   123        
Izlaz:  0             
\end{test}
\end{minitest}
\begin{minitest}
\begin{test}{Test 2}
Ulaz:   3245      
Izlaz:  2              
\end{test}
\end{minitest}
\begin{minitest}
\begin{test}{Test 3}
Ulaz:   100328
Izlaz:  1
\end{test}
\end{minitest}  
\end{Exercise}
\begin{Answer}[ref=212]
%\includecode{resenja/02_Bitovi/212.c}
\end{Answer}


\section{Rešenja}
\shipoutAnswer
