\chapter{Uvodni zadaci}

\section{Podela koda po datotekama}

\komentar{Milena: izdovjiti resenja u posebno poglavlje, da ne bude u folderu o bitovima, npr 00\_PodelaPoDatotekama}

\begin{Exercise}[label=001] % I sa vezbi
Napisati program za rad sa kompleksnim brojevima.
\begin{enumerate}
\item Definisati strukturu \kckod{KompleksanBroj} koja predstavlja kompleksan broj i sadrži realan i imaginaran deo kompleksnog broja.
\item Napisati funkciju \kckod{ucitaj\_kompleksan\_broj} koja učitava kompleksan broj sa standardnog ulaza.
\item Napisati funkciju \kckod{ispisi\_kompleksan\_broj} koja ispisuje kompleksan broj na standardni izlaz u fomatu (npr. broj čiji je realan deo \argf{2} a imaginarni \argf{-3} ispisati kao $(2-3i)$ na standardni izlaz).
\item Napisati funkciju \kckod{realan\_deo} koja računa vrednosti realnog dela broja.
\item Napisati funkciju \kckod{imaginaran\_deo} koja računa vrednosti imaginarnog dela broja.
\item Napisati funkciju \kckod{moduo} koja računa moduo kompleksnog broja.
\item Napisati funkciju \kckod{konjugovan} koja računa konjugovano-kompleksni broj svog argumenta.
\item Napisati funkciju \kckod{saberi} koja sabira dva kompleksna broja.
\item Napisati funkciju \kckod{oduzmi} koja oduzima dva kompleksna broja.
\item Napisati funkciju \kckod{mnozi} koja množi dva kompleksna broja.
\item Napisati funkciju \kckod{argument} koja računa argument kompleksnog broja.
\end{enumerate}
Napisati program koji testira prethodno napisane funkcije za dva kompleksna broja $z1$ i $z2$ koja se unose sa standardnog ulaza i ispisuje:
\begin{enumerate}
\item realni deo, imaginarni deo i moduo kompleksnog broja $z1$,
\item konjugovano kompleksan broj i argument broja $z2$,
\item zbir, razliku i proizvod brojeva $z1$ i $z2$.
\end{enumerate}

\begin{maxitest}
\begin{test}{Test 1}
Ulaz:   Unesite realan i imaginaran deo kompleksnog broja: 1 -3 
          Unesite realan i imaginaran deo kompleksnog broja: -1 4 
Izlaz:  (1.00 -3.00 i )
          realan_deo: 1
          imaginaran_deo: -3.000000
          moduo 3.162278

          (-1.00 + 4.00 i )
          Njegov konjugovano kompleksan broj: (-1.00 -4.00 i )
          Argument kompleksnog broja 1.815775

          (1.00 -3.00 i ) + (-1.00 + 4.00 i )  =  ( 1.00 i )
          (1.00 -3.00 i ) - (-1.00 + 4.00 i )  =  (2.00 -7.00 i )
          (1.00 -3.00 i ) * (-1.00 + 4.00 i )  =  (11.00 + 7.00 i ) 
\end{test}
\end{maxitest}

\linkresenje{001}

\end{Exercise}
\begin{Answer}[ref=001]
\includecode{resenja/02_Bitovi/001.c}
\end{Answer}

\begin{Exercise}[label=002] % III sa vezbi
Uraditi prethodni zadatak tako da su sve napisane funkcije za rad sa kompleknim brojevima zajedno sa definicijom strukture \kckod{KompleksanBroj} izdvojene u posebnu biblioteku, dok test program koristi tu biblioteku da za kompleksan broj unet sa standardnog ulaza ispiše polarni oblik unetog broja.

\begin{maxitest}
\begin{test}{Test 1}
Ulaz:   Unesite realan i imaginaran deo kompleksnog broja: -5 2 
Izlaz:   Polarni oblik kompleksnog broj je 5.39 *  e^i * 2.76
\end{test}
\end{maxitest}

\linkresenje{002}

\end{Exercise}
\begin{Answer}[ref=002]
\includecode{resenja/02_Bitovi/002/complex.c}
\includecode{resenja/02_Bitovi/002/complex.h}
\includecode{resenja/02_Bitovi/002/main.c}
\end{Answer}


\begin{Exercise}[label=003] % sa praktikuma
Napisati malu biblioteku za rad sa polinomima.

  \begin{enumerate}
  \item Definisati strukturu \kckod{Polinom} koja predstavlja polinom
    (stepena najviše 20). Struktura sadrži stepen i niz
    koeficijenata. (Diskutovati redosled navođenja koeficijenata u
    nizu). \komentar{Milna: preformulisati u uputstvo ili obrisati - diskusija se ne pominje na drugim mestima}
  \item Napisati funkciju koja ispisuje polinom na standardni izlaz u
    što lepšem obliku.
  \item Napisati funkciju koja učitava polinom sa standardnog
    ulaza.
  \item Napisati funkciju za izračunavanje vrednosti polinoma u
    datoj tački (diskutovati Hornerov algoritam i njegove
    prednosti). \komentar{Milna: preformulisati u uputstvo ili obrisati - diskusija se ne pominje na drugim mestima}
  \item Napisati funkciju koja sabira dva polinoma.
  \item Napisati funkciju koja množi dva polinoma.
  \end{enumerate}

Napisati program koji testira prethodno napisane funkcije...

\komentar{Dopuniti prethodnu recenicu, a nacin prevodjenja prebaciti u komentar kod resenja, ukoliko to vec nije dato uz resenje prethodnog zadatka.}
\iffalse
\begin{verbatim}
gcc -o test-polinom polinom.c test-polinom.c
\end{verbatim}
i 
\begin{verbatim}
gcc -c polinom.c 
gcc -c test-polinom.c
gcc -o test-polinom polinom.o test-polinom.o
\end{verbatim}
 \fi
%
%''Distri\-bu\-i\-ra\-ti'' datoteke \verb"polinom.h" i \verb"polinom.o"
%(ne \verb"polinom.c") - na primer premeŠtanjem u novi
%direktorijum. Napisati novi test program, prevesti ga i povezati.

\begin{maxitest}
\begin{test}{Upotreba programa 1}
  Unesite polinom (prvo stepen, pa zatim koeficijente od najveceg stepena do nultog):
  3 1 2 3 4
  1.00x^3+2.00x^2+3.00x+4.00
  Unesite tacku u kojoj racunate vrednost polinoma
  5
  Vrednost polimoma u tacki je 194.00
  Unesite drugi polinom (prvo stepen, pa zatim koeficijente od najveceg stepena do nultog):
  2 1 0 1
  Zbir polinoma 1.00x^3+2.00x^2+3.00x+4.00 i   1.00x^2+1.00 polinoma je :  1.00x^3+3.00x^2+3.00x+5.00  
  Zbir polinoma 1.00x^3+2.00x^2+3.00x+4.00 i   1.00x^2+1.00 polinoma je :   1.00x^5+2.00x^4+4.00x^3+6.00x^2+3.00x+4
  Unosite izvod polinoma koji zelite:
  2
  2. izvod polinoma 1.00x^3+2.00x^2+3.00x+4.00 je : 6.00x+4.00
\end{test}
\end{maxitest}

\linkresenje{003}

\end{Exercise}
\begin{Answer}[ref=003]
\includecode{resenja/02_Bitovi/003/polinom.c}
\includecode{resenja/02_Bitovi/003/polinom.h}
\includecode{resenja/02_Bitovi/003/main.c}
%\includecode{resenja/02_Bitovi/003/Makefile}
\end{Answer}

\begin{Exercise}[label=004] % sa praktikuma
Napraviti biblioteku za rad sa razlomcima.

  \begin{enumerate}

  \item Definisati strukturu za reprezentovanje razlomaka.

  \item Napisati funkcije za učitavanje i ispis razlomaka.

  \item Napisati funkcije koje vraćaju brojilac i imenilac.

  \item Napisati funkciju koja vraća vrednost razlomka kao  \kckod{double}
    vrednost.

  \item Napisati funkciju koja izračunava recipročnu vrednost
    razlomka.

  \item Napisati funkciju koja skraćuje dati razlomak.

  \item Napisati funkcije koje sabiraju, oduzimaju, množe i dele
    dva razlomka.

  \end{enumerate}

Napisati program koji testira prethodne funkcije tako što se sa standardnog ulaza unose dva razlomka \argf{r1} i \argf{r2} i na standardni izlaz se ispisuju skraćeni vrednosti razlomaka koji predstavljaju zbir, razliku, proizvod i količnik razlomka \argf{r1} i recipročne vrednosti razlomka \argf{r2}.

\begin{maxitest}
\begin{test}{Test 1}
Ulaz:   Unesite imenilac i brojilac: 1 2
          Unesite imenilac i brojilac: 3 1 
Izlaz:   Zbir je 5/6
           Razlika je 1/6
           Proizvod je 1/6
           Kolicnik je 3/2
\end{test}
\end{maxitest}

%\linkresenje{004}

\end{Exercise}
%\begin{Answer}[ref=004]
%%\includecode{resenja/02_Bitovi/004.c}
%\end{Answer}

\section{Algoritmi za rad sa bitovima}

\komentar{Milena: izbaci kratke opise zadataka sa pocetka resenja. Izbaci link na make alate. U komentarima imas velika slova na mestma gde su bila nekada nasa slova... Izbegavala bih komentare koji narusavaju citljivost koda, npr komentar na strukturu treba da bude pre strukture a ne u okviru strukture (to je kod polinoma). Print bits je dva puta detaljno komentarisana, dosta je jednom, a i to bi trebalo da bude malo krace. Zadatak 1.6 se ne poklapa sa resenjem. U resenju su date dve funkcije, a to se u tekstu zadatka ne trazi --- mozda bi bilo bolje razdvojiti to u dva zadatka. Zadatak 1.8 ima ponovo komentare za print\_bits, isto i 1.9... Svuda nadalje to obrisati, ostaviti uz funkciju samo kratak komentar tipa "funkcija stampa bitove..." tj tako nesto kratko ali pismeno. Resenje 1.11 --- nedostaje kratak komentar sta radi funkcija Broj01. Isto i u narednom broju parova. Obrisi suvisne nove redove. Toliko za sada... }

\begin{Exercise}[label=201]
Napisati funkciju \kckod{print\_bits} koja štampa bitove u binarnom zapisu celog broja $x$. Napisati program koja testira funkciju \kckod{print\_bits} za brojeve koji se sa standardnog ulaza zadaju u heksadekasnom formatu.


\begin{maxitest}
\begin{test}{Test 1}
Ulaz:   0x7F  
Izlaz:  0000 0000 0000 0000 0000 0000 0111 1111    
\end{test}
\end{maxitest}

\begin{maxitest}
\begin{test}{Test 2}
Ulaz:   0x80
Izlaz:  0000 0000 0000 0000 0000 0000 1000 0000 
\end{test}
\end{maxitest}

\begin{maxitest}
\begin{test}{Test 3}
Ulaz:   0x00FF00FF
Izlaz:  0000 0000 1111 1111 0000 0000 1111 1111
\end{test}
\end{maxitest}

\begin{maxitest}
\begin{test}{Test 4}
Ulaz:   0xFFFFFFFF
Izlaz:  1111 1111 1111 1111 1111 1111 1111 1111 
\end{test}
\end{maxitest}

\begin{maxitest}
\begin{test}{Test }
Ulaz:   0xABCDE123
Izlaz:  1010 1011 1100 1101 1110 0001 0010 0011
\end{test}
\end{maxitest}

\linkresenje{201}
\end{Exercise}
\begin{Answer}[ref=201]
\includecode{resenja/02_Bitovi/201.c}
\end{Answer}

\begin{Exercise}[label=202]
 Napisati funkciju koja broji bitove postavljene na \argf{1} u zapisu broja $x$. Napisati program koji testira tu funkciju za brojeve koji se sa standardnog ulaza zadaju u heksadekasnom formatu.

\begin{minitest}
\begin{test}{Test 1}
Ulaz:   0x7F  
Izlaz:  
  Broj bitova u zapisu je  7   
\end{test}
\end{minitest}
\begin{minitest}
\begin{test}{Test 2}
Ulaz:   0x80
Izlaz:  
  Broj bitova u zapisu je 1
\end{test}
\end{minitest}
\begin{minitest}
\begin{test}{Test 3}
Ulaz:   0x00FF00FF
Izlaz:  
  Broj bitova u zapisu je 16
\end{test}
\end{minitest}

\begin{minitest}
\begin{test}{Test 4}
Ulaz:   0xFFFFFFFF
Izlaz:  
  Broj bitova u zapisu je 32
\end{test}
\end{minitest}
\begin{minitest}
\begin{test}{Test 4}
Ulaz:   0xABCDE123
Izlaz:  
  Broj bitova u zapisu je 17
\end{test}
\end{minitest}

\linkresenje{202}
\end{Exercise}
\begin{Answer}[ref=202]
\includecode{resenja/02_Bitovi/202.c}
\end{Answer}


\begin{Exercise}[label=203]
Napisati funkciju \kckod{najveci} koja određuje najveći broj koji se može zapisati istim binarnim ciframa kao dati broj i funkciju \kckod{najmanji} koja određuje najmanji broj koji se može zapisati istim binarnim ciframa kao dati broj.

Napisati program koji testira prethodno napisane funkcije tako što prikazuje binarnu reprezentaciju brojeva koji se dobijaju nakon poziva funkcije \kckod{najveci}, ondosno \kckod{najmanji} za brojeve koji se sa standardnog ulaza zadaju u heksadekasnom formatu. 

\begin{maxitest}
\begin{test}{Test 1}
Ulaz:   0x7F  
Izlaz:  
 Najveci:
 1111 1110 0000 0000 0000 0000 0000 0000  
 Najmanji:
 0000 0000 0000 0000 0000 0000 0111 1110
\end{test}
\end{maxitest}

\begin{maxitest}
\begin{test}{Test 2}
Ulaz:   0x80
Izlaz:  
 Najveci:
 1000 0000 0000 0000 0000 0000 0000 0000  
 Najmanji:
 0000 0000 0000 0000 0000 0000 0000 0001
\end{test}
\end{maxitest}

\begin{maxitest}
\begin{test}{Test 3}
Ulaz:   0x00FF00FF
Izlaz:  
 Najveci:
 1111 1111 1111 1111 0000 0000 0000 0000  
 Najmanji:
 0000 0000 0000 0000 1111 1111 1111 1111
\end{test}
\end{maxitest}

\begin{maxitest}
\begin{test}{Test 4}
Ulaz:   0xFFFFFFFF
Izlaz:  
 Najveci:
 1111 1111 1111 1111 1111 1111 1111 1111   
 Najmanji:
 1111 1111 1111 1111 1111 1111 1111 1111
\end{test}
\end{maxitest}

\begin{maxitest}
\begin{test}{Test 4}
Ulaz:   0xABCDE123
Izlaz:  
 Najveci:
 1111 1111 1111 1111 1000 0000 0000 0000  
 Najmanji:
 0000 0000 0000 0001 1111 1111 1111 1111
\end{test}
\end{maxitest}

\linkresenje{203}
\end{Exercise}
\begin{Answer}[ref=203]
\includecode{resenja/02_Bitovi/203.c}
\end{Answer}


\begin{Exercise}[label=204]
Napisati program za rad sa bitovima.
\begin{enumerate}
\item Napisati funkciju funkciju koja određuje broj koji se dobija kada se $n$ bitova datog broja, počevši od pozicije $p$ postave na \argf{0}.
\item Napisati funkciju koja određuje broj koji se dobija kada se $n$ bitova datog broja, počevši od pozicije $p$ postave na \argf{1}.
\item Napisati funkciju koja određuje broj koji se dobija kada se $n$ bitova datog broja, počevši od pozicije $p$ i vraća ih kao bitove najmanje težine rezultata.
\item Napisati funkciju koja vraća broj koji se dobija upisivanjem poslednjih $n$ bitova broja $y$ u broj $x$, počevši od pozicije $p$.
\item Napisati funkciju koja vraća broj koji se dobija invertovanjem $n$ bitova broja $x$ počevši od pozicije $p$.
\item Napisati program koji testira prethodno napisane funkcije.
\end{enumerate}
Program treba da testira prethodno napisane funkcije nad neoznačenim celim brojem koji se unosi sa standardnog ulaza.
\emph{Napomena: Pozicije se broje počev od pozicije najnižeg bita, pri čemu se broji od nule .}

\begin{maxitest}
\begin{test}{Test 1}
Ulaz:   235 5 10 127  
Izlaz:  
  Broj   235                          = 00000000000000000000000011101011
  reset(  235,    5,   10)            = 00000000000000000000000000101011
  set(  235,    5,   10)              = 00000000000000000000011111101011
  get_bits(  235,    5,   10)         = 00000000000000000000000000000011
  y =                             127 = 00000000000000000000000001111111
  set_n_bits(  235,    5,   10,  127) = 00000000000000000000011111101011
  invert(  235,    5,   10)           = 00000000000000000000011100101011
\end{test}
\end{maxitest}

\linkresenje{204}
\end{Exercise}
\begin{Answer}[ref=204]
\includecode{resenja/02_Bitovi/204.c}
\end{Answer}


\begin{Exercise}[label=205]
Rotiranje ulevo podrazumeva pomeranje svih bitova za jednu poziciju ulevo, s tim što se bit sa pozicije najviše težine pomera na poziciju najmanje težine. Analogno, rotiranje udesno podrazumeva pomeranje svih bitova za jednu poziciju udesno, s tim što se bit sa pozicije najmanje težine pomera na poziciju najviše težine.
\begin{enumerate}
\item Napisati funkciju \kckod{rotate\_left} koja određuje broj koji se dobija rotiranjem \argf{k} puta u levo datog celog broja \argf{x}. 
\item Napisati funkciju \kckod{rotate\_right} koja određuje broj koji se dobija rotiranjem \argf{k} puta u desno datog celog neoznačenog broja \argf{x}. 
\item Napisati funkciju \kckod{rotate\_right\_signed} koja određuje broj koji se dobija rotiranjem \argf{k} puta u desno datog celog broja \argf{x}. 
\end{enumerate}
Napisati program koji testira prethodno napisane funkcije za broj \argf{x} i broj \argf{k} koji se sa standardnog ulaza unose u heksadekasnom formatu.

\begin{maxitest}
\begin{test}{Test 1}
Ulaz:   B10011A7 5   
Izlaz:  
  x                                      = 10110001000000000001000110100111
  rotate_left(2969571751,     5)         = 00100000000000100011010011110110
  rotate_right(2969571751,     5)        = 00111101100010000000000010001101
  rotate_right_signed(2969571751,     5) = 00111101100010000000000010001101
\end{test}
\end{maxitest}

\linkresenje{205}
\end{Exercise}
\begin{Answer}[ref=205]
\includecode{resenja/02_Bitovi/205.c}
\end{Answer}

\begin{Exercise}[label=206]
Napisati funkciju \kckod{mirror} koja određuje ceo broj čiji binarni zapis je slika u ogledalu binarnog zapisa argumenta funkcije. Napisati i program koji testira datu funkciju za brojeve koji se sa standardnog ulaza zadaju u heksadekasnom formatu, tako što najpre ispisuje binarnu reprezentaciju unetog broja, a potom i binarnu reprezentaciju broja dobijenog nakon poziva funkcije \kckod{mirror} za uneti broj.

\begin{maxitest}
\begin{test}{Test 1}
Ulaz:  255   
Izlaz:  
  00000000000000000000001001010101
  10101010010000000000000000000000
\end{test}
\end{maxitest}

\linkresenje{206}
\end{Exercise}
\begin{Answer}[ref=206]
\includecode{resenja/02_Bitovi/206.c}
\end{Answer}

%%%%%%%%%%%%%%%%%%%%%%%%%
% Zadaci sa praktikuma - obavezni zadaci 
%%%%%%%%%%%%%%%%%%%%%%%%%

\begin{Exercise}[label=207]
Napisati funkciju \kckod{int Broj01(unsigned int n)} koja za dati broj \argf{n} vraća \argf{1} ako u njegovom binarnom zapisu ima više jednica nego nula, a inače vraća \argf{0}.  Napisati program koji tu funkciju testira za broj koji se zadaje sa standardnog ulaza.

\begin{minitest}
\begin{test}{Test 1}
Ulaz:   10
Izlaz:  0 
\end{test}
\end{minitest}
\begin{minitest}
\begin{test}{Test 2}
Ulaz:   1024
Izlaz:  0 
\end{test}
\end{minitest}
\begin{minitest}
\begin{test}{Test 3}
Ulaz:   2147377146
Izlaz:  1 
\end{test}
\end{minitest}

\begin{minitest}
\begin{test}{Test 4}
Ulaz:   1111111115
Izlaz:  0 
\end{test}
\end{minitest}

\linkresenje{207}
\end{Exercise}
\begin{Answer}[ref=207]
\includecode{resenja/02_Bitovi/207.c}
\end{Answer}

\begin{Exercise}[label=208]
Napisati funkciju koja broji koliko se puta kombinacija
  \argf{11} (dve uzastopne jedinice) pojavljuje u binarnom zapisu
  celog neoznačenog broja $x$. Tri uzastopne jedinice se broje
  dva puta.  Napisati program koji tu funkciju testira za broj koji se
  zadaje sa standardnog ulaza.
  
\begin{minitest}
\begin{test}{Test 1}
Ulaz:   11  
Izlaz:  1    
\end{test}
\end{minitest}
\begin{minitest}
\begin{test}{Test 2}
Ulaz:   1024
Izlaz:  0 
\end{test}
\end{minitest}
\begin{minitest}
\begin{test}{Test 3}
Ulaz:   2147377146
Izlaz:  22
\end{test}
\end{minitest}

\begin{minitest}
\begin{test}{Test 4}
Ulaz:   1111111115
Izlaz:  6 
\end{test}
\end{minitest}

\linkresenje{208}
\end{Exercise}
\begin{Answer}[ref=208]
\includecode{resenja/02_Bitovi/208.c}
\end{Answer}


%%%
%Ovaj 209.c nema resenje za sad...
%%%
\begin{Exercise}[label=209]\marker+{2}
Napisati program koji sa standardnog ulaza učitava pozitivan
  ceo broj, a na standardni izlaz ispisuje vrednost tog broja sa
  razmenjenim vrednostima bitova na pozicijama $i$,
  $j$. Pozicije $i$, $j$ se učitavaju kao parametri
  komandne linije. Smatrati da je krajnji desni bit binarne
  reprezentacije \argf{0}-ti bit. Pri rešavanju nije dozvoljeno koristiti
  pomoćni niz niti aritmetičke operatore +,-,/,*,\%.

\begin{minitest}
\begin{test}{Test 1}
Poziv:  ./a.out 1 2 
Ulaz:   11             
Izlaz:  13            
\end{test}
\end{minitest}
\begin{minitest}
\begin{test}{Test 2}
Poziv: ./a.out 1 2
Ulaz:   1024     
Izlaz:  1024        
\end{test}
\end{minitest}
\begin{minitest}
\begin{test}{Test 3}
Poziv: ./a.out 12 12
Ulaz:   12345
Izlaz:  12345
\end{test}
\end{minitest}
\end{Exercise}
\begin{Answer}[ref=209]
%\includecode{resenja/02_Bitovi/209.c}
\end{Answer}

\begin{Exercise}[label=210]
  Napisati funkciju koja na osnovu neoznačenog broja $x$
  formira nisku $s$ koja sadrži heksadekadni zapis broja
  $x$, koristeći algoritam za brzo prevođenje binarnog u
  heksadekadni zapis (svake $4$ binarne cifre se zamenjuju jednom
  odgovarajućom heksadekadnom cifrom).  Napisati program koji tu
  funkciju testira za broj koji se zadaje sa standardnog ulaza.

\begin{minitest}
\begin{test}{Test 1}
Ulaz:   11             
Izlaz:  0000000B      
\end{test}
\end{minitest}
\begin{minitest}
\begin{test}{Test 2}
Ulaz:  1024        
Izlaz: 00000400  
\end{test}
\end{minitest}
\begin{minitest}
\begin{test}{Test 3}
Ulaz:  12345
Izlaz: 00003039
\end{test}
\end{minitest}

\linkresenje{210}
\end{Exercise}
\begin{Answer}[ref=210]
\includecode{resenja/02_Bitovi/210.c}
\end{Answer}

%%%%%%%%%%%%%%%%%%%%%%%%%
% Zadaci sa praktikuma - dodatni zadaci - oni nemaju rešenja
% možda \subsection{Dodatni zadaci} ili zadaci za vežbu
%%%%%%%%%%%%%%%%%%%%%%%%%

\begin{Exercise}[label=211]\marker+{2}
  Napisati funkciju koja za dva data neoznačena broja $x$
  i $y$ invertuje u podatku $x$ one bitove koji se poklapaju
  sa odgovarajućim bitovima u broju $y$. Ostali bitovi ostaju
  nepromenjeni.  Napisati program koji tu funkciju testira za brojeve
  koji se zadaju sa standardnog ulaza.
  
\begin{minitest}
\begin{test}{Test 1}
Ulaz:   123 10        
Izlaz:  4294967285    
\end{test}
\end{minitest}
\begin{minitest}
\begin{test}{Test 2}
Ulaz:  3251 0    
Izlaz: 4294967295    
\end{test}
\end{minitest}
\begin{minitest}
\begin{test}{Test 3}
Ulaz:   12541 1024
Izlaz:  4294966271
\end{test}
\end{minitest}
\end{Exercise}
\begin{Answer}[ref=211]
%\includecode{resenja/02_Bitovi/211.c}
\end{Answer}

\begin{Exercise}[label=212]\marker+{2}
Napisati funkciju koja računa koliko petica bi imao ceo
  neoznačen broj $x$ u oktalnom zapisu. Napisati program koji
  tu funkciju testira za broj koji se zadaje sa standardnog ulaza.
  
\begin{minitest}
\begin{test}{Test 1}
Ulaz:   123        
Izlaz:  0             
\end{test}
\end{minitest}
\begin{minitest}
\begin{test}{Test 2}
Ulaz:   3245      
Izlaz:  2              
\end{test}
\end{minitest}
\begin{minitest}
\begin{test}{Test 3}
Ulaz:   100328
Izlaz:  1
\end{test}
\end{minitest}  
\end{Exercise}
\begin{Answer}[ref=212]
%\includecode{resenja/02_Bitovi/212.c}
\end{Answer}
