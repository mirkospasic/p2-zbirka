\chapter{Sortiranje}

\section{Zadaci}

%-----------------------------------------------------------------
%-----------------------------------------------------------------
\begin{Exercise}[label=501]
  U datom nizu brojeva pronaći dva koja su na najmanjem
  rastojanju. Niz se zadaje sa standardnog ulaza, sve do kraja ulaza,
  i neće imati više od 256 elemenata. (uputstvo: prvo sortirati
  niz). Na izlaz ispisati njihovu razliku.
  
\begin{miditest}
\begin{test}{Test 1}
Ulaz:   23 64 123 76 22 7
Izlaz:  1
\end{test}
\end{miditest}

\begin{miditest}
\begin{test}{Test 2}
Ulaz:   21 654 65 123 65 12 61
Izlaz:  0
\end{test}
\end{miditest}
  
\end{Exercise}

\begin{Answer}[ref=501]
  \includecode{resenja/05_Sortiranje/501.c}
\end{Answer}
%-----------------------------------------------------------------
%-----------------------------------------------------------------
\begin{Exercise}[label=502]
  Napisati funkciju koja sortira slova unutar niske
  karaktera. Napisati program koji proverava da li su dve niske
  karaktera anagrami. Dve niske su anagrami ako se sastoje od istog
  broja istih karaktera. Niske se zadaju sa standardnog ulaza, i neće
  biti duže od 128 karaktera.
  
\begin{miditest}
\begin{test}{Test 1}
Ulaz:   anagram ramgana
Izlaz:  jesu
\end{test}
\end{miditest}
\begin{miditest}
\begin{test}{Test 2}
Ulaz:   anagram anagrm
Izlaz:  nisu
\end{test}
\end{miditest}
  
\end{Exercise}

%\begin{Answer}[ref=502]
%  \includecode{resenja/05_Sortiranje/502.c}
%\end{Answer}
%-----------------------------------------------------------------
%-----------------------------------------------------------------
\begin{Exercise}[label=503]
  Napisati program koji pronalazi broj koji se najviše puta
  pojavljivao u datom nizu (uputstvo: prvo sortirati niz a zatim
  naći najdužu sekvencu jednakih elemenata). Niz se zadaje sa
  standardnog ulaza sve do kraja ulaza, i neće biti duži od 256
  elemenata.
  
\begin{miditest}
\begin{test}{Test 1}
Ulaz:   4 23 5 2 4 6 7 34 6 4 5
Izlaz:  4
\end{test}
\end{miditest}
\begin{miditest}
\begin{test}{Test 2}
Ulaz:   2 4 6 2 6 7 99 1
Izlaz:  2
\end{test}
\end{miditest}
  
\end{Exercise}

%\begin{Answer}[ref=503]
%  \includecode{resenja/05_Sortiranje/503.c}
%\end{Answer}
%-----------------------------------------------------------------
%-----------------------------------------------------------------
\begin{Exercise}[label=504]
  Napisati funkciju koja proverava da li u datom nizu postoje dva
  elementa kojima je zbir zadati ceo broj (uputstvo: prvo sortirati
  niz). Napisati i program koji testira ovu funkciju, u kome se prvo
  učitava pomenuti broj, pa zatim niz ne veće dužine od 256
  sve do kraja ulaza.
  
\begin{miditest}
\begin{test}{Test 1}
Ulaz:   34 134 4 1 6 30 23
Izlaz:  da
\end{test}
\end{miditest}
\begin{miditest}
\begin{test}{Test 2}
Ulaz:   12 53 1 43 3 56 13
Izlaz:  ne
\end{test}
\end{miditest}
  
\end{Exercise}

%\begin{Answer}[ref=504]
%  \includecode{resenja/05_Sortiranje/504.c}
%\end{Answer}
%-----------------------------------------------------------------
%-----------------------------------------------------------------
\begin{Exercise}[label=505]
  Napisati funkciju potpisa \kckod{int merge(int *niz1, int dim1, int
    *niz2, int dim2, int *niz3, int dim3)} koja prima dva sortirana
  niza, i na osnovu njih pravi novi sortirani niz koji koji sadrži
  elemente oba niza. Treća dimenzija predstavlja veličinu niza u koji
  se smešta rezultat. Ako je ona manja od potrebne dužine, funkcija
  vraća -1, kao indikator neuspeha, inače vraća 0. Napisati i program
  koji testira funkciju, u kome se nizovi unose sa standardnog ulaza,
  sve dok se ne unese 0.
  
\begin{miditest}
\begin{test}{Test 1}
Ulaz:   3 6 7 11 14 35 0 3 5 8 0
Izlaz:  3 3 5 6 7 8 11 14 35
\end{test}
\end{miditest}

\begin{miditest}
\begin{test}{Test 2}
Ulaz:   1 4 7 0 9 11 23 54 75 0
Izlaz:  1 4 7 9 11 23 54 75
\end{test}
\end{miditest}
  
\end{Exercise}

%\begin{Answer}[ref=505]
%  \includecode{resenja/05_Sortiranje/505.c}
%\end{Answer}
%-----------------------------------------------------------------
%-----------------------------------------------------------------
\begin{Exercise}[label=506]
  Napraviti biblioteku \kckod{sort.h} i \kckod{sort.c} koja
  implementira algoritme sortiranja nizova celih brojeva. Biblioteka
  treba da sadrži bubble, insertion i shell sort. Upotrebiti
  biblioteku kako bi se napravilo poređenje efikasnosti različitih
  algoritama sortiranja. Efikasnost meriti na slučajno generisanim
  nizovima, na već sortiranim nizovima i na naopako sortiranim
  nizovima. Izbor algoritma, veličine i početnog rasporeda elemenata
  niza birati kroz argumente komandne linije.  Vreme meriti programom
  \kckod{time}. Analizirati porast vremena sa porastom dimenzije
  \argf{n}. Porediti vremena za različite implementacije istog
  algoritma (iterativna i rekurzivna).
  
\end{Exercise}

%\begin{Answer}[ref=506]
%  \includecode{resenja/05_Sortiranje/506.c}
%\end{Answer}
%-----------------------------------------------------------------
%-----------------------------------------------------------------
\begin{Exercise}[label=507]
  Napisati funkcije koje sortiraju niz struktura tačaka na
  osnovu sledećih kriterijuma:
\begin{enumerate}
\item njihovog rastojanja od koordinatnog početka,
\item x koordinata tačkaka,
\item y koordinata tačaka.
\end{enumerate}
Napisati program koji učitava niz tačaka iz datoteke čije se ime
zadaje kao argument komandne linije, i u zavisnosti od prisutnih
opcija u komandnoj liniji (\argf{-d}, \argf{-x} ili \argf{-y}),
sortira tačke po jednom od prethodna tri kriterijuma i rezultat
upisuje u datoteku čije se ime zadaje kao drugi argument komandne
linije. U ulaznoj datoteci nije zadato više od 128 tačaka.
  
\begin{miditest}
\begin{test}{Test 1}
Poziv:  a.out -x tacke.txt sorttacke.txt
Ulazna datoteka:   3 4
11 6
7 3
2 82
-1 6
Izlazna datoteka:  -1 6
2 82
3 4
7 3
11 6
\end{test}
\end{miditest}
  
\end{Exercise}

%\begin{Answer}[ref=507]
%  \includecode{resenja/05_Sortiranje/507.c}
%\end{Answer}
%-----------------------------------------------------------------
%-----------------------------------------------------------------
\begin{Exercise}[label=508]
  Definisana je struktura podataka
\begin{ckod}
typedef struct dete
{
      char ime[MAX_IME];
      char prezime[MAX_IME];
      unsigned godiste;
} Dete;
\end{ckod}
Napisati funkciju koja sortira niz dece po godištu, a kada su deca
istog godišta, tada ih sortira leksikografski po prezimenu i
imenu. Napisati program koji učitava podatke o deci koji se nalaze u
datoteci, čije se ime zadaje kao prvi argument komandne linije,
sortira ih i sortirani niz upisuje u datoteku čije se ime zadaje kao
drugi argument komandne linije. Pretpostaviti da u ulaznoj datoteci
nisu zadati podaci o više od 128 dece.
  
\begin{maxitest}
\begin{test}{Test 1}
Poziv: ./a.out ulaz.txt izlaz.txt
Ulazna datoteka:                 Izlazna datoteka:
Petar Petrovic 2007              Marija Antic 2007
Milica Antonic 2008              Ana Petrovic 2007
Ana Petrovic 2007                Petar Petrovic 2007
Ivana Ivanovic 2009              Milica Antonic 2008
Dragana Markovic 2010            Ivana Ivanovic 2009
Marija Antic 2007                Dragana Markovic 2010
\end{test}
\end{maxitest}
  
\end{Exercise}

%\begin{Answer}[ref=508]
%  \includecode{resenja/05_Sortiranje/508.c}
%\end{Answer}
%-----------------------------------------------------------------
%-----------------------------------------------------------------
\begin{Exercise}[label=509]
  Napisati funkciju koja sortira niz niski po broju suglasnika u
  niski, ukoliko reči imaju isti broj suglasnika tada po dužini niske,
  a ukoliko su i dužine jednake onda leksikografski.  Napisati program
  koji testira ovu funkciju za niske koje se zadaju u datoteci
  \kckod{niske.txt}.  Pretpostaviti da u nizu nema više od 128
  elemenata, kao da svaka niska sadrži najviše 32 karaktera.
  
\begin{maxitest}
\begin{test}{Test 1}
Ulazna datoteka:
ana petar andjela milos nikola aleksandar ljubica matej milica
Izlaz:
ana matej milos petar milica nikola andjela ljubica aleksandar
\end{test}
\end{maxitest}
  
\end{Exercise}

%\begin{Answer}[ref=509]
%  \includecode{resenja/05_Sortiranje/509.c}
%\end{Answer}
%-----------------------------------------------------------------
%-----------------------------------------------------------------
\begin{Exercise}[label=510]\marker*2
  Razmatrajmo dve operacije: operacija \kckod{U} je unos novog broja
  \kckod{x}, a operacija \kckod{N} određivanje \kckod{n}-tog po
  veličini od unetih brojeva. Implementirati program koji izvršava ove
  operacije. Može postojati najviše 100000 operacija unosa, a uneti
  elementi se mogu ponavljati, pri čemu se i ponavljanja računaju
  prilikom brojanja. Napomena: brojeve čuvati u sortiranom nizu i
  svaki naredni element umetati na svoje mesto. Optimizovati program,
  ukoliko se zna da neće biti više od 500 različitih unetih brojeva.
  
\begin{maxitest}
\begin{test}{Test 1}
Ulaz: U 2 U 0 U 6 U 4 N 1 U 8 N 2 N 5 U 2 N 3 N 5
Izlaz: 0 2 8 2 6
\end{test}
\end{maxitest}
  
\end{Exercise}

%\begin{Answer}[ref=510]
%  \includecode{resenja/05_Sortiranje/510.c}
%\end{Answer}
%-----------------------------------------------------------------
%-----------------------------------------------------------------
\begin{Exercise}[label=511]\marker*2
  Sa dve susedne stranice pravougaone livade dve grupe krtica
  istovremeno kreću da kopaju tunele (jedna grupa na gore, a druga na
  desno). Krtica prestaje da kopa ukoliko naiđe na već iskopan tunel
  (npr. krticu \argf{A} zaustavlja krtica \argf{C}, krticu \argf{C} i
  \argf{D} zaustavlja krtica \argf{B}, a krticu \argf{B} zaustavlja
  krtica \argf{E}). Za svaku krticu se zna udaljenost od ćoška livade
  i brzina kojom kopa.
\begin{verbatim}
     |EEEEEEEEEE
     |      B
     |DDDDDDB 
     |      B
     |CCCCCCB
     |   A  B
     |___A__B_______________
\end{verbatim}
Napisati program koji određuje koliko dugo svaka krtica kopa. Unose
se broj krtica u obe grupe, a zatim udaljenost i brzina za svaku
krticu. Izlaz je vreme za svaku krticu prikazano na dve decimale (-1
ukoliko se ne zaustavlja) u istom redosledu u kojem su krtice unete.

\begin{miditest}
\begin{test}{Test 1}
Ulaz:          Izlaz:
2 3            1.12
1   4          0.88
2   5.1        -1.00
4.5 3.6        1.00
1   1          -1.00
5   0.5
\end{test}
\end{miditest}
  
\end{Exercise}

%\begin{Answer}[ref=511]
%  \includecode{resenja/05_Sortiranje/511.c}
%\end{Answer}
%-----------------------------------------------------------------
%-----------------------------------------------------------------
\begin{Exercise}[label=512]\marker*2
  Šef u restoranu je neuredan i palačinke koje ispeče ne slaže redom
  po veličini. Konobar pre serviranja mora da sortira palačinke po
  veličini, a jedina operacija koju sme da izvodi je da obrne deo
  palačinki. Na primer:
\begin{verbatim}
    3    5    2    1
    4    4    1__  2
    5__  3    3    3
    1    1    4    4
    2    2__  5    5
\end{verbatim}
Napisati program koji u najviše \kckod{2n-3} okretanja sortira učitani
niz. (Uputstvo: imitirati selection sort i u svakom koraku dovesti
jednu palačinku na svoje mesto korišćenjem najviše dva okretanja.)
    
\end{Exercise}

%\begin{Answer}[ref=512]
%  \includecode{resenja/05_Sortiranje/512.c}
%\end{Answer}
%-----------------------------------------------------------------
%-----------------------------------------------------------------
\begin{Exercise}[label=513]
  Napisati program koji sa standardnog ulaza učitava dva stringa,
  \argf{s} i \argf{t} (dužine manje od 32 karaktera), sortira nizove
  njihovih karaktera (bibliotečkom \kckod{qsort} funkcijom), ispituje
  i štampa da li su \argf{s} i \argf{t} anagrami. (dva stringa su
  anagrami ako su sastavljeni od potpuno istih slova, samo različito
  raspoređenih)
  
\begin{minitest}
\begin{test}{Test 1}
Ulaz:   vrata vatra
Izlaz:  jesu
\end{test}
\end{minitest}
\begin{miditest}
\begin{test}{Test 2}
Ulaz:   qsort bsearch
Izlaz:  nisu
\end{test}
\end{miditest}
  
\end{Exercise}

%\begin{Answer}[ref=513]
%  \includecode{resenja/05_Sortiranje/513.c}
%\end{Answer}
%-----------------------------------------------------------------
%-----------------------------------------------------------------
\begin{Exercise}[label=514]
  Napisati program koji sa standardnog ulaza učitava prvo ceo broj
  \argf{n} (\argf{n} <= 10) a zatim niz \argf{S} od \argf{n} stringova
  (maksimalna dužina stringa je 32 karaktera). Sortirati niz \argf{S}
  (bibliotečkom funkcijom \kckod{qsort}) i proveriti da li u njemu ima
  identičnih stringova.
  
\begin{miditest}
\begin{test}{Test 1}
Ulaz:   4 prog search sort search
Izlaz:  ima
\end{test}
\end{miditest}
\begin{miditest}
\begin{test}{Test 2}
Ulaz:   3 test kol ispit
Izlaz:  nema
\end{test}
\end{miditest}
  
\end{Exercise}

%\begin{Answer}[ref=514]
%  \includecode{resenja/05_Sortiranje/514.c}
%\end{Answer}
%-----------------------------------------------------------------
%-----------------------------------------------------------------
\begin{Exercise}[label=515]
  Datoteka \kckod{studenti.txt} sadrži spisak studenata. Za svakog
  studenta poznat je nalog na Alas-u (oblika npr. \kckod{mr97125},
  \kckod{mm09001}), ime i prezime i broj poena. Napisati program koji
  sortira (korišćenjem funkcije \kckod{qsort}) studente po broju poena
  (ukoliko je prisutna opcija \argf{-p}) ili po nalogu (ukoliko je
  prisutna opcija \argf{-n}). Studenti se po nalogu sortiraju tako što
  se sortiraju na osnovu godine, zatim na osnovu smera, i na kraju na
  osnovu broja indeksa. Ukoliko je u komandnoj liniji uz opciju
  \argf{-n} naveden i nalog nekog studenta, funkcijom \kckod{bsearch}
  potražiti i prijaviti broj poena studenta sa tim nalogom. Sortirane
  studente upisati u datoteku \kckod{izlaz.txt}.
  
\begin{maxitest}
\begin{test}{Test 1}
Poziv: ./a.out -n mm13321
Datoteka:                         Izlaz:
mr14123 Marko Antic 20            mm13321 Marija Radic 12
mm13321 Marija Radic 12
ml13011 Ivana Mitrovic 19
ml13066 Pera Simic 15
mv14003 Jovan Jovanovic 17
\end{test}
\end{maxitest}
  
\end{Exercise}

%\begin{Answer}[ref=515]
%  \includecode{resenja/05_Sortiranje/515.c}
%\end{Answer}
%-----------------------------------------------------------------
%-----------------------------------------------------------------
\begin{Exercise}[label=516]
  Definisana je struktura:
  \begin{ckod}
    typedef struct { int dan; int mesec; int godina; } Datum;}
  \end{ckod}
  Napisati funkciju koja poredi dva datuma i program koji učitava
  datume iz datoteke koja se zadaje kao prvi argument komandne linije
  (ne više od 128 datuma), sortira ih pozivajući funkciju
  \kckod{qsort} iz standardne biblioteke i potom pozivanjem funkcije
  \kckod{bsearch} iz standardne biblioteke proverava da li datumi
  učitani sa standardnog ulaza (sve do kraja ulaza) postoje među
  prethodno unetim datumima.
  
\begin{maxitest}
\begin{test}{Test 1}
Poziv: ./a.out datoteka.txt
Datoteka:            Ulaz:             Izlaz:
1.1.2013             13.12.2016        postoji
13.12.2016           10.5.2015         ne postoji
11.11.2011           5.2.2009          postoji
3.5.2015
5.2.2009
\end{test}
\end{maxitest}
  
\end{Exercise}

%\begin{Answer}[ref=516]
%  \includecode{resenja/05_Sortiranje/516.c}
%\end{Answer}
%-----------------------------------------------------------------
%-----------------------------------------------------------------



\section{Rešenja}
\shipoutAnswer


