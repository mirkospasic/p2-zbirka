\section{Sortiranje}

%-----------------------------------------------------------------
%-----------------------------------------------------------------
\begin{Exercise}[label=501]
  U datom nizu brojeva pronaći dva broja koja su na najmanjem
  rastojanju. Niz se zadaje sa standardnog ulaza, sve do kraja ulaza,
  i neće sadržati više od $256$ i manje od $2$ elemenata. Na izlaz
  ispisati njihovu razliku.  \uputstvo{Prvo sortirati niz.}

\begin{minitest}
\begin{test}{1}
#\naslovUlaz#
#\ulaz{23 64 123 76 22 7}#

#\naslovIzlaz#
#\izlaz{1}#
\end{test}
\end{minitest}
\begin{minitest}
\begin{test}{2}
#\naslovUlaz#
#\ulaz{21 654 65 123 65 12 61}#
  
#\naslovIzlaz#
#\izlaz{0}#
\end{test}
\end{minitest}
\begin{minitest}
\begin{test}{3}
#\naslovUlaz#
#\ulaz{34 30}#

#\naslovIzlaz#
#\izlaz{4}#
\end{test}
\end{minitest}
  
\linkresenje{501}
\end{Exercise}

\begin{Answer}[ref=501]
  \includecode{resenja/05_Sortiranje/501.c}
\end{Answer}
%-----------------------------------------------------------------
%-----------------------------------------------------------------
\begin{Exercise}[label=502]
  Dve niske su anagrami ako se sastoje od istog broja istih
  karaktera. Napisati program koji proverava da li su dve niske
  karaktera anagrami.  Niske se zadaju sa standardnog ulaza i neće
  biti duže od $127$ karaktera.  \uputstvo{Napisati funkciju koja
  sortira slova unutar niske karaktera, a zatim za sortirane niske
  proveriti da li su identične.}
  
\begin{minitest}
\begin{upotreba}{1}
#\naslovInt#
#\izlaz{Unesite prvu nisku}\ulaz{anagram}#
#\izlaz{Unesite drugu nisku}\ulaz{ramgana}#
#\izlaz{jesu}#
\end{upotreba}
\end{minitest}
\begin{minitest}
\begin{upotreba}{2}
#\naslovInt#
#\izlaz{Unesite prvu nisku}\ulaz{anagram}#
#\izlaz{Unesite drugu nisku}\ulaz{anagrm}#
#\izlaz{nisu}#
\end{upotreba}
\end{minitest}
\begin{minitest}
\begin{upotreba}{3}
#\naslovInt#
#\izlaz{Unesite prvu nisku}\ulaz{test}#
#\izlaz{Unesite drugu nisku}\ulaz{tset}#
#\izlaz{jesu}#
\end{upotreba}
\end{minitest}
  
\linkresenje{502}
\end{Exercise}

\begin{Answer}[ref=502]
  \includecode{resenja/05_Sortiranje/502.c}
\end{Answer}
%-----------------------------------------------------------------
%-----------------------------------------------------------------
\begin{Exercise}[label=503]
  Napisati program koji pronalazi broj koji se najviše puta
  pojavljivao u datom nizu. Niz se zadaje sa standardnog ulaza sve do
  kraja ulaza i neće biti duži od $256$ i kraći od jednog
  elemenata. \uputstvo{Prvo sortirati niz, a zatim naći najdužu
  sekvencu jednakih elemenata.}

\begin{minitest}
\begin{test}{1}
#\naslovUlaz#
#\ulaz{4 23 5 2 4 6 7 34 6 4 5}#
  
#\naslovIzlaz#
#\izlaz{4}#
\end{test}
\end{minitest}
\begin{minitest}
\begin{test}{2}
#\naslovUlaz#
#\ulaz{2 4 6 2 6 7 99 1}#
  
#\naslovIzlaz#
#\izlaz{2}#
\end{test}
\end{minitest}
\begin{minitest}
\begin{test}{3}
#\naslovUlaz#
#\ulaz{123}#
  
#\naslovIzlaz#
#\izlaz{123}#
\end{test}
\end{minitest}

\linkresenje{503}
\end{Exercise}

\begin{Answer}[ref=503]
  \includecode{resenja/05_Sortiranje/503.c}
\end{Answer}
%-----------------------------------------------------------------
%-----------------------------------------------------------------
\begin{Exercise}[label=504]
  Napisati funkciju koja proverava da li u datom nizu postoje dva
  elementa kojima je zbir zadati ceo broj. Napisati i program koji
  testira ovu funkciju. U programu se prvo učitava broj, a zatim i niz
  (pretpostaviti da za niz neće biti uneto više od $256$ brojeva).
  Elementi niza se unose sve do kraja ulaza. \uputstvo{Prvo sortirati
  niz.}
  
\begin{minitest}
\begin{upotreba}{1}
#\naslovInt#
#\izlaz{Unesite trazeni zbir:}\ulaz{34}#
#\izlaz{Unesite elemente niza:}#
#\ulaz{134 4 1 6 30 23}#
#\izlaz{da}#
\end{upotreba}
\end{minitest}
\begin{minitest}
\begin{upotreba}{2}
#\naslovInt#
#\izlaz{Unesite trazeni zbir:}\ulaz{12}#
#\izlaz{Unesite elemente niza:}#
#\ulaz{53 1 43 3 56 13}#
#\izlaz{ne}#
\end{upotreba}
\end{minitest}
\begin{minitest}
\begin{upotreba}{3}
#\naslovInt#
#\izlaz{Unesite trazeni zbir:}\ulaz{52}#
#\izlaz{Unesite elemente niza:}#
#\ulaz{52}#
#\izlaz{ne}#
\end{upotreba}
\end{minitest}
  
\linkresenje{504}
\end{Exercise}

\begin{Answer}[ref=504]
  \includecode{resenja/05_Sortiranje/504.c}
\end{Answer}
%-----------------------------------------------------------------
%-----------------------------------------------------------------
\begin{Exercise}[label=507]
  Napraviti biblioteku \kckod{sort.h} i \kckod{sort.c} koja
  implementira algoritme sortiranja nizova celih brojeva. Biblioteka
  treba da sadrži \kckod{selection}, \kckod{merge}, \kckod{quick},
  \kckod{bubble}, \kckod{insertion} i \kckod{shell sort}. Upotrebiti
  biblioteku kako bi se napravilo poređenje efikasnosti različitih
  algoritama sortiranja. Efikasnost meriti na slučajno generisanim
  nizovima, na rastuće sortiranim nizovima i na opadajuće sortiranim
  nizovima. Izbor algoritma, veličine i početnog rasporeda elemenata
  niza birati kroz argumente komandne linije.  Vreme meriti programom
  \kckod{time}. Analizirati porast vremena sa porastom dimenzije
  \argf{n}.

\begin{miditest}
\begin{test}{1}
#\poziv{time a.out 100000 -i -o}#
  
#\naslovIzlaz#
#\izlaz{real    0m17.631s}#
#\izlaz{user    0m17.604s}#
#\izlaz{sys     0m0.000s}#
\end{test}
\end{miditest}
\begin{miditest}
\begin{test}{2}
#\poziv{time a.out 100000 -b -r}#
  
#\naslovIzlaz#
#\izlaz{real    0m0.005s}#
#\izlaz{user    0m0.004s}#
#\izlaz{sys     0m0.000s}#
\end{test}
\end{miditest}

\linkresenje{507}
\end{Exercise}

\begin{Answer}[ref=507]
  \includecode{resenja/05_Sortiranje/sort.h}
  \includecode{resenja/05_Sortiranje/sort.c}
  \includecode{resenja/05_Sortiranje/507.c}
\end{Answer}
%-----------------------------------------------------------------
%-----------------------------------------------------------------
\begin{Exercise}[label=505]
  Napisati funkciju potpisa \kckod{int merge(int *niz1, int dim1, int
    *niz2, int dim2, int *niz3, int dim3)} koja prima dva sortirana
  niza, i na osnovu njih pravi novi sortirani niz koji koji sadrži
  elemente oba niza. Treća dimenzija predstavlja veličinu niza u koji
  se smešta rezultat. Ako je ona manja od potrebne dužine, funkcija
  vraća -1 kao indikator neuspeha, inače vraća 0. Napisati zatim
  program koji testira ovu funkciju. Nizovi se unose sa standardnog
  ulaza sve dok se ne unese 0 i može se pretpostaviti da će njihove
  dimenzije biti manje od $256$.
  
\begin{miditest}
\begin{upotreba}{1}
#\naslovInt#  
#\izlaz{Unesite elemente prvog niza:}#
#\ulaz{3 6 7 11 14 35 0}#
#\izlaz{Unesite elemente drugog niza:}#
#\ulaz{3 5 8 0}#
#\izlaz{3 3 5 6 7 8 11 14 35}#
\end{upotreba}
\end{miditest}
\begin{miditest}
\begin{upotreba}{2}
#\naslovInt#   
#\izlaz{Unesite elemente prvog niza:}#
#\ulaz{1 4 7 0}#
#\izlaz{Unesite elemente drugog niza:}#
#\ulaz{9 11 23 54 75 0}#
#\izlaz{1 4 7 9 11 23 54 75}#
\end{upotreba}
\end{miditest}
  
\linkresenje{505}
\end{Exercise}

\begin{Answer}[ref=505]
  \includecode{resenja/05_Sortiranje/505.c}
\end{Answer}
%-----------------------------------------------------------------
%-----------------------------------------------------------------
\begin{Exercise}[label=506]
  Napisati program koji čita sadržaj dveju datoteka od kojih svaka
  sadrži spisak imena i prezimena studenata iz jedne od dve grupe,
  rastuće sortiran po imenima i kreira jedinstven spisak studenata
  sortiranih takođe po imenu rastuće.  Program dobija nazive datoteka
  iz komandne linije i jedinstveni spisak upisuje u datoteku
  \kckod{ceo-tok.txt}. Pretpostaviti da je ime studenta nije duže od
  $10$, a prezime od $15$ karaktera.


\begin{miditest}
\begin{test}{1}
#\poziv{./a.out prvi-deo.txt drugi-deo.txt}#
  
#\naslovDat{prvi-deo.txt}#
#\datoteka{Andrija Petrovic}#
#\datoteka{Anja Ilic}#
#\datoteka{Ivana Markovic}#
#\datoteka{Lazar Micic}#
#\datoteka{Nenad Brankovic}#
#\datoteka{Sofija Filipovic}#
#\datoteka{Vladimir Savic}#
#\datoteka{Uros Milic}#

#\naslovDat{drugi-deo.txt}#
#\datoteka{Aleksandra Cvetic}#
#\datoteka{Bojan Golubovic}#
#\datoteka{Dragan Markovic}#
#\datoteka{Filip Dukic}#
#\datoteka{Ivana Stankovic}#
#\datoteka{Marija Stankovic}#
#\datoteka{Ognjen Peric}#
\end{test}
\end{miditest}
\begin{miditest}
\begin{test}{1}
  

#\naslovDat{ceo-tok.txt}#
#\datoteka{Aleksandra Cvetic}#
#\datoteka{Andrija Petrovic}#
#\datoteka{Anja Ilic}#
#\datoteka{Bojan Golubovic}#
#\datoteka{Dragan Markovic}#
#\datoteka{Filip Dukic}#
#\datoteka{Ivana Stankovic}#
#\datoteka{Ivana Markovic}#
#\datoteka{Lazar Micic}#
#\datoteka{Marija Stankovic}#
#\datoteka{Nenad Brankovic}#
#\datoteka{Ognjen Peric}#
#\datoteka{Sofija Filipovic}#
#\datoteka{Uros Milic}#
#\datoteka{Vladimir Savic}#
\end{test}
\end{miditest}
  
\linkresenje{506}
\end{Exercise}

\begin{Answer}[ref=506]
  \includecode{resenja/05_Sortiranje/506.c}
\end{Answer}
%-----------------------------------------------------------------
%-----------------------------------------------------------------
\begin{Exercise}[label=508]
  Napisati funkcije koje sortiraju niz struktura tačaka na
  osnovu sledećih kriterijuma:
  \begin{enumerate}
  \item njihovog rastojanja od koordinatnog početka,
  \item \kckod{x} koordinata tačaka,
  \item \kckod{y} koordinata tačaka.
  \end{enumerate}
  Napisati program koji učitava niz tačaka iz datoteke čije se ime
  zadaje kao drugi argument komandne linije, i u zavisnosti od
  prisutnih opcija (prvi argument) u komandnoj liniji (\argf{-o},
  \argf{-x} ili \argf{-y}) sortira tačke po jednom od prethodna tri
  kriterijuma i rezultat upisuje u datoteku čije se ime zadaje kao
  treći argument komandne linije. U ulaznoj datoteci nije zadato više
  od 128 tačaka.
  
\begin{miditest}
\begin{test}{1}
#\poziv{./a.out -x in.txt out.txt}#
  
#\naslovDat{in.txt}#
#\datoteka{3 4}#
#\datoteka{11 6}#
#\datoteka{7 3}#
#\datoteka{2 82}#
#\datoteka{-1 6}#
  
#\naslovDat{out.txt}#
#\datoteka{-1 6}#
#\datoteka{2 82}#
#\datoteka{3 4}#
#\datoteka{7 3}#
#\datoteka{11 6}#
\end{test}
\end{miditest}
\begin{miditest}
\begin{test}{2}
#\poziv{./a.out -o in.txt out.txt}#
  
#\naslovDat{in.txt}#
#\datoteka{3 4}#
#\datoteka{11 6}#
#\datoteka{7 3}#
#\datoteka{2 82}#
#\datoteka{-1 6}#

#\naslovDat{out.txt}#
#\datoteka{3 4}#
#\datoteka{-1 6}#
#\datoteka{7 3}#
#\datoteka{11 6}#
#\datoteka{2 82}#
\end{test}
\end{miditest}
  
\linkresenje{508}
\end{Exercise}

\begin{Answer}[ref=508]
  \includecode{resenja/05_Sortiranje/508.c}
\end{Answer}
%-----------------------------------------------------------------
%-----------------------------------------------------------------
\begin{Exercise}[label=509]
  Napisati program koji učitava imena i prezimena građana (najviše
  njih $1000$) iz datoteke \kckod{biracki-spisak.txt} i kreira
  biračke spiskove. Jedan birački spisak je sortiran po imenu građana,
  a drugi po prezimenu. Program treba da ispisuje koliko građana ima
  isti redni broj u oba biračka spiska. Pretpostaviti da je za ime,
  odnosno prezime građana dovoljno $15$ karaktera.

\begin{minitest}
\begin{test}{1}
#\naslovDat{biracki-spisak.txt}#    
#\datoteka{Bojan Golubovic}#
#\datoteka{Andrija Petrovic}#
#\datoteka{Anja Ilic}#
#\datoteka{Aleksandra Cvetic}#
#\datoteka{Dragan Markovic}#
#\datoteka{Ivana Markovic}#
#\datoteka{Lazar Micic}#
#\datoteka{Marija Stankovic}#
#\datoteka{Filip Dukic}#

#\naslovIzlaz#
#\izlaz{3}#
\end{test}
\end{minitest}
\begin{minitest}
\begin{test}{2}
#\naslovDat{biracki-spisak.txt}#
#\datoteka{Milan Milicevic}#
  
#\naslovIzlaz#
#\izlaz{1}#
\end{test}
\end{minitest}
\begin{minitest}
\begin{test}{3}
#\naslovDat{biracki-spisak.txt}#
#\datoteka{[datoteka ne postoji]}#
  
#\naslovIzlaz#
#\izlaz{Problem pri otvaranju}#
#\izlaz{datoteke.}#
\end{test}
\end{minitest}
  
\linkresenje{509}
\end{Exercise}

\begin{Answer}[ref=509]
  \includecode{resenja/05_Sortiranje/509.c}
\end{Answer}
%-----------------------------------------------------------------
%-----------------------------------------------------------------
\begin{Exercise}[label=510]
  Definisana je struktura podataka
\begin{ckod}
typedef struct dete
{
      char ime[MAX_IME];
      char prezime[MAX_IME];
      unsigned godiste;
} Dete;
\end{ckod}
Napisati funkciju koja sortira niz dece po godištu, a kada su deca
istog godišta, tada ih sortira leksikografski po prezimenu i
imenu. Napisati program koji učitava podatke o deci koji se nalaze u
datoteci čije se ime zadaje kao prvi argument komandne linije,
sortira ih i sortirani niz upisuje u datoteku čije se ime zadaje kao
drugi argument komandne linije. Pretpostaviti da u ulaznoj datoteci
nisu zadati podaci o više od $128$ deteta.
  
\begin{miditest}
\begin{test}{1}
#\poziv{./a.out in.txt out.txt}#
  
#\naslovDat{in.out}#
#\datoteka{Petar Petrovic 2007}#
#\datoteka{Milica Antonic 2008}#
#\datoteka{Ana Petrovic 2007}#
#\datoteka{Ivana Ivanovic 2009}#
#\datoteka{Dragana Markovic 2010}#
#\datoteka{Marija Antic 2007}#
  
#\naslovDat{out.txt}#
#\datoteka{Marija Antic 2007}#
#\datoteka{Ana Petrovic 2007}#
#\datoteka{Petar Petrovic 2007}#
#\datoteka{Milica Antonic 2008}#
#\datoteka{Ivana Ivanovic 2009}#
#\datoteka{Dragana Markovic 2010}#
\end{test}
\end{miditest}
\begin{miditest}
\begin{test}{2}
#\poziv{./a.out in.txt out.txt}#

#\naslovDat{in.out}#
#\datoteka{Milijana Maric 2009}#

#\naslovDat{out.txt}#
#\datoteka{Milijana Maric 2009}#
\end{test}
\end{miditest}
  
%\linkresenje{510}
\end{Exercise}

%\begin{Answer}[ref=510]
%  \includecode{resenja/05_Sortiranje/510.c}
%\end{Answer}
%-----------------------------------------------------------------
%-----------------------------------------------------------------
\begin{Exercise}[label=511]
  Napisati funkciju koja sortira niz niski po broju suglasnika u
  niski. Ukoliko reči imaju isti broj suglasnika tada sortirati ih po
  dužini niske, a ukoliko su i dužine jednake onda
  leksikografski. Napisati program koji testira ovu funkciju za niske
  koje se zadaju u datoteci \kckod{niske.txt}.  Pretpostaviti da u
  nizu nema više od $128$ elemenata, kao i da svaka niska sadrži
  najviše $31$ karakter.
  
\begin{maxitest}
\begin{test}{1}
#\naslovDat{niske.txt}#
#\datoteka{ana petar andjela milos nikola aleksandar ljubica matej milica}#
  
#\naslovIzlaz#
#\izlaz{ana matej milos petar milica nikola andjela ljubica aleksandar}#
\end{test}
\end{maxitest}
  
\linkresenje{511}
\end{Exercise}

\begin{Answer}[ref=511]
  \includecode{resenja/05_Sortiranje/511.c}
\end{Answer}
%-----------------------------------------------------------------
%-----------------------------------------------------------------
\begin{Exercise}[label=512]
  Napisati program koji simulira rad kase u prodavnici. Kupci prilaze
  kasi, a prodavac unošenjem bar-koda kupljenog proizvoda dodaje
  njegovu cenu na ukupan račun. Na kraju, program ispisuje ukupnu
  vrednost svih proizvoda. Sve artikle, zajedno sa bar-kodovima,
  prozivođačima i cenama učitati iz datoteke
  \kckod{artikli.txt}. Pretraživanje niza artikala vršiti binarnom
  pretragom.
  
\begin{maxitest}
\begin{upotreba}{1}
#\naslovDat{artikli.txt}#
#\datoteka{1001 Keks Jaffa 120}#
#\datoteka{2530 Napolitanke Bambi 230}#
#\datoteka{0023 MedenoSrce Pionir 150}#
#\datoteka{2145 Pardon Marbo 70}#

#\naslovInt#
#\izlaz{Asortiman:}#
#\izlaz{KOD                Naziv artikla     Ime proizvodjaca       Cena}#
#\izlaz{        23           MedenoSrce               Pionir       150.00}#
#\izlaz{      1001                 Keks                Jaffa       120.00}#
#\izlaz{      2145               Pardon                Marbo        70.00}#
#\izlaz{      2530          Napolitanke                Bambi       230.00}#
#\izlaz{---------------------------}#
#\izlaz{- Za kraj za kraj rada kase, pritisnite CTRL+D!}#
#\izlaz{- Za nov racun unesite kod artikla!}#
#\izlaz{}#
#\ulaz{1001}#
#\izlaz{	Trazili ste:	Keks Jaffa       120.00}#
#\izlaz{Unesite kod artikla [ili 0 za prekid]:}\ulaz{23}#
#\izlaz{	Trazili ste:	MedenoSrce Pionir       150.00}#
#\izlaz{Unesite kod artikla [ili 0 za prekid]:}\ulaz{0}#
#\izlaz{}#
#\izlaz{	UKUPNO: 270.00 dinara.}#
#\izlaz{}#
#\izlaz{---------------------------}#
#\izlaz{- Za kraj za kraj rada kase, pritisnite CTRL+D!}#
#\izlaz{- Za nov racun unesite kod artikla!}#
#\izlaz{}#
#\ulaz{232}#
#\izlaz{	GRESKA: Ne postoji proizvod sa trazenim kodom!}#
#\izlaz{Unesite kod artikla [ili 0 za prekid]:}\ulaz{2530}#
#\izlaz{	Trazili ste:	Napolitanke Bambi       230.00}#
#\izlaz{Unesite kod artikla [ili 0 za prekid]:}\ulaz{0}#
#\izlaz{}#
#\izlaz{	UKUPNO: 230.00 dinara.}#
#\izlaz{}#
#\izlaz{---------------------------}#
#\izlaz{- Za kraj za kraj rada kase, pritisnite CTRL+D!}#
#\izlaz{- Za nov racun unesite kod artikla!}#
#\izlaz{}#
#\izlaz{Kraj rada kase!}#
\end{upotreba}
\end{maxitest}
  
\linkresenje{512}
\end{Exercise}

\begin{Answer}[ref=512]
  \includecode{resenja/05_Sortiranje/512.c}
\end{Answer}
%-----------------------------------------------------------------
%-----------------------------------------------------------------
\begin{Exercise}[label=513]
   Napisati program koji iz datoteke \kckod{aktivnost.txt} čita
   podatke o aktivnostima studenata na praktikumima i u datoteke
   \kckod{dat1.txt}, \kckod{dat2.txt} i \kckod{dat3.txt} upisuje redom
   tri spiska. Na prvom su studenti sortirani leksikografski po imenu
   rastuće. Na drugom su sortirani po ukupnom broju urađenih zadataka
   opadajuće, a ukoliko neki studenti imaju isti broj rešenih zadataka
   sortiraju se po dužini imena rastuće. Na trećem spisku kriterijum
   sortiranja je broj časova na kojima su bili opadajuće. Ukoliko neki
   studenti imaju isti broj časova, sortirati ih opadajuće po broju
   urađenih zadataka, a ukoliko se i on poklapa sortirati po prezimenu
   opadajuće. U datoteci se nalazi ime, prezime studenta, broj časova
   na kojima je prisustvovao, kao i ukupan broj urađenih
   zadataka. Pretpostaviti da studenata neće biti više od $500$ i da
   je za ime studenta dovoljno $20$, a za prezime $25$ karaktera.
  
\begin{miditest}
\begin{test}{1}
#\naslovDat{aktivnosti.txt}#
#\datoteka{Aleksandra Cvetic 4 6}#
#\datoteka{Bojan Golubovic 4 3}#
#\datoteka{Dragan Markovic 3 5}#
#\datoteka{Ivana Stankovic 3 1}#
#\datoteka{Marija Stankovic 1 3}#
#\datoteka{Ognjen Peric 1 2}#
#\datoteka{Uros Milic 2 5}#
#\datoteka{Andrija Petrovic 2 5}#
#\datoteka{Anja Ilic 3 1}#
#\datoteka{Lazar Micic 1 3}#
#\datoteka{Nenad Brankovic 2 4}#

#\naslovDat{dat1.txt}#
#\datoteka{Studenti sortirani po imenu}#
#\datoteka{leksikografski rastuce:}#
#\datoteka{Aleksandra Cvetic  4  6}#
#\datoteka{Andrija Petrovic  2  5}#
#\datoteka{Anja Ilic  3  1}#
#\datoteka{Bojan Golubovic  4  3}#
#\datoteka{Dragan Markovic  3  5}#
#\datoteka{Ivana Stankovic  3  1}#
#\datoteka{Lazar Micic  1  3}#
#\datoteka{Marija Stankovic  1  3}#
#\datoteka{Nenad Brankovic  2  4}#
#\datoteka{Ognjen Peric  1  2}#
#\datoteka{Uros Milic  2  5}#
\end{test}
\end{miditest}
\begin{miditest}
\begin{test}{1}
#\naslovDat{dat2.txt}#
#\datoteka{Studenti sortirani po broju zadataka}#
#\datoteka{opadajuce, pa po duzini imena rastuce:}#
#\datoteka{Aleksandra Cvetic  4  6}#
#\datoteka{Uros Milic  2  5}#
#\datoteka{Dragan Markovic  3  5}#
#\datoteka{Andrija Petrovic  2  5}#
#\datoteka{Nenad Brankovic  2  4}#
#\datoteka{Lazar Micic  1  3}#
#\datoteka{Bojan Golubovic  4  3}#
#\datoteka{Marija Stankovic  1  3}#
#\datoteka{Ognjen Peric  1  2}#
#\datoteka{Anja Ilic  3  1}#
#\datoteka{Ivana Stankovic  3  1}#
                                                
#\naslovDat{dat3.txt}#
#\datoteka{Studenti sortirani po prisustvu}#
#\datoteka{opadajuce, pa po broju zadataka,}#
#\datoteka{pa po prezimenima leksikografski}#
#\datoteka{opadajuce:}#
#\datoteka{Aleksandra Cvetic  4  6}#
#\datoteka{Bojan Golubovic  4  3}#
#\datoteka{Dragan Markovic  3  5}#
#\datoteka{Ivana Stankovic  3  1}#
#\datoteka{Anja Ilic  3  1}#
#\datoteka{Andrija Petrovic  2  5}#
#\datoteka{Uros Milic  2  5}#
#\datoteka{Nenad Brankovic  2  4}#
#\datoteka{Marija Stankovic  1  3}#
#\datoteka{Lazar Micic  1  3}#
#\datoteka{Ognjen Peric  1  2}#
\end{test}
\end{miditest}
  
\linkresenje{513}
\end{Exercise}

\begin{Answer}[ref=513]
  \includecode{resenja/05_Sortiranje/513.c}
\end{Answer}
%-----------------------------------------------------------------
%-----------------------------------------------------------------
\begin{Exercise}[label=514]
U datoteci ,,pesme.txt`` nalaze se informacije o gledanosti pesama na
Youtube-u. Format datoteke sa informacijama je sledeći:
\begin{itemize}
\item U prvoj liniji datoteke se nalazi ukupan broj pesama prisutnih u
  datoteci.
\item Svaki naredni red datoteke sadrži informacije o gledanosti
  pesama u formatu \kckod{izvođač - naslov, broj gledanja}.
\end{itemize}
Napisati program koji učitava informacije o pesmama i vrši sortiranje
pesama u zavisnosti od argumenata komandne linije na sledeći način:
\begin{itemize}
\item nema opcija, sortiranje se vrši po broju gledanja;
\item prisutna je opcija \kckod{-i}, sortiranje se vrši po imenima
  izvođača;
\item prisutna je opcija \kckod{-n}, sortiranje se vrši po naslovu
  pesama.
\end{itemize}
Na standardnom izlazu ispisati informacije o pesmama sortiranim na opisani
način. Uraditi zadatak bez pravljenja pretpostavki o maksimalnoj dužini
  imena izvođača i naslova pesme.

\begin{minitest}
\begin{test}{1}
#\poziv{./a.out}#
  
#\naslovDat{pesme.txt}#
#\datoteka{5}#
#\datoteka{Ana - Nebo, 2342}#
#\datoteka{Laza - Oblaci, 29}#
#\datoteka{Pera - Ptice, 327}#
#\datoteka{Jelena - Sunce, 92321}#
#\datoteka{Mika - Kisa, 5341}#

#\naslovIzlaz#
#\izlaz{Jelena - Sunce, 92321}#
#\izlaz{Mika - Kisa, 5341}#
#\izlaz{Ana - Nebo, 2342}#
#\izlaz{Pera - Ptice, 327}#
#\izlaz{Laza - Oblaci, 29}#
\end{test}
\end{minitest}
\begin{minitest}
\begin{test}{2}
#\poziv{./a.out -i}#
  
#\naslovDat{pesme.txt}#
#\datoteka{5}#
#\datoteka{Ana - Nebo, 2342}#
#\datoteka{Laza - Oblaci, 29}#
#\datoteka{Pera - Ptice, 327}#
#\datoteka{Jelena - Sunce, 92321}#
#\datoteka{Mika - Kisa, 5341}#

#\naslovIzlaz#
#\izlaz{Ana - Nebo, 2342}#
#\izlaz{Jelena - Sunce, 92321}#
#\izlaz{Laza - Oblaci, 29}#
#\izlaz{Mika - Kisa, 5341}#
#\izlaz{Pera - Ptice, 327}#
\end{test}
\end{minitest}
\begin{minitest}
\begin{test}{3}
#\poziv{./a.out -n}#
  
#\naslovDat{pesme.txt}#
#\datoteka{5}#
#\datoteka{Ana - Nebo, 2342}#
#\datoteka{Laza - Oblaci, 29}#
#\datoteka{Pera - Ptice, 327}#
#\datoteka{Jelena - Sunce, 92321}#
#\datoteka{Mika - Kisa, 5341}#

#\naslovIzlaz#
#\izlaz{Mika - Kisa, 5341}#
#\izlaz{Ana - Nebo, 2342}#
#\izlaz{Laza - Oblaci, 29}#
#\izlaz{Pera - Ptice, 327}#
#\izlaz{Jelena - Sunce, 92321}#
\end{test}
\end{minitest}

\linkresenje{514}
\end{Exercise}

\begin{Answer}[ref=514]
\includecode{resenja/05_Sortiranje/514.c}
\end{Answer}
%-----------------------------------------------------------------
%-----------------------------------------------------------------
\begin{Exercise}[difficulty=2, label=515]
  Razmatrajmo dve operacije: operacija \kckod{U} je unos novog broja
  \kckod{x}, a operacija \kckod{N} određivanje \kckod{n}-tog po
  veličini od unetih brojeva. Implementirati program koji izvršava ove
  operacije. Može postojati najviše $100000$ operacija unosa, a uneti
  elementi se mogu ponavljati, pri čemu se i ponavljanja računaju
  prilikom brojanja. \napomena{Brojeve čuvati u sortiranom nizu i
  svaki naredni element umetati na svoje mesto.} Optimizovati program,
  ukoliko se zna da neće biti više od $500$ različitih unetih brojeva.
  
\begin{maxitest}
\begin{upotreba}{1}
#\naslovInt#
#\izlaz{Unesi niz operacija:}\ulaz{U 2 U 0 U 6 U 4 N 1 U 8 N 2 N 5 U 2 N 3 N 5}#
#\izlaz{0 2 8 2 6}
\end{upotreba}
\end{maxitest}
  
%\linkresenje{515}
\end{Exercise}

%\begin{Answer}[ref=515]
%  \includecode{resenja/05_Sortiranje/515.c}
%\end{Answer}
%-----------------------------------------------------------------
%-----------------------------------------------------------------
\begin{Exercise}[difficulty=2, label=516]
  Šef u restoranu je neuredan i palačinke koje ispeče ne slaže redom
  po veličini. Konobar pre serviranja mora da sortira palačinke po
  veličini, a jedina operacija koju sme da izvodi je da obrne deo
  palačinki. Na primer, sledeća slika po kolonama predstavlja
  naslagane palačinke posle svakog okretanja. Na početku, palačinka
  veličine $2$ je na dnu, iznad nje se redom nalaze najmanja, najveća,
  itd... Na slici crtica predstavlja mesto iznad koga će konobar
  okrenuti palačinke. Prvi potez konobara je okretanje palačinki
  veličine $5$, $4$ i $3$ (prva kolona), i tada će veličine palačinki
  odozdo nagore biti $2$, $1$, $3$, $4$, $5$ (druga kolona). Posle još
  dva okretanja, palačinke će biti složene.
\begin{ckod}
    3    5    2    1
    4    4    1__  2
    5__  3    3    3
    1    1    4    4
    2    2__  5    5
\end{ckod}
Napisati program koji u najviše \kckod{2n-3} okretanja sortira učitani
niz. \uputstvo{Imitirati \kckod{selection sort} i u svakom koraku
dovesti jednu palačinku na svoje mesto korišćenjem najviše dva
okretanja.}

\begin{maxitest}
\begin{test}{1}
#\naslovUlaz#
#\ulaz{23 64 123 76 22 7 34 123 54562 12 453 342 5342 42 542 1 3 432 1 32 43}#

#\naslovIzlaz#
#\izlaz{1 1 3 7 12 22 23 32 34 42 43 64 76 123 123 342 432 453 542 5342 54562}#
\end{test}
\end{maxitest}

%\linkresenje{516}
\end{Exercise}

%\begin{Answer}[ref=516]
%  \includecode{resenja/05_Sortiranje/516.c}
%\end{Answer}
%-----------------------------------------------------------------
%-----------------------------------------------------------------
\section{Bibliotečke funkcije pretrage i sortiranja}
\begin{Exercise}[label=517]
  Napisati program koji ilustruje upotrebu bibiliotečkih funkcija za
  pretraživanje i sortiranje nizova i mogućnost zadavanja različitih
  kriterijuma sortiranja. Sa standardnog ulaza se unosi dimenzija niza
  celih brojeva (ne veća od $100$), a potom i sami elementi
  niza. Upotrebom funkcije \kckod{qsort} sortirati niz u rastućem
  poretku, sa standardnog ulaza učitati broj koji se traži u nizu, pa
  zatim funkcijama \kckod{bsearch} i \kckod{lfind} utvrditi da li se
  zadati broj nalazi u nizu. Na standardnom izlazu ispisati
  odgovarajuću poruku.
  
\begin{miditest}
\begin{upotreba}{1}
#\naslovInt#
#\izlaz{Uneti dimenziju niza:}\ulaz{10}#
#\izlaz{Uneti elemente niza:}#
#\ulaz{5 3 1 6 8 90 34 5 3 432}#
#\izlaz{Sortirani niz u rastucem poretku:}#
#\izlaz{1 3 3 5 5 6 8 34 90 432 }#
#\izlaz{Uneti element koji se trazi u nizu:}\ulaz{34}#
#\izlaz{Binarna pretraga: }#
#\izlaz{Element je nadjen na poziciji 7}#
#\izlaz{Linearna pretraga (lfind): }#
#\izlaz{Element je nadjen na poziciji 7}#
\end{upotreba}
\end{miditest}
\begin{miditest}
\begin{upotreba}{2}
#\naslovInt#
#\izlaz{Uneti dimenziju niza:}\ulaz{4}#
#\izlaz{Uneti elemente niza:}#
#\ulaz{4 2 5 7}#
#\izlaz{Sortirani niz u rastucem poretku:}#
#\izlaz{2 4 5 7 }#
#\izlaz{Uneti element koji se trazi u nizu:}\ulaz{3}#
#\izlaz{Binarna pretraga: }#
#\izlaz{Elementa nema u nizu!}#
#\izlaz{Linearna pretraga (lfind): }#
#\izlaz{Elementa nema u nizu!}#
\end{upotreba}
\end{miditest}
  
\linkresenje{517}
\end{Exercise}

\begin{Answer}[ref=517]
  \includecode{resenja/05_Sortiranje/517.c}
\end{Answer}
%-----------------------------------------------------------------
%-----------------------------------------------------------------
\begin{Exercise}[label=518]
  Napisati program koji sa standardnog ulaza učitava dimenziju niza
  celih brojeva (ne veću od 100), a potom i same elemente
  niza. Upotrebom funkcije \kckod{qsort} sortirati niz u rastućem
  poretku prema broju delilaca i tako dobijeni niz odštampati na
  standardnom izlazu.
  
\begin{maxitest}
\begin{upotreba}{1}
#\naslovInt#
#\izlaz{Uneti dimenziju niza:}\ulaz{10}#
#\izlaz{Uneti elemente niza:}#
#\ulaz{1 2 3 4 5 6 7 8 9 10}#
#\izlaz{Sortirani niz u rastucem poretku prema broju delilaca: 1 2 3 5 7 4 9 6 8 10}#
\end{upotreba}
\end{maxitest}
  
\linkresenje{518}
\end{Exercise}

\begin{Answer}[ref=518]
  \includecode{resenja/05_Sortiranje/518.c}
\end{Answer}
%-----------------------------------------------------------------
%-----------------------------------------------------------------
\begin{Exercise}[label=519]
   Korišćenjem bibiliotečke funkcije \kckod{qsort} napisati program
   koji sortira niz niski po sledećim kriterijumima:
   \begin{enumerate}
   \item leksikografski,
   \item po dužini.
   \end{enumerate}
   Niske se učitavaju iz fajla \kckod{niske.txt}, neće ih biti više od
   $1000$ i svaka će biti dužine najviše $30$ karaktera. Program prvo
   leksikografski sortira niz, primenjuje binarnu pretragu
   (\kckod{bsearch}) zarad traženja niske unete sa standardnog ulaza,
   a potom linearnu pretragu koristeći funkciju \kckod{lfind}. Na
   kraju, niske bivaju sortirane po dužini. Rezultate svih sortiranja
   i pretraga ispisati na standardnom izlazu.
   
\begin{maxitest}
\begin{upotreba}{1}
#\naslovDat{niske.txt}#
#\datoteka{ana petar andjela milos nikola aleksandar ljubica matej milica}#
  
#\naslovInt#
#\izlaz{Leksikografski sortirane niske:}#
#\izlaz{aleksandar ana andjela ljubica matej milica milos nikola petar }#
#\izlaz{Uneti trazenu nisku:}\ulaz{matej}#
#\izlaz{Niska "matej" je pronadjena u nizu na poziciji 4}#
#\izlaz{Niska "matej" je pronadjena u nizu na poziciji 4}#
#\izlaz{Niske sortirane po duzini:}#
#\izlaz{ana matej milos petar milica nikola andjela ljubica aleksandar}#
\end{upotreba}
\end{maxitest}
  
\linkresenje{519}
\end{Exercise}

\begin{Answer}[ref=519]
  \includecode{resenja/05_Sortiranje/519.c}
\end{Answer}
%-----------------------------------------------------------------
%-----------------------------------------------------------------
\begin{Exercise}[label=520]
  Uraditi prethodni zadatak \ref{519} sa dinamički alociranim niskama
  i sortiranjem niza pokazivača (umesto niza niski).
  
\linkresenje{520}
\end{Exercise}

\begin{Answer}[ref=520]
  \includecode{resenja/05_Sortiranje/520.c}
\end{Answer}
%-----------------------------------------------------------------
%-----------------------------------------------------------------
\begin{Exercise}[label=521]
  Napisati program koji korišćenjem bibliotečke funkcije \kckod{qsort}
  sortira studente prema broju poena osvojenih na kolokvijumu. Ukoliko
  više studenata ima isti broj bodova, sortirati ih po prezimenu
  leksikografski rastuće. Korisnik potom unosi broj bodova i prikazuje
  mu se jedan od studenata sa tim brojem bodova ili poruka ukoliko
  nema takvog. Potom, sa standardnog ulaza, unosi se prezime traženog
  studenta i prikazuje se osoba sa tim prezimenom ili poruka da se
  nijedan student tako ne preziva. Za pretraživanje koristiti
  odgovarajuće bibliotečke funkcije. Podaci o studentima čitaju se iz
  datoteke čije se ime zadaje preko argumenata komandne linije. Za
  svakog studenta u datoteci postoje ime, prezime i bodovi osvojeni na
  kolokvijumu. Pretpostaviti da neće biti vise od $500$ studenata i
  da je za ime i prezime svakog studenta dovoljno po $20$ karaktera.
  
\begin{miditest}
\begin{upotreba}{1}
#\poziv{./a.out kolokvijum.txt}#
  
#\naslovDat{Ulazna datoteka (kolokvijum.txt):}#
#\datoteka{Aleksandra Cvetic 15}#
#\datoteka{Bojan Golubovic 30}#
#\datoteka{Dragan Markovic 25}#
#\datoteka{Filip Dukic 20}#
#\datoteka{Ivana Stankovic 25}#
#\datoteka{Marija Stankovic 15}#
#\datoteka{Ognjen Peric 20}#
#\datoteka{Uros Milic 10}#
#\datoteka{Andrija Petrovic 0}#
#\datoteka{Anja Ilic 5}#
#\datoteka{Ivana Markovic 5}#
#\datoteka{Lazar Micic 20}#
#\datoteka{Nenad Brankovic 15}#
\end{upotreba}
\end{miditest}
\begin{miditest}
\begin{upotreba}{1}
#\naslovInt#
#\izlaz{Studenti sortirani po broju poena}#
#\izlaz{opadajuce, pa po prezimenu rastuce:}#
#\izlaz{Bojan Golubovic  30}#
#\izlaz{Dragan Markovic  25}#
#\izlaz{Ivana Stankovic  25}#
#\izlaz{Filip Dukic  20}#
#\izlaz{Lazar Micic  20}#
#\izlaz{Ognjen Peric  20}#
#\izlaz{Nenad Brankovic  15}#
#\izlaz{Aleksandra Cvetic  15}#
#\izlaz{Marija Stankovic  15}#
#\izlaz{Uros Milic  10}#
#\izlaz{Anja Ilic  5}# 
#\izlaz{Ivana Markovic  5}#
#\izlaz{Andrija Petrovic  0}#
#\izlaz{Unesite broj bodova:}\ulaz{20}#
#\izlaz{Pronadjen je student sa unetim}#
#\izlaz{brojem bodova: Filip Dukic 20}#
#\izlaz{Unesite prezime:}\ulaz{Markovic}#
#\izlaz{Pronadjen je student sa unetim}#
#\izlaz{prezimenom: Dragan Markovic 25}#
\end{upotreba}
\end{miditest}
  
\linkresenje{521}
\end{Exercise}

\begin{Answer}[ref=521]
  \includecode{resenja/05_Sortiranje/521.c}
\end{Answer}
%-----------------------------------------------------------------
%-----------------------------------------------------------------
\begin{Exercise}[label=522]
  Uraditi zadatak \ref{502}, ali korišćenjem bibliotečke \kckod{qsort}
  funkcije.
  
\linkresenje{522}
\end{Exercise}

\begin{Answer}[ref=522]
  \includecode{resenja/05_Sortiranje/522.c}
\end{Answer}
%-----------------------------------------------------------------
%-----------------------------------------------------------------
\begin{Exercise}[label=523]
  Napisati program koji sa standardnog ulaza učitava prvo ceo broj
  \argf{n} ($n \leq 10$), a zatim niz \argf{S} od \argf{n} stringova
  (maksimalna dužina stringa je $31$ karakter). Sortirati niz \argf{S}
  (bibliotečkom funkcijom \kckod{qsort}) i proveriti da li u njemu ima
  identičnih stringova.
  
\begin{minitest}
\begin{upotreba}{1}
#\naslovInt#
#\izlaz{Unesite broj niski:}\ulaz{4}#
#\izlaz{Unesite niske:}#
#\ulaz{prog search sort search}#
#\izlaz{ima}#
\end{upotreba}
\end{minitest}
\begin{minitest}
\begin{upotreba}{2}
#\naslovInt#
#\izlaz{Unesite broj niski:}\ulaz{3}#
#\izlaz{Unesite niske:}#
#\ulaz{test kol ispit}#
#\izlaz{nema}#
\end{upotreba}
\end{minitest}
\begin{minitest}
\begin{upotreba}{3}
#\naslovInt#
#\izlaz{Unesite broj niski:}\ulaz{5}#
#\izlaz{Unesite niske:}#
#\ulaz{a ab abc abcd abcde}#
#\izlaz{nema}#
\end{upotreba}
\end{minitest}
  
\linkresenje{523}
\end{Exercise}

\begin{Answer}[ref=523]
  \includecode{resenja/05_Sortiranje/523.c}
\end{Answer}
%-----------------------------------------------------------------
%-----------------------------------------------------------------
\begin{Exercise}[label=524]
  Datoteka \kckod{studenti.txt} sadrži spisak studenata. Za svakog
  studenta poznat je nalog na Alas-u (oblika npr. \kckod{mr97125},
  \kckod{mm09001}), ime i prezime i broj poena. I ime i prezime neće
  biti duže od 20 karaktera. Napisati program koji sortira
  (korišćenjem funkcije \kckod{qsort}) studente po broju poena
  opadajuće (ukoliko je prisutna opcija \argf{-p}) ili po nalogu
  (ukoliko je prisutna opcija \argf{-n}). Studenti se po nalogu
  sortiraju tako što se sortiraju na osnovu godine, zatim na osnovu
  smera, i na kraju na osnovu broja indeksa. Sortirane studente
  upisati u datoteku \kckod{izlaz.txt}. Ukoliko je u komandnoj liniji
  uz opciju \argf{-n} naveden i nalog nekog studenta, funkcijom
  \kckod{bsearch} potražiti i prijaviti broj poena studenta sa tim
  nalogom.
  
\begin{miditest}
\begin{test}{1}
#\poziv{./a.out -n mm13321}#
  
#\naslovDat{studenti.txt}#
#\datoteka{mr14123 Marko Antic 20}#
#\datoteka{mm13321 Marija Radic 12}#
#\datoteka{ml13011 Ivana Mitrovic 19}#
#\datoteka{ml13066 Pera Simic 15}#
#\datoteka{mv14003 Jovan Jovanovic 17}#

#\naslovDat{izlaz.txt}#
#\datoteka{ml13011 Ivana Mitrovic 19}#
#\datoteka{ml13066 Pera Simic 15}#
#\datoteka{mm13321 Marija Radic 12}#
#\datoteka{mr14123 Marko Antic 20}#
#\datoteka{mv14003 Jovan Jovanovic 17}#
  
#\naslovIzlaz#
#\izlaz{mm13321 Marija Radic 12}#
\end{test}
\end{miditest}
\begin{miditest}
\begin{test}{2}.
#\poziv{/a.out -p}#
  
#\naslovDat{studenti.txt}#
#\datoteka{mr14123 Marko Antic 20}#
#\datoteka{mm13321 Marija Radic 12}#
#\datoteka{ml13011 Ivana Mitrovic 19}#
#\datoteka{ml13066 Pera Simic 15}#
#\datoteka{mv14003 Jovan Jovanovic 17}#
  
#\naslovDat{izlaz.txt}#
#\datoteka{mr14123 Marko Antic 20}#
#\datoteka{ml13011 Ivana Mitrovic 19}#
#\datoteka{mv14003 Jovan Jovanovic 17}#
#\datoteka{ml13066 Pera Simic 15}#
#\datoteka{mm13321 Marija Radic 12}#
\end{test}
\end{miditest}
  
\linkresenje{524}
\end{Exercise}

\begin{Answer}[ref=524]
  \includecode{resenja/05_Sortiranje/524.c}
\end{Answer}
%-----------------------------------------------------------------
%-----------------------------------------------------------------
\begin{Exercise}[label=525]
  Definisana je struktura:
  \begin{ckod}
    typedef struct { int dan; int mesec; int godina; } Datum;
  \end{ckod}
  Napisati funkciju koja poredi dva datuma i program koji učitava
  datume iz datoteke koja se zadaje kao prvi argument komandne linije
  (ne više od $128$ datuma), sortira ih pozivajući funkciju
  \kckod{qsort} iz standardne biblioteke i potom pozivanjem funkcije
  \kckod{bsearch} iz standardne biblioteke proverava da li datumi
  učitani sa standardnog ulaza (sve do kraja ulaza) postoje među
  prethodno unetim datumima.
  
\begin{minitest}
\begin{upotreba}{1}
#\poziv{./a.out datoteka.txt}#
  
#\naslovDat{datoteka.txt}#
#\datoteka{1.1.2013.}#
#\datoteka{13.12.2016.}#
#\datoteka{11.11.2011.}#
#\datoteka{3.5.2015.}#
#\datoteka{5.2.2009.}#
\end{upotreba}
\end{minitest}
\begin{miditest}
\begin{upotreba}{1}
#\naslovInt#
#\izlaz{Unesi sledeci datum za pregragu:}\ulaz{13.12.2016.}#
#\izlaz{postoji}#
#\izlaz{Unesi sledeci datum za pregragu:}\ulaz{10.5.2015.}#
#\izlaz{ne postoji}#
#\izlaz{Unesi sledeci datum za pregragu:}\ulaz{5.2.2009.}#
#\izlaz{postoji}#
\end{upotreba}
\end{miditest}

%\linkresenje{525}
\end{Exercise}

%\begin{Answer}[ref=525]
%  \includecode{resenja/05_Sortiranje/525.c}
%\end{Answer}
%-----------------------------------------------------------------
%-----------------------------------------------------------------
\begin{Exercise}[label=526]
  Za zadatu celobrojnu matricu dimenzije $n \times m$ napisati
  funkciju koja vrši sortiranje vrsta matrice rastuće na osnovu sume
  elemenata u vrsti. Napisati potom program koji testira ovu funkciju. Sa
  standardnog ulaza se prvo unose dimenzije matrice, a zatim redom
  elementi matrice. Rezultujuću matricu ispisati na standardnom izlazu.

\begin{miditest}
\begin{test}{1}
#\naslovInt#
#\izlaz{Unesite dimenzije matrice:}\ulaz{3 2}#
#\izlaz{Unesite elemente matrice po vrstama:}#
#\ulaz{6 -5}#
#\ulaz{-4 3}#
#\ulaz{2 1}#
#\izlaz{Sortirana matrica je:}#
#\izlaz{-4 3}#
#\izlaz{6 -5}# 
#\izlaz{2 1}#  
\end{test}
\end{miditest}
\begin{miditest}
\begin{test}{2}
#\naslovInt#
#\izlaz{Unesite dimenzije matrice:}\ulaz{4 4}#
#\izlaz{Unesite elemente matrice po vrstama:}#
#\ulaz{34 12 54 642}#
#\ulaz{1 2 3 4}#
#\ulaz{53 2 1 5}#
#\ulaz{54 23 5 671}#
#\izlaz{Sortirana matrica je:}#
#\izlaz{1 2 3 4}#
#\izlaz{53 2 1 5}#
#\izlaz{34 12 54 642}#
#\izlaz{54 23 5 671}#
\end{test}
\end{miditest}

\linkresenje{526}
\end{Exercise}

\begin{Answer}[ref=526]
  \includecode{resenja/05_Sortiranje/526.c}
\end{Answer}
%-----------------------------------------------------------------
%-----------------------------------------------------------------
\begin{Exercise}[label=527]
  Za zadatu kvadratnu matricu dimenzije $n$ napisati funkciju koja
  sortira kolone matrice opadajuće na osnovu vrednosti prvog elementa
  u koloni.  Napisati program koji testira ovu funkciju. Sa
  standardnog ulaza se prvo unosi dimenzija matrice, a zatim redom
  elementi matrice.  Rezultujuću matricu ispisati na standardnom
  izlazu.


\begin{miditest}
\begin{upotreba}{1}
#\naslovInt#  
#\izlaz{Unesite dimenziju matrice:}\ulaz{2}#
#\izlaz{Unesite elemente matrice po vrstama:}#
#\ulaz{6 -5}#
#\ulaz{-4 3}#
#\izlaz{Sortirana matrica je:}#
#\izlaz{-5 6}#
#\izlaz{3 -4}#
\end{upotreba}
\end{miditest}
\begin{miditest}
\begin{upotreba}{2}
#\naslovInt#  
#\izlaz{Unesite dimenziju matrice:}\ulaz{4}#
#\izlaz{Unesite elemente matrice po vrstama:}#
#\ulaz{34 12 54 642}#
#\ulaz{1 2 3 4}#
#\ulaz{53 2 1 5}#
#\ulaz{54 23 5 671}#
#\izlaz{Sortirana matrica je:}#
#\izlaz{12 34 54 642}#
#\izlaz{2 1 3 4}#
#\izlaz{2 53 1 5}#
#\izlaz{23 54 5 671}#
\end{upotreba}
\end{miditest}
  
%\linkresenje{527}
\end{Exercise}

%\begin{Answer}[ref=527]
%  \includecode{resenja/05_Sortiranje/527.c}
%\end{Answer}
%-----------------------------------------------------------------
%-----------------------------------------------------------------

\section{Rešenja}
\shipoutAnswer


