\chapter{Sortiranje}

\section{Zadaci}

%-----------------------------------------------------------------
%-----------------------------------------------------------------
\begin{Exercise}[label=501]
  U datom nizu brojeva prona\'ci dva koja su na najmanjem
  rastojanju. Niz se zadaje sa standardnog ulaza, sve do kraja ulaza,
  i ne\' ce imati vi\v se od 256 elemenata. (uputstvo: prvo sortirati
  niz). Na izlaz ispisati njihovu razliku.
  
  \begin{miditest}
    \begin{test}{Test 1}
      Ulaz:   23 64 123 76 22 7
      Izlaz:  1
    \end{test}
  \end{miditest}
  
  \begin{miditest}
    \begin{test}{Test 2}
      Ulaz:   21 654 65 123 65 12 61
      Izlaz:  0
    \end{test}
  \end{miditest}
  
\end{Exercise}

\begin{Answer}[ref=501]
  \includecode{resenja/05_Sortiranje/501.c}
\end{Answer}
%-----------------------------------------------------------------
%-----------------------------------------------------------------
\begin{Exercise}[label=502]
  Napisati funkciju koja sortira slova unutar niske
  karaktera. Napisati program koji proverava da li su dve niske
  karaktera anagrami. Dve niske su anagrami ako se sastoje od istog
  broja istih karaktera. Niske se zadaju sa standardnog ulaza, i ne\'
  ce biti du\v ze od 128 karaktera.
  
  \begin{miditest}
    \begin{test}{Test 1}
      Ulaz:   anagram ramgana
      Izlaz:  jesu
    \end{test}
  \end{miditest}
  \begin{miditest}
    \begin{test}{Test 2}
      Ulaz:   anagram anagrm
      Izlaz:  nisu
    \end{test}
  \end{miditest}
  
\end{Exercise}

%\begin{Answer}[ref=502]
%  \includecode{resenja/05_Sortiranje/502.c}
%\end{Answer}
%-----------------------------------------------------------------
%-----------------------------------------------------------------
\begin{Exercise}[label=503]
  Napisati program koji pronalazi broj koji se najvi\v{s}e puta
  pojavljivao u datom nizu (uputstvo: prvo sortirati niz a zatim
  na\'ci najdu\v{z}u sekvencu jednakih elemenata). Niz se zadaje sa
  standardnog ulaza sve do kraja ulaza, i ne\' ce biti du\v zi od 256
  elemenata.
  
  \begin{miditest}
    \begin{test}{Test 1}
      Ulaz:   4 23 5 2 4 6 7 34 6 4 5
      Izlaz:  4
    \end{test}
  \end{miditest}
  \begin{miditest}
    \begin{test}{Test 2}
      Ulaz:   2 4 6 2 6 7 99 1
      Izlaz:  2
    \end{test}
  \end{miditest}
  
\end{Exercise}

%\begin{Answer}[ref=503]
%  \includecode{resenja/05_Sortiranje/503.c}
%\end{Answer}
%-----------------------------------------------------------------
%-----------------------------------------------------------------
\begin{Exercise}[label=504]
  Napisati funkciju koja proverava da li u datom nizu postoje dva
  elementa kojima je zbir zadati ceo broj (uputstvo: prvo sortirati
  niz). Napisati i program koji testira ovu funkciju, u kome se prvo
  u\v citava pomenuti broj, pa zatim niz ne ve\' ce du\v zine od 256
  sve do kraja ulaza.
  
  \begin{miditest}
    \begin{test}{Test 1}
      Ulaz:   34 134 4 1 6 30 23
      Izlaz:  da
    \end{test}
  \end{miditest}
  \begin{miditest}
    \begin{test}{Test 2}
      Ulaz:   12 53 1 43 3 56 13
      Izlaz:  ne
    \end{test}
  \end{miditest}
  
\end{Exercise}

%\begin{Answer}[ref=504]
%  \includecode{resenja/05_Sortiranje/504.c}
%\end{Answer}
%-----------------------------------------------------------------
%-----------------------------------------------------------------
\begin{Exercise}[label=505]
  Napisati funkciju koja prima dva sortirana niza, i na osnovu
  njih pravi novi sortirani niz koji koji sadr\v zi elemente oba niza
\begin{verbatim}
int merge(int *niz1, int dim1, int *niz2, int dim2, int *niz3, int dim3);
\end{verbatim}
Tre\' ca dimenzija predstavlja veli\v cinu niza u koji se sme\v sta
rezultat. Ako je ona manja od potrebne du\v zine, funkcija vra\' ca
-1, kao indikator neuspeha, ina\v ce vra\' ca 0. Napisati i program
koji testira funkciju, u kome se nizovi unose sa standardnog ulaza,
sve dok se ne unese 0.
  
  \begin{miditest}
    \begin{test}{Test 1}
      Ulaz:   3 6 7 11 14 35 0 3 5 8 0
      Izlaz:  3 3 5 6 7 8 11 14 35
    \end{test}
  \end{miditest}
  
  \begin{miditest}
    \begin{test}{Test 1}
      Ulaz:   1 4 7 0 9 11 23 54 75 0
      Izlaz:  1 4 7 9 11 23 54 75
    \end{test}
  \end{miditest}
  
\end{Exercise}

%\begin{Answer}[ref=505]
%  \includecode{resenja/05_Sortiranje/505.c}
%\end{Answer}
%-----------------------------------------------------------------
%-----------------------------------------------------------------
\begin{Exercise}[label=506]
  Napraviti biblioteku ,,sort.h'' i ,,sort.c'' koja implementira
  algoritme sortiranja nizova celih brojeva. Biblioteka treba da
  sadr\v zi bubble, insertion i shell sort. Upotrebiti biblioteku kako
  bi se napravilo pore\dj enje efikasnosti razli\v citih algoritama
  sortiranja. Efikasnost meriti na slu\v cajno generisanim nizovima,
  na ve\' c sortiranim nizovima i na naopako sortiranim
  nizovima. Izbor algoritma, veli\v cine i po\v cetnog rasporeda
  elemenata niza birati kroz argumente komandne linije.  Vreme meriti
  programom \verb|time|. Analizirati porast vremena sa porastom
  dimenzije $n$. Porediti vremena za razli\v cite implementacije istog
  algoritma (iterativna i rekurzivna).  
  
\end{Exercise}

%\begin{Answer}[ref=506]
%  \includecode{resenja/05_Sortiranje/506.c}
%\end{Answer}
%-----------------------------------------------------------------
%-----------------------------------------------------------------
\begin{Exercise}[label=507]
  Napisati funkcije koje sortiraju niz struktura ta\v{c}aka na
  osnovu
\begin{enumerate}
\item njihovog rastojanja od koordinatnog po\v{c}etka.
\item x koordinate date ta\v{c}ke.
\item y koordinate date ta\v{c}ke.
\end{enumerate}
Napisati program koji u\v{c}itava niz ta\v{c}aka iz datoteke \v{c}ije
se ime zadaje kao argument komandne linije, i u zavisnosti od
prisutnih opcija u komandnoj liniji, sortira ta\v{c}ke po jednom od
prethodna tri kriterijuma i rezultat upisuje u datoteku \v{c}ije se
ime zadaje kao drugi argument komandne linije. U ulaznoj datoteci nije
zadato vi\v{s}e od 100 ta\v{c}aka.
  
  \begin{miditest}
    \begin{test}{Test 1}
Poziv:  a.out -x tacke.txt sorttacke.txt
Ulazna datoteka:   3 4
                   11 6
                   7 3
                   2 82
                   -1 6
Izlazna datoteka:  -1 6
                   2 82
                   3 4
                   7 3
                   11 6
    \end{test}
  \end{miditest}
  
\end{Exercise}

%\begin{Answer}[ref=507]
%  \includecode{resenja/05_Sortiranje/507.c}
%\end{Answer}
%-----------------------------------------------------------------
%-----------------------------------------------------------------
\begin{Exercise}[label=508]
  Definisana je struktura podataka
\begin{verbatim}
typedef struct dete
{
      char ime[MAX_IME];
      char prezime[MAX_IME];
      unsigned godiste;
} Dete;
\end{verbatim}
Napisati funkciju koja srotira niz dece po godi\v{s}tu, a kada su deca
istog godi\v{s}ta, tada ih sortirati leksikografski po prezimenu i
imenu. Napisati program koji u\v{c}itava podatke o deci koji se nalaze
u datoteci \v{c}ije se ime zadaje kao prvi argument komandne linije,
sortira ih i sortirani niz upisuje u datoteku \v{s}ije se ime zadaje
kao drugi argument komandne linije. Pretpostaviti da u ulaznoj
datoteci nisu zadati podaci o vi\v{s}e od 100 dece.
  
  \begin{maxitest}
    \begin{test}{Test 1}
Poziv: ./a.out ulaz.txt izlaz.txt
Ulazna datoteka:                 Izlazna datoteka:
Petar Petrovic 2007              Marija Antic 2007
Milica Antonic 2008              Ana Petrovic 2007
Ana Petrovic 2007                Petar Petrovic 2007
Ivana Ivanovic 2009              Milica Antonic 2008
Dragana Markovic 2010            Ivana Ivanovic 2009
Marija Antic 2007                Dragana Markovic 2010
    \end{test}
  \end{maxitest}
  
\end{Exercise}

%\begin{Answer}[ref=508]
%  \includecode{resenja/05_Sortiranje/508.c}
%\end{Answer}
%-----------------------------------------------------------------
%-----------------------------------------------------------------
\begin{Exercise}[label=509]
  Napisati funkciju koja sortira niz niski po broju suglasnika u
  niski, ukoliko re\v{c}i imaju isti broj suglasnika tada po
  du\v{z}ini niske, a ukoliko su i du\v{z}ine jednake tada
  leksikografski.  Napisati program koji testira ovu funkciju za niske
  koje se zadaju u datoteci \verb|niske.txt|.  Pretpostaviti da u nizu
  nema vi\v{s}e od 100 elemenata, kao da svaka niska sadr\v{z}i
  najvi\v{s}e 20 karaktera.
  
  \begin{maxitest}
    \begin{test}{Test 1}
Ulazna datoteka:
ana petar andjela milos nikola aleksandar ljubica matej milica
Izlaz:
ana matej milos petar milica nikola andjela ljubica aleksandar
    \end{test}
  \end{maxitest}
  
\end{Exercise}

%\begin{Answer}[ref=509]
%  \includecode{resenja/05_Sortiranje/509.c}
%\end{Answer}
%-----------------------------------------------------------------
%-----------------------------------------------------------------
\begin{Exercise}[label=510]
  Razmatrajmo dve operacije: operacija U je unos novog broja $x$, a
  operacija N odre\dj ivanje $n$-tog po veli\v cini od unetih
  brojeva. Implementirati program koji izvr\v sava ove operacije (Mo\v
  ze postojati najvi\v se 100000 operacija unosa, a uneti elementi se
  mogu ponavljati, pri \v cemu se i ponavljanja ra\v cunaju prilikom
  brojanja). Napomena: brojeve \v cuvati u sortiranom nizu i svaki
  naredni element umetati na svoje mesto. Optimizovati program,
  ukoliko se zna da ne\' ce biti vi\v se od 500 razli\v citih unetih
  brojeva.
  
  \begin{maxitest}
    \begin{test}{Test 1}
Ulaz: U 2 U 0 U 6 U 4 N 1 U 8 N 2 N 5 U 2 N 3 N 5
Izlaz: 0 2 8 2 6
    \end{test}
  \end{maxitest}
  
\end{Exercise}

%\begin{Answer}[ref=510]
%  \includecode{resenja/05_Sortiranje/510.c}
%\end{Answer}
%-----------------------------------------------------------------
%-----------------------------------------------------------------
\begin{Exercise}[label=511]
  Sa dve susedne stranice pravougaone livade dve grupe krtica
  istovremeno kre\' cu da kopaju tunele (jedna grupa na gore, a
  druga na desno). Krtica prestaje da kopa ukoliko nai\dj e na ve\'
  c iskopan tunel (npr. krticu A zaustavlja krtica C, krticu C i D
  zaustavlja krtica B, a krticu B zaustavlja krtica E). Za svaku
  krticu se zna udaljenost od \' co\v ska livade i brzina kojom
  kopa.
\begin{verbatim}
     |EEEEEEEEEE
     |      B
     |DDDDDDB 
     |      B
     |CCCCCCB
     |   A  B
     |___A__B_______________
\end{verbatim}
Napisati program koji odre\dj uje koliko dugo svaka krtica kopa. Unose
se broj krtica u obe grupe, a zatim udaljenost i brzina za svaku
krticu. Izlaz je vreme za svaku krticu prikazano na dve decimale (-1
ukoliko se ne zaustavlja) u istom redosledu u kojem su krtice unete.

  \begin{miditest}
    \begin{test}{Test 1}
      Ulaz:          Izlaz:
      2 3            1.12
      1   4          0.88
      2   5.1        -1.00
      4.5 3.6        1.00
      1   1          -1.00
      5   0.5
    \end{test}
  \end{miditest}
  
\end{Exercise}

%\begin{Answer}[ref=511]
%  \includecode{resenja/05_Sortiranje/511.c}
%\end{Answer}
%-----------------------------------------------------------------
%-----------------------------------------------------------------
\begin{Exercise}[label=512]
  \v Sef u restoranu je neuredan i pala\v cinke koje ispe\v ce
  ne sla\v ze redom po veli\v cini. Konobar pre serviranja mora da
  sortira pala\v cinke po veli\v cini, a jedina operacija koju sme
  da izvodi je da obrne deo pala\v cinki. Na primer:
\begin{verbatim}
    3    5    2    1
    4    4    1__  2
    5__  3    3    3
    1    1    4    4
    2    2__  5    5
\end{verbatim}
Napisati program koji u najvi\v se $2n-3$ okretanja sortira u\v
citani niz. (Uputstvo: imitirati selection sort i u svakom koraku
dovesti jednu pala\v cinku na svoje mesto kori\v s\' cenjem
najvi\v se dva okretanja.)
    
\end{Exercise}

%\begin{Answer}[ref=512]
%  \includecode{resenja/05_Sortiranje/512.c}
%\end{Answer}
%-----------------------------------------------------------------
%-----------------------------------------------------------------
\begin{Exercise}[label=513]
  Napisati program koji sa standardnog ulaza u\v citava dva
  stringa, \verb|s| i \verb|t| (du\v zine manje od 20 karaktera),
  sortira nizove njihovih karaktera (bibliote\v ckom \verb|qsort|
  funkcijom), ispituje i \v stampa da li su \verb|s| i \verb|t|
  anagrami. (dva stringa su anagrami ako su sastavljeni od potpuno
  istih slova, samo razli\v cito raspore\dj enih)
  
  \begin{miditest}
    \begin{test}{Test 1}
Ulaz:   vrata vatra
Izlaz:  jesu
    \end{test}
  \end{miditest}
  \begin{miditest}
    \begin{test}{Test 2}
Ulaz:   qsort bsearch
Izlaz:  nisu
    \end{test}
  \end{miditest}
  
\end{Exercise}

%\begin{Answer}[ref=513]
%  \includecode{resenja/05_Sortiranje/513.c}
%\end{Answer}
%-----------------------------------------------------------------
%-----------------------------------------------------------------
\begin{Exercise}[label=514]
  Napisati program koji sa standardnog ulaza u\v citava prvo ceo broj
  $n$ ($n <= 10$) a zatim niz $S$ od $n$ stringova (maksimalna du\v
  zina stringa je $20$ karaktera). Sortirati niz $S$ (bibliote\v{c}kom
  funkcijom \verb|qsort|) i proveriti da li u njemu ima identi\v{c}nih
  stringova.
  
  \begin{miditest}
    \begin{test}{Test 1}
      Ulaz:   4 prog search sort search
      Izlaz:  ima
    \end{test}
  \end{miditest}
  \begin{miditest}
    \begin{test}{Test 2}
      Ulaz:   3 test kol ispit
      Izlaz:  nema
    \end{test}
  \end{miditest}
  
\end{Exercise}

%\begin{Answer}[ref=514]
%  \includecode{resenja/05_Sortiranje/514.c}
%\end{Answer}
%-----------------------------------------------------------------
%-----------------------------------------------------------------
\begin{Exercise}[label=515]
  Datoteka \texttt{studenti.txt} sadr\v zi spisak studenata. Za svakog
  studenta poznat je nalog na Alas-u (oblika npr. \texttt{mr97125},
  \texttt{mm09001}), ime i prezime i broj poena. Napisati program koji
  sortira (kori\v s\' cenjem funkcije \verb|qsort|) studente po broju
  poena (ukoliko je prisutna opcija \verb"-p") ili po nalogu (ukoliko
  je prisutna opcija \verb"-n"). Studenti se po nalogu sortiraju tako
  \v sto se sortiraju na osnovu godine, zatim na osnovu smera, i na
  kraju na osnovu broja indeksa. Ukoliko je u komandnoj liniji uz
  opciju \verb"-n" naveden i nalog nekog studenta, funkcijom
  \verb|bsearch| potra\v ziti i prijaviti broj poena studenta sa tim
  nalogom. Sortirane studente upisati u datoteku $izlaz.txt$.
  
  \begin{maxitest}
    \begin{test}{Test 1}
Poziv: ./a.out -n mm13321
Datoteka:                         Izlaz:
mr14123 Marko Antic 20            mm13321 Marija Radic 12
mm13321 Marija Radic 12
ml13011 Ivana Mitrovic 19
ml13066 Pera Simic 15
mv14003 Jovan Jovanovic 17
    \end{test}
  \end{maxitest}
  
\end{Exercise}

%\begin{Answer}[ref=515]
%  \includecode{resenja/05_Sortiranje/515.c}
%\end{Answer}
%-----------------------------------------------------------------
%-----------------------------------------------------------------
\begin{Exercise}[label=516]
Definisana je struktura:
\begin{verbatim}
     typedef struct { int dan; int mesec; int godina; } Datum;
\end{verbatim}

Napisati funkciju koja poredi dva datuma i program koji u\v{c}itava
datume iz datoteke koja se zadaje kao prvi argument komandne linije
(ne vi\v{s}e od 100 datuma), sortira ih pozivaju\'ci funkciju
\verb|qsort| iz standardne biblioteke i potom pozivanjem funkcije
\verb|bsearch| iz standardne biblioteke proverava da li datumi u\v
citani sa standardnog ulaza (sve do kraja ulaza) postoje me\dj u
prethodno unetim datumima.
  
  \begin{maxitest}
    \begin{test}{Test 1}
Poziv: ./a.out datoteka.txt
Datoteka:  1.1.2013       Ulaz:  13.12.2016     Izlaz:  postoji
           13.12.2016            10.5.2015              ne postoji
           11.11.2011            5.2.2009               postoji
           3.5.2015
           5.2.2009
    \end{test}
  \end{maxitest}
  
\end{Exercise}

%\begin{Answer}[ref=516]
%  \includecode{resenja/05_Sortiranje/516.c}
%\end{Answer}
%-----------------------------------------------------------------
%-----------------------------------------------------------------



\section{Rešenja}
\shipoutAnswer


