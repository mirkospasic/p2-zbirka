\chapter*{Predgovor}

U okviru kursa {\em Programiranje 2} na Matematičkom fakultetu vežbaju se zadaci 
koji imaju za cilj da studente nauče rekurzivnom pristupu rešavanja problema, 
ispravnom radu sa pokazivačima i dinamički alociranom memorijom, osnovnim algoritmima 
pretraživanja i sortiranja, kao i radu sa dinamičkim strukturama podataka, 
poput listi i stabala. Zadaci koji se nalaze u ovoj zbirci predstavljaju 
objedinjen skup zadataka sa vežbi i praktikuma ovog kursa, kao i primere 
zadataka sa održanih ispita. 
% Elektronska verzija zbirke, dostupna je 
% (besplatno) u okviru strane kursa \url{www.programiranje2.matf.bg.ac.rs}, 
% a tu je dostupan i radni repozitorijum elektronskih verzija rešenja zadataka.
Elektronska verzija zbirke i propratna rešenja u elektronskom formatu, dostupna su 
besplatno u okviru strane kursa \url{www.programiranje2.matf.bg.ac.rs} u skladu sa navedenom licencom.


U prvom poglavlju zbirke obrađene su uvodne teme koje obuhvataju osnovne tehnike koje se koriste u rešavanju svih ostalih zadataka u zbirci: podela koda po datotekama i rekurzivni pristup rešavanju problema. Takođe, u okviru ovog poglavlja dati su i osnovni algoritmi za rad sa bitovima. Drugo poglavlje je posvećeno pokazivačima: pokazivačkoj aritmetici, višedimenzionim nizovima, dinamičkoj alokaciji memorije i radu sa pokazivačima na funkcije. Treće poglavlje obrađuje algoritme pretrage i sortiranja, a četvrto dinamičke strukture podataka: liste i stabla. Dodatak sadrži najvažnije ispitne rokove iz jedne akademske godine. Većina zadataka je rešena, a teži zadaci su obeleženi zvezdicom.


Autori velikog broja zadataka ove zbirke su ujedno i autori same zbirke, ali postoje 
i zadaci za koje se ne može tačno utvrditi ko je originalni autor jer su zadacima 
davali svoje doprinose različiti asistenti koji su držali vežbe iz ovog kursa u 
prethodnih desetak godina. Zbog toga smatramo da je naš osnovni doprinos 
što smo objedinili, precizno formulisali, rešili i detaljno iskomentarisali 
sve najvažnije zadatke koji su potrebni za uspešno savlađivanje koncepata 
koji se obrađuju u okviru kursa. Takođe, formulacije zadataka smo obogatili
primerima koji upotpunjuju razumevanje zahteva zadataka i koji omogućavaju
čitaocu zbirke da proveri sopstvena rešenja. Primeri su dati u obliku 
testova i interakcija sa programom. Testovi su svedene prirode i obuhvataju 
samo jednostavne ulaze i izlaze iz programa. Interakcija sa programom obuhvata 
naizmeničnu interakciju čovek-računar u kojoj su ulazi i izlazi isprepletani. U zadacima koji zahtevaju rad sa argumentima komandne linije, navedeni su i primeri poziva programa, a u zadacima koji demonstriraju rad sa datotekama, i primeri ulaznih ili izlaznih datoteka. Test primeri koji su navedeni uz ispitne zadatke u dodatku su oni koji su korišćni za 
početno testiranje (koje prethodi ocenjivanju) studentskih radova na ispitima.

%Nakon formulacije zadataka navedeni su i test primeri, koje ne samo da upotpunjuju razumevanje zahteva zadatka uz koji stoje, već mogu i poslužiti da čitalac ove zbirke proveri (sopstvena) rešenja zadataka. Test primeri koji su navedeni uz ispitne zadatke u dodatku su upravo oni koji su korišćeni za testiranje studentskih radova koje prethodi ocenjivanju radova na ispitima za kurs  {\em Programiranje 2} na Matematičkom fakultetu. 

%\newpage
Neizmerno zahvaljujemo recenzentima, Gordani Pavlović Lažetić i Draganu Uroševiću, na veoma pažljivom čitanju rukopisa i na brojnim korisnim sugestijama. Takođe, zahvaljujemo studentima koji su svojim aktivnim učešćem u nastavi pomogli i doprineli uobličavanju ovog materijala. 

Svi komentari i sugestije na sadržaj zbirke su dobrodošli i osećajte se slobodnim da ih pošaljete elektronskom poštom bilo kome od autora\footnote{Adrese autora su: \texttt{milena}, \texttt{jgraovac}, \texttt{nina}, \texttt{aspasic}, \texttt{mirko}, \texttt{andjelkaz}, sa nastavkom \texttt{@matf.bg.ac.rs}}. 



\bigskip

\begin{flushright}
{\em Autori}
\end{flushright}