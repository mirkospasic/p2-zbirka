
\chapter*{Predgovor}

U okviru kursa {\em Programiranje 2} na Matematičkom fakultetu vežbaju se zadaci 
koji imaju za cilj da studente nauče rekurzivnom pristupu rešavanju problema, 
ispravnom radu sa pokazivačima i dinamički alociranom memorijom, osnovnim algoritmima 
pretraživanja i sortiranja, kao i radu sa dinamičkim strukturama podataka, 
poput listi i stabala. Zadaci koji se nalaze u ovoj zbirci predstavljaju 
objedinjen skup zadataka sa vežbi i praktikuma ovog kursa, kao i primere 
zadataka sa održanih ispita. Elektronska verzija zbirke, dostupna je 
(besplatno) u okviru strane kursa \url{www.programiranje2.matf.bg.ac.rs}, 
a tu je dostupan i radni repozitorijum elektronskih verzija rešenja zadataka.

U prvom poglavlju zbirke obrađene su uvodne teme koje obuhvataju osnovne tehnike koje se koriste u rešavanju svih ostalih zadataka u zbirci: podela koda po datotekama i rekurzivni pristup rešvanju problema. Takođe, u okviru ovog poglavlja dati su i osnovni algoritmi za rad sa bitovima. Drugo poglavlje je posvećeno pokazivačima, pokazivačkoj aritmetici, višedimenzionim nizovima, dinamičkoj alokaciji memorije i radu sa pokazivačima na funkcije. Treće poglavlje obrađuje algoritme pretrage i sortiranja, a četvrto dinamičke strukture podataka: liste i stabla. U poslednjem poglavlju dati su zadaci sa ispitnih rokova u toku jedne školske godine. Većina zdataka je rešeno, teži zadaci su obeleženi sa zvezdicom, a posebno teški zadaci sa dve zvezdice.


Autori velikog broja zadataka ove zbirke su ujedno i autori same zbirke, ali postoje 
i zadaci za koje se ne može tačno utvrditi ko je originalni autor jer su zadacima 
davali svoje doprinose različiti asistenti koji su držali vežbe iz ovog kursa u 
prethodnih desetak godina. Zbog toga smatramo da je naš osnovni doprinos 
što smo objedinili, precizno formulisali, rešili i detaljno iskomentarisali 
sve najvažnije zadatke koji su potrebni za uspešno savlađivanje koncepata 
koji se obrađuju u okviru kursa. 

\newpage
Zahvaljujemo recenzentima, Gordani Pavlović Lažetić i Draganu Uroševiću, na veoma pažljivom čitanju rukopisa i na brojnim korisnim sugestijama. Takođe, zahvaljujemo studentima koji su svojim aktivnim učešćem u nastavi pomogli i doprineli u obličavanju ovog materijala. 

Svi komentari i sugestije na zadatke i rešenja zbirke su dobrodošli i osećajte se slobodno da ih pošaljete elektronskom poštom bilo kome od autora\footnote{Adrese autora su: \texttt{milena}, \texttt{jgraovac}, \texttt{aspasic}, \texttt{mirko}, \texttt{nina}, \texttt{andjelkaz} sa nastavkom \texttt{@matf.bg.ac.rs}}. 



%Da dodamo ko je radio na kom poglavlju?


\bigskip

\begin{flushright}
{\em Autori}
\end{flushright}
