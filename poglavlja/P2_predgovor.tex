
\chapter*{Predgovor}

U okviru kursa {\em Programiranje 2} na Matematičkom fakultetu vežbaju se zadaci 
koji imaju za cilj da studente nauče rekurzivnom pristupu rešavanju problema, 
ispravnom radu sa pokazivačima i dinamički alociranom memorijom,  osnovnim algoritmima 
pretraživanja i sortiranja, kao i radu sa dinamičkim strukturama podataka, 
poput listi i stabala. Zadaci koji se nalaze u ovoj zbirci predstavljaju 
objedinjen skup zadataka sa vežbi i praktikuma ovog kursa, kao i primere 
zadataka sa kolokvijuma i ispita. Elektronska verzija zbirke, dostupna je u okviru
strane kursa \url{www.programiranje2.matf.bg.ac.rs}, a tu je dostupan i radni 
repozitorijum elektronskih verzija rešenja zadataka.

Autori velikog broja zadataka ove zbirke su ujedno i autori same zbirke, ali postoje 
i zadaci za koje se ne može tačno utvrditi ko je originalni autor jer su zadacima 
davali svoje doprinose različiti asistenti koji su držali vežbe iz ovog kursa u 
prethodnih desetak godina, pomenimo tu, pre svega, Milana Bankovića i doc dr 
Filipa Marića. Zbog toga smatramo da je naš osnovni doprinos što smo objedinili, 
precizno formulisali i rešili sve najvažnije zadatke koji su potrebni za 
uspešno savlađivanje koncepata koji se obrađuju u okviru kursa. 

...

%Zahvaljujemo se recenzentima na ..., kao i studentima koji su svojim aktivnim učešćem u nastavi pomogli i doprineli u obličavanju ovog materijala.  

%Prvo poglavlje nosi naziv Uvodni zadaci, s obzirom da obuhva gradivo sa pocetka kursa {\em Programiranje 2} na Matematičkom fakultetu i čine ga tri celine: podela koda po datotekama, algoritmi za rad sa bitovima i rekurzija.

%Da dodamo kratko o svakom poglavlju i mozda ko je radio na kom poglavlju?


\bigskip

\begin{flushright}
{\em Autori}
\end{flushright}
