
\chapter{Algoritmi pretrage i sortiranja}

\section{Algoritmi pretrage}

%-----------------------------------------------------------------
%-----------------------------------------------------------------
\begin{Exercise}[label=3_01]
  Napisati iterativne funkcije pretraga nizova. Svaka funkcija treba
  da vrati indeks pozicije na kojoj je pronađen traženi broj ili
  vrednost $-1$ ukoliko broj nije pronađen.
  \begin{enumerate}  
  \item Napisati funkciju \kckod{linarna\_pretraga} koja vrši linearnu pretragu niza 
    celih brojeva \argf{a}, dužine \argf{n}, tražeći u njemu broj
    \argf{x}.  
  \item Napisati funkciju \kckod{binarna\_pretraga} koja vrši binarnu pretragu
    sortiranog niza \argf{a}, dužine \argf{n}, tražeći u njemu broj \argf{x}.
  \item Napisati funkciju \kckod{interpolaciona\_pretraga} koja vrši interpolacionu pretragu
    sortiranog niza \argf{a}, dužine \argf{n}, tražeći u njemu broj \argf{x}.
  \end{enumerate}
  Napisati i program koji generiše rastući niz slučajnih brojeva dimenzije
  $n$ i pozivajući napisane funkcije traži broj $x$. Programu se kao prvi 
  argument komandne linije prosleđuje prirodan broj \kckod{n} koji nije veći od $1000000$ i broj \kckod{x} kao drugi
  argument komandne linije. Potrebna vremena za izvršavanje ovih
  funkcija dopisati u datoteku \kckod{vremena.txt}.

  
\begin{minitest}
\begin{test}{1}
#\poziv{./a.out 1000000 23542}#
  
#\naslovIzlaz#
#\izlaz{Linearna pretraga:}#
#\izlaz{Element nije u nizu}#
#\izlaz{Binarna pretraga:}#
#\izlaz{Element nije u nizu}#
#\izlaz{Interpolaciona pretraga:}#
#\izlaz{Element nije u nizu}#
\end{test}
\end{minitest}
\begin{minitest}
\begin{test2}{1}
    

  #\naslovDat{vremena.txt}#

  #\datoteka{Dimenzija niza: 1000000}#
  #\datoteka{	Linearna:            3615091 ns}#
  #\datoteka{	Binarna:                1536 ns}#
  #\datoteka{	Interpolaciona:          558 ns}#
\end{test2}
\end{minitest}

\begin{minitest}
\begin{test}{2}
#\poziv{./a.out 100000 37842}#
  
#\naslovIzlaz#
#\izlaz{Linearna pretraga:}#
#\izlaz{Element nije u nizu}#
#\izlaz{Binarna pretraga:}#
#\izlaz{Element nije u nizu}#
#\izlaz{Interpolaciona pretraga:}#
#\izlaz{Element nije u nizu}#
\end{test}
\end{minitest}
\begin{minitest}
\begin{test2}{1}


  #\naslovDat{vremena.txt}#

  #\datoteka{Dimenzija niza: 1000000}#
  #\datoteka{	Linearna:            3615091 ns}#
  #\datoteka{	Binarna:                1536 ns}#
  #\datoteka{	Interpolaciona:          558 ns}#
  #\datoteka{}#
  #\datoteka{Dimenzija niza: 100000}#
  #\datoteka{	Linearna:             360803 ns}#
  #\datoteka{	Binarna:                1187 ns}#
  #\datoteka{	Interpolaciona:          628 ns}#
\end{test2}
\end{minitest}

\linkresenje{3_01}

\end{Exercise}

\begin{Answer}[ref=3_01]
  \includecode{resenja/3_AlgoritmiPretrageISortiranja/3_01.c}
\end{Answer}

%-----------------------------------------------------------------
%-----------------------------------------------------------------
\begin{Exercise}[label=3_02]
  Napisati rekurzivne funkcije algoritama linearne, binarne i
  interpolacione pretrage i program koji ih testira za brojeve koji se
  unose sa standardnog ulaza. Linearnu pretragu implementirati na dva
  načina, svođenjem pretrage na prefiks i na sufiks
  niza. Prvo se unosi broj koji se traži, a zatim sortirani
  elementi niza sve do kraja ulaza. Pretpostaviti da niz brojeva koji se unosi neće biti duži od
  $1024$ elemenata.

\begin{miditest}
\begin{upotreba}{1}
#\naslovInt#
#\izlaz{Unesite trazeni broj:} \ulaz{11}#
#\izlaz{Unesite sortiran niz elemenata:}# 
#\ulaz{2 5 6 8 10 11 23}#
#\izlaz{Linearna pretraga}#
#\izlaz{Pozicija elementa je 5.}#
#\izlaz{Binarna pretraga}#
#\izlaz{Pozicija elementa je 5.}#
#\izlaz{Interpolaciona pretraga}#
#\izlaz{Pozicija elementa je 5.}#
\end{upotreba}
\end{miditest}
\begin{miditest}
\begin{upotreba}{2}
#\naslovInt#
#\izlaz{Unesite trazeni broj:} \ulaz{14}#
#\izlaz{Unesite sortiran niz elemenata:}#
#\ulaz{10 32 35 43 66 89 100}#
#\izlaz{Linearna pretraga}#
#\izlaz{Element se ne nalazi u nizu.}#
#\izlaz{Binarna pretraga}#
#\izlaz{Element se ne nalazi u nizu.}#
#\izlaz{Interpolaciona pretraga}#
#\izlaz{Element se ne nalazi u nizu.}#
\end{upotreba}
\end{miditest}

\linkresenje{3_02}

\end{Exercise}

\begin{Answer}[ref=3_02]
  \includecode{resenja/3_AlgoritmiPretrageISortiranja/3_02.c}
\end{Answer}
%-----------------------------------------------------------------
%-----------------------------------------------------------------
\begin{Exercise}[label=3_03]
  Napisati program koji preko argumenta komandne linije dobija ime
  datoteke koja sadrži sortirani spisak studenta po broju indeksa
  rastuće. Za svakog studenta u jednom redu stoje informacije o
  indeksu, imenu i prezimenu. Program učitava spisak studenata u niz i
  traži od korisnika indeks ili prezime studenta čije informacije se
  potom prikazuju na ekranu. U slučaju više studenata sa istim
  prezimenom prikazati informacije o prvom takvom. Odabir kriterijuma
  pretrage se vrši kroz poslednji argument komandne linije, koji može
  biti \kckod{-indeks} ili \kckod{-prezime}. U slučaju neuspešnih
  pretragi, štampati odgovarajuću poruku. Pretrage implementirati u
  vidu iterativnih funkcija što manje složenosti. Pretpostaviti da u
  datoteci neće biti više od $128$ studenata i da su imena i prezimena
  svih kraća od $16$ slova.
  
\begin{miditest}
\begin{upotreba}{1}
#\poziv{./a.out datoteka.txt -indeks}#
  
#\naslovDat{datoteka.txt}#
#\datoteka{20140003 Marina Petrovic}#
#\datoteka{20140012 Stefan Mitrovic}#
#\datoteka{20140032 Dejan Popovic}#
#\datoteka{20140049 Mirko Brankovic}#
#\datoteka{20140076 Sonja Stevanovic}#
#\datoteka{20140104 Ivan Popovic}#
#\datoteka{20140187 Vlada Stankovic}#
#\datoteka{20140234 Darko Brankovic}#
\end{upotreba}
\end{miditest}
\begin{miditest}
\begin{test2}{1}
  
  
#\naslovInt#
#\izlaz{Unesite indeks studenta}# 
#\izlaz{cije informacije zelite:}#
#\ulaz{20140076}#
#\izlaz{Indeks: 20140076,}# 
#\izlaz{Ime i prezime: Sonja Stevanovic}#
\end{test2}
\end{miditest}

\begin{miditest}
\begin{upotreba}{2}
#\poziv{./a.out datoteka.txt -prezime}#
  
#\naslovDat{datoteka.txt}#
#\datoteka{20140003 Marina Petrovic}#
#\datoteka{20140012 Stefan Mitrovic}#
#\datoteka{20140032 Dejan Popovic}#
#\datoteka{20140049 Mirko Brankovic}#
#\datoteka{20140076 Sonja Stevanovic}#
#\datoteka{20140104 Ivan Popovic}#
#\datoteka{20140187 Vlada Stankovic}#
#\datoteka{20140234 Darko Brankovic}#
\end{upotreba}
\end{miditest}
\begin{miditest}
\begin{test2}{1}
  
  
#\naslovInt#
#\izlaz{Unesite prezime studenta }#
#\izlaz{cije informacije zelite:}#
#\ulaz{Popovic}#
#\izlaz{Indeks: 20140032,}# 
#\izlaz{Ime i prezime: Dejan Popovic}#
\end{test2}
\end{miditest}

\linkresenje{3_03}

\end{Exercise}

\begin{Answer}[ref=3_03]
  \includecode{resenja/3_AlgoritmiPretrageISortiranja/3_03.c}
\end{Answer}
%-----------------------------------------------------------------
%-----------------------------------------------------------------
\begin{Exercise}[label=3_04]
  Modifikovati zadatak \ref{3_03} tako da tražene funkcije
  budu rekurzivne.

\linkresenje{3_04}

\end{Exercise}

\begin{Answer}[ref=3_04]
  \includecode{resenja/3_AlgoritmiPretrageISortiranja/3_04.c}
\end{Answer}
%-----------------------------------------------------------------
%-----------------------------------------------------------------
\begin{Exercise}[label=3_05]
  U datoteci koja se zadaje kao prvi argument komandne linije, nalaze
  se koordinate tačaka. U zavisnosti od prisustva opcija komandne
  linije (\kckod{-x} ili \kckod{-y}), pronaći onu koja je najbliža
  $x$, ili $y$ osi, ili koordinatnom početku, ako nije
  prisutna nijedna opcija. Pretpostaviti da je broj tačaka u datateci
  veći od $0$ i ne veći od $1024$.
  
\begin{minitest}
\begin{test}{1}
#\poziv{./a.out dat.txt -x}#

#\naslovDat{dat.txt}#
#\datoteka{12 53}#
#\datoteka{2.342 34.1}#
#\datoteka{-0.3 23}#
#\datoteka{-1 23.1}#
#\datoteka{123.5 756.12}#

#\naslovIzlaz#
#\izlaz{-0.3 23}#
\end{test}
\end{minitest}
\begin{minitest}
\begin{test}{2}
#\poziv{./a.out dat.txt}#
  
#\naslovDat{dat.txt}#
#\datoteka{12 53}#
#\datoteka{2.342 34.1}#
#\datoteka{-0.3 23}#
#\datoteka{-1 2.1}#
#\datoteka{123.5 756.12}#

#\naslovIzlaz#
#\izlaz{-1 2.1}#
\end{test}
\end{minitest}
\begin{minitest}
\begin{test}{3}.
#\poziv{./a.out dat.txt -y}#

#\naslovDat{dat.txt}#
#\datoteka{12 53}#
#\datoteka{2.342 34.1}#
#\datoteka{-0.3 0.23}#
#\datoteka{-1 2.1}#
#\datoteka{123.5 756.12}#
  
#\naslovIzlaz#
#\izlaz{-0.3 0.23}#
\end{test}
\end{minitest}

\linkresenje{3_05}

\end{Exercise}

\begin{Answer}[ref=3_05]
  \includecode{resenja/3_AlgoritmiPretrageISortiranja/3_05.c}
\end{Answer}
%-----------------------------------------------------------------
%-----------------------------------------------------------------
\begin{Exercise}[label=3_06]
  Napisati funkciju koja određuje nulu funkcije \kckod{cos(x)} na
  intervalu $[0,2]$ metodom polovljenja intervala. Algoritam se
  završava kada se vrednost kosinusne funkcije razlikuje za najviše
  $0.001$ od nule. \uputstvo{Korisiti algoritam analogan algoritmu
    binarne pretrage, metod polovljenja intervala.}  \napomena{Ovaj
    metod se može primeniti na funkciju \kckod{cos(x)} na intervalu
    $[0,2]$ zato što je ona na ovom intervalu neprekidna, i
    vrednosti funkcije na krajevima intervala su različitog znaka.}

% Ovde ne moze biti vise od jednog test primera
\begin{minitest}
\begin{test}{1}
#\naslovIzlaz#
#\izlaz{1.57031}#
\end{test}
\end{minitest}

\linkresenje{3_06}

\end{Exercise}

\begin{Answer}[ref=3_06]
  \includecode{resenja/3_AlgoritmiPretrageISortiranja/3_06.c}
\end{Answer}

%-----------------------------------------------------------------
%-----------------------------------------------------------------
\begin{Exercise}[label=3_07]
  Napisati funkciju koja metodom polovljenja intervala određuje nulu
  izabrane funkcije na proizvoljnom intervalu sa tačnošću
  $epsilon$. Ime funkcije se zadaje kao prvi agrument komandne
  linije, a interval i tačnost se unose sa standardnog
  ulaza. Pretpostaviti da je izabrana funkcija na tom intervalu
  neprekidna. \uputstvo{U okviru algoritma pretrage koristiti
    pokazivač na odgovarajuću funkciju (na primer, kao u zadatku
    \ref{2_27}).}

\begin{miditest}
\begin{upotreba}{1}
#\poziv{./a.out cos}#
  
#\naslovInt#
#\izlaz{Unesite krajeve intervala:} \ulaz{0 2}#
#\izlaz{Unesite preciznost:} \ulaz{0.001}#
#\izlaz{1.57031}#
\end{upotreba}
\end{miditest}
\begin{miditest}
\begin{upotreba}{2}
#\poziv{./a.out sin}#
  
#\naslovInt#
#\izlaz{Unesite krajeve intervala:} \ulaz{1 5}#
#\izlaz{Unesite preciznost:} \ulaz{0.00001}#
#\izlaz{3.1416}#
\end{upotreba}
\end{miditest}

\begin{miditest}
\begin{upotreba}{3}
#\poziv{./a.out tan}#
      
#\naslovInt#
#\izlaz{Unesite krajeve intervala:} \ulaz{-1.1 1}#
#\izlaz{Unesite preciznost:} \ulaz{0.00001}#
#\izlaz{3.8147e-06}#
\end{upotreba}
\end{miditest}
\begin{miditest}
\begin{upotreba}{4}
#\poziv{./a.out sin}#  

#\naslovInt#
#\izlaz{Unesite krajeve intervala:} \ulaz{1 3}#
#\izlaz{Funkcija sin na intervalu [1, 3]}#
#\izlaz{ne zadovoljava uslove}#
\end{upotreba}
\end{miditest}
  
\linkresenje{3_07}

\end{Exercise}

\begin{Answer}[ref=3_07]
  \includecode{resenja/3_AlgoritmiPretrageISortiranja/3_07.c}
\end{Answer}

%-----------------------------------------------------------------
%-----------------------------------------------------------------
\begin{Exercise}[label=3_08]
Napisati funkciju koja u rastuće sortiranom nizu celih brojeva
binarnom pretragom pronalazi indeks prvog elementa većeg od
nule. Ukoliko nema elemenata većih od nule, funkcija kao rezultat
vraća $-1$. Napisati program koji testira ovu funkciju za rastući
niz celih brojeva koji se učitavaju sa standardnog ulaza. Niz neće biti
duži od $256$, i njegovi elementi se unose sve do kraja ulaza.

\begin{minitest}
\begin{test}{1}
#\naslovUlaz#
#\ulaz{-151 -44 5 12 13 15}#

#\naslovIzlaz#
#\izlaz{2}#
\end{test}
\end{minitest}
\begin{minitest}
\begin{test}{2}
#\naslovUlaz#
#\ulaz{-100 -15 -11 -8 -7 -5}#

#\naslovIzlaz#
#\izlaz{-1}#
\end{test}
\end{minitest}
\begin{minitest}
\begin{test}{3}
#\naslovUlaz#
#\ulaz{-100 -15 0 13 55 124}#
#\ulaz{258 315 516 7000}#
  
#\naslovIzlaz#
#\izlaz{3}#
\end{test}
\end{minitest}

\linkresenje{3_08}

\end{Exercise}
\begin{Answer}[ref=3_08]
  \includecode{resenja/3_AlgoritmiPretrageISortiranja/3_08.c}
\end{Answer}
%-----------------------------------------------------------------
%-----------------------------------------------------------------
\begin{Exercise}[label=3_09]
Napisati funkciju koja u opadajuće sortiranom nizu celih brojeva
binarnom pretragom pronalazi indeks prvog elementa manjeg od
nule. Ukoliko nema elemenata manjih od nule, funkcija kao rezultat
vraća $-1$. Napisati program koji testira ovu funkciju za
opadajući niz celih brojeva koji se učitavaju sa standardnog ulaza. Niz
neće biti duži od $256$, i njegovi elementi se unose sve do kraja
ulaza.

\begin{minitest}
\begin{test}{1}
#\naslovUlaz#
#\ulaz{151 44 5 -12 -13 -15}#

#\naslovIzlaz#
#\izlaz{3}#
\end{test}
\end{minitest}
\begin{minitest}
\begin{test}{2}
#\naslovUlaz#
#\ulaz{100 55 15 0 -15 -124}#
#\ulaz{-155 -258 -315 -516}#

#\naslovIzlaz#
#\izlaz{4}#
\end{test}
\end{minitest}
\begin{minitest}
\begin{test}{3}
#\naslovUlaz#
#\ulaz{100 15 11 8 7 5 4 3 2}#

#\naslovIzlaz#
#\izlaz{-1}#
\end{test}
\end{minitest}

\linkresenje{3_09}

\end{Exercise}

\begin{Answer}[ref=3_09]
  \includecode{resenja/3_AlgoritmiPretrageISortiranja/3_09.c}
\end{Answer}
%-----------------------------------------------------------------
%-----------------------------------------------------------------
\begin{Exercise}[label=3_10]
  Napisati funkciju koja određuje ceo deo logaritma za osnovu 2 datog
  neoznačenog celog broja koristeći samo bitske i relacione
  operatore.
  \begin{enumerate}
  \item Napisati funkciju linearne složenosti koja određuje
    logaritam pomeranjem broja udesno.
  \item Napisati funkciju logaritmske složenosti koja određuje
    logaritam koristeći binarnu pretragu.
  \end{enumerate}
  Tražene funkcije testirati programom koji pozitivan broj učitava sa
  standardnog ulaza, a logaritam ispisuje na standardnom izlazu.

\begin{minitest}
\begin{test}{1}
#\naslovUlaz#
#\ulaz{4}#
  
#\naslovIzlaz#
#\izlaz{2 2}#
\end{test}
\end{minitest}
\begin{minitest}
\begin{test}{2}
#\naslovUlaz#
#\ulaz{17}#
  
#\naslovIzlaz#
#\izlaz{4 4}#
\end{test}
\end{minitest}
\begin{minitest}
\begin{test}{3}
#\naslovUlaz#
#\ulaz{1031}#
  
#\naslovIzlaz#
#\izlaz{10 10}#
\end{test}
\end{minitest}

\linkresenje{3_10}

\end{Exercise}

\begin{Answer}[ref=3_10]
  \includecode{resenja/3_AlgoritmiPretrageISortiranja/3_10.c}
\end{Answer}
%-----------------------------------------------------------------
%-----------------------------------------------------------------
\begin{Exercise}[difficulty=1, label=3_11]
  U prvom kvadrantu dato je $1 \leq N \leq 10000$ duži svojim
  koordinatama (duži mogu da se seku, preklapaju, itd.). Napisati
  program koji pronalazi najmanji ugao $0 \leq \alpha \leq 90^\circ$,
  na dve decimale, takav da je suma dužina duži sa obe strane
  polupoluprave iz koordinatnog početka pod uglom $\alpha$ jednak
  (neke duži bivaju presečene, a neke ne). Program prvo učitava broj
  $N$, a zatim i same koordinate temena duži. \uputstvo{Vršiti
  binarnu pretragu intervala $[0, 90^\circ]$.}
  
\begin{minitest}
\begin{upotreba}{1}
#\naslovInt#
#\izlaz{Unesi broj tacaka:}\ulaz{2}#
#\izlaz{Unesi koordinate tacaka:}#
#\ulaz{2 0 2 1}#
#\ulaz{1 2 2 2}#
#\izlaz{26.57}#
\end{upotreba}
\end{minitest}
\begin{minitest}
\begin{upotreba}{2}
#\naslovInt#
#\izlaz{Unesi broj tacaka:}\ulaz{2}#
#\izlaz{Unesi koordinate tacaka:}#
#\ulaz{1 0 1 1}#
#\ulaz{0 1 1 1}#
#\izlaz{45}#
\end{upotreba}
\end{minitest}
\begin{minitest}
\begin{upotreba}{3}
#\naslovInt#
#\izlaz{Unesi broj tacaka:}\ulaz{3}#
#\izlaz{Unesi koordinate tacaka:}#
#\ulaz{1 0 1 1}#
#\ulaz{2 0 2 1}#
#\ulaz{1 2 2 2}#
#\izlaz{26.57}#
\end{upotreba}
\end{minitest}

\end{Exercise}

%-----------------------------------------------------------------
%-----------------------------------------------------------------


\section{Algoritmi sortiranja}

%-----------------------------------------------------------------
%-----------------------------------------------------------------
\begin{Exercise}[label=3_12]
  Napraviti biblioteku koja implementira algoritme sortiranja nizova
  celih brojeva. Biblioteka treba da sadrži algoritam sortiranja
  izborom (engl. \kckod{selection sort}), sortiranja spajanjem
  (engl. \kckod{merge sort}), brzog sortiranja (engl. \kckod{quick
    sort}), mehurastog sortiranja (engl. \kckod{bubble sort}),
  sortiranja direktnim umetanjem (engl. \kckod{insertion sort}) i
  sortiranja umetanjem sa inkrementom (engl. \kckod{shell
    sort}). Upotrebiti biblioteku kako bi se napravilo poređenje
  efikasnosti različitih algoritama sortiranja. Efikasnost meriti na
  slučajno generisanim nizovima, na rastuće sortiranim nizovima i na
  opadajuće sortiranim nizovima. Izbor algoritma, veličine i početnog
  rasporeda elemenata niza birati kroz argumente komandne
  linije. Moguće opcije kojima se bira algoritam sortiranja su:
  \kckod{-m} za sortiranje spajanjem, \kckod{-q} za brzo sortiranje,
  \kckod{-b} za mehurasto, \kckod{-i} za sortiranje direktnim
  umetanjem ili \kckod{-s} za sortiranje umetanjem sa inkrementom. U
  slučaju da nije prisutna ni jedna od ovih opcija, niz sortirati
  algoritmom sortiranja izborom. Niz koji se sortira generisati
  neopadajuće ako je prisutna opcija \kckod{-r}, nerastuće ako je
  prisutna opcija \kckod{-o} ili potpuno slučajno ako nema nijedne
  opcije. Vreme meriti programom \kckod{time}. Analizirati porast
  vremena sa porastom dimenzije $n$.

\begin{minitest}
\begin{test}{1}
#\poziv{time ./a.out 200000}#
  
#\naslovIzlaz#
#\izlaz{real    0m42.168s}#
#\izlaz{user    0m42.100s}#
#\izlaz{sys     0m0.000s}#
\end{test}
\end{minitest}
\begin{minitest}
\begin{test}{2}
#\poziv{time ./a.out 400000}#
  
#\naslovIzlaz#
#\izlaz{real    2m48.395s}#
#\izlaz{user    2m48.128s}#
#\izlaz{sys     0m0.000s}#
\end{test}
\end{minitest}
\begin{minitest}
\begin{test}{3}
#\poziv{time ./a.out 800000}#
  
#\naslovIzlaz#
#\izlaz{real    11m13.703s}#
#\izlaz{user    11m12.636s}#
#\izlaz{sys     0m0.000s}#
\end{test}
\end{minitest}

\begin{minitest}
\begin{test}{4}
#\poziv{time ./a.out 800000 -r}#
  
#\naslovIzlaz#
#\izlaz{real    11m21.533s}#
#\izlaz{user    11m20.436s}#
#\izlaz{sys     0m0.020s}#
\end{test}
\end{minitest}
\begin{minitest}
\begin{test}{5}
#\poziv{time ./a.out 800000 -q}#
  
#\naslovIzlaz#
#\izlaz{real    0m0.159s}#
#\izlaz{user    0m0.156s}#
#\izlaz{sys     0m0.000s}#
\end{test}
\end{minitest}
\begin{minitest}
\begin{test}{6}
#\poziv{time ./a.out 800000 -m}#
  
#\naslovIzlaz#
#\izlaz{real    0m0.137s}#
#\izlaz{user    0m0.136s}#
#\izlaz{sys     0m0.000s}#
\end{test}
\end{minitest}

\linkresenje{3_12}
\end{Exercise}

\begin{Answer}[ref=3_12]
  \includecodeLib{resenja/biblioteke/sortiranje/sort.h}{sort.h}
  \includecodeLib{resenja/biblioteke/sortiranje/sort.c}{sort.c}
  \includecodeLib{resenja/3_AlgoritmiPretrageISortiranja/3_12.c}{main.c}
\end{Answer}
%-----------------------------------------------------------------
%-----------------------------------------------------------------
\begin{Exercise}[label=3_13]
  Dve niske su anagrami ako se sastoje od istog broja istih
  karaktera. Napisati program koji proverava da li su dve niske
  karaktera anagrami. Niske se zadaju sa standardnog ulaza i neće
  biti duže od $127$ karaktera.  \uputstvo{Napisati funkciju koja
  sortira slova unutar niske karaktera, a zatim za sortirane niske
  proveriti da li su identične.}
  
\begin{minitest}
\begin{upotreba}{1}
#\naslovInt#
#\izlaz{Unesite prvu nisku}\ulaz{anagram}#
#\izlaz{Unesite drugu nisku}\ulaz{ramgana}#
#\izlaz{jesu}#
\end{upotreba}
\end{minitest}
\begin{minitest}
\begin{upotreba}{2}
#\naslovInt#
#\izlaz{Unesite prvu nisku}\ulaz{anagram}#
#\izlaz{Unesite drugu nisku}\ulaz{anagrm}#
#\izlaz{nisu}#
\end{upotreba}
\end{minitest}
\begin{minitest}
\begin{upotreba}{3}
#\naslovInt#
#\izlaz{Unesite prvu nisku}\ulaz{test}#
#\izlaz{Unesite drugu nisku}\ulaz{tset}#
#\izlaz{jesu}#
\end{upotreba}
\end{minitest}
  
\linkresenje{3_13}
\end{Exercise}

\begin{Answer}[ref=3_13]
  \includecode{resenja/3_AlgoritmiPretrageISortiranja/3_13.c}
\end{Answer}
%-----------------------------------------------------------------
%-----------------------------------------------------------------
\begin{Exercise}[label=3_14]
  U datom nizu brojeva treba pronaći dva broja koja su na najmanjem
  rastojanju. Niz se zadaje sa standardnog ulaza, sve do kraja ulaza,
  ali neće sadržati više od $256$ i manje od $2$ elemenata. Na izlaz
  ispisati razliku pronađena dva broja. \uputstvo{Prvo sortirati niz.}
  \napomena{Koristiti biblioteku za sortiranje celih brojeva iz
    zadatka \ref{3_12}.}

\begin{minitest}
\begin{test}{1}
#\naslovUlaz#
#\ulaz{23 64 123 76 22 7}#

#\naslovIzlaz#
#\izlaz{1}#
\end{test}
\end{minitest}
\begin{minitest}
\begin{test}{2}
#\naslovUlaz#
#\ulaz{21 654 65 123 65 12 61}#
  
#\naslovIzlaz#
#\izlaz{0}#
\end{test}
\end{minitest}
\begin{minitest}
\begin{test}{3}
#\naslovUlaz#
#\ulaz{34 30}#

#\naslovIzlaz#
#\izlaz{4}#
\end{test}
\end{minitest}
  
\linkresenje{3_14}
\end{Exercise}

\begin{Answer}[ref=3_14]
  \\
  \napomena{Rešenje koristi biblioteku za sortiranje celih brojeva iz
    zadatka \ref{3_12}.}
  \includecode{resenja/3_AlgoritmiPretrageISortiranja/3_14.c}
\end{Answer}
%-----------------------------------------------------------------
%-----------------------------------------------------------------
\begin{Exercise}[label=3_15]
  Napisati program koji pronalazi broj koji se najviše puta
  pojavljivao u datom nizu. Niz se zadaje sa standardnog ulaza sve do
  kraja ulaza i neće biti duži od $256$ i kraći od jednog
  elemenata. \uputstvo{Prvo sortirati niz, a zatim naći najdužu
    sekvencu jednakih elemenata.}  \napomena{Koristiti biblioteku za
    sortiranje celih brojeva iz zadatka \ref{3_12}.}

\begin{minitest}
\begin{test}{1}
#\naslovUlaz#
#\ulaz{4 23 5 2 4 6 7 34 6 4 5}#
  
#\naslovIzlaz#
#\izlaz{4}#
\end{test}
\end{minitest}
\begin{minitest}
\begin{test}{2}
#\naslovUlaz#
#\ulaz{2 4 6 2 6 7 99 1}#
  
#\naslovIzlaz#
#\izlaz{2}#
\end{test}
\end{minitest}
\begin{minitest}
\begin{test}{3}
#\naslovUlaz#
#\ulaz{123}#
  
#\naslovIzlaz#
#\izlaz{123}#
\end{test}
\end{minitest}

\linkresenje{3_15}
\end{Exercise}

\begin{Answer}[ref=3_15]
  \\
  \napomena{Rešenje koristi biblioteku za sortiranje celih brojeva iz
    zadatka \ref{3_12}.}
  \includecode{resenja/3_AlgoritmiPretrageISortiranja/3_15.c}
\end{Answer}
%-----------------------------------------------------------------
%-----------------------------------------------------------------
\begin{Exercise}[label=3_16]
  Napisati funkciju koja proverava da li u datom nizu postoje dva
  elementa čiji zbir je jednak zadatom celom broju. Napisati i program
  koji testira ovu funkciju. U programu se prvo učitava broj, a zatim
  i niz.  Elementi niza se unose sve do kraja ulaza.  Pretpostaviti da
  u niz neće biti uneto više od $256$ brojeva.  \uputstvo{Prvo
    sortirati niz.}  \napomena{Koristiti biblioteku za sortiranje
    celih brojeva iz zadatka \ref{3_12}.}
  
\begin{minitest}
\begin{upotreba}{1}
#\naslovInt#
#\izlaz{Unesite trazeni zbir:}\ulaz{34}#
#\izlaz{Unesite elemente niza:}#
#\ulaz{134 4 1 6 30 23}#
#\izlaz{da}#
\end{upotreba}
\end{minitest}
\begin{minitest}
\begin{upotreba}{2}
#\naslovInt#
#\izlaz{Unesite trazeni zbir:}\ulaz{12}#
#\izlaz{Unesite elemente niza:}#
#\ulaz{53 1 43 3 56 13}#
#\izlaz{ne}#
\end{upotreba}
\end{minitest}
\begin{minitest}
\begin{upotreba}{3}
#\naslovInt#
#\izlaz{Unesite trazeni zbir:}\ulaz{52}#
#\izlaz{Unesite elemente niza:}#
#\ulaz{52}#
#\izlaz{ne}#
\end{upotreba}
\end{minitest}
  
\linkresenje{3_16}
\end{Exercise}

\begin{Answer}[ref=3_16]
  \\
  \napomena{Rešenje koristi biblioteku za sortiranje celih brojeva iz
    zadatka \ref{3_12}.}
  \includecode{resenja/3_AlgoritmiPretrageISortiranja/3_16.c}
\end{Answer}

%-----------------------------------------------------------------
%-----------------------------------------------------------------

\begin{Exercise}[label=3_17]
  Napisati funkciju potpisa \kckod{int merge(int *niz1, int dim1, int
    *niz2, int dim2, int *niz3, int dim3)} koja prima dva sortirana
  niza, i na osnovu njih pravi novi sortirani niz koji koji sadrži
  elemente oba niza. Treća dimenzija predstavlja veličinu niza u koji
  se smešta rezultat. Ako je ona manja od potrebne dužine, funkcija
  vraća -1 kao indikator neuspeha, inače vraća 0. Napisati zatim
  program koji testira ovu funkciju. Nizovi se unose sa standardnog
  ulaza sve dok se ne unese 0 i može se pretpostaviti da će njihove
  dimenzije biti manje od $256$.
  
\begin{miditest}
\begin{upotreba}{1}
#\naslovInt#  
#\izlaz{Unesite elemente prvog niza:}#
#\ulaz{3 6 7 11 14 35 0}#
#\izlaz{Unesite elemente drugog niza:}#
#\ulaz{3 5 8 0}#
#\izlaz{3 3 5 6 7 8 11 14 35}#
\end{upotreba}
\end{miditest}
\begin{miditest}
\begin{upotreba}{2}
#\naslovInt#   
#\izlaz{Unesite elemente prvog niza:}#
#\ulaz{1 4 7 0}#
#\izlaz{Unesite elemente drugog niza:}#
#\ulaz{9 11 23 54 75 0}#
#\izlaz{1 4 7 9 11 23 54 75}#
\end{upotreba}
\end{miditest}
  
\linkresenje{3_17}
\end{Exercise}

\begin{Answer}[ref=3_17]
  \includecode{resenja/3_AlgoritmiPretrageISortiranja/3_17.c}
\end{Answer}
%-----------------------------------------------------------------
%-----------------------------------------------------------------
\begin{Exercise}[label=3_18]
  Napisati program koji čita sadržaj dveju datoteka od kojih svaka
  sadrži spisak imena i prezimena studenata iz jedne od dve grupe,
  rastuće sortiran po imenima, a u slučaju istog imena po prezimenima,
  i kreira jedinstven spisak studenata sortiranih takođe po istom
  kriterijumu. Program dobija nazive datoteka iz komandne linije i
  jedinstveni spisak upisuje u datoteku
  \kckod{ceo-tok.txt}. Pretpostaviti da je ime studenta nije duže od
  $10$, a prezime od $15$ karaktera.


\begin{miditest}
\begin{test}{1}
#\poziv{./a.out prvi-deo.txt drugi-deo.txt}#
  
#\naslovDat{prvi-deo.txt}#
#\datoteka{Andrija Petrovic}#
#\datoteka{Anja Ilic}#
#\datoteka{Ivana Markovic}#
#\datoteka{Lazar Micic}#
#\datoteka{Nenad Brankovic}#
#\datoteka{Sofija Filipovic}#
#\datoteka{Uros Milic}#
#\datoteka{Vladimir Savic}#

#\naslovDat{drugi-deo.txt}#
#\datoteka{Aleksandra Cvetic}#
#\datoteka{Bojan Golubovic}#
#\datoteka{Dragan Markovic}#
#\datoteka{Filip Dukic}#
#\datoteka{Ivana Stankovic}#
#\datoteka{Marija Stankovic}#
#\datoteka{Ognjen Peric}#
\end{test}
\end{miditest}
\begin{miditest}
\begin{test2}{1}
  
  
  

#\naslovDat{ceo-tok.txt}#
#\datoteka{Aleksandra Cvetic}#
#\datoteka{Andrija Petrovic}#
#\datoteka{Anja Ilic}#
#\datoteka{Bojan Golubovic}#
#\datoteka{Dragan Markovic}#
#\datoteka{Filip Dukic}#
#\datoteka{Ivana Markovic}#
#\datoteka{Ivana Stankovic}#
#\datoteka{Lazar Micic}#
#\datoteka{Marija Stankovic}#
#\datoteka{Nenad Brankovic}#
#\datoteka{Ognjen Peric}#
#\datoteka{Sofija Filipovic}#
#\datoteka{Vladimir Savic}#
#\datoteka{Uros Milic}#
\end{test2}
\end{miditest}
  
\linkresenje{3_18}
\end{Exercise}

\begin{Answer}[ref=3_18]
  \includecode{resenja/3_AlgoritmiPretrageISortiranja/3_18.c}
\end{Answer}
%-----------------------------------------------------------------
%-----------------------------------------------------------------
\begin{Exercise}[label=3_19]
  Napisati funkcije koje sortiraju niz struktura tačaka na
  osnovu sledećih kriterijuma:
% milena: ovo je zbog preloma ovako
%  \begin{enumerate}
%  \item 
 (i) njihovog rastojanja od koordinatnog početka,
%  \item 
(ii)  \kckod{x} koordinata tačaka,
%  \item 
(iii)  \kckod{y} koordinata tačaka.
%  \end{enumerate}
  Napisati program koji učitava niz tačaka iz datoteke čije se ime
  zadaje kao drugi argument komandne linije, i u zavisnosti od
  prisutnih opcija (prvi argument) u komandnoj liniji (\kckod{-o},
  \kckod{-x} ili \kckod{-y}) sortira tačke po jednom od prethodna tri
  kriterijuma i rezultat upisuje u datoteku čije se ime zadaje kao
  treći argument komandne linije. U ulaznoj datoteci nije zadato više
  od 128 tačaka.
  
\begin{miditest}
\begin{test}{1}
#\poziv{./a.out -x in.txt out.txt}#
  
#\naslovDat{in.txt}#
#\datoteka{3 4}#
#\datoteka{11 6}#
#\datoteka{7 3}#
#\datoteka{2 82}#
#\datoteka{-1 6}#
  
#\naslovDat{out.txt}#
#\datoteka{-1 6}#
#\datoteka{2 82}#
#\datoteka{3 4}#
#\datoteka{7 3}#
#\datoteka{11 6}#
\end{test}
\end{miditest}
\begin{miditest}
\begin{test}{2}
#\poziv{./a.out -o in.txt out.txt}#
  
#\naslovDat{in.txt}#
#\datoteka{3 4}#
#\datoteka{11 6}#
#\datoteka{7 3}#
#\datoteka{2 82}#
#\datoteka{-1 6}#

#\naslovDat{out.txt}#
#\datoteka{3 4}#
#\datoteka{-1 6}#
#\datoteka{7 3}#
#\datoteka{11 6}#
#\datoteka{2 82}#
\end{test}
\end{miditest}
  
\linkresenje{3_19}
\end{Exercise}

\begin{Answer}[ref=3_19]
  \includecode{resenja/3_AlgoritmiPretrageISortiranja/3_19.c}
\end{Answer}
%-----------------------------------------------------------------
%-----------------------------------------------------------------
\begin{Exercise}[label=3_20]
  Napisati program koji učitava imena i prezimena građana (najviše
  njih $1000$) iz datoteke \kckod{biracki-spisak.txt} i kreira biračke
  spiskove. Jedan birački spisak je sortiran po imenu građana, a drugi
  po prezimenu. Program treba da ispisuje koliko građana ima isti
  redni broj u oba biračka spiska. Pretpostaviti da je za ime, odnosno
  prezime građana dovoljno $15$ karaktera, i da se nijedno ime i
  prezime ne pojavljuje više od jednom.

\begin{minitest}
\begin{test}{1}
#\naslovDat{biracki-spisak.txt}#    
#\datoteka{Bojan Golubovic}#
#\datoteka{Andrija Petrovic}#
#\datoteka{Anja Ilic}#
#\datoteka{Aleksandra Cvetic}#
#\datoteka{Dragan Markovic}#
#\datoteka{Ivana Markovic}#
#\datoteka{Lazar Micic}#
#\datoteka{Marija Stankovic}#
#\datoteka{Filip Dukic}#

#\naslovIzlaz#
#\izlaz{3}#
\end{test}
\end{minitest}
\begin{minitest}
\begin{test}{2}
#\naslovDat{biracki-spisak.txt}#
#\datoteka{Milan Milicevic}#
  
#\naslovIzlaz#
#\izlaz{1}#
\end{test}
\end{minitest}
\begin{minitest}
\begin{test}{3}
#\naslovDat{Datoteka biracki-spisak.txt}#
#\naslovDat{ne postoji}#
  
#\naslovIzlazZaGresku#
#\izlaz{Neupesno otvaranje}#
#\izlaz{datoteke}#
#\izlaz{biracki-spisak.txt.}#
\end{test}
\end{minitest}
  
\linkresenje{3_20}
\end{Exercise}

\begin{Answer}[ref=3_20]
  \includecode{resenja/3_AlgoritmiPretrageISortiranja/3_20.c}
\end{Answer}
%-----------------------------------------------------------------
%-----------------------------------------------------------------
\begin{Exercise}[label=3_21]
\iffalse
  Definisana je struktura podataka
\begin{ckod}
typedef struct dete
{
      char ime[MAX_IME];
      char prezime[MAX_IME];
      unsigned godiste;
} Dete;
\end{ckod} 
\fi
Definisati strukturu koja čuva imena, prezimena i godišta dece. Pretpostaviti da su imena i prezimena niske karaktera koje nisu duže od 30 karaktera.
Napisati funkciju koja sortira niz dece po godištu, a decu
istog godišta sortira leksikografski po prezimenu i
imenu. Napisati program koji učitava podatke o deci koji se nalaze u
datoteci čije se ime zadaje kao prvi argument komandne linije,
sortira ih i sortirani niz upisuje u datoteku čije se ime zadaje kao
drugi argument komandne linije. Pretpostaviti da u ulaznoj datoteci
nisu zadati podaci o više od $128$ dece.
  
\begin{miditest}
\begin{test}{1}
#\poziv{./a.out in.txt out.txt}#
  
#\naslovDat{in.out}#
#\datoteka{Petar Petrovic 2007}#
#\datoteka{Milica Antonic 2008}#
#\datoteka{Ana Petrovic 2007}#
#\datoteka{Ivana Ivanovic 2009}#
#\datoteka{Dragana Markovic 2010}#
#\datoteka{Marija Antic 2007}#
\end{test}
\end{miditest}
\begin{miditest}
\begin{test2}{1}




#\naslovDat{out.txt}#
#\datoteka{Marija Antic 2007}#
#\datoteka{Ana Petrovic 2007}#
#\datoteka{Petar Petrovic 2007}#
#\datoteka{Milica Antonic 2008}#
#\datoteka{Ivana Ivanovic 2009}#
#\datoteka{Dragana Markovic 2010}#
\end{test2}
\end{miditest}

\begin{miditest}
\begin{test}{2}
#\poziv{./a.out in.txt out.txt}#

#\naslovDat{in.out}#
#\datoteka{Milijana Maric 2009}#
\end{test}
\end{miditest}
\begin{miditest}
\begin{test2}{1}




#\naslovDat{out.txt}#
#\datoteka{Milijana Maric 2009}#
\end{test2}
\end{miditest}
  
%\linkresenje{3_21}
\end{Exercise}

%\begin{Answer}[ref=3_21]
%  \includecode{resenja/3_AlgoritmiPretrageISortiranja/3_21.c}
%\end{Answer}
%-----------------------------------------------------------------
%-----------------------------------------------------------------
\begin{Exercise}[label=3_22]
  Napisati funkciju koja sortira niz niski po broju suglasnika u
  niski. Ukoliko reči imaju isti broj suglasnika tada sortirati ih po
  dužini niske rastuće, a ukoliko su i dužine jednake onda
  leksikografski rastuće. Napisati program koji testira ovu funkciju za niske
  koje se zadaju u datoteci \kckod{niske.txt}.  Pretpostaviti da u
  nizu nema više od $128$ elemenata, kao i da svaka niska sadrži
  najviše $31$ karakter.
  
\begin{maxitest}
\begin{test}{1}
#\naslovDat{niske.txt}#
#\datoteka{ana petar andjela milos nikola aleksandar ljubica matej milica}#
  
#\naslovIzlaz#
#\izlaz{ana matej milos petar milica nikola andjela ljubica aleksandar}#
\end{test}
\end{maxitest}
  
\linkresenje{3_22}
\end{Exercise}

\begin{Answer}[ref=3_22]
  \includecode{resenja/3_AlgoritmiPretrageISortiranja/3_22.c}
\end{Answer}
%-----------------------------------------------------------------
%-----------------------------------------------------------------
\begin{Exercise}[label=3_23]
  Napisati program koji simulira rad kase u prodavnici. Kupci prilaze
  kasi, a prodavac unošenjem bar-koda kupljenog proizvoda dodaje
  njegovu cenu na ukupan račun. Na kraju, program ispisuje ukupnu
  vrednost svih proizvoda. Sve artikle, zajedno sa bar-kodovima,
  prozivođačima i cenama učitati iz datoteke
  \kckod{artikli.txt}. Pretraživanje niza artikala vršiti binarnom
  pretragom.
  
\begin{maxitest}
\begin{upotreba}{1}
#\naslovDat{artikli.txt}#
#\datoteka{1001 Keks Jaffa 120}#
#\datoteka{2530 Napolitanke Bambi 230}#
#\datoteka{0023 MedenoSrce Pionir 150}#
#\datoteka{2145 Pardon Marbo 70}#

#\naslovInt#
#\izlaz{Asortiman:}#
#\izlaz{KOD                Naziv artikla     Ime proizvodjaca       Cena}#
#\izlaz{        23           MedenoSrce               Pionir       150.00}#
#\izlaz{      1001                 Keks                Jaffa       120.00}#
#\izlaz{      2145               Pardon                Marbo        70.00}#
#\izlaz{      2530          Napolitanke                Bambi       230.00}#
#\izlaz{---------------------------}#
#\izlaz{- Za kraj za kraj rada kase, pritisnite CTRL+D!}#
#\izlaz{- Za nov racun unesite kod artikla!}#
#\izlaz{}#
#\ulaz{1001}#
#\izlaz{	Trazili ste:	Keks Jaffa       120.00}#
#\izlaz{Unesite kod artikla [ili 0 za prekid]:}\ulaz{23}#
#\izlaz{	Trazili ste:	MedenoSrce Pionir       150.00}#
#\izlaz{Unesite kod artikla [ili 0 za prekid]:}\ulaz{0}#
#\izlaz{}#
#\izlaz{	UKUPNO: 270.00 dinara.}#
#\izlaz{}#
#\izlaz{---------------------------}#
#\izlaz{- Za kraj za kraj rada kase, pritisnite CTRL+D!}#
#\izlaz{- Za nov racun unesite kod artikla!}#
#\izlaz{}#
#\ulaz{232}#
#\izlaz{	GRESKA: Ne postoji proizvod sa trazenim kodom!}#
#\izlaz{Unesite kod artikla [ili 0 za prekid]:}\ulaz{2530}#
#\izlaz{	Trazili ste:	Napolitanke Bambi       230.00}#
#\izlaz{Unesite kod artikla [ili 0 za prekid]:}\ulaz{0}#
#\izlaz{}#
#\izlaz{	UKUPNO: 230.00 dinara.}#
#\izlaz{}#
#\izlaz{---------------------------}#
#\izlaz{- Za kraj za kraj rada kase, pritisnite CTRL+D!}#
#\izlaz{- Za nov racun unesite kod artikla!}#
#\izlaz{}#
#\izlaz{Kraj rada kase!}#
\end{upotreba}
\end{maxitest}
  
\linkresenje{3_23}
\end{Exercise}

\begin{Answer}[ref=3_23]
  \includecode{resenja/3_AlgoritmiPretrageISortiranja/3_23.c}
\end{Answer}
%-----------------------------------------------------------------
%-----------------------------------------------------------------
\begin{Exercise}[label=3_24]
   Napisati program koji iz datoteke \kckod{aktivnost.txt} čita
   podatke o aktivnostima studenata na praktikumima i u datoteke
   \kckod{dat1.txt}, \kckod{dat2.txt} i \kckod{dat3.txt} upisuje redom
   tri spiska. Na prvom su studenti sortirani leksikografski po imenu
   rastuće. Na drugom su sortirani po ukupnom broju urađenih zadataka
   opadajuće, a ukoliko neki studenti imaju isti broj rešenih zadataka
   sortiraju se po dužini imena rastuće. Na trećem spisku kriterijum
   sortiranja je broj časova na kojima su bili opadajuće. Ukoliko neki
   studenti imaju isti broj časova, sortirati ih opadajuće po broju
   urađenih zadataka, a ukoliko se i on poklapa sortirati po prezimenu
   opadajuće. U datoteci se nalazi ime, prezime studenta, broj časova
   na kojima je prisustvovao, kao i ukupan broj urađenih
   zadataka. Pretpostaviti da studenata neće biti više od $500$ i da
   je za ime studenta dovoljno $20$, a za prezime $25$ karaktera.
  
\begin{miditest}
\begin{test}{1}
#\naslovDat{aktivnosti.txt}#
#\datoteka{Aleksandra Cvetic 4 6}#
#\datoteka{Bojan Golubovic 4 3}#
#\datoteka{Dragan Markovic 3 5}#
#\datoteka{Ivana Stankovic 3 1}#
#\datoteka{Marija Stankovic 1 3}#
#\datoteka{Ognjen Peric 1 2}#
#\datoteka{Uros Milic 2 5}#
#\datoteka{Andrija Petrovic 2 5}#
#\datoteka{Anja Ilic 3 1}#
#\datoteka{Lazar Micic 1 3}#
#\datoteka{Nenad Brankovic 2 4}#

#\naslovDat{dat1.txt}#
#\datoteka{Studenti sortirani po imenu}#
#\datoteka{leksikografski rastuce:}#
#\datoteka{Aleksandra Cvetic  4  6}#
#\datoteka{Andrija Petrovic  2  5}#
#\datoteka{Anja Ilic  3  1}#
#\datoteka{Bojan Golubovic  4  3}#
#\datoteka{Dragan Markovic  3  5}#
#\datoteka{Ivana Stankovic  3  1}#
#\datoteka{Lazar Micic  1  3}#
#\datoteka{Marija Stankovic  1  3}#
#\datoteka{Nenad Brankovic  2  4}#
#\datoteka{Ognjen Peric  1  2}#
#\datoteka{Uros Milic  2  5}#
\end{test}
\end{miditest}
\begin{miditest}
\begin{test2}{1}

    
#\naslovDat{dat2.txt}#
#\datoteka{Studenti sortirani po broju zadataka}#
#\datoteka{opadajuce, pa po duzini imena rastuce:}#
#\datoteka{Aleksandra Cvetic  4  6}#
#\datoteka{Uros Milic  2  5}#
#\datoteka{Dragan Markovic  3  5}#
#\datoteka{Andrija Petrovic  2  5}#
#\datoteka{Nenad Brankovic  2  4}#
#\datoteka{Lazar Micic  1  3}#
#\datoteka{Bojan Golubovic  4  3}#
#\datoteka{Marija Stankovic  1  3}#
#\datoteka{Ognjen Peric  1  2}#
#\datoteka{Anja Ilic  3  1}#
#\datoteka{Ivana Stankovic  3  1}#
                                                
#\naslovDat{dat3.txt}#
#\datoteka{Studenti sortirani po prisustvu}#
#\datoteka{opadajuce, pa po broju zadataka,}#
#\datoteka{pa po prezimenima leksikografski}#
#\datoteka{opadajuce:}#
#\datoteka{Aleksandra Cvetic  4  6}#
#\datoteka{Bojan Golubovic  4  3}#
#\datoteka{Dragan Markovic  3  5}#
#\datoteka{Ivana Stankovic  3  1}#
#\datoteka{Anja Ilic  3  1}#
#\datoteka{Andrija Petrovic  2  5}#
#\datoteka{Uros Milic  2  5}#
#\datoteka{Nenad Brankovic  2  4}#
#\datoteka{Marija Stankovic  1  3}#
#\datoteka{Lazar Micic  1  3}#
#\datoteka{Ognjen Peric  1  2}#
\end{test2}
\end{miditest}
  
\linkresenje{3_24}
\end{Exercise}

\begin{Answer}[ref=3_24]
  \includecode{resenja/3_AlgoritmiPretrageISortiranja/3_24.c}
\end{Answer}
%-----------------------------------------------------------------
%-----------------------------------------------------------------
\begin{Exercise}[label=3_25]
U datoteci \kckod{pesme.txt} nalaze se informacije o gledanosti pesama na
Youtube-u. Format datoteke sa informacijama je sledeći:
\begin{itemize}
\item U prvoj liniji datoteke se nalazi ukupan broj pesama prisutnih u
  datoteci.
\item Svaki naredni red datoteke sadrži informacije o gledanosti
  pesama u formatu \kckod{izvođač - naslov, broj gledanja}.
\end{itemize}
Napisati program koji učitava informacije o pesmama i vrši sortiranje
pesama u zavisnosti od argumenata komandne linije na sledeći način:
\begin{itemize}
\item nema opcija, sortiranje se vrši po broju gledanja;
\item prisutna je opcija \kckod{-i}, sortiranje se vrši po imenima
  izvođača;
\item prisutna je opcija \kckod{-n}, sortiranje se vrši po naslovu
  pesama.
\end{itemize}
Na standardnom izlazu ispisati informacije o pesmama sortiranim na opisani
način. Uraditi zadatak bez pravljenja pretpostavki o maksimalnoj dužini
  imena izvođača i naslova pesme.

\begin{minitest}
\begin{test}{1}
#\poziv{./a.out}#
  
#\naslovDat{pesme.txt}#
#\datoteka{5}#
#\datoteka{Ana - Nebo, 2342}#
#\datoteka{Laza - Oblaci, 29}#
#\datoteka{Pera - Ptice, 327}#
#\datoteka{Jelena - Sunce, 92321}#
#\datoteka{Mika - Kisa, 5341}#

#\naslovIzlaz#
#\izlaz{Jelena - Sunce, 92321}#
#\izlaz{Mika - Kisa, 5341}#
#\izlaz{Ana - Nebo, 2342}#
#\izlaz{Pera - Ptice, 327}#
#\izlaz{Laza - Oblaci, 29}#
\end{test}
\end{minitest}
\begin{minitest}
\begin{test}{2}
#\poziv{./a.out -i}#
  
#\naslovDat{pesme.txt}#
#\datoteka{5}#
#\datoteka{Ana - Nebo, 2342}#
#\datoteka{Laza - Oblaci, 29}#
#\datoteka{Pera - Ptice, 327}#
#\datoteka{Jelena - Sunce, 92321}#
#\datoteka{Mika - Kisa, 5341}#

#\naslovIzlaz#
#\izlaz{Ana - Nebo, 2342}#
#\izlaz{Jelena - Sunce, 92321}#
#\izlaz{Laza - Oblaci, 29}#
#\izlaz{Mika - Kisa, 5341}#
#\izlaz{Pera - Ptice, 327}#
\end{test}
\end{minitest}
\begin{minitest}
\begin{test}{3}
#\poziv{./a.out -n}#
  
#\naslovDat{pesme.txt}#
#\datoteka{5}#
#\datoteka{Ana - Nebo, 2342}#
#\datoteka{Laza - Oblaci, 29}#
#\datoteka{Pera - Ptice, 327}#
#\datoteka{Jelena - Sunce, 92321}#
#\datoteka{Mika - Kisa, 5341}#

#\naslovIzlaz#
#\izlaz{Mika - Kisa, 5341}#
#\izlaz{Ana - Nebo, 2342}#
#\izlaz{Laza - Oblaci, 29}#
#\izlaz{Pera - Ptice, 327}#
#\izlaz{Jelena - Sunce, 92321}#
\end{test}
\end{minitest}

\linkresenje{3_25}
\end{Exercise}

\begin{Answer}[ref=3_25]
\includecode{resenja/3_AlgoritmiPretrageISortiranja/3_25.c}
\end{Answer}
%-----------------------------------------------------------------
%-----------------------------------------------------------------
\begin{Exercise}[difficulty=1, label=3_26]
  Razmatrajmo dve operacije: operacija \kckod{U} je unos novog broja
  \kckod{x}, a operacija \kckod{N} određivanje $n$-tog po veličini od
  unetih brojeva. Implementirati program koji izvršava ove
  operacije. Može postojati najviše $100000$ operacija unosa, a uneti
  elementi se mogu ponavljati, pri čemu se i ponavljanja računaju
  prilikom brojanja. Optimizovati program, ukoliko se zna da neće biti
  više od $500$ različitih unetih brojeva. \napomena{Brojeve čuvati u
    sortiranom nizu i svaki naredni element umetati na svoje mesto.}
  
\begin{maxitest}
\begin{upotreba}{1}
#\naslovInt#
#\izlaz{Unesi niz operacija:}\ulaz{U 2 U 0 U 6 U 4 N 1 U 8 N 2 N 5 U 2 N 3 N 5}#
#\izlaz{0 2 8 2 6}
\end{upotreba}
\end{maxitest}
  
%\linkresenje{3_26}
\end{Exercise}

%\begin{Answer}[ref=3_26]
%  \includecode{resenja/3_AlgoritmiPretrageISortiranja/3_26.c}
%\end{Answer}
%-----------------------------------------------------------------
%-----------------------------------------------------------------
\begin{Exercise}[difficulty=1, label=3_27]
  Šef u restoranu je neuredan i palačinke koje ispeče ne slaže redom
  po veličini. Konobar pre serviranja mora da sortira palačinke po
  veličini, a jedina operacija koju sme da izvodi je da obrne deo
  palačinki. Na primer, sledeća slika po kolonama predstavlja
  naslagane palačinke posle svakog okretanja. Na početku, palačinka
  veličine $2$ je na dnu, iznad nje se redom nalaze najmanja, najveća,
  itd... Na slici crtica predstavlja mesto iznad koga će konobar
  okrenuti palačinke. Prvi potez konobara je okretanje palačinki
  veličine $5$, $4$ i $3$ (prva kolona), i tada će veličine palačinki
  odozdo nagore biti $2$, $1$, $3$, $4$, $5$ (druga kolona). Posle još
  dva okretanja, palačinke će biti složene.
\begin{ckod}
    3    5    2    1
    4    4    1__  2
    5__  3    3    3
    1    1    4    4
    2    2__  5    5
\end{ckod}
Napisati program koji u najviše $2n-3$ okretanja sortira učitani
niz. \uputstvo{Imitirati \kckod{selection sort} i u svakom koraku
  dovesti jednu palačinku na svoje mesto korišćenjem najviše dva
  okretanja.}

\begin{maxitest}
\begin{test}{1}
#\naslovUlaz#
#\ulaz{23 64 123 76 22 7 34 123 54562 12 453 342 5342 42 542 1 3 432 1 32 43}#

#\naslovIzlaz#
#\izlaz{1 1 3 7 12 22 23 32 34 42 43 64 76 123 123 342 432 453 542 5342 54562}#
\end{test}
\end{maxitest}

%\linkresenje{3_27}
\end{Exercise}

%\begin{Answer}[ref=3_27]
%  \includecode{resenja/3_AlgoritmiPretrageISortiranja/3_27.c}
%\end{Answer}
%-----------------------------------------------------------------
%-----------------------------------------------------------------
\begin{Exercise}[label=3_28]
  Za zadatu celobrojnu matricu dimenzije $n \times m$ napisati
  funkciju koja vrši sortiranje vrsta matrice rastuće na osnovu sume
  elemenata u vrsti. Napisati potom program koji testira ovu
  funkciju. Sa standardnog ulaza se prvo unose dimenzije matrice, a
  zatim redom elementi matrice. Rezultujuću matricu ispisati na
  standardnom izlazu.  \napomena{Koristiti biblioteku za rad sa
    celobrojnim matricama iz zadatka \ref{2_19}.}

\begin{miditest}
\begin{test}{1}
#\naslovInt#
#\izlaz{Unesite dimenzije matrice:}\ulaz{3 2}#
#\izlaz{Unesite elemente matrice po vrstama:}#
#\ulaz{6 -5}#
#\ulaz{-4 3}#
#\ulaz{2 1}#
#\izlaz{Sortirana matrica je:}#
#\izlaz{-4 3}#
#\izlaz{6 -5}# 
#\izlaz{2 1}#  
\end{test}
\end{miditest}
\begin{miditest}
\begin{test}{2}
#\naslovInt#
#\izlaz{Unesite dimenzije matrice:}\ulaz{4 4}#
#\izlaz{Unesite elemente matrice po vrstama:}#
#\ulaz{34 12 54 642}#
#\ulaz{1 2 3 4}#
#\ulaz{53 2 1 5}#
#\ulaz{54 23 5 671}#
#\izlaz{Sortirana matrica je:}#
#\izlaz{1 2 3 4}#
#\izlaz{53 2 1 5}#
#\izlaz{34 12 54 642}#
#\izlaz{54 23 5 671}#
\end{test}
\end{miditest}

\linkresenje{3_28}
\end{Exercise}

\begin{Answer}[ref=3_28]
  \\
  \napomena{Rešenje koristi biblioteku za rad sa celobrojnim matricama
    iz zadatka \ref{2_19}.}
  \includecode{resenja/3_AlgoritmiPretrageISortiranja/3_28.c}
\end{Answer}
%-----------------------------------------------------------------
%-----------------------------------------------------------------
\begin{Exercise}[label=3_29]
  Za zadatu kvadratnu matricu dimenzije $n$ napisati funkciju koja
  sortira kolone matrice opadajuće na osnovu vrednosti prvog elementa
  u koloni.  Napisati program koji testira ovu funkciju. Sa
  standardnog ulaza se prvo unosi dimenzija matrice, a zatim redom
  elementi matrice.  Rezultujuću matricu ispisati na standardnom
  izlazu.  \napomena{Koristiti biblioteku za rad sa celobrojnim matricama iz
    zadatka \ref{2_19}.}


\begin{miditest}
\begin{upotreba}{1}
#\naslovInt#  
#\izlaz{Unesite dimenziju matrice:}\ulaz{2}#
#\izlaz{Unesite elemente matrice po vrstama:}#
#\ulaz{6 -5}#
#\ulaz{-4 3}#
#\izlaz{Sortirana matrica je:}#
#\izlaz{-5 6}#
#\izlaz{3 -4}#
\end{upotreba}
\end{miditest}
\begin{miditest}
\begin{upotreba}{2}
#\naslovInt#  
#\izlaz{Unesite dimenziju matrice:}\ulaz{4}#
#\izlaz{Unesite elemente matrice po vrstama:}#
#\ulaz{34 12 54 642}#
#\ulaz{1 2 3 4}#
#\ulaz{53 2 1 5}#
#\ulaz{54 23 5 671}#
#\izlaz{Sortirana matrica je:}#
#\izlaz{12 34 54 642}#
#\izlaz{2 1 3 4}#
#\izlaz{2 53 1 5}#
#\izlaz{23 54 5 671}#
\end{upotreba}
\end{miditest}
  
%\linkresenje{3_29}
\end{Exercise}

%\begin{Answer}[ref=3_29]
%  \includecode{resenja/3_AlgoritmiPretrageISortiranja/3_29.c}
%\end{Answer}
%-----------------------------------------------------------------
%-----------------------------------------------------------------
\section{Bibliotečke funkcije pretrage i sortiranja}

%-----------------------------------------------------------------
\begin{Exercise}[label=3_30]
  Napisati program u kome se prvo inicijalizuje statički niz struktura
  osoba sa članovima ime i prezime, a zatim se učitava jedan
  karakter i pronalazi i štampa jedna struktura iz niza osoba čije prezime počinje tim
  karakterom. Ako takva osoba ne postoji, štampati $-1$ na standardnom
  izlazu za greške.
  Niz struktura ima manje od $100$ elemenata i uređen je u rastućem leksikografskom poretku po prezimenima.
  Pretaživanje niza vršiti bibliotečkom funkcijom \kckod{bsearch}.
Na primer, niz osoba može da bude inicijalizovan na sledeći način:
\begin{ckod}
Osoba niz_osoba[]={{"Mika", "Antic"},
                   {"Dobrica", "Eric"},
                   {"Desanka", "Maksimovic"},
                   {"Dusko", "Radovic"},
                   {"Ljubivoje", "Rsumovic"}};
\end{ckod}
  
  
\begin{minitest}
\begin{test}{1}
#\naslovUlaz#
#\ulaz{R}#
  
#\naslovIzlaz#
#\izlaz{Dusko Radovic}#
\end{test}
\end{minitest}
\begin{minitest}
\begin{test}{2}
#\naslovUlaz#
#\ulaz{E}#
  
#\naslovIzlaz#
#\izlaz{Dobrica Eric}#
\end{test}
\end{minitest}
\begin{minitest}
\begin{test}{3}
#\naslovUlaz#
#\ulaz{X}#
  
#\naslovIzlaz#
#\izlaz{-1}#
\end{test}
\end{minitest}

%\linkresenje{3_30}

\end{Exercise}

%\begin{Answer}[ref=3_30]
%  \includecode{resenja/04_Pretrazivanje/3_30.c}
%\end{Answer}
%-----------------------------------------------------------------



\begin{Exercise}[label=3_31]
  Napisati program koji ilustruje upotrebu bibliotečkih funkcija za
  pretraživanje i sortiranje nizova i mogućnost zadavanja različitih
  kriterijuma sortiranja. Sa standardnog ulaza se unosi dimenzija niza
  celih brojeva, ne veća od $100$, a potom i sami elementi
  niza. Upotrebom funkcije \kckod{qsort} sortirati niz u rastućem
  poretku, sa standardnog ulaza učitati broj koji se traži u nizu, pa
  zatim funkcijama \kckod{bsearch} i \kckod{lfind} utvrditi da li se
  zadati broj nalazi u nizu. Na standardnom izlazu ispisati
  odgovarajuću poruku.
  
\begin{miditest}
\begin{upotreba}{1}
#\naslovInt#
#\izlaz{Uneti dimenziju niza:}\ulaz{11}#
#\izlaz{Uneti elemente niza:}#
#\ulaz{5 3 1 6 8 90 34 5 3 432 34}#
#\izlaz{Sortirani niz u rastucem poretku:}#
#\izlaz{1 3 3 5 5 6 8 34 34 90 432 }#
#\izlaz{Uneti element koji se trazi u nizu:}\ulaz{34}#
#\izlaz{Binarna pretraga: }#
#\izlaz{Element je nadjen na poziciji 8}#
#\izlaz{Linearna pretraga (lfind): }#
#\izlaz{Element je nadjen na poziciji 7}#
\end{upotreba}
\end{miditest}
\begin{miditest}
\begin{upotreba}{2}
#\naslovInt#
#\izlaz{Uneti dimenziju niza:}\ulaz{4}#
#\izlaz{Uneti elemente niza:}#
#\ulaz{4 2 5 7}#
#\izlaz{Sortirani niz u rastucem poretku:}#
#\izlaz{2 4 5 7 }#
#\izlaz{Uneti element koji se trazi u nizu:}\ulaz{3}#
#\izlaz{Binarna pretraga: }#
#\izlaz{Elementa nema u nizu!}#
#\izlaz{Linearna pretraga (lfind): }#
#\izlaz{Elementa nema u nizu!}#
\end{upotreba}
\end{miditest}
  
\linkresenje{3_31}
\end{Exercise}

\begin{Answer}[ref=3_31]
  \includecode{resenja/3_AlgoritmiPretrageISortiranja/3_31.c}
\end{Answer}
%-----------------------------------------------------------------
%-----------------------------------------------------------------
\begin{Exercise}[label=3_32]
  Napisati program koji sa standardnog ulaza učitava dimenziju niza
  celih brojeva (ne veću od 100), a potom i same elemente
  niza. Upotrebom funkcije \kckod{qsort} sortirati niz u rastućem
  poretku prema broju delilaca i tako dobijeni niz odštampati na
  standardnom izlazu.
  
\begin{minitest}
\begin{upotreba}{1}
#\naslovInt#
#\izlaz{Uneti dimenziju niza:}\ulaz{10}#
#\izlaz{Uneti elemente niza:}#
#\ulaz{1 2 3 4 5 6 7 8 9 10}#
#\izlaz{Sortirani niz u rastucem}# 
#\izlaz{poretku prema broju delilaca:}# 
#\izlaz{1 2 3 5 7 4 9 6 8 10}#
\end{upotreba}
\end{minitest}
\begin{minitest}
\begin{upotreba}{2}
#\naslovInt#
#\izlaz{Uneti dimenziju niza:}\ulaz{1}#
#\izlaz{Uneti elemente niza:}#
#\ulaz{234}#
#\izlaz{Sortirani niz u rastucem}# 
#\izlaz{poretku prema broju delilaca:}# 
#\izlaz{234}#
\end{upotreba}
\end{minitest}
\begin{minitest}
\begin{upotreba}{3}
#\naslovInt#
#\izlaz{Uneti dimenziju niza:}\ulaz{0}#
#\izlaz{Uneti elemente niza:}#
#\izlaz{Sortirani niz u rastucem}# 
#\izlaz{poretku prema broju}#
#\izlaz{delilaca:}# 
#\izlaz{}#
\end{upotreba}
\end{minitest}
  
\linkresenje{3_32}
\end{Exercise}

\begin{Answer}[ref=3_32]
  \includecode{resenja/3_AlgoritmiPretrageISortiranja/3_32.c}
\end{Answer}
%-----------------------------------------------------------------
%-----------------------------------------------------------------
\begin{Exercise}[label=3_33]
   Korišćenjem bibliotečke funkcije \kckod{qsort} napisati program
   koji sortira niz niski po sledećim kriterijumima:
   \begin{enumerate}
   \item leksikografski,
   \item po dužini.
   \end{enumerate}
   Niske se učitavaju iz datoteke \kckod{niske.txt}. Pretpostaviti da datoteka ne sadrži više od 
   $1000$ niski kao i da je svaka niska dužine najviše $30$ karaktera. Program prvo
   leksikografski sortira niz, primenjuje binarnu pretragu
   (\kckod{bsearch}) zarad traženja niske unete sa standardnog ulaza,
   a potom traži istu nisku koristeći funkciju \kckod{lfind} u nizu koji je neposredno pre toga
   sortiran po dužini. Rezultate svih sortiranja
   i pretraga ispisati na standardnom izlazu.

   
\begin{maxitest}
\begin{upotreba}{1}
#\naslovDat{niske.txt}#
#\datoteka{ana petar andjela milos nikola aleksandar ljubica matej milica}#
  
#\naslovInt#
#\izlaz{Leksikografski sortirane niske:}#
#\izlaz{aleksandar ana andjela ljubica matej milica milos nikola petar }#
#\izlaz{Uneti trazenu nisku:}\ulaz{matej}#
#\izlaz{Niska "matej" je pronadjena u nizu na poziciji 4}#
#\izlaz{Niske sortirane po duzini:}#
#\izlaz{ana matej milos petar milica nikola andjela ljubica aleksandar}#
#\izlaz{Niska "matej" je pronadjena u nizu na poziciji 1}#
\end{upotreba}
\end{maxitest}
  
\linkresenje{3_33}
\end{Exercise}

\begin{Answer}[ref=3_33]
  \includecode{resenja/3_AlgoritmiPretrageISortiranja/3_33.c}
\end{Answer}
%-----------------------------------------------------------------
%-----------------------------------------------------------------
\begin{Exercise}[label=3_34]
  Uraditi zadatak \ref{3_33} sa dinamički alociranim niskama
  i sortiranjem niza pokazivača, umesto niza niski.
  
\linkresenje{3_34}
\end{Exercise}

\begin{Answer}[ref=3_34]
  \includecode{resenja/3_AlgoritmiPretrageISortiranja/3_34.c}
\end{Answer}
%-----------------------------------------------------------------
%-----------------------------------------------------------------
\begin{Exercise}[label=3_35]
  Napisati program koji korišćenjem bibliotečke funkcije \kckod{qsort}
  sortira studente prema broju poena osvojenih na kolokvijumu. Ukoliko
  više studenata ima isti broj bodova, sortirati ih po prezimenu
  leksikografski rastuće. Korisnik potom unosi broj bodova i prikazuje
  mu se jedan od studenata sa tim brojem bodova ili poruka ukoliko
  nema takvog. Potom, sa standardnog ulaza, unosi se prezime traženog
  studenta i prikazuje se osoba sa tim prezimenom ili poruka da se
  nijedan student tako ne preziva. Za pretraživanje koristiti
  odgovarajuće bibliotečke funkcije. Podaci o studentima čitaju se iz
  datoteke čije se ime zadaje preko argumenata komandne linije. Za
  svakog studenta u datoteci postoje ime, prezime i bodovi osvojeni na
  kolokvijumu. Pretpostaviti da neće biti više od $500$ studenata i
  da je za ime i prezime svakog studenta dovoljno po $20$ karaktera.
  
\begin{miditest}
\begin{upotreba}{1}
#\poziv{./a.out kolokvijum.txt}#
  
#\naslovDat{Ulazna datoteka (kolokvijum.txt):}#
#\datoteka{Aleksandra Cvetic 15}#
#\datoteka{Bojan Golubovic 30}#
#\datoteka{Dragan Markovic 25}#
#\datoteka{Filip Dukic 20}#
#\datoteka{Ivana Stankovic 25}#
#\datoteka{Marija Stankovic 15}#
#\datoteka{Ognjen Peric 20}#
#\datoteka{Uros Milic 10}#
#\datoteka{Andrija Petrovic 0}#
#\datoteka{Anja Ilic 5}#
#\datoteka{Ivana Markovic 5}#
#\datoteka{Lazar Micic 20}#
#\datoteka{Nenad Brankovic 15}#
\end{upotreba}
\end{miditest}
\begin{miditest}
\begin{test2}{1}

    
#\naslovInt#
#\izlaz{Studenti sortirani po broju poena}#
#\izlaz{opadajuce, pa po prezimenu rastuce:}#
#\izlaz{Bojan Golubovic  30}#
#\izlaz{Dragan Markovic  25}#
#\izlaz{Ivana Stankovic  25}#
#\izlaz{Filip Dukic  20}#
#\izlaz{Lazar Micic  20}#
#\izlaz{Ognjen Peric  20}#
#\izlaz{Nenad Brankovic  15}#
#\izlaz{Aleksandra Cvetic  15}#
#\izlaz{Marija Stankovic  15}#
#\izlaz{Uros Milic  10}#
#\izlaz{Anja Ilic  5}# 
#\izlaz{Ivana Markovic  5}#
#\izlaz{Andrija Petrovic  0}#
#\izlaz{Unesite broj bodova:}\ulaz{20}#
#\izlaz{Pronadjen je student sa unetim}#
#\izlaz{brojem bodova: Filip Dukic 20}#
#\izlaz{Unesite prezime:}\ulaz{Markovic}#
#\izlaz{Pronadjen je student sa unetim}#
#\izlaz{prezimenom: Dragan Markovic 25}#
\end{test2}
\end{miditest}
  
\linkresenje{3_35}
\end{Exercise}

\begin{Answer}[ref=3_35]
  \includecode{resenja/3_AlgoritmiPretrageISortiranja/3_35.c}
\end{Answer}
%-----------------------------------------------------------------
%-----------------------------------------------------------------
\begin{Exercise}[label=3_36]
  Uraditi zadatak \ref{3_13}, ali korišćenjem bibliotečke \kckod{qsort}
  funkcije.
  
\linkresenje{3_36}
\end{Exercise}

\begin{Answer}[ref=3_36]
  \includecode{resenja/3_AlgoritmiPretrageISortiranja/3_36.c}
\end{Answer}
%-----------------------------------------------------------------
%-----------------------------------------------------------------
\begin{Exercise}[label=3_37]
  Napisati program koji sa standardnog ulaza učitava prvo ceo broj
  $n$ ($n \leq 10$), a zatim niz $S$ od $n$ niski.
  Maksimalna dužina svake niske je $31$ karakter. Sortirati niz $S$
  bibliotečkom funkcijom \kckod{qsort} i proveriti da li u njemu ima
  identičnih niski.
  
\begin{minitest}
\begin{upotreba}{1}
#\naslovInt#
#\izlaz{Unesite broj niski:}\ulaz{4}#
#\izlaz{Unesite niske:}#
#\ulaz{prog search sort search}#
#\izlaz{ima}#
\end{upotreba}
\end{minitest}
\begin{minitest}
\begin{upotreba}{2}
#\naslovInt#
#\izlaz{Unesite broj niski:}\ulaz{3}#
#\izlaz{Unesite niske:}#
#\ulaz{test kol ispit}#
#\izlaz{nema}#
\end{upotreba}
\end{minitest}
\begin{minitest}
\begin{upotreba}{3}
#\naslovInt#
#\izlaz{Unesite broj niski:}\ulaz{5}#
#\izlaz{Unesite niske:}#
#\ulaz{a ab abc abcd abcde}#
#\izlaz{nema}#
\end{upotreba}
\end{minitest}
  
\linkresenje{3_37}
\end{Exercise}

\begin{Answer}[ref=3_37]
  \includecode{resenja/3_AlgoritmiPretrageISortiranja/3_37.c}
\end{Answer}
%-----------------------------------------------------------------
%-----------------------------------------------------------------
\begin{Exercise}[label=3_38]
  Datoteka \kckod{studenti.txt} sadrži spisak studenata. Za svakog
  studenta poznat je nalog na Alas-u (oblika npr. \kckod{mr15125},
  \kckod{mm14001}), ime, prezime i broj poena. Ni ime, ni prezime neće
  biti duže od 20 karaktera. Napisati program koji korišćenjem funkcije \kckod{qsort} sortira
  studente po broju poena opadajuće, ukoliko je prisutna opcija \kckod{-p}, ili po nalogu, 
  ukoliko je prisutna opcija \kckod{-n}. Studenti se po nalogu
  sortiraju tako što se sortiraju na osnovu godine, zatim na osnovu
  smera, i na kraju na osnovu broja indeksa. Sortirane studente
  upisati u datoteku \kckod{izlaz.txt}. Ukoliko je u komandnoj liniji
  uz opciju \kckod{-n} naveden i nalog nekog studenta, funkcijom
  \kckod{bsearch} potražiti i prijaviti broj poena studenta sa tim
  nalogom.
  
\begin{miditest}
\begin{test}{1}
#\poziv{./a.out -n mm13321}#
  
#\naslovDat{studenti.txt}#
#\datoteka{mr14123 Marko Antic 20}#
#\datoteka{mm13321 Marija Radic 12}#
#\datoteka{ml13011 Ivana Mitrovic 19}#
#\datoteka{ml13066 Pera Simic 15}#
#\datoteka{mv14003 Jovan Jovanovic 17}#

#\naslovDat{izlaz.txt}#
#\datoteka{ml13011 Ivana Mitrovic 19}#
#\datoteka{ml13066 Pera Simic 15}#
#\datoteka{mm13321 Marija Radic 12}#
#\datoteka{mr14123 Marko Antic 20}#
#\datoteka{mv14003 Jovan Jovanovic 17}#
  
#\naslovIzlaz#
#\izlaz{mm13321 Marija Radic 12}#
\end{test}
\end{miditest}
\begin{miditest}
\begin{test}{2}.
#\poziv{/a.out -p}#
  
#\naslovDat{studenti.txt}#
#\datoteka{mr14123 Marko Antic 20}#
#\datoteka{mm13321 Marija Radic 12}#
#\datoteka{ml13011 Ivana Mitrovic 19}#
#\datoteka{ml13066 Pera Simic 15}#
#\datoteka{mv14003 Jovan Jovanovic 17}#
  
#\naslovDat{izlaz.txt}#
#\datoteka{mr14123 Marko Antic 20}#
#\datoteka{ml13011 Ivana Mitrovic 19}#
#\datoteka{mv14003 Jovan Jovanovic 17}#
#\datoteka{ml13066 Pera Simic 15}#
#\datoteka{mm13321 Marija Radic 12}#
\end{test}
\end{miditest}
  
\linkresenje{3_38}
\end{Exercise}

\begin{Answer}[ref=3_38]
  \includecode{resenja/3_AlgoritmiPretrageISortiranja/3_38.c}
\end{Answer}
%-----------------------------------------------------------------
%-----------------------------------------------------------------
\begin{Exercise}[label=3_39]
Definisati strukturu \kckod{Datum}. \iffalse
 Definisana je struktura:
  \begin{ckod}
    typedef struct { int dan; int mesec; int godina; } Datum;
  \end{ckod}\fi
  Napisati funkciju koja poredi dva datuma hronološki. Potom, napisati
  i program koji učitava datume iz datoteke koja se zadaje kao prvi
  argument komandne linije (ne više od $128$ datuma), sortira ih
  pozivajući funkciju \kckod{qsort} iz standardne biblioteke i
  pozivanjem funkcije \kckod{bsearch} iz standardne biblioteke
  proverava da li datumi učitani sa standardnog ulaza postoje među
  prethodno unetim datumima. Datumi se učitavaju sve do kraja ulaza.
  
\begin{miditest}
\begin{upotreba}{1}
#\poziv{./a.out datoteka.txt}#
  
#\naslovDat{datoteka.txt}#
#\datoteka{1.1.2013.}#
#\datoteka{13.12.2016.}#
#\datoteka{11.11.2011.}#
#\datoteka{3.5.2015.}#
#\datoteka{5.2.2009.}#
\end{upotreba}
\end{miditest}
\begin{miditest}
\begin{test2}{1}

    
#\naslovInt#
#\izlaz{Unesi sledeci datum:}\ulaz{13.12.2016.}#
#\izlaz{postoji}#
#\izlaz{Unesi sledeci datum:}\ulaz{10.5.2015.}#
#\izlaz{ne postoji}#
#\izlaz{Unesi sledeci datum:}\ulaz{5.2.2009.}#
#\izlaz{postoji}#
\end{test2}
\end{miditest}

%\linkresenje{3_39}
\end{Exercise}

%\begin{Answer}[ref=3_39]
%  \includecode{resenja/3_AlgoritmiPretrageISortiranja/3_39.c}
%\end{Answer}
%-----------------------------------------------------------------
%-----------------------------------------------------------------

\section{Rešenja}
\shipoutAnswer


