
\chapter{Ispitni rokovi}

\section{Zadaci}

\subsection*{Programiranje 2, praktični deo ispita, jun 2015.}


\begin{Exercise}[label=901]

Kao argument komandne linije zadaje se ime ulazne datoteke u kojoj se nalaze niske. U prvoj liniji datoteke nalazi se informacija o broju niski, a zatim u narednim linijama po jedna niska ne duža od \argf{50} karaktera.
  
Napisati program u kojem se dinamički alocira memorija za zadati niz niski, a zatim se na standardnom izlazu u redosledu suprotnom od redosleda čitanja ispisuju sve niske koje počinju velikim slovom. 

U slučaju pojave bilo kakve greške na standardnom izlazu ispisati vrednost \argf{-1} i prekinuti izvršavanje programa.

\begin{minitest}
\begin{test}{Test 1}
Sadržaj datoteke:  
 5                             
 Programiranje	                               
 Matematika		                         
 12345
 dInAmiCnArEc
 Ispit
Izlaz:           
 Ispit                                                                
 Matematika
 Programiranje
\end{test}
\end{minitest}
\begin{minitest}
\begin{test}{Test 2}
Sadržaj datoteke:               
  2                                             
  maksimalano               
  poena
Izlaz: 
\end{test}
\end{minitest}
\begin{minitest}
\begin{test}{Test 3}
Problem: 
  datoteka 
   ne postoji
Izlaz:    
  -1       
\end{test}
\end{minitest}


\end{Exercise}
\begin{Answer}[ref=901]
\includecode{resenja/09_IspitniRokovi/901.c}
\end{Answer}


\begin{Exercise}[label=902]

Data je biblioteka za rad sa binarnim pretrazivačkim stablima čiji čvorovi sadrže cele brojeve. 
Napisati funkciju   \kckod{int sumirajN (Cvor * koren, int n)}
koja izračunava zbir svih čvorova koji se nalaze na $n$-tom nivou stabla (koren se nalazi na nultom nivou, njegova deca na prvom nivou i tako redom). 
Ispravnost napisane funkcije testirati na osnovu zadate  \kckod{main} funkcije i biblioteke za rad sa pretraživačkim stablima.

Napisati program koji sa standardnog ulaza učitava najpre prirodan broj $n$, a potom i brojeve sve do pojave nule koje smešta u stablo i ispisuje rezultat pozivanja funkcije \kckod{prebrojN} za broj $n$ i tako kreirano stablo. U slučaju greške na standardni izlaz za grešku ispisati \argf{-1}.


\begin{miditest}
\begin{test}{Test 1}
 Ulaz:                                                        
  2 8 10 3 6 14 13 7 4 0     
 Izlaz:                                                  
  20                                                        
\end{test}
\end{miditest}
\begin{miditest}
\begin{test}{Test 2}
Ulaz:	                               
 0 50 14 5 2 4 56 8 52 7 1 0            
Izlaz:                               
 50                               
\end{test}
\end{miditest}
\end{Exercise}
\begin{Answer}[ref=902]
\includecode{resenja/09_IspitniRokovi/902.c}
\includecode{resenja/09_IspitniRokovi/902stabla.c}
\includecode{resenja/09_IspitniRokovi/902stabla.h}
\end{Answer}

\begin{Exercise}[label=903]
Sa standardnog ulaza učitava se broj vrsta i broj kolona celobrojne matrice $A$, a zatim i elementi matrice $A$. Napisati program koji će ispisati indeks kolone u kojoj se nalazi najviše negativnih elemenata. Ukoliko postoji više takvih kolona, ispisati indeks prve kolone. Može se pretpostaviti da je broj vrsta i broj kolona manji od \argf{50}. U slučaju greške ispisati vrednost \argf{-1} na standardni izlaz za greške. 

\begin{minitest}
\begin{test}{Test 1}
Ulaz:                         
  4                             
  5                                                               
  1  2  3  4  5               
 -1  2 -3  4 -5                
 -5 -4 -3 -2  1               
 -1  0  0  0  0 
 Izlaz:                         
  0                                                    
\end{test}
\end{minitest}
\begin{minitest}
\begin{test}{Test 2}
Ulaz:                         
 2                               
 3                                 
 0 0 -5
 1 2 -4
Izlaz:   
\end{test}
\end{minitest}
\begin{minitest}
\begin{test}{Test 3}
Ulaz:                         
 -2
Izlaz (na stderr):
 -1
\end{test}
\end{minitest}


\end{Exercise}
\begin{Answer}[ref=903]
\includecode{resenja/09_IspitniRokovi/903.c}
\end{Answer}

\subsection*{Programiranje 2, praktični deo ispita, jul 2015.}


\begin{Exercise}[label=904]

 Napisati program koji kao prvi arugment komandne linije prima ime dokumenta u kome treba prebrojati sva pojavljivanja tražene niske (bez preklapanja) koja se navodi kao drugi argument komandne linije (iskoristiti funkciju standardne biblioteke \kckod{strstr}). U slučaju
  bilo kakve greške ispisati \argf{-1} na standardni izlaz za greške.
  Pretpostaviti da linije datoteke neće biti duže od \argf{127}
  karaktera.\\
  Potpis funkcije strstr:\\
  \kckod{char *strstr(const char *haystack, const char *needle);}\\
  Funkcija traži prvo pojavljivanje podniske needle u nisci
  haystack, i vraća pokazivač na početak podniske, ili
  NULL ako podniska nije pronađena.

\begin{miditest}
\begin{test}{Test 1}
Poziv:  ./a.out fajl.txt test          
Datoteka:   Ovo je test primer.     
        U njemu se rec test javlja
        vise puta. testtesttest
Izlaz:  5                       
\end{test}
\end{miditest}
\begin{miditest}
\begin{test}{Test 2}
Poziv:   ./a.out      
Izlaz  (na stderr):  -1       
\end{test}
\end{miditest}


\begin{miditest}
\begin{test}{Test 3}
Poziv:   ./a.out fajl.txt foo
Datoteka:   (ne postoji)      
Izlaz (na stderr):    -1              
\end{test}
\end{miditest}
\begin{miditest}
\begin{test}{Test 4}
Poziv:  ./a.out fajl.txt .    
Datoteka:   (prazna)      
Izlaz:    0              
\end{test}
\end{miditest}
\end{Exercise}
\begin{Answer}[ref=904]
\includecode{resenja/09_IspitniRokovi/904.c}
\end{Answer}


\begin{Exercise}[label=905]

Na početku datoteke ''trouglovi.txt'' nalazi se broj trouglova čije su koordinate temena zapisane u nastavku datoteke. Napisati
  program koji učitva trouglove, i ispisuje ih na standardni izlaz
  sortirane po površini opadajuće (koristiti Heronov obrazac: 
  $P = \sqrt{s*(s-a)*(s-b)*(s-c)}$, gde je $s$ poluobim trougla). U slučaju bilo kakve greške ispisati \argf{-1} na standardni izlaz za greške. Ne praviti nikave pretpostavke o broju trouglova u datoteci, i proveriti da li je datoteka ispravno zadata.

\begin{miditest}
\begin{test}{Test 1}
Datoteka:  4                         
           0 0 0 1.2 1 0          
           0.3 0.3 0.5 0.5 0.9 1
           -2 0 0 0 0 1
           2 0 2 2 -1 -1
Izlaz:     2 0 2 2 -1 -1             
           -2 0 0 0 0 1
           0 0 0 1.2 1 0
           0.3 0.3 0.5 0.5 0.9 1                                                   
\end{test}
\end{miditest}
\begin{minitest}
\begin{test}{Test 2}
Datoteka:  3             
          1.2 3.2 1.1 4.3
Izlaz:     -1                                            
\end{test}
\end{minitest}

\begin{miditest}
\begin{test}{Test 3}
Datoteka:   (nema datoteke)
Izlaz:     -1                                            
\end{test}
\end{miditest}
\begin{minitest}
\begin{test}{Test 4}
Datoteka:   0
Izlaz:                                                 
\end{test}
\end{minitest}
\end{Exercise}

\begin{Answer}[ref=905]
%\includecode{resenja/09_IspitniRokovi/905.c}
%\includecode{resenja/09_IspitniRokovi/905stabla.c}
%\includecode{resenja/09_IspitniRokovi/905stabla.h}  
\end{Answer}

\begin{Exercise}[label=906]
  Data je biblioteka za rad sa binarnim pretraživačkim stablima
  celih brojeba. Napisati funkciju\\ 
\kckod{int f3(Cvor *koren, int n)}\\
  koja u datom stablu prebrojava čvorove na $n$-tom nivou, koji
  imaju tačno jednog potomka. Pretpostaviti da se koren nalazi na
  nivou \argf{0}. Ispravnost napisane funkcije testirati na osnovu zadate
  \kckod{main} funkcije i biblioteke za rad sa stablima.

\begin{minitest}
\begin{test}{Test 1}
Ulaz:   
 1 5 3 6 1 4 7 9    
Izlaz:  
 1                                                            
\end{test}
\end{minitest}
\begin{minitest}
\begin{test}{Test 2}
Ulaz:   
 2 5 3 6 1 0 4 7 9  
Izlaz:  
 2          
\end{test}
\end{minitest}
\begin{minitest}
\begin{test}{Test 3}
Ulaz:  
 0 4 2 5   
Izlaz: 
 0       
\end{test}
\end{minitest}


\begin{minitest}
\begin{test}{Test 4}
Ulaz:  
  3
Izlaz: 
 0       
\end{test}
\end{minitest}
\begin{minitest}
\begin{test}{Test 5}
Ulaz:  
  -1 4 5 1 7 
Izlaz: 
 0       
\end{test}
\end{minitest}

\end{Exercise}
\begin{Answer}[ref=906]
%\includecode{resenja/09_IspitniRokovi/906.c}
\end{Answer}

\section{Rešenja}
\shipoutAnswer
