\chapter{Ispitni rokovi}

\section{Programiranje 2, praktični deo ispita, jun 2015.}

\begin{Exercise}[label=901]
Kao argument komandne linije zadaje se ime ulazne datoteke u kojoj se nalaze niske. U prvoj liniji datoteke nalazi se informacija o broju niski, a zatim u narednim linijama po jedna niska ne duža od \argf{50} karaktera.
  
Napisati program u kojem se dinamički alocira memorija za zadati niz niski, a zatim se na standardnom izlazu u redosledu suprotnom od redosleda čitanja ispisuju sve niske koje počinju velikim slovom. 

U slučaju pojave bilo kakve greške na standardnom izlazu ispisati vrednost \argf{-1} i prekinuti izvršavanje programa.

\begin{miditest}
\begin{test}{1}
#\poziv{./a.out ulaz.txt}#

#\naslovDat{ulaz.txt}#
#\datoteka{5}#
#\datoteka{Programiranje}#
#\datoteka{Matematika}#
#\datoteka{12345}#
#\datoteka{dInAmiCnArEc}#
#\datoteka{Ispit}#
  
#\naslovIzlaz#
#\izlaz{Ispit}#
#\izlaz{Matematika}#
#\izlaz{Programiranje}#
\end{test}
\end{miditest}
\begin{minitest}
\begin{test}{2}
#\poziv{./a.out ulaz.txt}#

#\naslovDat{ulaz.txt}#
#\datoteka{2}#
#\datoteka{maksimalano}#
#\datoteka{poena}#

#\naslovIzlaz#
#\izlaz{}#
\end{test}
\end{minitest}


\begin{miditest}
\begin{test}{3}
#\poziv{./a.out ulaz.txt}#

#\naslovDat{Datoteka ulaz.txt ne postoji}#

#\naslovIzlaz#
#\izlaz{  -1}#
\end{test}
\end{miditest}
\begin{miditest}
\begin{test}{4}
#\poziv{./a.out}#

#\naslovIzlaz#
#\izlaz{-1}#
\end{test}
\end{miditest}

\linkresenje{901}
\end{Exercise}
\begin{Answer}[ref=901]
\includecode{resenja/09_IspitniRokovi/901.c}
\end{Answer}


\begin{Exercise}[label=902]
Data je biblioteka za rad sa binarnim pretraživačkim stablima čiji čvorovi sadrže cele brojeve. 
Napisati funkciju   \kckod{int sumirajN (Cvor * koren, int n)}
koja izračunava zbir svih čvorova koji se nalaze na $n$-tom nivou stabla (koren se nalazi na nultom nivou, njegova deca na prvom nivou i tako redom). 
Ispravnost napisane funkcije testirati na osnovu zadate  \kckod{main} funkcije i biblioteke za rad sa pretraživačkim stablima.

Napisati program koji sa standardnog ulaza učitava najpre prirodan broj $n$, a potom i brojeve sve do pojave nule koje smešta u stablo i ispisuje rezultat pozivanja funkcije \kckod{prebrojN} za broj $n$ i tako kreirano stablo. U slučaju greške na standardni izlaz za grešku ispisati \argf{-1}.


\begin{miditest}
\begin{test}{1}
#\naslovUlaz#
#\ulaz{2 8 10 3 6 14 13 7 4 0}#
#\naslovIzlaz#
#\izlaz{ 20}#
\end{test}
\end{miditest}
\begin{miditest}
\begin{test}{2}
#\naslovUlaz#
#\ulaz{0 50 14 5 2 4 56 8 52 7 1 0}#
#\naslovIzlaz#
#\izlaz{ 50}#
\end{test}
\end{miditest}

\linkresenje{902}
\end{Exercise}
\begin{Answer}[ref=902]
%\includecode{resenja/09_IspitniRokovi/902/stabla.h}
%\includecodeLib{resenja/09_IspitniRokovi/902/stabla.h}{stabla.h}
%\includecodeLib{resenja/09_IspitniRokovi/902/stabla.c}{stabla.c}
\napomena{Rešenje koristi biblioteku za rad sa binarnim pretraživačkim stablima iz zadatka \ref{701}.}\\
\includecodeLib{resenja/09_IspitniRokovi/902/main.c}{main.c}
%\includecode{resenja/09_IspitniRokovi/902/stabla.c}
%\includecode{resenja/09_IspitniRokovi/902/main.c}
\end{Answer}

\begin{Exercise}[label=903]
Sa standardnog ulaza učitava se broj vrsta i broj kolona celobrojne matrice $A$, 
a zatim i elementi matrice $A$. Napisati program koji će ispisati indeks kolone u kojoj se nalazi najviše negativnih elemenata. 
Ukoliko postoji više takvih kolona, ispisati indeks prve kolone. 
Može se pretpostaviti da je broj vrsta i broj kolona manji od \argf{50}. 
U slučaju greške ispisati vrednost \argf{-1} na standardni izlaz za greške. 

\begin{minitest}
\begin{test}{1}
#\naslovUlaz#
#\ulaz{4 5}#
#\ulaz{1  2  3  4  5}#
#\ulaz{ -1  2 -3  4 -5 }#
#\ulaz{ -5 -4 -3 -2  1}#
#\ulaz{-1  0  0  0  0 }#
#\naslovIzlaz#
#\izlaz{0}#
\end{test}
\end{minitest}
\begin{minitest}
\begin{test}{2}
#\naslovUlaz#
#\ulaz{2 3}#
#\ulaz{0 0 -5}#
#\ulaz{1 2 -4}#
#\naslovIzlaz#
#\izlaz{2}#
\end{test}
\end{minitest}
\begin{minitest}
\begin{test}{3}
#\naslovUlaz#
#\ulaz{-2}#
#\naslovIzlaz#
#\izlaz{-1}#
\end{test}
\end{minitest}

\linkresenje{903}
\end{Exercise}
\begin{Answer}[ref=903]
\includecode{resenja/09_IspitniRokovi/903.c}
\end{Answer}

\section{Programiranje 2, praktični deo ispita, jul 2015.}

\begin{Exercise}[label=904]
Napisati program koji kao prvi arugment komandne linije prima ime dokumenta u kome treba prebrojati sva pojavljivanja tražene niske (bez preklapanja) koja se navodi kao drugi argument komandne linije (iskoristiti funkciju standardne biblioteke \kckod{strstr}). U slučaju
bilo kakve greške ispisati \argf{-1} na standardni izlaz za greške.
Pretpostaviti da linije datoteke neće biti duže od \argf{127}
karaktera.\\
Potpis funkcije \kckod{strstr}:\\
\kckod{char *strstr(const char *haystack, const char *needle);}\\
Funkcija traži prvo pojavljivanje podniske needle u nisci
haystack, i vraća pokazivač na početak podniske, ili
NULL ako podniska nije pronađena.

\begin{miditest}
\begin{test}{1}
#\poziv{./a.out ulaz.txt test}#

#\naslovDat{ulaz.txt}#
#\datoteka{ Ovo je test primer. }#
#\datoteka{ U njemu se rec test javlja}#
#\datoteka{vise puta. testtesttest}#

#\naslovIzlaz#
#\izlaz{5}#
\end{test}
\end{miditest}
\begin{miditest}
\begin{test}{2}
#\poziv{./a.out}#

#\naslovIzlaz#
#\izlaz{(na stderr) -1}#
\end{test}
\end{miditest}

\begin{miditest}
\begin{test}{3}
#\poziv{./a.out ulaz.txt foo}#

#\naslovDat{Datoteka ulaz.txt ne postoji}#

#\naslovIzlaz#
#\izlaz{(na stderr) -1}#
\end{test}
\end{miditest}
\begin{miditest}
\begin{test}{4}
#\poziv{ ./a.out ulaz.txt .  }#

#\naslovDat{Datoteka ulaz.txt je prazna}#

#\naslovIzlaz#
#\izlaz{0}#
\end{test}
\end{miditest}

\linkresenje{904}
\end{Exercise}
\begin{Answer}[ref=904]
\includecode{resenja/09_IspitniRokovi/904.c}
\end{Answer}


\begin{Exercise}[label=905]
Na početku datoteke ,,trouglovi.txt'' nalazi se broj trouglova čije su koordinate temena zapisane u nastavku datoteke. Napisati
  program koji učitva trouglove, i ispisuje ih na standardni izlaz
  sortirane po površini opadajuće (koristiti Heronov obrazac: 
  $P = \sqrt{s*(s-a)*(s-b)*(s-c)}$, gde je $s$ poluobim trougla). U slučaju bilo kakve greške ispisati \argf{-1} na standardni izlaz za greške. Ne praviti nikave pretpostavke o broju trouglova u datoteci, i proveriti da li je datoteka ispravno zadata.

\begin{miditest}
\begin{test}{1}
#\naslovDat{trouglovi.txt}#
#\datoteka{4}#
#\datoteka{ 0 0 0 1.2 1 0 }#
#\datoteka{ 0.3 0.3 0.5 0.5 0.9 1}#
#\datoteka{-2 0 0 0 0 1}#
#\datoteka{-2 0 0 0 0 1}#

#\naslovIzlaz#
#\izlaz{2 0 2 2 -1 -1}#
#\izlaz{-2 0 0 0 0 1}#
#\izlaz{0 0 0 1.2 1 0}#
#\izlaz{0.3 0.3 0.5 0.5 0.9 1}#
\end{test}
\end{miditest}
\begin{minitest}
\begin{test}{2}
#\naslovDat{trouglovi.txt}#
#\datoteka{3}#
#\datoteka{ 1.2 3.2 1.1 4.3}#

#\naslovIzlaz#
#\izlaz{-1}#
\end{test}
\end{minitest}

\begin{miditest}
\begin{test}{3}
#\naslovDat{Datoteka trouglovi.txt ne postoji}#

#\naslovIzlaz#
#\izlaz{-1}#
\end{test}
\end{miditest}
\begin{minitest}
\begin{test}{4}
#\naslovDat{trouglovi.txt}#
#\datoteka{0}#

#\naslovIzlaz#
#\izlaz{}#
\end{test}
\end{minitest}

\linkresenje{905}
\end{Exercise}

\begin{Answer}[ref=905]
\includecode{resenja/09_IspitniRokovi/905.c}
\end{Answer}

\begin{Exercise}[label=906]
Data je biblioteka za rad sa binarnim pretraživačkim stablima celih brojeva. Napisati funkciju\\ 
\kckod{int f3(Cvor *koren, int n)}\\
  koja u datom stablu prebrojava čvorove na $n$-tom nivou, koji
  imaju tačno jednog potomka. Pretpostaviti da se koren nalazi na
  nivou \argf{0}. Ispravnost napisane funkcije testirati na osnovu zadate
  \kckod{main} funkcije i biblioteke za rad sa stablima.

\begin{minitest}
\begin{test}{1}
#\naslovUlaz#
#\ulaz{ 1 5 3 6 1 4 7 9}#
#\naslovIzlaz#
#\izlaz{1}#
\end{test}
\end{minitest}
\begin{minitest}
\begin{test}{2}
#\naslovUlaz#
#\ulaz{2 5 3 6 1 0 4 7 9}#
#\naslovIzlaz#
#\izlaz{2}#
\end{test}
\end{minitest}
\begin{minitest}
\begin{test}{3}
#\naslovUlaz#
#\ulaz{0 4 2 5}#
#\naslovIzlaz#
#\izlaz{0}#
\end{test}
\end{minitest}

\begin{minitest}
\begin{test}{4}
#\naslovUlaz#
#\ulaz{ 3}#
#\naslovIzlaz#
#\izlaz{0}#
\end{test}
\end{minitest}
\begin{minitest}
\begin{test}{5}
#\naslovUlaz#
#\ulaz{-1 4 5 1 7}#
#\naslovIzlaz#
#\izlaz{0}#
\end{test}
\end{minitest}

\linkresenje{906}
\end{Exercise}
\begin{Answer}[ref=906]
%%\includecode{resenja/09_IspitniRokovi/906/stabla.h}  
%\includecodeLib{resenja/09_IspitniRokovi/906/stabla.h}{stabla.h}
%\includecodeLib{resenja/09_IspitniRokovi/906/stabla.c}{stabla.c}
\napomena{Rešenje koristi biblioteku za rad sa binarnim pretraživačkim stablima iz zadatka \ref{701}.}\\
\includecodeLib{resenja/09_IspitniRokovi/906/main.c}{main.c}
%%\includecode{resenja/09_IspitniRokovi/906/stabla.c}
%%\includecode{resenja/09_IspitniRokovi/906/main.c}
\end{Answer}


\section{Programiranje 2, praktični deo ispita, septembar 2015.}

\begin{Exercise}[label=907]
Sa standardnog ulaza se učitavaju neoznačeni celi brojevi \texttt{x} i \texttt{n}. Na
   standardni izlaz ispisati neoznačen ceo broj koji se dobija od broja \texttt{x} kada se njegov binarni zapis
   rotira za \texttt{n} mesta udesno (na primer, ako je binarni zapis broja \texttt{x} jednak \texttt{00000000000000000000000000001111},
   i ako je \texttt{n=1} tada na standardni izlaz treba ispisati neo\v{z}načen broj čiji je binarni zapis jednak \texttt{10000000000000000000000000000111}).

\begin{minitest}
\begin{test}{1}
#\naslovUlaz#
#\ulaz{6 1}#
#\naslovIzlaz#
#\izlaz{3}#
\end{test}
\end{minitest}
\begin{minitest}
\begin{test}{2}
#\naslovUlaz#
#\ulaz{15 3}#
#\naslovIzlaz#
#\izlaz{3758096385}#
\end{test}
\end{minitest}
\begin{minitest}
\begin{test}{3}
#\naslovUlaz#
#\ulaz{31 100}#
#\naslovIzlaz#
#\izlaz{4026531841}#
\end{test}
\end{minitest}

\begin{minitest}
\begin{test}{4}
#\naslovUlaz#
#\ulaz{4 0}#
#\naslovIzlaz#
#\izlaz{4}#
\end{test}
\end{minitest}
\begin{minitest}
\begin{test}{5}
#\naslovUlaz#
#\ulaz{0 5}#
#\naslovIzlaz#
#\izlaz{0}#
\end{test}
\end{minitest}

\linkresenje{907}
\end{Exercise}
\begin{Answer}[ref=907]
\includecode{resenja/09_IspitniRokovi/907.c}
\end{Answer}


\begin{Exercise}[label=908]
Napisati funkciju \kckod{void dopuni\_listu(Cvor ** adresa\_glave)}
koja samo čvorovima koji imaju sledbenika u jednostruko povezanoj listi realnih brojeva,
  dodaje između čvora i njegovog sledbenika nov čvor čija vrednost je aritmetička sredina njihovih vrednosti.
 Ispravnost napisane funkcije testirati koristeći dostupnu biblioteku za rad sa listama i \kckod{main} funkciju koja najpre
 učitava elemente liste, poziva pomenutu funkciju i ispisuje sadržaj liste.

\begin{maxitest}
\begin{test}{1}
#\naslovUlaz#
#\ulaz{1 2 3 4 5}#
#\naslovIzlaz#
#\izlaz{1.00 1.50 2.00 2.50 3.00 3.50 4.00 4.50 5.00}#
\end{test}
\end{maxitest}

\begin{minitest}
\begin{test}{2}
#\naslovUlaz#
#\ulaz{12}#
#\naslovIzlaz#
#\izlaz{12.00}#
\end{test}
\end{minitest}
\begin{minitest}
\begin{test}{3}
#\naslovUlaz#
#\ulaz{prazna lista}#
#\naslovIzlaz#
#\izlaz{}#
\end{test}
\end{minitest}
\begin{minitest}
\begin{test}{4}
#\naslovUlaz#
#\ulaz{13.3 15.8}#
#\naslovIzlaz#
#\izlaz{13.30 14.55}#
\end{test}
\end{minitest}

\linkresenje{908}
\end{Exercise}

\begin{Answer}[ref=908]
%\includecode{resenja/09_IspitniRokovi/908/liste.h}
\includecodeLib{resenja/09_IspitniRokovi/908/liste.h}{liste.h}
\includecodeLib{resenja/09_IspitniRokovi/908/liste.c}{liste.c}
\includecodeLib{resenja/09_IspitniRokovi/908/main.c}{main.c}
%\includecode{resenja/09_IspitniRokovi/908/liste.c}
%\includecode{resenja/09_IspitniRokovi/908/main.c}
\end{Answer}

\begin{Exercise}[label=909]
Sa standardnog ulaza se učitava dimenzija \kckod{n} kvadratne celobrojne
    matrice \argf{A} ($n>0$), a zatim i elementi matrice \argf{A}. Napisati program koji
    proverava da li je data kvadratna matrica magični kvadrat
    (magični kvadrat je kvadratna matrica kod koje je suma brojeva
    u svakom redu i svakoj koloni jednaka). Ukoliko jeste, ispisati na
    standardnom izlazu sumu brojeva jedne vrste ili kolone te matrice,
    a ukoliko nije ispisati "-". Broj vrsta i broj kolona matrice nije
    unapred poznat. U slučaju greške ispisati \argf{-1}.

\begin{minitest}
\begin{test}{1}
#\naslovUlaz#
#\ulaz{4}#
#\ulaz{1 2 3 4}#
#\ulaz{2 1 4 3}#
#\ulaz{3 4 2 1}#
#\ulaz{4 3 1 2}#
#\naslovIzlaz#
#\izlaz{10}#
\end{test}
\end{minitest}
\begin{minitest}
\begin{test}{2}
#\naslovUlaz#
#\ulaz{3}#
#\ulaz{1 1 1}#
#\ulaz{1 1 1}#
#\ulaz{1 1 1}#
#\naslovIzlaz#
#\izlaz{3}#
\end{test}
\end{minitest}
\begin{minitest}
\begin{test}{3}
#\naslovUlaz#
#\ulaz{2}#
#\ulaz{1 1}#
#\ulaz{2 2}#
#\naslovIzlaz#
#\izlaz{-}#
\end{test}
\end{minitest}

\begin{minitest}
\begin{test}{4}
#\naslovUlaz#
#\ulaz{2}#
#\ulaz{1 2}#
#\ulaz{1 2}#
#\naslovIzlaz#
#\izlaz{-}#
\end{test}
\end{minitest}
\begin{minitest}
\begin{test}{5}
#\naslovUlaz#
#\ulaz{1}#
#\ulaz{5}#
#\naslovIzlaz#
#\izlaz{5}#
\end{test}
\end{minitest}
\begin{minitest}
\begin{test}{6}
#\naslovUlaz#
#\ulaz{0}#
#\naslovIzlaz#
#\izlaz{-1}#
\end{test}
\end{minitest}

\linkresenje{909}
\end{Exercise}
\begin{Answer}[ref=909]
\includecode{resenja/09_IspitniRokovi/909.c}
\end{Answer}

\section{Rešenja}
\shipoutAnswer
