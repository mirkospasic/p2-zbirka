\section{Stabla}

% 1.zadatak
\begin{Exercise}[label=701]
Napisati program za rad sa binarnim pretraživačkim stablima.
\begin{enumerate}
\item Definisati strukturu \kckod{Cvor} kojom se opisuje čvor binarnog pretraživačkog stabla koja sadrži ceo broj \kckod{broj} i pokazivače \kckod{levo} i \kckod{desno} redom na levo i desno podstablo\footnote{U zadacima ove glave u kojima nije eksplicitno naglašen sadržaj čvorova stabla, podrazumevaće se ova struktura.}.

\item Napisati funkciju \kckod{Cvor* napravi\_cvor(int broj)} koja alocira memoriju za novi čvor stabla i vrši njegovu inicijalizaciju zadatim celim brojem \argf{broj}.

\item Napisati funkciju \kckod{void dodaj\_u\_stablo(Cvor** koren, int broj)} koja u stablo na koje pokazuje argument \argf{koren} dodaje ceo broj \argf{broj}.

\item Napisati funkciju \kckod{Cvor* pretrazi\_stablo(Cvor* koren, int broj)} koja proverava da li se ceo broj \argf{broj} nalazi u stablu sa korenom \argf{koren}. Funkcija vraća pokazivač na čvor stabla koji sadrži traženu vrednost ili NULL ukoliko takav čvor ne postoji.

\item Napisati funkciju \kckod{Cvor* pronadji\_najmanji(Cvor* koren)} koja pronalazi čvor koji sadrži najmanju vrednost u stablu sa korenom \argf{koren}. 

\item Napisati funkciju \kckod{Cvor* pronadji\_najveci(Cvor* koren)} koja pronalazi čvor koji sadrži najveću vrednost u stablu sa korenom \argf{koren}.

\item Napisati funkciju \kckod{void obrisi\_element(Cvor** koren, int broj)} koja briše čvor koji sadrži vrednost \argf{broj} iz stabla na koje pokazuje argument \argf{koren}.

\item Napisati funkciju \kckod{void ispisi\_stablo\_infiksno(Cvor* koren)} koja infiksno ispisuje sadržaj stabla sa korenom \argf{koren}. Infiksni ispis podrazumeva ispis levog podstabla, korena, a zatim i desnog podstabla.

\item Napisati funkciju \kckod{void ispisi\_stablo\_prefiksno(Cvor* koren)} koja prefiksno ispisuje sadržaj stabla sa korenom \argf{koren}. Prefiksni ispis podrazumeva ispis korena, levog podstabla, a zatim i desnog podstabla.

\item Napisati funkciju \kckod{void ispisi\_stablo\_postfiksno(Cvor* koren)} koja postfiksno ispisuje sadržaj stabla sa korenom \argf{koren}. Postfiksni ispis podrazumeva ispis levog podstabla, desnog podstabla, a zatim i korena.

\item Napisati funkciju \kckod{void oslobodi\_stablo(Cvor** koren)} koja oslobađa memoriju zauzetu stablom na koje pokazuje argument \argf{koren}.
\end{enumerate}

Korišćenjem prethodnih funkcija, napisati program koji sa standardnog ulaza učitava cele brojeve sve do kraja ulaza, dodaje ih u binarno pretraživačko stablo i ispisuje stablo u svakoj od navedenih notacija. Zatim omogućiti unos još dva cela broja i demonstrirati rad funkcije za pretragu nad prvim unetim brojem i rad funkcije za brisanje elemenata nad drugim unetim brojem. 

\begin{maxitest}
\begin{test}{Test 1}
Poziv: ./a.out
Ulaz:
	Unesite brojeve (CRL+D za kraj unosa): 7 2 1 9 32 18
Izlaz:
	Infiksni ispis: 1 2 7 9 18 32
	Prefiksni ispis: 7 2 1 9 32 18
	Postfiksni ispis: 1 2 18 32 9 7
	Trazi se broj: 11
	Broj se ne nalazi u stablu!
	Brise se broj: 7
	Rezultujuce stablo: 1 2 9 18 32
\end{test}
\end{maxitest}

\begin{maxitest}
\begin{test}{Test 2}
Poziv: ./a.out
Ulaz:
	Unesite brojeve (CRL+D za kraj unosa): 8 -2 6 13 24 -3
Izlaz:
	Infiksni ispis:  -3 -2 6 8 13 24
	Prefiksni ispis: 8 -2 -3 6 13 24
	Postfiksni ispis: -3 6 -2 24 13 8 
	Trazi se broj: 6
	Broj se nalazi u stablu!
	Brise se broj: 14
	Rezultujuce stablo: -3 -2 6 8 13 24
\end{test}
\end{maxitest}
\end{Exercise}

\begin{Answer}[ref=701]
%\includecode{resenja/07_Drveta/701.c}
\end{Answer}



%2.zadatak
\begin{Exercise}[label=702]
Napisati program koji izračunava i na standardnom izlazu ispisuje broj pojavljivanja svake reči datoteke čije se ime zadaje kao argument komandne linije. Program realizovati korišćenjem binarnog pretraživackog stabla uređenog leksikografski prema rečima ne uzimajući u obzir razliku između malih i velikih slova. Ukoliko prilikom pokretanja programa korisnik ne navede ime ulazne datoteke ispisati poruku \kckod{Nedostaje ime ulazne datoteke!}.

\komentar{Milena: dodati i test primer sa pokretanjem bez ulazne datoteke}

\begin{miditest}
\begin{test}{Test 1}
Poziv: ./a.out test.txt
Datoteka test.txt:
	Sunce utorak raCunar SUNCE programiranje jabuka PROGramiranje sunCE JABUka
Izlaz:
	jabuka: 2
	programiranje: 2
	racunar: 1
	sunce: 3
	utorak: 1
\end{test}
\end{miditest}

\begin{miditest}
\begin{test}{Test 2}
Poziv: ./a.out suma.txt
Datoteka suma.txt:
	lipa zova hrast ZOVA breza LIPA
Izlaz:
	breza: 1
	hrast: 1
	lipa: 2
	zova: 2
\end{test}
\end{miditest}

\begin{miditest}
\begin{test}{Test 3}
Poziv: ./a.out 
Izlaz:
	Nedostaje ime ulazne datoteke!
\end{test}
\end{miditest}

\end{Exercise}


\begin{Answer}[ref=702]
%\includecode{resenja/07_Drveta/702.c}
\end{Answer}

%3.zadatak

%4.zadatak
\begin{Exercise}[label=704]
U svakoj liniji datoteke čije se ime zadaje sa standardnog ulaza nalazi se ime osobe, prezime osobe i njen broj telefona, npr.~\kckod{Pera Peric 064/123-4567}. Napisati program koji korišćenjem binarnog pretraživačkog stabla implementira mapu koja sadrži navedene informacije i koja će omogućiti pretragu brojeva telefona za zadata imena i prezimena. Imena i prezimena se unose sve do unosa reči \kckod{KRAJ}, a za svaki od unetih podataka ispisuje se ili broj telefona ili obaveštenje da traženi broj nije u imeniku. Može se pretpostaviti da imena, prezimena i brojevi telefona neće biti duži od $30$ karaktera.  

\begin{maxitest}
\begin{test}{Upotreba programa 1}
Poziv: ./a.out 
Datoteka imenik.txt:
   Pera Peric 011/3240-987
   Marko Maric 064/1234-987
   Mirko Maric 011/589-333
   Sanja Savkovic 063/321-098
   Zika Zikic 021/759-858
Ulaz:
	Unesite ime datoteke: imenik.txt
	Unesite ime i prezime: Pera Peric
Izlaz:	
	Broj je: 011/3240-987
Ulaz:	
	Unesite ime i prezime: Marko Markovic
Izlaz:
	Broj nije u imeniku!
Ulaz:	
	Unesite ime i prezime: KRAJ
\end{test}
\end{maxitest}
\end{Exercise}

\begin{Answer}[ref=704]
%\includecode{resenja/07_Drveta/704.c}
\end{Answer}


%5.zadatak
\begin{Exercise}[label=705]
U datoteci \kckod{prijemni.txt} nalaze se podaci o prijemnom ispitu učenika jedne osnovne škole tako što je u svakom redu navedeno ime i prezime učenika (niz najviše $50$ karaktera), broj poena na osnovu uspeha (realan broj), broj poena na prijemnom ispitu iz matematike (realan broj) i broj poena na prijemnom ispitu iz maternjeg jezika (realan broj). Za učenika koji u zbiru osvoji manje od $10$ poena na oba prijemna ispita smatra se da nije položio prijemni. Napisati program koji na osnovu podataka iz ove datoteke formira i prikazuje rang listu učenika. Rang lista sadrži redni broj učenika, njegovo ime i prezime, broj poena na osnovu uspeha, broj poena na prijemnom ispitu iz matematike, broj poena na prijemnom ispitu iz maternjeg jezika i ukupan broj poena i sortirana je opadajuće po ukupnom broju poena. Na rang listi se prvo navode oni učenici koji su položili prijemni ispit, a potom i učenici koji ga nisu položili. Između ovih dveju grupa učenika postoji i horizontalna linija koja ih vizuelno razdvaja. 
\begin{maxitest}
\begin{test}{Test 1}
Poziv: ./a.out
Datoteka prijemni.txt:
   Marko Markovic 45.4 12.3 11         
   Milan Jevremovic 35.2 1.3 9         
   Maja Agic 60 19 20                  
   Nadica Zec 54.2 10 15.8             
   Jovana Milic 23.3 2 5.6
Izlaz:
	1. Maja Agic 60 19 20 99
	2. Nadica Zec 54.2 10 15.8 80
	3. Marko Markovic 45.4 12.3 11 68.7
	4. Milan Jevremovic 35.2 1.3 9 45.5
	-----------------------------------
	5. Jovana Milic 23.3 2 5.6 30.9
\end{test}
\end{maxitest}
\end{Exercise}
\begin{Answer}[ref=705]
%\includecode{resenja/07_Drveta/705.c}
\end{Answer}

%6.zadatak
\begin{Exercise}[label=706, difficulty=1]
Napisati program koji implementira podsetnik za rođendane. Informacije o rođendanima se nalaze u datoteci čije se ime zadaje kao argument komandne linije u formatu \kckod{Ime Prezime DD.MM.YYYY.} - za svaku osobu po jedna linija datoteke. Korisnik unosi datum u naznačenom formatu, a program pronalazi i ispisuje ime i prezime osobe čiji je rođendan zadatog datuma ili ime i prezime osobe koja prva sledeća slavi rođendan. Ovaj postupak treba ponavljati dokle god korisnik ne unese komandu za kraj rada. Informacije o rođendanima uneti u mapu koja je implementirana preko binarnog pretraživačkog stabla i uređena po datumima. Može se pretpostaviti da će svi korišćeni datumi biti validni i u formatu \kckod{DD.MM.YYYY}.
\begin{maxitest}
\begin{test}{Upotreba programa 1}
Poziv: a.out
Datoteka rodjendani.txt:
	Marko Markovic 12.12.1990.
	Milan Jevremovic 04.06.1989.
	Maja Agic 23.04.2000.
	Nadica Zec 01.01.1993.
	Jovana Milic 05.05.1990.
Ulaz:
	Unesite datum: 23.04.
Izlaz:
	Slavljenik: Maja Agic 
Ulaz:	
	Unesite datum: 01.01.
Izlaz:
	Slavljenik: Nadica Zec
Ulaz:
	Unesite datum: 01.05.
Izlaz:	
	Slavljeni: Jovana Milic 05.05.
Ulaz:	
	Unesite datum: CTRL+D
\end{test}
\end{maxitest}
\end{Exercise}

\begin{Answer}[ref=706]
%\includecode{resenja/07_Drveta/706.c}
\end{Answer}



%7.zadatak
\begin{Exercise}[label=707]
Dva binarna stabla su identična ako su ista po strukturi i sadržaju tj. ako oba korena imaju isti sadržaj i identična odgovarajuća podstabla. Napistati funkciju \kckod{int identitet(Cvor* koren1, Cvor* koren2)} koja proverava da li su binarna stabla \argf{koren1} i \argf{koren2} koja sadrže cele brojeve identična, a zatim i glavni program koji testira njen rad. Elemente pojedinačnih stabla unositi sa standardnog ulaza sve do pojave broja $0$.

\begin{maxitest}
\begin{test}{Test 1}
Poziv: ./a.out
Ulaz:
	Prvo stablo: 10 5 15 3 2 4 30 12 14 13 0
	Drugo stablo: 10 15 5 3 4 2 12 14 13 30 0
Izlaz:
	Stabla jesu identicna.
\end{test}
\end{maxitest}

\begin{maxitest}
\begin{test}{Test 2}
Poziv: ./a.out
Ulaz:
	Prvo stablo: 10 5 15 4 3 2 30 12 14 13 0
	Drugo stablo: 10 15 5 3 4 2 12 14 13 30 0
Izlaz:
	Stabla nisu identicna.
\end{test}
\end{maxitest}

\end{Exercise}

\begin{Answer}[ref=707]
%\includecode{resenja/07_Drveta/707.c}
\end{Answer}


%8.zadatak
\begin{Exercise}[label=708, difficulty=1]
Napisati program koji za dva binarna pretraživačka stabla čiji se elementi zadaju sa standardnog ulaza, sve do kraja ulaza, ispisuje uniju, presek i razliku stabla. Unija dva stabala je stablo koje sadrži vrednosti iz oba stabla uračunata tačno po jednom. Presek dva stabala je stablo koje sadrži vrednosti koje se pojavljuju i u prvom i u drugom stablu. Razlika dva stabla je stablo koje sadrži sve vrednosti prvog stabla koje se ne pojavljuju u drugom stablu. 

\begin{miditest}
\begin{test}{Test 1}
Poziv: ./a.out
Ulaz:
	Prvo stablo: 1 7 8 9 2 2
	Drugo stablo: 3 9 6 11 1
Izlaz:
	Unija: 1 2 3 6 7 8 9 11
	Presek: 1 9 
	Razlika: 2 7 8 
\end{test}
\end{miditest}

\end{Exercise}

\begin{Answer}[ref=708]
%\includecode{resenja/07_Drveta/708.c}
\end{Answer}


%9.zadatak
\begin{Exercise}[label=709]
Napisati funkciju \kckod{void sortiraj(int a[], int n)} koja sortira niz celih brojeva \argf{a} dimenzije \argf{n} korišćenjem binarnog pretraživačkog stabla. Napisati i program koji sa standardnog ulaza učitava ceo broj \argf{n} manji od 50 i niz \argf{a} celih brojeva dužine \argf{n}, poziva funkciju \kckod{sortiraj} i rezultat ispisuje na standardnom izlazu.  

\begin{miditest}
\begin{test}{Test 1}
Poziv: ./a.out
Ulaz:
	n: 7 
	a: 1 11 8 6 37 25 30
Izlaz:
	1 6 8 11 25 30 37 
\end{test}
\end{miditest}

\begin{maxitest}
\begin{test}{Test 2}
Poziv: ./a.out
Ulaz:
	n: 55 
Izlaz:
	Greska: pogresna dimenzija niza!
\end{test}
\end{maxitest}
\end{Exercise}

\begin{Answer}[ref=709]
%\includecode{resenja/07_Drveta/709.c}
\end{Answer}


%10.zadatak
\begin{Exercise}[label=710]
Dato je binarno pretraživačko stablo celih brojeva.
\begin{enumerate}
\item Napisati funkciju koja izračunava broj čvorova stabla.
\item Napisati funkciju koja izračunava broj listova stabla.
\item Napisati funkciju koja štampa pozitivne vrednosti listova stabla.
\item Napisati funkciju koja izračunava zbir čvorova stabla.
\item Napisati funkciju koja izračunava najveći element stabla.
\item Napisati funkciju koja izračunava dubinu stabla.
\item Napisati funkciju koja izračunava broj čvorova na $i$-tom nivou stabla.
\item Napisati funkciju koja ispisuje sve elemente na $i$-tom nivou stabla.
\item Napisati funkciju koja izračunava maksimalnu vrednost na $i$-tom nivou stabla.
\item Napisati funkciju koja izračunava zbir čvorova na $i$-tom nivou stabla.
\item Napisati funkciju koja izračunava zbir svih vrednosti stabla koje su manje ili jednake od date vrednosti $x$.
\end{enumerate}
Napisati program koji testira prethodne funkcije. Stablo formirati na osnovu vrednosti koje se unose
sa standardnog ulaza, sve do kraja ulaza, a vrednosti parametara $i$ i $x$ pročitati kao argumente komandne linije. 

\begin{maxitest}
\begin{test}{Test 2}
Poziv: ./a.out 2 15
Ulaz: 
	10 5 15 3 2 4 30 12 14 13
Izlaz: 
	broj cvorova: 10
	broj listova: 4
	pozitivni listovi: 2 4 13 30
	zbir cvorova: 108
	najveci element: 30
	dubina stabla: 5
	broj cvorova na 2. nivou: 3
	elementi na 2. nivou: 3 12 30
	maksimalni na 2. nivou: 30
	zbir na 2. nivou: 45
	zbir elemenata manjih ili jednakih od 15: 7
\end{test}
\end{maxitest}

\end{Exercise}

\begin{Answer}[ref=710]
%\includecode{resenja/07_Drveta/710.c}
\end{Answer}



%11.zadatak

\begin{Exercise}[label=711]
Dato je binarno pretraživačko stablo celih brojeva.
\begin{enumerate}
\item Napisati funkciju koja pronalazi čvor u stablu sa maksimalnim proizvodom vrednosti iz desnog podstabla.
\item Napisati funkciju koja pronalazi čvor u stablu sa najmanjom sumom vrednosti iz levog podstabla.
\item Napisati funkciju  koja štampa sadržaj svih čvorova stabla na putanji od korena do najdubljeg čvora.
\item Napisati funkciju koja štampa sadržaj svih čvorova stabla na putanji od korena do čvora koji ima najmanju vrednost u stablu.
\end{enumerate}
Napisati program koji testira gorenavedene funkcije. Stablo formirati na osnovu vrednosti koje se unose
sa standardnog ulaza, sve do kraja ulaza.


\begin{maxitest}
\begin{test}{Test 1}
Poziv: ./a.out
Ulaz: 
	10 5 15 3 2 4 30 12 14 13
Izlaz: 
	Cvor sa maksimalnim desnim proizvodom: 10
	Cvor sa najmanjom levom sumom: 2
	Putanja do najdubljeg cvora: 10 15 12 14 13
	Putanja do najmanjeg cvora: 10 5 3 2
\end{test}
\end{maxitest}
\end{Exercise}

\begin{Answer}[ref=711]
%\includecode{resenja/07_Drveta/711.c}
\end{Answer}


%12.zadatak
\begin{Exercise}[label=712]
Napisati program koji ispisuje sadržaj binarnog pretraživačkog stabla po nivoima. 

\begin{maxitest}
\begin{test}{Test 1}
Poziv: ./a.out
Ulaz: 
	10 5 15 3 2 4 30 12 14 13
Izlaz: 
	0.nivo: 10
	1.nivo: 5 15
	2.nivo: 3 12 30
	3.nivo: 2 4 14
	4.nivo: 13
\end{test}
\end{maxitest}

\end{Exercise}

\begin{Answer}[ref=712]
%\includecode{resenja/07_Drveta/712.c}
\end{Answer}



%13.zadatak
\begin{Exercise}[label=713, difficulty=1]
Dva binarna stabla su {\em slična kao u ogledalu} ako su ili oba prazna ili ako oba nisu prazna i levo podstablo svakog stabla je {\em slično kao u ogledalu} desnom podstablu onog drugog (bitna je struktura stabala, ali ne i njihov sadržaj). Napisati funkciju koja proverava da li su dva binarna pretraživačka stabla {\em slična kao u ogledalu}, a potom i program koji testira rad funkcije nad stablima čiji se elementi unose sa standardnog ulaza sve do unosa broja 0 i to redom za prvo stablo, pa zatim i za drugo stablo. 

\begin{miditest}
\begin{test}{Test 1}
Poziv: ./a.out
Ulaz: 
	Prvo stablo: 11 20 5 3 0
	Drugo stablo: 8 14 30 1 0
Izlaz: 
	Stabla su slicna kao u ogledalu.
\end{test}
\end{miditest}

\begin{miditest}
\begin{test}{Test 2}
Poziv: ./a.out
Ulaz: 
	Prvo stablo: 11 20 5 3 0
	Drugo stablo: 8 20 15 0
Izlaz: 
	Stabla nisu slicna kao u ogledalu.
\end{test}
\end{miditest}
\end{Exercise}

\begin{Answer}[ref=713]
%\includecode{resenja/07_Drveta/713.c}
\end{Answer}


%14.zadatak
\begin{Exercise}[label=714]
AVL-stablo je binarno stablo pretrage kod koga apsolutna razlika visina levog i desnog podstabla svakog elementa
nije veća od jedan. Napisati funkciju \kckod{int avl(Cvor* koren)} koja izračunava broj čvorova stabla sa korenom \argf{koren} koji ispunjavaju uslov za AVL stablo. Napisati zatim i glavni program koji ispisuje rezultat \kckod{avl} funkcije za stablo čiji se elementi unose sa standardnog ulaza sve do kraja ulaza.

\begin{miditest}
\begin{test}{Test 1}
Poziv: ./a.out
Ulaz: 
	10 5 15 2 11 16 1 13
Izlaz: 
	7
\end{test}
\end{miditest}
\begin{miditest}
\begin{test}{Test 2}
Poziv: ./a.out
Ulaz: 
	16 30 40 24 10 18 45 22
Izlaz: 
	6
\end{test}
\end{miditest}
\end{Exercise}

\begin{Answer}[ref=714]
%\includecode{resenja/07_Drveta/714.c}
\end{Answer}


%15.zadatak
\begin{Exercise}[label=715]
Binarno stablo se naziva HEAP ako je kompletno (svaki njegov čvor, izuzev listova, ima i levog i desnog potomka) i za svaki čvor u stablu važi da je njegova vrednost veća od vrednosti svih ostalih čvorova u njegovim podstablima. Napisati funkciju \kckod{int heap(Cvor* koren)} koja proverava da li je dato binarno stablo celih brojeva HEAP. Napisati zatim i glavni program koji ispisuje rezultat \kckod{heap} funkcije za stablo čiji se elementi unose sa standardnog ulaza sve do kraja ulaza.

\begin{miditest}
\begin{test}{Test 1}
Poziv: ./a.out
Ulaz: 
	100 19 36 17 3 25 1 2 7
Izlaz: 
	Stablo je heap.
\end{test}
\end{miditest}

\end{Exercise}

\begin{Answer}[ref=715]
%\includecode{resenja/07_Drveta/715.c}
\end{Answer}


\section{Rešenja}
\shipoutAnswer
