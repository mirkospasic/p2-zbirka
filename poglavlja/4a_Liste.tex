\chapter{Dinamičke strukture podataka}

\section{Liste}

%=========================================================================
\begin{Exercise}[label=4_01]
Napisati biblioteku za rad sa jednostruko povezanom listom čiji čvorovi sadrže cele brojeve. 
\begin{enumerate}
\item Definisati strukturu \kckod{Cvor} kojom se predstavlja čvor liste. Čvor treba da sadrži ceo broj \argf{vrednost} i pokazivač na sledeći čvor liste.

\item Napisati funkciju \kckod{Cvor *napravi\_cvor(int broj)} koja kao argument dobija ceo broj, kreira nov čvor liste, inicijalizuje mu polja i vraća njegovu adresu.

 \item Napisati funkciju \kckod{int dodaj\_na\_pocetak\_liste(Cvor ** adresa\_glave, int broj)} koja dodaje novi čvor sa vrednošću \argf{broj} na početak liste, čija glava se nalazi na adresi \argf{adresa\_glave}.

 \item Napisati funkciju \kckod{Cvor *pronadji\_poslednji(Cvor * glava)} koja pronalazi poslednji čvor u listi.

 \item Napisati funkciju \kckod{int dodaj\_na\_kraj\_liste(Cvor ** adresa\_glave, int broj)} koja dodaje novi čvor sa vrednošću \argf{broj} na kraj liste.

 \item Napisati funkciju \kckod{Cvor *pronadji\_mesto\_umetanja(Cvor * glava, int broj)} koja vraća pokazivač na čvor u neopadajuće uređenoj listi iza kojeg bi trebalo dodati nov čvor sa vrednošću \argf{broj}.

 \item Napisati funkciju \kckod{int dodaj\_iza(Cvor * tekuci, int broj)} koja iza čvora \argf{tekuci} dodaje novi čvor sa vrednošću \argf{broj}.

 \item Napisati funkciju \kckod{int dodaj\_sortirano(Cvor ** adresa\_glave, int broj)} koja dodaje novi elemenat u neopadajuće uređenu listu tako da se očuva postojeće uređenje.

 \item Napisati funkciju \kckod{void ispisi\_listu(Cvor * glava)} koja ispisuje čvorove liste uokvirene zagradama [, ] i međusobno razdvojene zapetama.

 \item Napisati funkciju \kckod{Cvor *pretrazi\_listu(Cvor * glava, int broj)} koja proverava da li se u listi nalazi čvor čija je vrednost jednaka argumentu \argf{broj}. Funkcija vraća pokazivač na pronađeni čvor ili $NULL$ ukoliko ga ne pronađe.

 \item Napisati funkciju \kckod{Cvor *pretrazi\_sortiranu\_listu(Cvor * glava, int broj)} koja proverava da li se u listi nalazi čvor sa vrednošću \argf{broj}, pri čemu se pretpostavlja da se pretražuje neopadajuće uređena lista.

 \item Napisati funkciju \kckod{void obrisi\_cvor(Cvor ** adresa\_glave, int broj)} koja briše sve čvorove u listi koji imaju vrednost jednaku argumentu \argf{broj}.

 \item Napisati funkciju \kckod{void obrisi\_cvor\_sortirane\_liste(Cvor ** adre\-sa\-\_\-glave, int broj)} koja briše sve čvorove u listi koji imaju vrednost jednaku argumentu \argf{broj}, pri čemu se pretpostavlja da se briše iz neopadajuće uređene liste.

 \item Napisati funkciju \kckod{void oslobodi\_listu(Cvor ** adresa\_glave)} koja oslobađa dinamički zauzetu memoriju za čvorove liste.
 \end{enumerate}

Funkcije dodavanja novog elementa u postojeću listu poput, \kckod{dodaj\_na\_pocetak\_liste}, \kckod{dodaj\_na\_kraj\_liste} i \kckod{dodaj\_sortirano}, 
treba da vrate $0$, ukoliko je sve bilo u redu, odnosno $1$, ukoliko se dogodila greška prilikom alokacije memorije za nov čvor.
\napomena{Sve funkcije za rad sa listom implementirati iterativno.}

Napisati programe koji koriste jednostruko povezanu listu za čuvanje elemenata koji se unose sa standardnog ulaza.  Unošenje novih brojeva u listu prekida se učitavanjem kraja ulaza (\kckod{EOF}). Svako dodavanje novog broja u listu ispratiti ispisivanjem trenutnog sadržaja liste. 

\begin{enumerate}
\item[(1)] U programu se učitani celi brojevi dodaju na početak liste. 
    Unosi se ceo broj koji se traži u unetoj listi i na ekran se ispisuje rezultat pretrage. 

\begin{miditest}
\begin{upotreba}{1}
#\naslovInt#
#\izlaz{Unesite brojeve (CTRL+D za kraj unosa): }#
#\ulaz{2}#
#\izlaz{Lista: [2]}#
#\ulaz{3}#
#\izlaz{Lista: [3, 2]}#
#\ulaz{14}#
#\izlaz{Lista: [14, 3, 2]}#
#\ulaz{5}#
#\izlaz{Lista: [5, 14, 3, 2]}#
#\ulaz{3}#
#\izlaz{Lista: [3, 5, 14, 3, 2]}#
#\ulaz{17}#
#\izlaz{Lista: [17, 3, 5, 14, 3, 2]}#

#\izlaz{Unesite broj koji se trazi:} \ulaz{5}#
#\izlaz{Trazeni broj 5 je u listi!}#
\end{upotreba}
\end{miditest}
\begin{miditest}
\begin{upotreba}{2}
#\naslovInt#
#\izlaz{Unesite brojeve (CTRL+D za kraj unosa): }#
#\ulaz{23}#
#\izlaz{Lista: [23]}#
#\ulaz{14}#
#\izlaz{Lista: [14, 23]}#
#\ulaz{35}#
#\izlaz{Lista: [35, 14, 23]}#

#\izlaz{Unesite broj koji se trazi:} \ulaz{8}#
#\izlaz{Broj 8 se ne nalazi u listi!}#
\end{upotreba}
\end{miditest}
% \begin{miditest}
% \begin{upotreba}{3}
% #\poziv{./a.out}#
% 
% #\naslovInt#
% #\izlaz{Unesite brojeve (CTRL+D za kraj unosa): }#
% 
% #\izlaz{Unesite broj koji se trazi:} \ulaz{1}#
% #\izlaz{Broj 1 se ne nalazi u listi!}#
% \end{upotreba}
% \end{miditest}


\item[(2)] U programu se učitani celi brojevi dodaju na kraj liste. 
    Unosi se ceo broj čija se sva pojavljivanja u listi brišu. Na ekran se ispisuje sadržaj liste nakon brisanja.

\begin{miditest}
\begin{upotreba}{1}
#\naslovInt#
#\izlaz{Unesite brojeve (CTRL+D za kraj unosa): }#
#\ulaz{2}#
#\izlaz{Lista: [2]}#
#\ulaz{3}#
#\izlaz{Lista: [2, 3]}#
#\ulaz{14}#
#\izlaz{Lista: [2, 3, 14]}#
#\ulaz{3}#
#\izlaz{Lista: [2, 3, 14, 3]}#
#\ulaz{3}#
#\izlaz{Lista: [2, 3, 14, 3, 3]}#
#\ulaz{17}#
#\izlaz{Lista: [2, 3, 14, 3, 3, 17]}#
#\ulaz{3}#
#\izlaz{Lista: [2, 3, 14, 3, 3, 17, 3]}#

#\izlaz{Unesite broj koji se brise:} \ulaz{3}#
#\izlaz{Lista nakon brisanja:  [2, 14, 17]}#
\end{upotreba}
\end{miditest}
\begin{miditest}
\begin{upotreba}{2}
#\naslovInt#
#\izlaz{Unesite brojeve (CTRL+D za kraj unosa): }#
#\ulaz{23}#
#\izlaz{Lista: [23]}#
#\ulaz{14}#
#\izlaz{Lista: [23, 14]}#
#\ulaz{35}#
#\izlaz{Lista: [23, 14, 35]}#

#\izlaz{Unesite broj koji se brise:} \ulaz{3}#
#\izlaz{Lista nakon brisanja:  [23, 14, 35]}#
\end{upotreba}
\end{miditest}  
% \begin{miditest}
% \begin{upotreba}{3}
% #\poziv{./a.out}#
% 
% #\naslovInt#
% #\izlaz{Unesite brojeve (CTRL+D za kraj unosa): }#
% 
% #\izlaz{Unesite broj koji se brise:} \ulaz{12}#
% #\izlaz{Lista nakon brisanja:  []}#
% \end{upotreba}
% \end{miditest} 


\item[(3)] U programu se učitani celi brojevi dodaju u listu tako da vrednosti budu uređene u neopadajućem poretku. 
    Unosi se ceo broj koji se traži u unetoj listi i na ekran se ispisuje rezultat pretrage. 
    Potom se unosi još jedan ceo broj čija se sva pojavljivanja u listi brišu i prikazuje se aktuelni sadržaj liste nakon brisanja.
    \napomena{Prilikom pretraživanja liste i brisanja čvora liste koristiti činjenicu da je lista uređena.}
    
\begin{miditest}
\begin{upotreba}{1}
#\naslovInt#
#\izlaz{Unesite brojeve (CTRL+D za kraj unosa): }#
#\ulaz{2}#
#\izlaz{Lista: [2]}#
#\ulaz{3}#
#\izlaz{Lista: [2, 3]}#
#\ulaz{14}#
#\izlaz{Lista: [2, 3, 14]}#
#\ulaz{3}#
#\izlaz{Lista: [2, 3, 3, 14]}#
#\ulaz{3}#
#\izlaz{Lista: [2, 3, 3, 3, 14]}#
#\ulaz{5}#
#\izlaz{Lista: [2, 3, 3, 3, 5, 14]}#

#\izlaz{Unesite broj koji se trazi:} \ulaz{14}#
#\izlaz{Trazeni broj 14 je u listi!}#

#\izlaz{Unesite broj koji se brise:} \ulaz{3}#
#\izlaz{Lista nakon brisanja:  [2, 5, 14]}#
\end{upotreba}
\end{miditest}
\begin{miditest}
\begin{upotreba}{2}
#\naslovInt#
#\izlaz{Unesite brojeve (CTRL+D za kraj unosa): }#
#\ulaz{23}#
#\izlaz{Lista: [23]}#
#\ulaz{14}#
#\izlaz{Lista: [14, 23]}#
#\ulaz{35}#
#\izlaz{Lista: [14, 23, 35]}#

#\izlaz{Unesite broj koji se trazi:} \ulaz{8}#
#\izlaz{Broj 8 se ne nalazi u listi!}#

#\izlaz{Unesite broj koji se brise:} \ulaz{3}#
#\izlaz{Lista nakon brisanja:  [14, 23, 35]}#
\end{upotreba}
\end{miditest}
% \begin{miditest}
% \begin{upotreba}{3}
% #\poziv{./a.out}#
% 
% #\naslovInt#
% #\izlaz{Unesite brojeve (CTRL+D za kraj unosa): }#
% 
% #\izlaz{Unesite broj koji se trazi:} \ulaz{1}#
% #\izlaz{Broj 1 se ne nalazi u listi!}#
% 
% #\izlaz{Unesite broj koji se brise:} \ulaz{5}#
% #\izlaz{Lista nakon brisanja:  []}#
% \end{upotreba}
% \end{miditest}

\end{enumerate}
\linkresenje{4_01}
\end{Exercise}
\begin{Answer}[ref=4_01]
\includecodeLib{resenja/biblioteke/lista_iterativna/lista.h}{lista.h}
\includecodeLib{resenja/biblioteke/lista_iterativna/lista.c}{lista.c}
\includecodeLib{resenja/4_DinamickeStrukture/401_i_402/main_a.c}{main\_a.c}
\includecodeLib{resenja/4_DinamickeStrukture/401_i_402/main_b.c}{main\_b.c}
\includecodeLib{resenja/4_DinamickeStrukture/401_i_402/main_c.c}{main\_c.c}
\end{Answer}

%=========================================================================
\skrati{3}
\begin{Exercise}[label=4_02]
Napisati biblioteku za rad sa jednostruko povezanim listama koja sadrži sve funkcije iz zadatka \ref{4_01}, ali tako da funkcije budu implementirane rekurzivno. 
\napomena{Koristiti \kckod{main} programe i test primere iz zadatka \ref{4_01}.}
\linkresenje{4_02}
\end{Exercise}
\begin{Answer}[ref=4_02]
\includecodeLib{resenja/biblioteke/lista_rekurzivna/lista.h}{lista.h}
\includecodeLib{resenja/biblioteke/lista_rekurzivna/lista.c}{lista.c}
\end{Answer}

%========================================================================
\begin{Exercise}[label=4_03]
Napisati program koji prebrojava pojavljivanja etiketa \kckod{HTML}
datoteke čije se ime zadaje kao argument komandne linije. Rezultat prebrojavanja 
ispisati na standardni izlaz. Etikete smeštati u listu, a za formiranje liste koristiti strukturu \kckod{Element} koja sadrži neoznačen broj pojavljivanja etiketi, nisku karaktera koja može da prihvati etiketu veličine do $20$ karaktera i pokazivač na sledeći element liste.
\iffalse
\begin{ckod} 
 typedef struct _Element
 {
   unsigned broj_pojavljivanja;
   char etiketa[20];
   struct _Element *sledeci;
 } Element;
\end{ckod}
\fi

\begin{miditest}
\begin{test}{2}
#\poziv{./a.out datoteka.html}#

#\naslovDat{datoteka.html}#
#\datoteka{<html>}#                       
#\datoteka{  <head><title>Primer</title></head>}#
#\datoteka{  <body>}#
#\datoteka{  \ \ <h1>Naslov</h1>}#
#\datoteka{  \ \ Danas je lep i suncan dan. <br>}#
#\datoteka{  \ \ A sutra ce biti jos lepsi.}#
#\datoteka{  \ \ <a link='http://www.google.com'> Link 1</a>}#     
#\datoteka{  \ \ <a link='http://www.math.rs'> Link 2</a>}# 
#\datoteka{  </body>}# 
#\datoteka{</html>}# 
\end{test}
\end{miditest}
\begin{minitest}
\begin{test2}{1}


#\naslovIzlaz#
#\izlaz{a - 4}#
#\izlaz{br - 1}#
#\izlaz{h1 - 2}#
#\izlaz{body - 2}#
#\izlaz{title - 2}#
#\izlaz{head - 2}#
#\izlaz{html - 2}#
\end{test2}
\end{minitest}

\begin{miditest}
\begin{test}{1}
#\poziv{./a.out datoteka.html}#

#\naslovDat{Datoteka datoteka.html ne postoji.}#

#\naslovIzlazZaGresku#
#\izlaz{Greska: Neuspesno otvaranje}#
#\izlaz{datoteke datoteka.html.}#
\end{test}
\end{miditest}

% \begin{miditest}
% \begin{test}{3}
% #\poziv{./a.out}#
% 
% #\naslovIzlazZaGresku#
% #\izlaz{Greska: Program se poziva }#
% #\izlaz{sa: ./a.out datoteka.html!}#
% \end{test}
% \end{miditest}
% \begin{miditest}
% \begin{test}{4}
% #\poziv{./a.out datoteka.html}#
% 
% #\naslovDat{datoteka.html}#
% #\datoteka{Datoteka je prazna.}#
% \end{test}
% \end{miditest}
\linkresenje{4_03}
\end{Exercise}
\begin{Answer}[ref=4_03]
\includecode{resenja/4_DinamickeStrukture/4_03.c}
\end{Answer}

%=========================================================================
\begin{Exercise}[label=4_04]
U datoteci se nalaze podaci o studentima. U svakom redu datoteke nalazi se indeks, ime i prezime studenta. 
Napisati program kome se preko argumenata komandne linije prosleđuje ime datoteke sa studentskim podacima koje program treba da pročita i smesti u listu. 
Nakon završenog učitavanja svih podataka o studentima, sa standardnog ulaza unose se, jedan po jedan, indeksi studenata koji se traže u učitanoj listi. 
Posle svakog unetog indeksa, program ispisuje poruku $da$ ili $ne$, u zavisnosti od toga da li u listi postoji student sa unetim indeksom ili ne. 
Prekid unosa indeksa se vrši unošenjem karaktera za kraj ulaza (\kckod{EOF}). Poruke o greškama ispisivati na standardni izlaz za greške.
\uputstvo{Pretpostaviti da je $10$ karaktera dovoljno za zapis indeksa i da je $20$ karaktera maksimalna dužina bilo imena bilo prezimena studenta.}

% Dodavanje novog studenta izdvojiti u funkciju \kckod{void dodaj\_na\_pocetak\_liste(Cvor** glava, char* broj\_indeksa, char* ime, char* prezime)}.
% Napisati rekurzivnu funkciju \kckod{Cvor* pretrazi\_listu(Cvor* glava, char* broj\_indeksa)} 
% koja određuje da li student sa zadatim brojem indeksa pripada listi ili ne.
% Nakon završenog pretraživanja rekurzivnom funkcijom \kckod{void oslobodi\_listu(Cvor** glava)} osloboditi memoriju koju je lista sa studentima zauzimala.


\begin{miditest}
\begin{upotreba}{1}
#\poziv{./a.out studenti.txt}#

#\naslovDat{studenti.txt}#
#\datoteka{123/2014 Marko Lukic}#
#\datoteka{3/2014   Ana Sokic}#
#\datoteka{43/2013  Jelena Ilic}#
#\datoteka{41/2009  Marija Zaric}#
#\datoteka{13/2010  Milovan Lazic}#

#\naslovInt#           
#\ulaz{3/2014} \izlaz{da: Ana Sokic}#
#\ulaz{235/2008} \izlaz{ne}#
#\ulaz{41/2009} \izlaz{da: Marija Zaric}#
\end{upotreba}
\end{miditest}
\begin{miditest}
\begin{upotreba}{2}
#\poziv{./a.out  studenti.txt}#

#\naslovDat{Datoteka studenti.txt je prazna}#

#\naslovInt#
#\ulaz{3/2014} \izlaz{ne}#
#\ulaz{235/2008} \izlaz{ne}#
#\ulaz{41/2009} \izlaz{ne}#
\end{upotreba}
\end{miditest}
% \begin{miditest}
% \begin{upotreba}{3}
% #\poziv{./a.out}#
% 
% #\naslovIzlazZaGresku# 
% #\izlaz{Greska: Program se poziva sa: }#
% #\izlaz{./a.out ime_datoteke}#
% \end{upotreba}
% \end{miditest}
% \begin{miditest}
% \begin{upotreba}{4}
% #\poziv{./a.out  studenti.txt}#
% 
% #\naslovDat{Datoteka studenti.txt ne postoji.}#
% 
% #\naslovIzlazZaGresku#
% #\izlaz{Greska: Neuspesno otvaranje datoteke}#
% #\izlaz{studenti.txt.}#
% \end{upotreba}
% \end{miditest}  

\linkresenje{4_04}
\end{Exercise}
\begin{Answer}[ref=4_04]
\includecode{resenja/4_DinamickeStrukture/4_04.c}
\end{Answer}


%=========================================================================

\begin{Exercise}[difficulty=1,label=4_05]
Data je datoteka \kckod{brojevi.txt} koja sadrži cele brojeve.
\begin{enumerate}
 \item Napisati funkciju koja iz zadate datoteke učitava brojeve i smešta ih u listu.
 \item Napisati funkciju koja u jednom prolazu kroz zadatu listu celih brojeva 
pronalazi maksimalan strogo rastući podniz.
\end{enumerate}
Napisati program koji u datoteku \kckod{rezultat.txt} upisuje nađeni strogo rastući podniz.
%\komentar{Milena: I ovde me muci sto bi zadatak mogao da se resi i bez koriscenja listi... }


\begin{minitest}
\begin{test}{1}
#\naslovDat{brojevi.txt}#
#\datoteka{43 12 15 16 4 2 8}#

#\naslovIzlaz#
#\naslovDat{rezultat.txt}#
#\datoteka{12 15 16}#
\end{test}
\end{minitest}
\begin{minitest}
\begin{test}{2}
#\naslovDat{Datoteka brojevi.txt}#
#\naslovDat{ne postoji.}#

#\naslovIzlazZaGresku#
#\datoteka{Greska: Neuspesno otvaranje}#
#\datoteka{datoteke brojevi.txt.}#
\end{test}
\end{minitest}
\begin{minitest}
\begin{test}{3}
#\naslovDat{Datoteka brojevi.txt je prazna}#

#\naslovIzlaz#
#\naslovDat{rezultat.txt}#
#\datoteka{Rezultat.txt ce biti prazna.}#
\end{test}
\end{minitest}
\end{Exercise}

%\begin{Answer}[ref=4_05]
% \includecode{resenja/06_Liste/4_05.C}
%\end{Answer}


%=========================================================================

\begin{Exercise}[difficulty=1, label=4_06]
%\komentar{Milena: malo me muci u ovom zadatku sto nema neki smisao. Naime, ako se samo vrsi ucitavanje iz datoteka i ispisivanje,  onda su ove liste zapravo visak jer isti rezultat moze da se dobije i bez koriscenja listi. Zato mi fali da program uradi nesto sto ne bi mogao da uradi bez koriscenja listi, npr da na osnovu unetog broja ispisuje svaki n-ti broj rezultujuce liste pa to u nekoj petlji da korisnik moze da ispisuje za razlicite unete n ili tako nesto... }
Napisati program koji objedinjuje dve sortirane liste u jednu sortiranu listu. Funkcija ne treba da 
kreira nove, već da samo preraspodeli postojeće čvorove. Prva lista se učitava iz datoteke čije ime se zadaje kao prvi argument komandne linije, a druga iz datoteke čije se ime zadaje kao drugi argument komandne linije. Rezultujuću listu ispisati na standardni izlaz.


\begin{miditest}
\begin{test}{1}
#\poziv{./a.out dat1.txt dat2.txt}#

#\naslovDat{dat1.txt}#
#\datoteka{2 4 6 10 15}#

#\naslovDat{dat2.txt}#
#\datoteka{5 6 11 12 14 16}#

#\naslovIzlaz#
#\izlaz{[2, 4, 5, 6, 6, 10, 11, 12, 14, 15, 16]}#
\end{test}
\end{miditest}
\begin{miditest}
\begin{test}{2}
#\poziv{./a.out dat1.txt dat2.txt}#

#\naslovDat{dat1.txt}#
#\datoteka{2 4 6 10 15}#

#\naslovDat{Datoteka dat2.txt ne postoji.}#

#\naslovIzlazZaGresku#
#\izlaz{Greska: Neuspesno otvaranje }#
#\izlaz{datoteke dat2.txt.}#
\end{test}
\end{miditest}
% \begin{miditest}
% \begin{test}{3}
% #\poziv{./a.out dat1.txt dat2.txt}#
% 
% #\naslovDat{Datoteka dat1.txt ne postoji.}#
% 
% #\naslovDat{dat2.txt}#
% #\datoteka{5 6 11 12 14 16}#
% 
% #\naslovIzlazZaGresku#
% #\izlaz{Greska: Neuspesno otvaranje }#
% #\izlaz{datoteke dat1.txt.}#
% \end{test}
% \end{miditest}  
% \begin{miditest}
% \begin{test}{4}
% #\poziv{./a.out dat1.txt dat2.txt}#
% 
% #\naslovDat{dat1.txt}#
% #\datoteka{2 4 6 10 15}#
% 
% #\naslovDat{dat2.txt}#
% #\datoteka{Datoteka je prazna.}#
% 
% #\naslovIzlaz#
% #\izlaz{[2, 4, 6, 10, 15]}#
% \end{test}
% \end{miditest}

\begin{miditest}
\begin{test}{3}
#\poziv{./a.out dat1.txt dat2.txt}#

#\naslovDat{Datoteka dat1.txt je prazna}#

#\naslovDat{dat2.txt}#
#\datoteka{5 6 11 12 14 16}#

#\naslovIzlaz#
#\izlaz{[5, 6, 11, 12, 14, 16]}#
\end{test}
\end{miditest}
\begin{miditest}
\begin{test}{4}
#\poziv{./a.out dat1.txt}#

#\naslovIzlazZaGresku# 
#\izlaz{Greska: Program se poziva sa:}# 
#\izlaz{./a.out dat1.txt dat2.txt!}#
\end{test}
\end{miditest}

\linkresenje{4_06}  
\end{Exercise}
\begin{Answer}[ref=4_06]
\\
\napomena{Rešenje koristi biblioteku za rad sa listama iz zadatka \ref{4_01}.}
\includecode{resenja/4_DinamickeStrukture/4_06.c}
%\ifpdf \else \newpage \fi
\end{Answer}

%% Dodatni
%=========================================================================
\begin{Exercise}[difficulty=1,label=4_07]
Date su dve jednostruko povezane liste $L1$ i $L2$. Napisati funkciju koja od 
ovih listi formira novu listu $L$ koja sadrži naizmenično raspoređene čvorove 
listi $L1$ i $L2$: prvi čvor iz $L1$, prvi čvor iz $L2$, drugi čvor $L1$,
drugi čvor $L2$, itd. Ne formirati nove čvorove, već samo postojeće rasporediti u jednu listu. Prva lista se učitava iz datoteke čije se ime zadaje kao prvi argument komandne linije, a druga iz datoteke čije se ime zadaje kao 
drugi argument komandne linije. Rezultujuću listu ispisati na standardni izlaz.\\
%\komentar{Milena: Sta ako je neka lista duza? To precizirati. I ovde me muci sto nedostaje neki smisao zadatku, nesto sto ne bi moglo da se uradi da nismo koristili liste. }
\napomena{Koristiti testove 2 - 6 za zadatak \ref{4_06}.}

\begin{miditest}
\begin{test}{1}
#\poziv{./a.out dat1.txt dat2.txt}#

#\naslovDat{dat1.txt}#
#\datoteka{2 4 6 10 15}#

#\naslovDat{dat2.txt}#
#\datoteka{5 6 11 12 14 16}#

#\naslovIzlaz#
#\izlaz{2 5 4 6 6 11 10 12 15 14 16}#
\end{test}
\end{miditest}
\end{Exercise}
%\begin{Answer}[ref=4_07]
% \includecode{resenja/06_Liste/4_07.c}
%\end{Answer}

%=========================================================================
\begin{Exercise}[label=4_08]
Sadržaj datoteke je aritmetički izraz koji može sadržati zagrade \{, [ i (. 
Napisati program koji učitava sadržaj datoteke \kckod{izraz.txt} i korišćenjem steka 
utvrđuje da li su zagrade u aritmetičkom izrazu dobro uparene. Program štampa odgovarajuću poruku na standardni izlaz.

\begin{miditest}
\begin{test}{1}
#\naslovDat{izraz.txt}#
#\datoteka{\{[23 + 5344] * (24 - 234)\} - 23}#
  
#\naslovIzlaz#
#\izlaz{Zagrade su ispravno uparene.}#
\end{test}
\end{miditest}
\begin{miditest}
\begin{test}{2}
#\naslovDat{izraz.txt}#
#\datoteka{\{[23 + 5] * (9 * 2)\} - \{23\}}#

#\naslovIzlaz#
#\izlaz{Zagrade su ispravno uparene.}# 
\end{test}
\end{miditest}

\begin{miditest}
\begin{test}{3}
#\naslovDat{izraz.txt}#
#\datoteka{\{[2 + 54) / (24 * 87)\} + (234 + 23)}#

#\naslovIzlaz#
#\izlaz{Zagrade nisu ispravno uparene.}#
\end{test}
\end{miditest}
\begin{miditest}
\begin{test}{4}
#\naslovDat{izraz.txt}#
#\datoteka{\{(2 - 14) / (23 + 11)\}\} * (2 + 13)}#

#\naslovIzlaz#
#\izlaz{Zagrade nisu ispravno uparene.}#
\end{test}
\end{miditest}

\begin{miditest}
\begin{test}{5}
#\naslovDat{Datoteka izraz.txt je prazna}#

#\naslovIzlaz#
#\izlaz{Zagrade su ispravno uparene.}#
\end{test}
\end{miditest}
\begin{miditest}
\begin{test}{6}
#\naslovDat{Datoteka izraz.txt ne postoji.}# 

#\naslovIzlazZaGresku#
#\izlaz{Greska: Neuspesno otvaranje }#
#\izlaz{datoteke izraz.txt!}#
\end{test}
\end{miditest}
\linkresenje{4_08}
\end{Exercise}
\begin{Answer}[ref=4_08]
\includecode{resenja/4_DinamickeStrukture/4_08.c}
\ifpdf \else \newpage \fi
\end{Answer}

%=========================================================================
\begin{Exercise}[label=4_09]
Napisati program koji proverava ispravnost uparivanja etiketa u \kckod{HTML} datoteci. Ime datoteke se zadaje kao argument komandne linije.
Poruke o greškama ispisivati na standardni izlaz za greške.
\uputstvo{Za rešavanje problema koristiti stek implementiran preko liste čiji čvorovi sadrže \kckod{HTML} etikete.}


\begin{miditest}      
\begin{test}{1}
#\poziv{./a.out datoteka.html}#

#\naslovDat{datoteka.html}#                     
#\datoteka{<html>}#                                
#\datoteka{  <head>}#
#\datoteka{  \ \ <title>Primer</title>}#
#\datoteka{  </head>}#
#\datoteka{  <body>}#
#\datoteka{  </body>}#

#\naslovIzlaz#
#\izlaz{Etikete nisu pravilno uparene}#
#\izlaz{(etiketa <html> nije zatvorena)}# 
\end{test}
\end{miditest}
\begin{miditest}
\begin{test}{2}
#\poziv{./a.out datoteka.html}#

#\naslovDat{datoteka.html}#                                                 
#\datoteka{  <head>}#
#\datoteka{  \ \ <title>Primer</title>}#
#\datoteka{  </head>}#
#\datoteka{  <body>}#
#\datoteka{  </body>}#
#\datoteka{</html>}#  

#\naslovIzlaz#
#\izlaz{Etikete nisu pravilno uparene}#
#\izlaz{(nadjena je etiketa </html>}#
#\izlaz{koja nije otvorena)}# 
\end{test}
\end{miditest}

\begin{miditest}
\begin{test}{3}
#\poziv{./a.out datoteka.html}#

#\naslovDat{datoteka.html}#                     
#\datoteka{<html>}#
#\datoteka{  <head>}#
#\datoteka{  \ \ <title>Primer</title>}#
#\datoteka{  </head>}#
#\datoteka{  <body>}#
#\datoteka{  \ \ <h1>Naslov</h1>}#
#\datoteka{  \ \ Danas je lep i suncan dan. <br>}#
#\datoteka{  \ \ Sutra ce biti jos lepsi.}#
#\datoteka{  \ \ <a link='http://www.math.rs'>Link</a>}#
#\datoteka{  </body>}#
#\datoteka{</html>}#

#\naslovIzlaz#
#\izlaz{Etikete su pravilno uparene!}#
\end{test}
\end{miditest}
\begin{miditest}    
\begin{test}{4}
#\poziv{./a.out datoteka.html}#

#\naslovDat{datoteka.html}#                     
#\datoteka{<html>}#
#\datoteka{  <head>}#
#\datoteka{  \ \ <title>Primer</title>}#
#\datoteka{  </head>}#
#\datoteka{  <body>}#
#\datoteka{</html>}#

#\naslovIzlaz#
#\izlaz{Etikete nisu pravilno uparene}#
#\izlaz{(nadjena je etiketa </html>, a}#
#\izlaz{ poslednja otvorena je <body>)}#

\end{test}
\end{miditest}

\begin{miditest}
\begin{test}{5}
#\poziv{./a.out datoteka.html}#

#\naslovDat{Datoteka datoteka.html ne postoji.}#

#\naslovIzlazZaGresku#
#\izlaz{Greska: Neuspesno otvaranje}# 
#\izlaz{datoteke datoteka.html.}#
\end{test}
\end{miditest}
\begin{miditest}
\begin{test}{6}
#\poziv{./a.out datoteka.html}#

#\naslovDat{datoteka.html je prazna}#                     

#\naslovIzlaz#
#\izlaz{Etikete su pravilno uparene!}#
\end{test}
\end{miditest}
\linkresenje{4_09}
\end{Exercise}
\begin{Answer}[ref=4_09]
\includecodeLib{resenja/biblioteke/stek/stek.h}{stek.h}
\includecodeLib{resenja/biblioteke/stek/stek.c}{stek.c}
\includecodeLib{resenja/4_DinamickeStrukture/4_09.c}{main.c}
\ifpdf \else \newpage \fi
\end{Answer}
%=========================================================================
\begin{Exercise}[label=4_10]
Napisati program koji pomaže službeniku u radu na šalteru.
Službenik najpre evidentira sve korisničke $JMBG$ brojeve (niske koje sadrže po $13$ karaktera) i zahteve (niska koja sadrži najviše $999$ karaktera). 
Prijem zahteva korisnika se prekida unošenjem karaktera za kraj ulaza (\kckod{EOF}).
Službenik redom pregleda zahteve i odlučuje da li zahtev obrađuje odmah ili kasnije. Program mu postavlja pitanje 
\kckod{Da li korisnika  vracate na kraj reda?} i ukoliko on da odgovor $Da$, 
korisnik se stavlja na kraj reda, čime se obrada njegovog zahteva odlaže. Ukoliko odgovor nije $Da$, službenik obrađuje zahtev i podatke o korisniku dopisuje na kraj datoteke \kckod{izvestaj.txt}. Ova datoteka, za svaki obrađen zahtev, sadrži $JMBG$ i zahtev usluženog korisnika.
Posle svakog $petog$ usluženog korisnika, službeniku se nudi mogućnost da prekine sa radom, nevezano od broja korisnika koji i dalje čekaju u redu. 
\uputstvo{Za čuvanje korisničkih zahteva koristiti red implementiran korišćenjem listi.}

\begin{maxitest}
\begin{upotreba}{1}
#\naslovInt#
#\izlaz{Sluzbenik evidentira korisnicke zahteve:}# 
#\izlaz{Novi zahtev [CTRL+D za kraj]}#
#\izlaz{  JMBG:} \ulaz{1234567890123}#
#\izlaz{  Opis problema:} \ulaz{Otvaranje racuna}#

#\izlaz{Novi zahtev [CTRL+D za kraj]}#
#\izlaz{  JMBG:} \ulaz{2345678901234}#
#\izlaz{  Opis problema:} \ulaz{Podizanje novca}#

#\izlaz{Novi zahtev [CTRL+D za kraj]}#
#\izlaz{  JMBG:} \ulaz{3456789012345}#
#\izlaz{  Opis problema:} \ulaz{Reklamacija}#

#\izlaz{Novi zahtev [CTRL+D za kraj]}#
#\izlaz{  JMBG:}#

#\izlaz{Sledeci je korisnik sa JMBG: 1234567890123}#
#\izlaz{i zahtevom: Otvaranje racuna}#
#\izlaz{  Da li ga vracate na kraj reda? [Da/Ne]} \ulaz{Da}#

#\izlaz{Sledeci je korisnik sa JMBG: 2345678901234}#
#\izlaz{i zahtevom: Podizanje novca}#
#\izlaz{  Da li ga vracate na kraj reda? [Da/Ne]} \ulaz{Ne}#

#\izlaz{Sledeci je korisnik sa JMBG: 3456789012345}#
#\izlaz{i zahtevom: Reklamacija}#
#\izlaz{  Da li ga vracate na kraj reda? [Da/Ne]} \ulaz{Da}#

#\izlaz{Sledeci je korisnik sa JMBG: 1234567890123}#
#\izlaz{i zahtevom: Otvaranje racuna}#
#\izlaz{  Da li ga vracate na kraj reda? [Da/Ne]} \ulaz{Da}#

#\izlaz{Sledeci je korisnik sa JMBG: 3456789012345}#
#\izlaz{i zahtevom: Reklamacija}#
#\izlaz{  Da li ga vracate na kraj reda? [Da/Ne]} \ulaz{Ne}#

#\izlaz{Da li je kraj smene? [Da/Ne]} \ulaz{Ne}#

#\izlaz{Sledeci je korisnik sa JMBG: 1234567890123}#
#\izlaz{i zahtevom: Otvaranje racuna}#
#\izlaz{  Da li ga vracate na kraj reda? [Da/Ne]} \ulaz{Ne}#

#\naslovDat{izvestaj.txt}#
#\datoteka{  JMBG: 2345678901234     Zahtev: Podizanje novca}#
#\datoteka{  JMBG: 3456789012345     Zahtev: Reklamacija}#
#\datoteka{  JMBG: 1234567890123     Zahtev: Otvaranje racuna}#
\end{upotreba}
\end{maxitest}
\linkresenje{4_10}
\end{Exercise}
\begin{Answer}[ref=4_10]
\includecodeLib{resenja/biblioteke/red/red.h}{red.h}
\includecodeLib{resenja/biblioteke/red/red.c}{red.c}
\includecodeLib{resenja/4_DinamickeStrukture/4_10.c}{main.c}
\ifpdf \else \newpage \fi
\end{Answer}

%=========================================================================
\begin{Exercise}[label=4_11]
Napisati biblioteku za rad sa dvostruko povezanom listom celih brojeva koja ima iste funkcionalnosti kao biblioteka iz zadatka \ref{4_01}. 
Dopuniti bibilioteku novim funkcijama.
\begin{enumerate}
 \item Napisati funkciju \kckod{void obrisi\_tekuci(Cvor ** adresa\_glave, Cvor ** adresa\_kraja, Cvor * tekuci)} koja briše čvor na koji pokazuje pokazivač \argf{tekuci} iz liste čiji se pokazivač na čvor koji je glava liste nalazi na adresi \argf{adresa\_glave} i poslednji čvor liste na adresi \argf{adresa\_kraja}.
 \item Napisati funkciju \kckod{void ispisi\_listu\_unazad(Cvor * kraj)} koja ispisuje sadržaj liste od poslednjeg čvora ka glavi liste.
\end{enumerate}

Sve funkcije za rad sa listom implementirati iterativno. Zbog efikasnog izvršavanja operacija dodavanja na kraj liste i ispisivanja liste unazad treba, pored pokazivača na glavu liste, čuvati i pokazivač na poslednji čvor liste.
\napomena{Koristiti test primere iz zadatka \ref{4_01}}
\linkresenje{4_11}
\end{Exercise}
\begin{Answer}[ref=4_11]
\includecodeLib{resenja/biblioteke/dvostruko_povezana_lista/dvostruko_povezana_lista.h}{dvostruko\_povezana\_lista.h}
\includecodeLib{resenja/biblioteke/dvostruko_povezana_lista/dvostruko_povezana_lista.c}{dvostruko\_povezana\_lista.c}
\includecodeLib{resenja/4_DinamickeStrukture/4_11_a.c}{main\_a.c}
\includecodeLib{resenja/4_DinamickeStrukture/4_11_b.c}{main\_b.c}
\includecodeLib{resenja/4_DinamickeStrukture/4_11_c.c}{main\_c.c}
\end{Answer}


%=========================================================================
\begin{Exercise}[difficulty=1,label=4_12]
Grupa od $n$ plesača na kostimima ima brojeve od $1$ do $n$. 
Plesači najpre formiraju krug tako da brojevi sa njihovih kostima rastu u smeru kazaljke na satu. Plesač sa brojem $1$ stavlja levu ruku na rame plesača sa brojem $2$, a desnu na svoj kuk i tako redom. Plesač sa brojem $n$ svoju levu ruku spušta na rame plesača sa brojem $1$, a desnu na svoj kuk i tako zatvara krug.
Svoju plesnu tačku izvode tako što iz formiranog kruga najpre izlazi $k$-ti plesač. 
Odbrojava se počevši od plesača označenog brojem $1$ u smeru kretanja kazaljke na satu. 
Preostali plesači obrazuju manji krug tako što $k-1$-vi stavlja ruku na rame $k+1$-og i zatvara krug iz kog opet izlazi $k$-ti plesač. Odbrojavanje sada počinje od
sledećeg suseda prethodno izbačenog, opet u smeru kazaljke na satu. Izlasci iz kruga se nastavljaju
sve dok svi plesači ne budu isključeni. 
Celi brojevi $n$, $k$ ($k < n$) se učitavaju sa standardnog ulaza. 
Napisati program koji će na standardni izlaz ispisati redne brojeve plesača u redosledu napuštanja kruga. 
\uputstvo{Pri implementaciji koristiti jednostruko povezanu kružnu listu.}


\begin{minitest}
\begin{test}{1}
#\naslovUlaz#
#\ulaz{5 3}#

#\naslovIzlaz# 
#\izlaz{3 1 5 2 4}#
\end{test}
\end{minitest}
\begin{minitest}
\begin{test}{2}
#\naslovUlaz#
#\ulaz{8 4}#

#\naslovIzlaz# 
#\izlaz{4 8 5 2 1 3 7 6}# 
\end{test}
\end{minitest}
\begin{minitest}
\begin{test}{3}
#\naslovUlaz#
#\ulaz{3 8}#

#\naslovIzlazZaGresku# 
#\izlaz{Greska: n mora biti uvek vece}#
#\izlaz{od k, a 3 < 8!}#
\end{test}
\end{minitest}
\end{Exercise}
%\begin{Answer}[ref=4_12]
% \includecode{resenja/06_Liste/4_12.c}
%\end{Answer}

%=========================================================================
\begin{Exercise}[difficulty=1,label=613]
Grupa od $n$ plesača na kostimima ima brojeve od $1$ do $n$. 
Plesači najpre formiraju krug tako da brojevi sa njihovih kostima rastu u smeru kazaljke na satu. Svaki plesač levu ruku stavlja na rame plesača sa sledećim većim brojem, a desnu na rame plesača sa prvim manjim brojem. Plesač sa brojem $1$ stavlja levu ruku na rame plesača sa brojem $2$, a desnu na rame plesača sa brojem $n$. Plesač sa brojem $n$ svoju desnu ruku spušta na rame plesača sa brojem $n-1$, a levu na rame plesača sa brojem $1$ i tako zatvara krug.
Plesači izvode svoju plesnu tačku tako što iz formiranog kruga najpre izlazi $k$-ti plesač.
Odbrojava se počevši od plesača označenog brojem $1$ u smeru kretanja kazaljke na satu. 
Preostali plesači obrazuju manji krug iz kog opet izlazi $k$-ti plesač. Odbrojavanje sada počinje od
sledećeg suseda prethodno izbačenog, uz promenu smera. Ukoliko se prilikom prethodnog izbacivanja odbrojavalo 
u smeru kazaljke na satu sada će se obrojavati u suprotnom smeru, i obrnuto. Izlasci iz kruga se nastavljaju
sve dok svi plesači ne budu isključeni. 
Celi brojevi $n$, $k$ ($k < n$) se učitavaju sa standardnog ulaza. 
Napisati program koji će na standardni izlaz ispisati redne brojeve plesača u redosledu napuštanja kruga. 
\uputstvo{Pri implementaciji koristiti dvostruko povezanu kružnu listu.}
%\napomena{Iskoristiti test 3 iz \ref{4_12}. zadatka.}

\begin{minitest}
\begin{test}{1}
#\naslovUlaz#
#\ulaz{5 3}#

#\naslovIzlaz# 
#\izlaz{3 5 4 2 1}#
\end{test}
\end{minitest}
\begin{minitest}
\begin{test}{2}
#\naslovUlaz#
#\ulaz{8 4}#

#\naslovIzlaz# 
#\izlaz{4 8 5 7 6 3 2 1}# 
\end{test}
\end{minitest}
\begin{minitest}
\begin{test}{3}
#\naslovUlaz#
#\ulaz{5 8}#

#\naslovIzlazZaGresku# 
#\izlaz{Greska: n mora biti uvek vece}#
#\izlaz{od k, a 5 < 8!}#
\end{test}
\end{minitest}
\end{Exercise}
%\begin{Answer}[ref=613]
% \includecode{resenja/06_Liste/4_13.c}
%\end{Answer}

