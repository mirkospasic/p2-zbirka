\section{Rekurzija}

%%%\subsection{Rekurzivne funkcije nad brojevima}

\begin{Exercise}[label=102]
Napisati rekurzivnu funkciju koja izračunava  $x^k$,  za dati ceo broj $x$ i prirodan broj $k$.
Napisati program koji testira napisanu funkciju za vrednosti koje se unose sa standardnog ulaza.
 
\begin{minitest}
\begin{test}{Test 1}
Ulaz:   2 10
Izlaz:  1024
\end{test}
\end{minitest}
\linkresenje{102}
\end{Exercise}
\begin{Answer}[ref=102]
\includecode{resenja/01_Rekurzija/102.c}
\end{Answer}

\begin{Exercise}[label=104]
 Koristeći uzajamnu (posrednu) rekurziju napisati naredne dve funkcije:
 \begin{itemize}
\item fukciju \kckod{paran} koja proverava da li je broj cifara nekog broja paran i vraća 1 ako jeste, a 0 inače;
\item i fukciju \kckod{neparan} koja vraća 1 ukoliko je broj cifara nekog broja neparan, a 0 inače.
 \end{itemize}
 Napisati program koji testira napisanu funkciju tako što se za heksadekadnu vrednost koja se unosi sa standardnog ulaza ispisuje da li je paran ili neparan.
 
\begin{miditest}
\begin{test}{Test 1}
Ulaz:  11
Izlaz: Uneti broj ima paran broj cifara
\end{test}
\end{miditest}
\begin{miditest}
\begin{test}{Test 2}
Ulaz:  123
Izlaz: Uneti broj ima neparan 
       broj cifara
\end{test}
\end{miditest}

\linkresenje{104}
\end{Exercise}
\begin{Answer}[ref=104]
\includecode{resenja/01_Rekurzija/104.c}
\end{Answer}

\begin{Exercise}[label=105]
  Napisati repno-rekurzivnu funkciju koja izračunava faktorijel broja $n$. Napisati program koji testira napisanu funkciju za proizvoljan broj $n$ ($n \le 12$) unet sa standardnog ulaza.
  
\begin{miditest}
\begin{test}{Test 1}
Ulaz:   Unesite n (<= 12): 5
Izlaz:  5! = 120
\end{test}
\end{miditest}

\linkresenje{105}
\end{Exercise}
\begin{Answer}[ref=105]
\includecode{resenja/01_Rekurzija/105.c}
\end{Answer}


\begin{Exercise}[label=106]
Elementi funkcije $F$ izračunavaju se na osnovu sledećih rekurentnih relacija:
 $$F(0) = 0$$
 $$F(1) = 1$$
 $$F(n) = a* F(n-1) + b*F(n-2)$$
%\begin{enumerate}
%\item Napisati rekurzivnu funkciju  koja računa $n$-ti element u nizu $F$.
%\item 
Napisati rekurzivnu funkciju  koja računa $n$-ti element u nizu $F$ ali tako da se problemi manje dimenzije rešavaju samo jedan put.
%\end{enumerate}
Napisati program koji testira napisane funkcije za poizvoljan broj $n$ ($n \in \mathbb N$) unet sa standardnog ulaza.
  
\begin{miditest}
\begin{test}{Test 1}
Ulaz:   Unesi koeficijente
        2 3
        Unesi koji clan niza racunamo
        5
Izlaz:  F(5) = 61
\end{test}
\end{miditest}

\linkresenje{106}
\end{Exercise}
\begin{Answer}[ref=106]
%\includecode{resenja/01_Rekurzija/106.c}
\end{Answer}

\begin{Exercise}[label=111]
Napisati rekurzivnu funkciju koja sabira dekadne cifre datog
  celog broja $x$. Napisati program koji testira ovu funkciju, za broj
  koji se unosi sa standardnog ulaza.
  
\begin{minitest}
\begin{test}{Test 1}
Ulaz:    123  
Izlaz:   6 
\end{test}
\end{minitest}
\begin{minitest}
\begin{test}{Test 2}
Ulaz:    23156    
Izlaz:   17 
\end{test}
\end{minitest}
\begin{minitest}
\begin{test}{Test 3}
Ulaz:   1432
Izlaz:   10        
\end{test}
\end{minitest}

\begin{minitest}
\begin{test}{Test 4}
Ulaz:   1       
Izlaz:  1       
\end{test}
\end{minitest}
\begin{minitest}
\begin{test}{Test 5}
Ulaz:   0
Izlaz:  0
\end{test}
\end{minitest}

\linkresenje{111}
\end{Exercise}
\begin{Answer}[ref=111]
%\includecode{resenja/01_Rekurzija/111.c}
\end{Answer}

%%%\subsection{Rekurzivne funkcije za rad sa nizovima}

\begin{Exercise}[label=101]
Napisati rekurzivnu funkciju koja sumira elemente niza celih brojeva.  Napisati program koji testira napisane funkcije. Sa standardnog ulaza učitati dimenziju $n$  ($0 < n \leq 100$) celobrojnog
niza, a zatim i elemente niza. Na standardni izlaz ispisati
rezultat primene napisane funkcije nad učitanim
nizom.

\begin{miditest}
\begin{test}{Test 1}
Ulaz:  5 1 2 3 4 5   
Izlaz: Suma elemenata je 15
\end{test}
\end{miditest}

\linkresenje{101}
\end{Exercise}
\begin{Answer}[ref=101]
\includecode{resenja/01_Rekurzija/101.c}
\end{Answer}

\begin{Exercise}[label=109]
Napisati rekurzivnu funkciju koja određuje maksimum niza
  celih brojeva. Napisati program koji testira ovu funkciju za niz
  koji se unosi sa standardnog ulaza. Niz neće imati više od $256$ elemenata, i
  njegovi elementi se unose sve do kraja ulaza.
  
\begin{miditest}
\begin{test}{Test 1}
Ulaz:   3 2 1 4 21    
Izlaz: 21              
\end{test}
\end{miditest}
\begin{miditest}
\begin{test}{Test 2}
Ulaz:   2 -1 0 -5 -10
Izlaz: 2                
\end{test}
\end{miditest}

\begin{miditest}
\begin{test}{Test 3}
Ulaz:  1 11 3 5 8 1    
Izlaz: 11
\end{test}
\end{miditest}
\begin{minitest}
\begin{test}{Test 4}
Ulaz:   5
Izlaz:  5
\end{test}
\end{minitest}

\linkresenje{109}
\end{Exercise}
\begin{Answer}[ref=109]
\includecode{resenja/01_Rekurzija/109.c}
\end{Answer}

\begin{Exercise}[label=110]
Napisati rekurzivnu funkciju \kckod{skalarno} koja izračunava skalarni
  proizvod dva data vektora.  Napisati program koji testira ovu
  funkciju, za nizove koji se unose sa standardnog ulaza. Nizovi neće imati više od $256$ elemenata. Prvo se unosi dimenzija nizova, a zatim i sami njihovi elementi.
  
\begin{miditest}
\begin{test}{Test 1}
Ulaz:    3 1 2 3 1 2 3 
Izlaz:   14                 
\end{test}
\end{miditest}
\begin{miditest}
\begin{test}{Test 2}
Ulaz:   2 3 5 2 6       
Izlaz:  36              
\end{test}
\end{miditest}


\begin{minitest}
\begin{test}{Test 3}
Ulaz:   0
Izlaz:  0
\end{test}
\end{minitest}

\linkresenje{110}
\end{Exercise}
\begin{Answer}[ref=110]
\includecode{resenja/01_Rekurzija/110.c}
\end{Answer}


\begin{Exercise}[label=112]
Napisati rekurzivnu funkciju \kckod{br\_pojave} 
  koja računa broj pojavljivanja
  elementa $x$ u nizu $a$ dužine $n$. Napisati program koji testira
  ovu funkciju, za $x$ i niz koji se unose sa standardnog
  ulaza. Niz neće imati više od $256$ elemenata. Prvo se unosi $x$, a zatim
  elementi niza sve do kraja ulaza.
  
\begin{miditest}
\begin{test}{Test 1}
Ulaz:    4 1 2 3 4     
Izlaz:   1              
\end{test}
\end{miditest}
\begin{miditest}
\begin{test}{Test 2}
Ulaz:    11 3 2 11 14 11 43 1      
Izlaz:   2                      
\end{test}
\end{miditest}

\begin{minitest}
\begin{test}{Test 3}
Ulaz:    1 3 21 5 6
Izlaz:   0    
\end{test}
\end{minitest}

\linkresenje{112}
\end{Exercise}
\begin{Answer}[ref=112]
\includecode{resenja/01_Rekurzija/112.c}
\end{Answer}

\begin{Exercise}[label=114]
Napisati rekurzivnu funkciju \kckod{tri\_uzastopna\_clana}
  kojom se proverava da li su tri
  zadata broja uzastopni članovi niza. Potom, napisati program koji
  je testira. Sa standardnog ulaza se unose najpre tri tražena
  broja, a zatim elementi niza, sve do kraja ulaza. Pretpostaviti da
neće biti uneto više od $256$ brojeva.
  
\begin{miditest}
\begin{test}{Test 1}
Ulaz:     1 2 3 4 1 2 3 4 5 
Izlaz:    da                     
\end{test}
\end{miditest}
\begin{miditest}
\begin{test}{Test 2}
Ulaz:     1 2 3 11 1 2 4 3 6 
Izlaz:    ne                    
\end{test}
\end{miditest}

\begin{miditest}
\begin{test}{Test 3}
Ulaz:     1 2 3 1 2
Izlaz:    ne 
\end{test}
\end{miditest}

\linkresenje{114}
\end{Exercise}
\begin{Answer}[ref=114]
\includecode{resenja/01_Rekurzija/114.c}
\end{Answer}

%%%\subsection{Rekurzivne funkcije za rad sa bitovima}

\begin{Exercise}[label=103]
Napisati rekurzivnu funkciju koja vraća broj bitova koji su postavljeni na $1$, 
u binarnoj reprezentaciji njenog celobrojnog argumenta.  
Napisati program koji testira napisanu funkciju za broj koji se učitava sa 
standardnog ulaza u heksadekadnom formatu. 


\begin{minitest}
\begin{test}{Test 1}
Ulaz:  0x7F    
Izlaz:  7             
\end{test}
\end{minitest}
\begin{minitest}
\begin{test}{Test 2}
Ulaz:  0x80
Izlaz:  1         
\end{test}
\end{minitest}
\begin{minitest}
\begin{test}{Test 3}
Ulaz:   0x00FF00FF
Izlaz:  16
\end{test}
\end{minitest}

\begin{minitest}
\begin{test}{Test 4}
Ulaz:   0xFFFFFFFF       
Izlaz:  32
\end{test}
\end{minitest}  

\linkresenje{103}
\end{Exercise}
\begin{Answer}[ref=103]
\includecode{resenja/01_Rekurzija/103.c}
\end{Answer}

\begin{Exercise}[label=115]%\marker+{2}
Napisati rekurzivnu funkciju koja štampa bitovsku
  reprezentaciju neoznačenog celog broja, i program koji je
  testira za vrednost koja se zadaje sa standardnog ulaza.

\begin{maxitest}
\begin{test}{Test 1}
Ulaz:       10                                 
Izlaz:      00000000000000000000000000001010                    
\end{test}
\end{maxitest}
\end{Exercise}
%\begin{Answer}[ref=115]
%\includecode{resenja/01_Rekurzija/115.c}
%\end{Answer}

\begin{Exercise}[label=119]
Napisati rekurzivnu funkciju za određivanje
najveće cifre u oktalnom zapisu
neoznačenog celog broja korišćenjem bitskih operatora.
\uputstvo{binarne cifre grupisati u podgrupe od po tri cifre,
počev od bitova najmanje težine.}

\begin{minitest}
\begin{test}{Test 1}
Ulaz:  5
Izlaz: 5
\end{test}
\end{minitest}
\begin{minitest}
\begin{test}{Test 2}
Ulaz:  125
Izlaz: 7
\end{test}
\end{minitest}

\begin{minitest}
\begin{test}{Test 3}
Ulaz:  8
Izlaz: 1
\end{test}
\end{minitest}
\begin{minitest}
\begin{test}{Test 4}
Ulaz:  10
Izlaz: 2
\end{test}
\end{minitest}

\linkresenje{119}
\end{Exercise}
\begin{Answer}[ref=119]
\includecode{resenja/01_Rekurzija/119.c}
\end{Answer}

\begin{Exercise}[label=120]
Napisati rekurzivnu funkciju za određivanje (dekadne vrednosti)
najveće cifre u heksadekadnom zapisu neoznačenog celog broja
korišćenjem bitskih operatora. \uputstvo{binarne cifre
grupisati u podgrupe od po četiri cifre, počev od bitova
najmanje težine.}

\begin{minitest}
\begin{test}{Test 1}
Ulaz:  5
Izlaz: 5
\end{test}
\end{minitest}
\begin{minitest}
\begin{test}{Test 2}
Ulaz:  16
Izlaz: 1
\end{test}
\end{minitest}
\begin{minitest}
\begin{test}{Test 3}
Ulaz:  18
Izlaz: 2
\end{test}
\end{minitest}

\begin{minitest}
\begin{test}{Test 4}
Ulaz:  165
Izlaz: 10
\end{test}
\end{minitest}

\linkresenje{120}
\end{Exercise}
\begin{Answer}[ref=120]
\includecode{resenja/01_Rekurzija/120.c}
\end{Answer}

%%%\subsection{Rekurzivne funkcije - razni zadaci}

\begin{Exercise}[label=113]
Napisati rekurzivnu funkciju \kckod{palindrom} koja ispituje da li je data niska
  palindrom. Napisati program koji testira ovu funkciju. Pretposatviti
  da niska neće neće imati više od $31$ karaktera, i da se unosi sa standardnog
  ulaza.
  
\begin{miditest}
\begin{test}{Test 1}
Ulaz:    programiranje    
Izlaz:   ne                   
\end{test}
\end{miditest}
\begin{miditest}
\begin{test}{Test 2}
Ulaz:    anavolimilovana 
Izlaz:   da                   
\end{test}
\end{miditest}

\begin{minitest}
\begin{test}{Test 3}
Ulaz:    a        
Izlaz:   da 
\end{test}
\end{minitest}
\begin{minitest}
\begin{test}{Test 4}
Ulaz:    aba   
Izlaz:   da 
\end{test}
\end{minitest}
\begin{minitest}
\begin{test}{Test }
Ulaz:    aa
Izlaz:   da 
\end{test}
\end{minitest}

\linkresenje{113}
\end{Exercise}
\begin{Answer}[ref=113]
\includecode{resenja/01_Rekurzija/113.c}
\end{Answer}

\begin{Exercise}[label=107, difficulty=1]
Napisati rekurzivnu funkciju koja prikazuje sve permutacije skupa $\{1, 2, ... ,n\}$. Napisati program koji testira napisanu funkciju za poizvoljan prirodan broj $n$ ($n \le 50$) unet sa standardnog ulaza.

\begin{miditest}
\begin{test}{Test 1}
Ulaz:   Unesite duzinu permutacije: 3
Izlaz:  1 2 3
        1 3 2
        2 1 3
        2 3 1
        3 1 2
        3 2 1
\end{test}
\end{miditest}

\linkresenje{107}
\end{Exercise}
\begin{Answer}[ref=107]
\includecode{resenja/01_Rekurzija/107.c}
\end{Answer}

\begin{Exercise}[label=108, difficulty=1]
 Paskalov trougao se dobija tako što mu je svako polje
(izuzev jedinica po krajevima) zbir jednog polja levo i
jednog polja iznad.
\begin{verbatim}
               1
             1   1
           1   2   1
         1   3   3   1
       1   4   6   4   1
     1   5   10  10  5  1
\end{verbatim}
\begin{enumerate}
\item Napisati rekurzivnu funkciju koja izračunava vrednost binomnog koeficijenta $\binom{n}{k}$, tj. vrednost polja \argf{(n, k)}, gde je $n$ redni broj hipotenuze, a $k$ redni broj elementa u tom redu (na toj hipotenuzi). Brojanje počinje od nule. Na primer vrednost polja \argf{(4, 2)} je $6$. 
\item Napisati rekurzivnu funkciju koja izračunava $d_n$ kao sumu elemenata $n$-te hipotenuze Paskalovog trougla.
\end{enumerate}

Napisati program koji za unetu veličinu Paskalovog trougla i hipotenuzu 
najpre iscrtava Paskalov trougao a zatim sumu elemenata hipotenuze.

\begin{miditest}
\begin{test}{Test 1}
Ulaz:   5 3
Izlaz:  
               1
             1   1
           1   2   1
         1   3   3   1
       1   4   6   4   1
     1   5   10  10  5  1

     8
\end{test}
\end{miditest}
\begin{miditest}
\begin{test}{Test 2}
Ulaz:   6 5
Izlaz:  
  		           1
               1   1
             1   2   1
           1   3   3   1
         1   4   6   4   1
       1   5   10  10   5  1
     1   6  15   20  15  6   1

     32
\end{test}
\end{miditest}

\linkresenje{108}
\end{Exercise}
\begin{Answer}[ref=108]
\includecode{resenja/01_Rekurzija/108.c}
\end{Answer}

%%%%%%%%%%%%%%%%%%%%%%%%%
% Zadaci sa praktikuma - dodatni zadaci - oni nemaju rešenja
% možda \subsection{Dodatni zadaci} ili zadaci za vezbu
%%%%%%%%%%%%%%%%%%%%%%%%%


\begin{Exercise}[label=116]%\marker+{2}
Napisati rekurzivnu funkciju koja prikazuje sve varijacije sa
   ponavljanjem dužine $n$ skupa $\{a, b\}$, i program koji je
   testira, za $n$ koje se unosi sa standardnog ulaza.

\begin{miditest}
\begin{test}{Test 1}
Ulaz:    3
Izlaz:   a a a
         a a b
         a b a
         a b b
         b a a
         b a b
         b b a
         b b b
\end{test}
\end{miditest}
\end{Exercise}
%\begin{Answer}[ref=116]
%\includecode{resenja/01_Rekurzija/116.c}
%\end{Answer}

\begin{Exercise}[label=117]%\marker+{2}
{\em Hanojske kule}: Data su tri
  vertikalna štapa, na jednom se nalazi $n$ diskova poluprečnika
  $1$,$2$,$3$,... do $n$, tako da se najveći nalazi na dnu, a
  najmanji na vrhu. Ostala dva štapa su prazna. Potrebno je
  premestiti diskove na drugi štap tako da budu u istom redosledu, pri čemu se ni u jednom
  trenutku ne sme staviti veći disk preko manjeg, a preostali štap se koristi kao pomoćni štap prilikom
  premeštanja. \\
  Napisati program koji za proizvoljnu vrednost $n$, koja se unosi sa standardnog ulaza, prikazuje proces premeštanja diskova.

\end{Exercise}
%\begin{Answer}[ref=117]
%\includecode{resenja/01_Rekurzija/117.c}
%\end{Answer}

\begin{Exercise}[label=118]%\marker+{2}
{\em Modifikacija Hanojskih kula}: Data su četiri
  vertikalna štapa, na jednom se nalazi $n$ diskova poluprečnika
  $1$,$2$,$3$,... do $n$, tako da se najveći nalazi na dnu, a
  najmanji na vrhu. Ostala tri štapa su prazna. Potrebno je
  premestiti diskove na drugi štap tako da budu u istom redosledu,
  premestajući jedan po jedan disk, pri čemu se ni u jednom
  trenutku ne sme staviti veći disk preko manjeg, pri čemu se
  preostala dva štapa koriste kao pomoćni štapovi prilikom
  premeštanja.\\
  Napisati program koji za proizvoljnu vrednost $n$, koja se unosi sa standardnog ulaza, prikazuje proces premeštanja diskova.

\end{Exercise}
%\begin{Answer}[ref=118]
%\includecode{resenja/01_Rekurzija/118.c}
%\end{Answer}



\section{Rešenja}
\shipoutAnswer
