\section{Rekurzija}

%%%\subsection{Rekurzivne funkcije nad brojevima}

\begin{Exercise}[label=101]
Napisati rekurzivnu funkciju koja izračunava  $x^k$, za dati ceo broj $x$ i prirodan broj $k$
\begin{enumerate}
\item tako da rešenje bude linearne složenosti,
\item tako da rešenje bude logaritamske složenosti.
\end{enumerate}
Napisati program koji testira napisane funkciju tako što se sa standardnog ulaza najpre unosi redni broj funckije koja se primenjuje, a zatim i ceo broj $x$ i prirodan broj $k$, a potom se na standarni izlaz ispisuje rezultat primene odgovarajuće funkcije na unete brojeve.
 
\begin{miditest}
\begin{upotreba}{1}
#\naslovInt#
#\izlaz{Unesite redni broj funkcije (1/2):}# 
#\ulaz{1}#
#\izlaz{Unesite broj x:} \ulaz{ 2}#
#\izlaz{Unesite broj k:} \ulaz{ 10}#
#\izlaz{1024}#
\end{upotreba}
\end{miditest}
\begin{miditest}
\begin{upotreba}{2}
#\naslovInt#
#\izlaz{Unesite redni broj funkcije (1/2):}# 
#\ulaz{2}#
#\izlaz{Unesite broj x:} \ulaz{ 9}#
#\izlaz{Unesite broj k:} \ulaz{ 4}#
#\izlaz{6561}#
\end{upotreba}
\end{miditest}    
 
\linkresenje{101}
\end{Exercise}
\begin{Answer}[ref=101]
\includecode{resenja/01_Rekurzija/101.c}
\end{Answer}

\begin{Exercise}[label=102]
Koristeći uzajamnu (posrednu) rekurziju napisati:
 \begin{enumerate}
\item funkciju \kckod{paran} koja proverava da li je broj cifara nekog broja paran i vraća 1 ako jeste, a 0 inače;
\item i funkciju \kckod{neparan} koja vraća 1, ukoliko je broj cifara nekog broja neparan, a 0 inače.
 \end{enumerate}
Napisati program koji testira napisane funkcije tako što za heksadekadni broj koji se unosi sa standardnog ulaza ispisuje da li je broj njegovih cifara paran ili neparan.
 
 \begin{miditest}
\begin{test}{1}
#\naslovUlaz#
#\ulaz{11}#
#\naslovIzlaz#
#\izlaz{Uneti broj ima paran broj cifara.}#
\end{test}
\end{miditest}
\begin{miditest}
\begin{test}{2}
#\naslovUlaz#
#\ulaz{123}#
#\naslovIzlaz#
#\izlaz{Uneti broj ima neparan broj cifara.}#
\end{test}
\end{miditest}
 
\linkresenje{102}
\end{Exercise}
\begin{Answer}[ref=102]
\includecode{resenja/01_Rekurzija/102.c}
\end{Answer}

\begin{Exercise}[label=103]
Napisati repno-rekurzivnu funkciju koja izračunava faktorijel broja $n$. Napisati program koji testira napisanu funkciju za proizvoljan broj $n$ ($n \le 12$) unet sa standardnog ulaza.

\begin{miditest}
\begin{upotreba}{1}
#\naslovInt#
#\izlaz{Unesite n (<= 12):} \ulaz{5}#
#\izlaz{5! = 120}#
\end{upotreba}
\end{miditest}
\begin{miditest}
\begin{upotreba}{2}
#\naslovInt#
#\izlaz{Unesite n (<= 12):} \ulaz{0}#
#\izlaz{0! = 1}#
\end{upotreba}
\end{miditest}

\linkresenje{103}
\end{Exercise}
\begin{Answer}[ref=103]
\includecode{resenja/01_Rekurzija/103.c}
\end{Answer}


\begin{Exercise}[label=104]
Elementi niza $F$ izračunavaju se na osnovu sledećih rekurentnih relacija:
 $$F(0) = 0$$
 $$F(1) = 1$$
 $$F(n) = a* F(n-1) + b*F(n-2)$$
Napisati funkciju koja računa $n$-ti element u nizu $F$
\begin{enumerate}
\item iterativno,
\item tako da funkcija bude rekurzivna i da koristi navedene rekurentne relacije,
\item tako da funkcija bude rekurzivna ali da se problemi manje dimenzije rešavaju samo jedan put.
\end{enumerate}
Napisati program koji testira napisane funkciju tako što se sa standardnog ulaza najpre unosi redni broj funckije koja se primenjuje, a zatim i vrednosti koeficijenata $a$ i $b$ i prirodan broj $n$. Na standardni izlaz se ispisuje rezultat primene odabrane funkcije na unete vrednosti koeficijenata $a$ i $b$ i prirodan broj  $n$.  \napomena{Niz  $F$ definisan na ovaj način predstavlja uopštenje Fibonačijevih brojeva.}
  
\begin{miditest}
\begin{upotreba}{1}
#\naslovInt#
#\izlaz{Unesite redni broj funkcije koju zelite:}# 
#\izlaz{1 - iterativna}# 
#\izlaz{2 - rekurzivna}# 
#\izlaz{3 - rekurzivna napredna}# 
#\ulaz{1}#
#\izlaz{Unesite koeficijente:} \ulaz{ 2 3}#
#\izlaz{Unesite koji clan niza se racuna:} \ulaz{ 5}#
#\izlaz{F(5) = 61}#
\end{upotreba}
\end{miditest}
\begin{miditest}
\begin{upotreba}{2}
#\naslovInt#
#\izlaz{Unesite redni broj funkcije koju zelite:}# 
#\izlaz{1 - iterativna}# 
#\izlaz{2 - rekurzivna}# 
#\izlaz{3 - rekurzivna napredna}# 
#\ulaz{3}#
#\izlaz{Unesite koeficijente:} \ulaz{4 2}#
#\izlaz{Unesite koji clan niza se racuna:} \ulaz{8}#
#\izlaz{F(8) = 31360}#
\end{upotreba}
\end{miditest}

\linkresenje{104}
\end{Exercise}
\begin{Answer}[ref=104]
\includecode{resenja/01_Rekurzija/104.c}
\end{Answer}

\begin{Exercise}[label=105]
Napisati rekurzivnu funkciju koja sabira dekadne cifre datog celog broja $x$. Napisati program koji testira ovu funkciju, za broj koji se unosi sa standardnog ulaza.
  
\begin{minitest}
\begin{test}{1}
#\naslovUlaz#
#\ulaz{123}#
#\naslovIzlaz#
#\izlaz{6}#
\end{test}
\end{minitest}
\begin{minitest}
\begin{test}{2}
#\naslovUlaz#
#\ulaz{23156}#
#\naslovIzlaz#
#\izlaz{17}#
\end{test}
\end{minitest}
\begin{minitest}
\begin{test}{3}
#\naslovUlaz#
#\ulaz{1432}#
#\naslovIzlaz#
#\izlaz{10}#
\end{test}
\end{minitest}      
 
\begin{minitest}
\begin{test}{4}
#\naslovUlaz#
#\ulaz{1}#
#\naslovIzlaz#
#\izlaz{1}#
\end{test}
\end{minitest}
\begin{minitest}
\begin{test}{5}
#\naslovUlaz#
#\ulaz{0}#
#\naslovIzlaz#
#\izlaz{0}#
\end{test}
\end{minitest}      

\linkresenje{105}
\end{Exercise}
\begin{Answer}[ref=105]
\includecode{resenja/01_Rekurzija/105.c}
\end{Answer}

%%%\subsection{Rekurzivne funkcije za rad sa nizovima}

\begin{Exercise}[label=106]
Napisati rekurzivnu funkciju koja sumira elemente niza celih brojeva
\begin{enumerate}
\item sabirajući elemente počev od početka niza ka kraju niza,
\item sabirajući elemente počev od kraja niza ka početku niza.
\end{enumerate}
Napisati program koji testira napisane funkcije. Sa standardnog ulaza učitati redni broj funkcije, zatim dimenziju $n$  ($0 < n \leq 100$) celobrojnog niza, a potom i elemente niza. Na standardni izlaz ispisati rezultat primene odabrane funkcije nad učitanim nizom.

\begin{miditest}
\begin{upotreba}{1}
#\naslovInt#
#\izlaz{Unesite redni broj funkcije (1 ili 2):} \ulaz{1}#
#\izlaz{Unesite dimenziju niza:} \ulaz{5}#
#\izlaz{Unesite elemente niza:}#
#\ulaz{1 2 3 4 5}#
#\izlaz{Suma elemenata je 15}#
\end{upotreba}
\end{miditest}
\begin{miditest}
\begin{upotreba}{2}
#\naslovInt#
#\izlaz{Unesite redni broj funkcije (1 ili 2):} \ulaz{2}#
#\izlaz{Unesite dimenziju niza:} \ulaz{4}#
#\izlaz{Unesite elemente niza:}#
#\ulaz{-5 2 -3 6}#
#\izlaz{Suma elemenata je 0}#
\end{upotreba}
\end{miditest}

\linkresenje{106}
\end{Exercise}
\begin{Answer}[ref=106]
\includecode{resenja/01_Rekurzija/106.c}
\end{Answer}

\begin{Exercise}[label=107]
Napisati rekurzivnu funkciju koja određuje maksimum niza celih brojeva. Napisati program koji testira ovu funkciju za niz koji se unosi sa standardnog ulaza. Niz neće imati više od $256$ elemenata. Njegovi elementi se unose sve do unosa kraja ulaza (EOF).
  
 \begin{minitest}
\begin{test}{1}
#\naslovUlaz#
#\ulaz{3 2 1 4 21}#
#\naslovIzlaz#
#\izlaz{21}#
\end{test}
\end{minitest}
\begin{minitest}
\begin{test}{2}
#\naslovUlaz#
#\ulaz{ 2 -1 0 -5 -10}#
#\naslovIzlaz#
#\izlaz{2}#
\end{test}
\end{minitest}
\begin{minitest}
\begin{test}{3}
#\naslovUlaz#
#\ulaz{1 11 3 5 8 1}#
#\naslovIzlaz#
#\izlaz{11}#
\end{test}
\end{minitest}
% \begin{minitest}
% \begin{test}{4}
% #\naslovUlaz#
% #\ulaz{ 5}#
% #\naslovIzlaz#
% #\izlaz{5}#
% \end{test}
% \end{minitest}

\linkresenje{107}
\end{Exercise}
\begin{Answer}[ref=107]
\includecode{resenja/01_Rekurzija/107.c}
\end{Answer}

\begin{Exercise}[label=108]
Napisati rekurzivnu funkciju koja izračunava skalarni proizvod dva data vektora.  Napisati program koji testira ovu funkciju, za nizove koji se unose sa standardnog ulaza. Prvo se unosi dimenzija nizova, a zatim i njihovi elementi. Nizovi neće imati više od $256$ elemenata.

\begin{miditest}
\begin{upotreba}{1}
#\naslovInt#
#\izlaz{Unesite dimenziju nizova:} \ulaz{3}#
#\izlaz{Unesite elemente prvog niza:}#
#\ulaz{1 2 3}#
#\izlaz{Unesite elemente drugog niza:}#
#\ulaz{1 2 3}#
#\izlaz{Skalarni proizvod je 14}#
\end{upotreba}
\end{miditest}
\begin{miditest}
\begin{upotreba}{2}
#\naslovInt#
#\izlaz{Unesite dimenziju nizova:} \ulaz{2}#
#\izlaz{Unesite elemente prvog niza:}#
#\ulaz{3 5}#
#\izlaz{Unesite elemente drugog niza:}#
#\ulaz{2 6}#
#\izlaz{Skalarni proizvod je 36}#
\end{upotreba}
\end{miditest}
  
%\begin{minitest}
%\begin{test}{3}
%#\naslovUlaz#
%#\ulaz{ 0 }#
%#\naslovIzlaz#
%#\izlaz{0}#
%\end{test}
%\end{minitest}  

\linkresenje{108}
\end{Exercise}
\begin{Answer}[ref=108]
\includecode{resenja/01_Rekurzija/108.c}
\end{Answer}

\begin{Exercise}[label=109]
Napisati rekurzivnu funkciju koja računa broj pojavljivanja
elementa $x$ u nizu $a$ dužine $n$. Napisati program koji testira ovu funkciju za broj $x$ i niz  $a$ koji se unose sa standardnog ulaza. Prvo se unosi $x$, a zatim elementi niza sve do unosa kraja ulaza. Niz neće imati više od $256$ elemenata. 

\begin{miditest}
\begin{upotreba}{1}
#\naslovInt#
#\izlaz{Unesite ceo broj:}#
#\ulaz{4}#
#\izlaz{Unesite elemente niza:}#
#\ulaz{1 2 3 4}#
#\izlaz{Broj pojavljivanja je 1}#
\end{upotreba}
\end{miditest}
\begin{miditest}
\begin{upotreba}{2}
#\naslovInt#
#\izlaz{Unesite ceo broj:}#
#\ulaz{11}#
#\izlaz{Unesite elemente niza:}#
#\ulaz{3 2 11 14 11 43 1}#
#\izlaz{Broj pojavljivanja je 2}#
\end{upotreba}
\end{miditest}


\begin{miditest}
\begin{upotreba}{3}
#\naslovInt#
#\izlaz{Unesite ceo broj:}#
#\ulaz{1}#
#\izlaz{Unesite elemente niza:}#
#\ulaz{3 21 5 6}#
#\izlaz{Broj pojavljivanja je 0}#
\end{upotreba}
\end{miditest}

\linkresenje{109}
\end{Exercise}
\begin{Answer}[ref=109]
\includecode{resenja/01_Rekurzija/109.c}
\end{Answer}

\begin{Exercise}[label=110]
Napisati rekurzivnu funkciju kojom se proverava da li su tri zadata broja uzastopni članovi niza. Potom, napisati program koji
  je testira. Sa standardnog ulaza se unose najpre tri tražena
  broja, a zatim elementi niza, sve do kraja ulaza. Pretpostaviti da
neće biti uneto više od $256$ brojeva.
  
\begin{miditest}
\begin{upotreba}{1}
#\naslovInt#
#\izlaz{Unesite tri cela broja:}#
#\ulaz{1 2 3}#
#\izlaz{Unesite elemente niza:}#
#\ulaz{4 1 2 3 4 5}#
#\izlaz{Uneti brojevi jesu uzastopni clanovi niza.}#
\end{upotreba}
\end{miditest}
\begin{miditest}
\begin{upotreba}{2}
#\naslovInt#
#\izlaz{Unesite tri cela broja:}#
#\ulaz{1 2 3}#
#\izlaz{Unesite elemente niza:}#
#\ulaz{11 1 2 4 3 6}#
#\izlaz{Uneti brojevi nisu uzastopni clanovi niza.}#
\end{upotreba}
\end{miditest}
%
%\begin{miditest}
%\begin{test}{Test 3}
%Ulaz:     1 2 3 1 2
%Izlaz:    ne 
%\end{test}
%\end{miditest}

\linkresenje{110}
\end{Exercise}
\begin{Answer}[ref=110]
\includecode{resenja/01_Rekurzija/110.c}
\end{Answer}

%%%\subsection{Rekurzivne funkcije za rad sa bitovima}

\begin{Exercise}[label=111]
Napisati rekurzivnu funkciju koja vraća broj bitova koji su postavljeni na $1$, 
u binarnoj reprezentaciji njenog celobrojnog argumenta.  
Napisati program koji testira napisanu funkciju za broj koji se učitava sa 
standardnog ulaza u heksadekadnom formatu. 

\begin{minitest}
\begin{test}{1}
#\naslovUlaz#
#\ulaz{0x7F}#
#\naslovIzlaz#
#\izlaz{7}#
\end{test}
\end{minitest}
%\begin{minitest}
%\begin{test}{2}
%#\naslovUlaz#
%#\ulaz{0x80}#
%#\naslovIzlaz#
%#\izlaz{1}#
%\end{test}
%\end{minitest}
\begin{minitest}
\begin{test}{2}
#\naslovUlaz#
#\ulaz{0x00FF00FF}#
#\naslovIzlaz#
#\izlaz{16}#
\end{test}
\end{minitest}  
\begin{minitest}
\begin{test}{3}
#\naslovUlaz#
#\ulaz{0xFFFFFFFF}#
#\naslovIzlaz#
#\izlaz{32}#
\end{test}
\end{minitest}  

\linkresenje{111}
\end{Exercise}
\begin{Answer}[ref=111]
\includecode{resenja/01_Rekurzija/111.c}
\end{Answer}

\begin{Exercise}[label=112]%\marker+{2}
Napisati rekurzivnu funkciju koja štampa bitovsku
  reprezentaciju neoznačenog celog broja, i program koji je
  testira za vrednost koja se zadaje sa standardnog ulaza.

\begin{miditest}
\begin{test}{1}
#\naslovUlaz#
#\ulaz{10}#
#\naslovIzlaz#
#\izlaz{00000000000000000000000000001010}#
\end{test}
\end{miditest}
\begin{miditest}
\begin{test}{2}
#\naslovUlaz#
#\ulaz{0}#
#\naslovIzlaz#
#\izlaz{00000000000000000000000000000000}#
\end{test}
\end{miditest}
\end{Exercise}

%\begin{Answer}[ref=112]
%\includecode{resenja/01_Rekurzija/112.c}
%\end{Answer}

\begin{Exercise}[label=113]
Napisati rekurzivnu funkciju za određivanje
najveće cifre u oktalnom zapisu
neoznačenog celog broja korišćenjem bitskih operatora.
\uputstvo{Binarne cifre grupisati u podgrupe od po tri cifre,
počev od bitova najmanje težine.}

\begin{minitest}
\begin{test}{1}
#\naslovUlaz#
#\ulaz{5}#
#\naslovIzlaz#
#\izlaz{5}#
\end{test}
\end{minitest}
\begin{minitest}
\begin{test}{2}
#\naslovUlaz#
#\ulaz{125}#
#\naslovIzlaz#
#\izlaz{7}#
\end{test}
\end{minitest}
\begin{minitest}
\begin{test}{3}
#\naslovUlaz#
#\ulaz{8}#
#\naslovIzlaz#
#\izlaz{1}#
\end{test}
\end{minitest}  

\linkresenje{113}
\end{Exercise}
\begin{Answer}[ref=113]
\includecode{resenja/01_Rekurzija/113.c}
\end{Answer}

\begin{Exercise}[label=114]
Napisati rekurzivnu funkciju za određivanje (dekadne vrednosti)
najveće cifre u heksadekadnom zapisu neoznačenog celog broja
korišćenjem bitskih operatora. \uputstvo{Binarne cifre
grupisati u podgrupe od po četiri cifre, počev od bitova
najmanje težine.}

\begin{minitest}
\begin{test}{1}
#\naslovUlaz#
#\ulaz{5}#
#\naslovIzlaz#
#\izlaz{5}#
\end{test}
\end{minitest}
\begin{minitest}
\begin{test}{2}
#\naslovUlaz#
#\ulaz{16}#
#\naslovIzlaz#
#\izlaz{1}#
\end{test}
\end{minitest}
\begin{minitest}
\begin{test}{3}
#\naslovUlaz#
#\ulaz{18}#
#\naslovIzlaz#
#\izlaz{2}#
\end{test}
\end{minitest}  
%\begin{minitest}
%\begin{test}{Test 4}
%Ulaz:  165
%Izlaz: 10
%\end{test}
%\end{minitest}

\linkresenje{114}
\end{Exercise}
\begin{Answer}[ref=114]
\includecode{resenja/01_Rekurzija/114.c}
\end{Answer}

%%%\subsection{Rekurzivne funkcije - razni zadaci}

\begin{Exercise}[label=115]
Napisati rekurzivnu funkciju \kckod{palindrom} koja ispituje da li je data niska
  palindrom. Napisati program koji testira ovu funkciju na nisci koja se unosi sa standardnog
  ulaza. Pretpostaviti da niska neće neće imati više od $31$ karaktera.
  
\begin{minitest}
\begin{test}{1}
#\naslovUlaz#
#\ulaz{a}#
#\naslovIzlaz#
#\izlaz{da}#
\end{test}
\end{minitest}  
\begin{minitest}
\begin{test}{2}
#\naslovUlaz#
#\ulaz{aa}#
#\naslovIzlaz#
#\izlaz{da}#
\end{test}
\end{minitest}
\begin{minitest}
\begin{test}{3}
#\naslovUlaz#
#\ulaz{aba}#
#\naslovIzlaz#
#\izlaz{da}#
\end{test}
\end{minitest}  
 
\begin{minitest}
\begin{test}{4}
#\naslovUlaz#
#\ulaz{programiranje}#
#\naslovIzlaz#
#\izlaz{ne}#
\end{test}
\end{minitest}
\begin{miditest}
\begin{test}{5}
#\naslovUlaz#
#\ulaz{anavolimilovana}#
#\naslovIzlaz#
#\izlaz{da}#
\end{test}
\end{miditest}
 
\linkresenje{115}
\end{Exercise}
\begin{Answer}[ref=115]
\includecode{resenja/01_Rekurzija/115.c}
\end{Answer}

\begin{Exercise}[label=116, difficulty=1]
Napisati rekurzivnu funkciju koja prikazuje sve permutacije skupa $\{1, 2, ... ,n\}$. Napisati program koji testira napisanu funkciju za proizvoljan prirodan broj $n$ ($n \le 15$) unet sa standardnog ulaza.

\begin{minitest}
\begin{test}{1}
#\naslovUlaz#
#\ulaz{2}#
#\naslovIzlaz#
#\izlaz{1 2}#
#\izlaz{2 1}#
\end{test}
\end{minitest}  
\begin{minitest}
\begin{test}{2}
#\naslovUlaz#
#\ulaz{3}#
#\izlaz{1 2 3}#
#\izlaz{1 3 2}#
#\izlaz{2 1 3}#
#\izlaz{2 3 1}#
#\izlaz{3 1 2}#
#\izlaz{3 2 1}#
\end{test}
\end{minitest}
\begin{minitest}
\begin{test}{3}
#\naslovUlaz#
#\ulaz{-5}#
#\izlaz{Duzina }#
#\izlaz{permutacije}#
#\izlaz{mora biti}#
#\izlaz{broj iz }#
#\izlaz{intervala}#
#\izlaz{[0, 50]!}#
\end{test}
\end{minitest}


\linkresenje{116}
\end{Exercise}
\begin{Answer}[ref=116]
\includecode{resenja/01_Rekurzija/116.c}
\end{Answer}

\begin{Exercise}[label=117, difficulty=1]
Paskalov trougao sadrži brojeve čije se vrednosti računaju tako što svako polje ima vrednost
 zbira dve vrednosti koje su u susedna dva polja iznad. Izuzetak su jedinice na krajevima. Vrednosti
 brojeva Paskalovog trougla odgovaraju binomnim koeficijentima tj.~vrednost polja \argf{(n, k)}, gde je $n$ redni broj hipotenuze, a $k$ redni broj elementa u tom redu (na toj hipotenuzi) odgovara binomnom koeficijentu $\binom{n}{k}$, pri čemu brojanje počinje od nule. Na primer, vrednost polja \argf{(4, 2)} je $6$. 

\begin{verbatim}
               1
             1   1
           1   2   1
         1   3   3   1
       1   4   6   4   1
     1   5   10  10  5   1
\end{verbatim}

\begin{enumerate}
\item Napisati rekurzivnu funkciju koja izračunava vrednost binomnog koeficijenta $\binom{n}{k}$ koristeći osobine Paskalovog trougla. 
\item Napisati rekurzivnu funkciju koja izračunava $d_n$ kao sumu elemenata $n$-te hipotenuze Paskalovog trougla.
\end{enumerate}

Napisati program koji za unetu veličinu Paskalovog trougla i redni broj hipotenuze
najpre iscrtava Paskalov trougao, a zatim štampa sumu elemenata hipotenuze.

\begin{miditest}
\begin{test}{1}
#\naslovUlaz#
#\ulaz{5 3}#
#\naslovIzlaz#
            #\izlaz{1}#
          #\izlaz{1}#   #\izlaz{1}#
        #\izlaz{1}#   #\izlaz{2}#   #\izlaz{1}#
      #\izlaz{1}#   #\izlaz{3}#   #\izlaz{3}#   #\izlaz{1}#
    #\izlaz{1}#   #\izlaz{4}#   #\izlaz{6}#   #\izlaz{4}#   #\izlaz{1}#
  #\izlaz{1}#   #\izlaz{5}#   #\izlaz{10}#  #\izlaz{10}#  #\izlaz{5}#  #\izlaz{1}#
#\izlaz{}#
#\izlaz{8}#
\end{test}
\end{miditest}
\begin{miditest}
\begin{test}{2}
#\naslovUlaz#
#\ulaz{6 5}#
#\naslovIzlaz#
            #\izlaz{1}#
          #\izlaz{1}#   #\izlaz{1}#
        #\izlaz{1}#   #\izlaz{2}#   #\izlaz{1}#
      #\izlaz{1}#   #\izlaz{3}#   #\izlaz{3}#   #\izlaz{1}#
    #\izlaz{1}#   #\izlaz{4}#   #\izlaz{6}#   #\izlaz{4}#   #\izlaz{1}#
  #\izlaz{1}#   #\izlaz{5}#   #\izlaz{10}#  #\izlaz{10}#  #\izlaz{5}#  #\izlaz{1}#
#\izlaz{1}#   #\izlaz{6}#   #\izlaz{15}#  #\izlaz{20}# #\izlaz{15}#  #\izlaz{6}#  #\izlaz{1}#
#\izlaz{}#
#\izlaz{32}#
\end{test}
\end{miditest}

\linkresenje{117}
\end{Exercise}
\begin{Answer}[ref=117]
\includecode{resenja/01_Rekurzija/117.c}
\end{Answer}

%%%%%%%%%%%%%%%%%%%%%%%%%
% Zadaci sa praktikuma - dodatni zadaci - oni nemaju rešenja
% možda \subsection{Dodatni zadaci} ili zadaci za vezbu
%%%%%%%%%%%%%%%%%%%%%%%%%


\begin{Exercise}[label=118]%\marker+{2}
Napisati rekurzivnu funkciju koja prikazuje sve varijacije sa
   ponavljanjem dužine $n$ skupa $\{a, b\}$, i program koji je
   testira, za $n$ koje se unosi sa standardnog ulaza.

\begin{miditest}
\begin{test}{1}
#\naslovUlaz#
#\ulaz{2}#
#\naslovIzlaz#
#\izlaz{a a}#
#\izlaz{a b}#
#\izlaz{b a}#
#\izlaz{b b}#
\end{test}
\end{miditest}
\begin{miditest}
\begin{test}{2}
#\naslovUlaz#
#\ulaz{3}#
#\naslovIzlaz#
#\izlaz{a a a }#
#\izlaz{a a b}#
#\izlaz{a b a}#
#\izlaz{a b b}#
#\izlaz{b a a}#
#\izlaz{b a b}#
#\izlaz{b b a}#
#\izlaz{b b b}#
\end{test}
\end{miditest}
\end{Exercise}
%\begin{Answer}[ref=118]
%\includecode{resenja/01_Rekurzija/118.c}
%\end{Answer}

\begin{Exercise}[label=119]%\marker+{2}
{\em Hanojske kule}: Data su tri
  vertikalna štapa, na jednom se nalazi $n$ diskova poluprečnika
  $1$, $2$, $3$,... do $n$, tako da se najveći nalazi na dnu, a
  najmanji na vrhu. Ostala dva štapa su prazna. Potrebno je
  premestiti diskove na drugi štap tako da budu u istom redosledu, pri čemu se ni u jednom
  trenutku ne sme staviti veći disk preko manjeg, a preostali štap se koristi kao pomoćni štap prilikom
  premeštanja. \\
  Napisati program koji za proizvoljnu vrednost $n$, koja se unosi sa standardnog ulaza, prikazuje proces premeštanja diskova.

\end{Exercise}
%\begin{Answer}[ref=119]
%\includecode{resenja/01_Rekurzija/119.c}
%\end{Answer}

\begin{Exercise}[label=120]%\marker+{2}
{\em Modifikacija Hanojskih kula}: Data su četiri
  vertikalna štapa, na jednom se nalazi $n$ diskova poluprečnika
  $1$, $2$, $3$,... do $n$, tako da se najveći nalazi na dnu, a
  najmanji na vrhu. Ostala tri štapa su prazna. Potrebno je
  premestiti diskove na drugi štap tako da budu u istom redosledu,
  premestajući jedan po jedan disk, pri čemu se ni u jednom
  trenutku ne sme staviti veći disk preko manjeg, pri čemu se
  preostala dva štapa koriste kao pomoćni štapovi prilikom
  premeštanja.\\
  Napisati program koji za proizvoljnu vrednost $n$, koja se unosi sa standardnog ulaza, prikazuje proces premeštanja diskova.

\end{Exercise}
%\begin{Answer}[ref=120]
%\includecode{resenja/01_Rekurzija/120.c}
%\end{Answer}



\section{Rešenja}
\shipoutAnswer
