


\section{Rekurzija}

\begin{Exercise}[label=101]
Napisati rekurzivnu funkciju koja sumira elemente niza celih brojeva.  Napisati program koji testira napisane funkcije. Sa standardnog ulaza učitati dimenziju $n$  ($0 < n \leq 100$) celobrojong
niza, a zatim i elemente niza. Na standardni izlaz ispisati
rezultat primene napisane funkcije nad učitanim
nizom.
%Napisati program koji testira napisanu funkciju za niz \argf{10, 2, 3, 45, 21}.
\end{Exercise}
\begin{Answer}[ref=101]
\includecode{resenja/01_Rekurzija/101.c}
\end{Answer}

\begin{Exercise}[label=102]
 Napisati rekurzivnu funkciju koja izračunava  $x^k$,  za dati realni broj $x$ i prirodan broj $k$. Napisati program koji testira napisanu funkciju za vrednosti koje se unose sa standardnog ulaza.
\end{Exercise}
\begin{Answer}[ref=102]
\includecode{resenja/01_Rekurzija/102.c}
\end{Answer}

\begin{Exercise}[label=104]
 Koristeći uzajamnu (posrednu) rekurziju napisati naredne dve funkcije:
 \begin{itemize}
\item fukciju \kckod{paran} koja proverava da li je broj cifara nekog broja paran i vraća 1 ako jeste, a 0 inače;
\item i fukciju \kckod{neparan} koja vraća 1 ukoliko je broj cifara nekog broja neparan, a 0 inače.
 \end{itemize}
 Napisati program koji testira napisanu funkciju tako što se za heksadekadnu vrednost koja se unosi sa standardnog ulaza ispisuje da li je paran ili neparan.
\end{Exercise}
\begin{Answer}[ref=104]
%\includecode{resenja/01_Rekurzija/104.c}
\end{Answer}

\begin{Exercise}[label=105]
  Napisati repno-rekurzivnu funkciju koja izračunava faktorijel broja n. Napisati program koji testira napisanu funkciju za poizvoljan broj $n$ ($n \le 12$) unet sa standardnog ulaza.
\end{Exercise}
\begin{Answer}[ref=105]
%\includecode{resenja/01_Rekurzija/105.c}
\end{Answer}


\begin{Exercise}[label=106]
Elementi funkcije $F$ izračunavaju se na osnovu sledećih rekurentnih relacija:
 $$F(0) = 0$$
 $$F(1) = 1$$
 $$F(n) = a* F(n-1) + b*F(n-2)$$
\begin{enumerate}
\item Napisati rekurzivnu funkciju  koja računa $n$-ti element u nizu $F$.
\item Napisati rekurzivnu funkciju  koja računa $n$-ti element u nizu $F$ ali tako da se problemi manje dimenzije rešavaju samo jedan put.
\end{enumerate}
  Napisati program koji testira napisane funkcije za poizvoljan broj $n$ ($n \in \mathbb N$) unet sa standardnog ulaza.
\end{Exercise}
\begin{Answer}[ref=106]
%\includecode{resenja/01_Rekurzija/106.c}
\end{Answer}

\begin{Exercise}[label=107]
Napisati rekurzivnu funkciju koja prikazuje sve permutacije skupa $\{1, 2, ... ,n\}$. Napisati program koji testira napisanu funkciju za poizvoljan prirodan broj $n$ ($n \le 50$) unet sa standardnog ulaza.
\end{Exercise}
\begin{Answer}[ref=107]
%\includecode{resenja/01_Rekurzija/107.c}
\end{Answer}

\begin{Exercise}[label=108]
 Paskalov trougao se dobija tako što mu je svako polje
(izuzev jedinica po krajevima) zbir jednog polja levo i
jednog polja iznad.
\begin{verbatim}
               1
             1   1
           1   2   1
         1   3   3   1
       1   4   6   4   1
     1   5   10  10  5  1
\end{verbatim}
\begin{enumerate}
\item Napisati rekurzivnu funkciju koja izračunava $d_n$ kao sumu elemenata n-te hipotenuze Paskalovog trougla.
\item Napisati rekurzivnu funkciju koja izračunava vrednost polja \argf{(i, j)}. \komentar{Milena: dodati sta je i a sta j tj odakle se broji}
\end{enumerate}

\end{Exercise}
\begin{Answer}[ref=108]
%\includecode{resenja/01_Rekurzija/108.c}
\end{Answer}

%%%%%%%%%%%%%%%%%%%%%%%%%
% Zadaci sa praktikuma - obavezni zadaci 
%%%%%%%%%%%%%%%%%%%%%%%%%

\begin{Exercise}[label=109]
Napisati rekurzivnu funkciju koja određuje maksimum niza
  celih brojeva. Napisati program koji testira ovu funkciju, za niz
  koji se unosi sa standardnog ulaza. Niz neće imati više od $256$ elemenata, i
  njegovi elementi se unose sve do kraja ulaza.
  
\begin{miditest}
\begin{test}{Test 1}
Ulaz:   3 2 1 4 21    
Izlaz:  21              
\end{test}
\end{miditest}
\begin{miditest}
\begin{test}{Test 2}
Ulaz:   2 -1 0 -5 -10
Izlaz:  2                
\end{test}
\end{miditest}

\begin{miditest}
\begin{test}{Test 3}
Ulaz:  1 11 3 5 8 1    
Izlaz:  11
\end{test}
\end{miditest}
\begin{minitest}
\begin{test}{Test 4}
Ulaz:   5
Izlaz:  5
\end{test}
\end{minitest}
\end{Exercise}
\begin{Answer}[ref=109]
%\includecode{resenja/01_Rekurzija/109.c}
\end{Answer}

\begin{Exercise}[label=110]
Napisati rekurzivnu funkciju koja izračunava skalarni
  proizvod dva data vektora.  Napisati program koji testira ovu
  funkciju, za nizove koji se unose sa standardnog ulaza. Nizovi neće imati više od $256$ elemenata. Prvo se unosi dimenzija nizova, a
  zatim i sami njihovi elementi.
  
\begin{miditest}
\begin{test}{Test 1}
Ulaz:    3 1 2 3 1 2 3 
Izlaz:   14                 
\end{test}
\end{miditest}
\begin{minitest}
\begin{test}{Test 2}
Ulaz:   2 3 5 2 6       
Izlaz:   36              
\end{test}
\end{minitest}


\begin{minitest}
\begin{test}{Test 3}
Ulaz:   0
Izlaz:   0
\end{test}
\end{minitest}
\end{Exercise}
\begin{Answer}[ref=110]
%\includecode{resenja/01_Rekurzija/110.c}
\end{Answer}


\begin{Exercise}[label=111]
Napisati rekurzivnu funkciju koja sabira dekadne cifre datog
  celog broja $x$. Napisati program koji testira ovu funkciju, za broj
  koji se unosi sa standardnog ulaza.
  
\begin{minitest}
\begin{test}{Test 1}
Ulaz:    123  
Izlaz:   6 
\end{test}
\end{minitest}
\begin{minitest}
\begin{test}{Test 2}
Ulaz:    23156    
Izlaz:   17 
\end{test}
\end{minitest}
\begin{minitest}
\begin{test}{Test 3}
Ulaz:   1432
Izlaz:   10        
\end{test}
\end{minitest}

\begin{minitest}
\begin{test}{Test 4}
Ulaz:   1       
Izlaz:  1       
\end{test}
\end{minitest}
\begin{minitest}
\begin{test}{Test 5}
Ulaz:   0
Izlaz:  0
\end{test}
\end{minitest}
\end{Exercise}
\begin{Answer}[ref=111]
%\includecode{resenja/01_Rekurzija/111.c}
\end{Answer}


\begin{Exercise}[label=112]
Napisati rekurzivnu funkciju koja računa broj pojavljivanja
  elementa $x$ u nizu $a$ dužine $n$. Napisati program koji testira
  ovu funkciju, za $x$ i niz koji se unose sa standardnog
  ulaza. Niz neće imati više od $256$ elemenata. Prvo se unosi $x$, a zatim
  elementi niza sve do kraja ulaza.
  
\begin{minitest}
\begin{test}{Test 1}
Ulaz:    4 1 2 3 4     
Izlaz:   1              
\end{test}
\end{minitest}
\begin{miditest}
\begin{test}{Test 2}
Ulaz:   11 3 2 11 14 11 43 1      
Izlaz:   2                      
\end{test}
\end{miditest}

\begin{minitest}
\begin{test}{Test 3}
Ulaz:  1 3 21 5 6
Izlaz:  0    
\end{test}
\end{minitest}
\end{Exercise}
\begin{Answer}[ref=112]
%\includecode{resenja/01_Rekurzija/112.c}
\end{Answer}


\begin{Exercise}[label=113]
Napisati rekurzivnu funkciju koja ispituje da li je data niska
  palindrom. Napisati program koji testira ovu funkciju. Pretposatviti
  da niska neće neće imati više od $31$ karaktera, i da se unosi sa standardnog
  ulaza.
  
\begin{miditest}
\begin{test}{Test 1}
Ulaz:    programiranje    
Izlaz:   ne                   
\end{test}
\end{miditest}
\begin{miditest}
\begin{test}{Test 2}
Ulaz:    anavolimilovana 
Izlaz:   da                   
\end{test}
\end{miditest}

\begin{minitest}
\begin{test}{Test 3}
Ulaz:    a        
Izlaz:   da 
\end{test}
\end{minitest}
\begin{minitest}
\begin{test}{Test 4}
Ulaz:    aba   
Izlaz:   da 
\end{test}
\end{minitest}
\begin{minitest}
\begin{test}{Test }
Ulaz:    aa
Izlaz:   da 
\end{test}
\end{minitest}
\end{Exercise}
\begin{Answer}[ref=113]
%\includecode{resenja/01_Rekurzija/113.c}
\end{Answer}


\begin{Exercise}[label=114]
Napisati rekurzivnu funkciju kojom se proverava da li su tri
  zadata broja uzastopni članovi niza. Potom, napisati program koji
  je testira. Sa standardnog ulaza se unose najpre tri tražena
  broja, a zatim elementi niza, sve do kraja ulaza. Pretpostaviti da
neće biti uneto više od $256$ brojeva.
  
\begin{miditest}
\begin{test}{Test 1}
Ulaz:     1 2 3 4 1 2 3 4 5 
Izlaz:    da                     
\end{test}
\end{miditest}
\begin{miditest}
\begin{test}{Test 2}
Ulaz:     1 2 3 11 1 2 4 3 6 
Izlaz:    ne                    
\end{test}
\end{miditest}

\begin{miditest}
\begin{test}{Test 3}
Ulaz:     1 2 3 1 2
Izlaz:    ne 
\end{test}
\end{miditest}
\end{Exercise}
\begin{Answer}[ref=114]
%\includecode{resenja/01_Rekurzija/114.c}
\end{Answer}

%%%%%%%%%%%%%%%%%%%%%%%%%
% Zadaci sa praktikuma - dodatni zadaci - oni nemaju rešenja
% možda \subsection{Dodatni zadaci} ili zadaci za vezbu
%%%%%%%%%%%%%%%%%%%%%%%%%



\begin{Exercise}[label=116]
Napisati rekurzivnu funkciju koja prikazuje sve varijacije sa
   ponavljanjem dužine $n$ skupa $\{a, b\}$, i program koji je
   testira, za $n$ koje se unosi sa standardnog ulaza.

\begin{miditest}
\begin{test}{Test 1}
Ulaz:    3
Izlaz:   a a a
         a a b
         a b a
         a b b
         b a a
         b a b
         b b a
         b b b
\end{test}
\end{miditest}
\end{Exercise}
%\begin{Answer}[ref=116]
%\includecode{resenja/01_Rekurzija/116.c}
%\end{Answer}

\begin{Exercise}[label=117]
{\em Hanojske kule}: Data su tri
  vertikalna štapa, na jednom se nalazi $n$ diskova poluprečnika
  $1$,$2$,$3$,... do $n$, tako da se najveći nalazi na dnu, a
  najmanji na vrhu. Ostala dva štapa su prazna. Potrebno je
  premestiti diskove na drugi štap tako da budu u istom redosledu, pri čemu se ni u jednom
  trenutku ne sme staviti veći disk preko manjeg, a preostali štap se koristi kao pomoćni štap prilikom
  premeštanja. \\
  Napisati program koji za proizvoljnu vrednost $n$, koja se unosi sa standardnog ulaza, prikazuje proces premeštanja diskova.

\end{Exercise}
%\begin{Answer}[ref=117]
%\includecode{resenja/01_Rekurzija/117.c}
%\end{Answer}

\begin{Exercise}[label=118]
{\em Modifikacija Hanojskih kula}: Data su četiri
  vertikalna štapa, na jednom se nalazi $n$ diskova poluprečnika
  $1$,$2$,$3$,... do $n$, tako da se najveći nalazi na dnu, a
  najmanji na vrhu. Ostala tri štapa su prazna. Potrebno je
  premestiti diskove na drugi štap tako da budu u istom redosledu,
  premestajući jedan po jedan disk, pri čemu se ni u jednom
  trenutku ne sme staviti veći disk preko manjeg, pri čemu se
  preostala dva štapa koriste kao pomoćni štapovi prilikom
  premeštanja.\\
  Napisati program koji za proizvoljnu vrednost $n$, koja se unosi sa standardnog ulaza, prikazuje proces premeštanja diskova.

\end{Exercise}
%\begin{Answer}[ref=118]
%\includecode{resenja/01_Rekurzija/118.c}
%\end{Answer}


\section{Rešenja}
\shipoutAnswer
